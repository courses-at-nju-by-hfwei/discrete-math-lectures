% relation-property.tex

%%%%%%%%%%%%%%%
\begin{frame}{}
  \begin{center}
    \teal{\Large 7 Properties}

    \[
      \red{R \subseteq X \times X}
    \]
  \end{center}
\end{frame}
%%%%%%%%%%%%%%%

%%%%%%%%%%%%%%%
\begin{frame}{}
  \begin{definition}[自反的 (Reflexive)]
    $R \subseteq X \times X$ is \red{\it reflexive} if
    \[
      \forall a \in X.\; (a, a) \in R
    \]

    \fig{width = 0.20\textwidth}{figs/reflexive}
  \end{definition}

  \pause
  \[
    \le \;\subseteq \R \times \R \text{ is reflexive}
  \]
  \pause
  \[
    \text{三角形上的\red{全等关系}是自反的}
  \]
\end{frame}
%%%%%%%%%%%%%%%

%%%%%%%%%%%%%%%
\begin{frame}{}
  \begin{definition}[反自反 (Irreflexive)]
    $R \subseteq X \times X$ is \red{\it irreflexive} if
    \[
      \forall a \in X.\; (a, a) \notin R
    \]
  \end{definition}

  \pause
  \[
    < \;\subseteq \R \times \R \text{ is irreflexive}
  \]
  \pause
  \[
    > \;\subseteq \R \times \R \text{ is irreflexive}
  \]
\end{frame}
%%%%%%%%%%%%%%%

%%%%%%%%%%%%%%%
\begin{frame}{}
  \[
    \blue{A = \set{1, 2, 3} \qquad R \subseteq A \times A}
  \]

  \[
    \set{(1, 1), (1, 3), (2, 1), (2, 2), (3, 3)}
  \]

  \pause
  \[
    \set{(1, 2), (2, 3), (3, 1)}
  \]

  \pause
  \[
    \set{(1, 2), (2, 2), (2, 3), (3, 1)}
  \]
\end{frame}
%%%%%%%%%%%%%%%

%%%%%%%%%%%%%%%
\begin{frame}{}
  \begin{definition}[对称 (Symmetric)]
    $R \subseteq X \times X$ is \red{\it symmetric} if
    \[
      \forall a, b \in X.\; a R b \to b R a
    \]

    \fig{width = 0.25\textwidth}{figs/symmetric}

    \pause
    \[
      \forall a, b \in X.\; a R b \leftrightarrow b R a
    \]
  \end{definition}
\end{frame}
%%%%%%%%%%%%%%%

%%%%%%%%%%%%%%%
\begin{frame}{}
  \[
    \blue{A = \set{1, 2, 3} \qquad R \subseteq A \times A}
  \]

  \pause
  \[
    \set{(1, 1), (1, 2), (1, 3), (2, 1), (3, 1), (3, 3)}
  \]

  \pause
  \[
    \set{(1, 2), (2, 3), (2, 2), (3, 1)}
  \]

  \pause
  \[
    \set{(1, 1), (2, 2), (3, 3)}
  \]

  \pause
  \[
    \set{(1, 2), (2, 1), (2, 3)}
  \]
\end{frame}
%%%%%%%%%%%%%%%

%%%%%%%%%%%%%%%
\begin{frame}{}
  \begin{definition}[反对称 (AntiSymmetric)]
    $R \subseteq X \times X$ is \red{\it antisymmetric} if
    \[
      \forall a, b \in X.\; (a R b \land b R a) \to a = b
    \]
  \end{definition}

  \pause
  \[
    \red{\ge}\; \text{is antisymmetric}
  \]
  \pause
  \[
    \red{\mid}\; \text{is antisymmetric}
  \]
\end{frame}
%%%%%%%%%%%%%%%

%%%%%%%%%%%%%%%
\begin{frame}{}
  \[
    \blue{A = \set{1, 2, 3} \qquad R \subseteq A \times A}
  \]

  \pause
  \[
    \set{(1, 1), (1, 2), (1, 3), (2, 1), (3, 1), (3, 3)}
  \]

  \pause
  \[
    \set{(1, 2), (2, 3), (2, 2), (3, 1)}
  \]

  \pause
  \[
    \set{(1, 1), (2, 2), (3, 3)}
  \]

  \pause
  \[
    \set{(1, 2), (2, 1), (2, 3)}
  \]
\end{frame}
%%%%%%%%%%%%%%%

%%%%%%%%%%%%%%%
\begin{frame}{}
  \begin{definition}[传递的 (Transitive)]
    $R \subseteq X \times X$ is \red{\it transitive} if
    \[
      \forall a, b, c \in X.\; (a R b \land b R c \to a R c)
    \]

    \fig{width = 0.30\textwidth}{figs/transitive}
  \end{definition}
\end{frame}
%%%%%%%%%%%%%%%

%%%%%%%%%%%%%%%
\begin{frame}{}
  \[
    \blue{A = \set{1, 2, 3} \qquad R \subseteq A \times A}
  \]

  \pause
  \[
    \set{(1, 1), (1, 2), (1, 3), (2, 1), (2, 2), (2, 3)}
  \]

  \pause
  \[
    \set{(1, 2), (2, 3), (3, 1)}
  \]

  \pause
  \[
    \set{(1, 3)}
  \]

  \pause
  \[
    \emptyset
  \]
\end{frame}
%%%%%%%%%%%%%%%

%%%%%%%%%%%%%%%
\begin{frame}{}
  \begin{definition}[连接的 (Connex)]
    $R \subseteq X \times X$ is \red{\it connex} if
    \[
      \forall a, b \in X.\; (a R b \lor b R a)
    \]
  \end{definition}

  \pause
  \vspace{0.60cm}
  \begin{definition}[三分的 (Trichotomous)]
    $R \subseteq X \times X$ is \red{\it trichotomous} if
    \[
      \forall a, b \in X.\;
        (\text{\red{exactly one of}}\; a R b, b R a, \;\text{or}\; a = b \text{ holds})
    \]
  \end{definition}
\end{frame}
%%%%%%%%%%%%%%%

%%%%%%%%%%%%%%%
\begin{frame}
  \begin{theorem}
    \[
      R \text{ is reflexive} \iff I \subseteq R
    \]
  \end{theorem}
  \[
    I = \set{(a, a) \in A \times A \mid a \in A}
  \]

  \pause
  \vspace{0.60cm}
  \begin{theorem}
    \[
      R \text{ is symmetric} \iff R^{-1} = R
    \]
  \end{theorem}
\end{frame}
%%%%%%%%%%%%%%%

%%%%%%%%%%%%%%%
\begin{frame}{}
  \begin{theorem}
    \[
      R \text{ is transitive} \iff R \circ R \;\red{\subseteq}\; R
    \]
  \end{theorem}

  \pause
  \[
    R = \set{(1, 2), (2, 3), (1, 3), (4, 4)}
  \]

  \pause
  \red{对任意 $(a, b)$,}
  \setcounter{equation}{0}
  \begin{align}
    &(a, b) \in R \circ R \\[6pt]
    \uncover<3->{\implies& \exists c.\; (a, c) \in R \land (b, c) \in R \\[6pt]}
    \uncover<4->{\implies& (a, b) \in R}
  \end{align}

  \vspace{0.30cm}
  \uncover<5->{
    \red{对任意 $a, b, c$}
    \[
      (a, b) \in R \land (b, c) \in R
      \uncover<6->{\implies (a, c) \in R \circ R}
      \uncover<7->{\implies (a, c) \in R}
    \]
  }
\end{frame}
%%%%%%%%%%%%%%%

%%%%%%%%%%%%%%%
\begin{frame}{}
  \begin{theorem}
    \[
      \teal{R \text{ is symmetric and transitive} \iff R = R^{-1} \circ R}
    \]
  \end{theorem}
\end{frame}
%%%%%%%%%%%%%%%