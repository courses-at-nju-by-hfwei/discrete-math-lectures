% relation-property.tex

%%%%%%%%%%%%%%%
\begin{frame}{}
  \begin{center}
    \teal{\Large 7 Properties}
  \end{center}
\end{frame}
%%%%%%%%%%%%%%%

%%%%%%%%%%%%%%%
\begin{frame}{}
  \[
    \red{R \subseteq X \times X}
  \]

  \begin{definition}[Reflexive]
    \[
      \forall a \in X: (a, a) \in R
    \]

    \fig{width = 0.20\textwidth}{figs/reflexive}
  \end{definition}

  \pause
  \vspace{0.30cm}
  \begin{definition}[Irreflexive]
    \[
      \forall a \in X: (a, a) \notin R
    \]
  \end{definition}
\end{frame}
%%%%%%%%%%%%%%%

%%%%%%%%%%%%%%%
\begin{frame}{}
  \[
    \blue{A = \set{1, 2, 3}, R \subseteq A \times A}
  \]

  \[
    \set{(1, 1), (1, 3), (2, 1), (2, 2), (3, 3)}
  \]

  \pause
  \[
    \set{(1, 2), (2, 3), (3, 1)}
  \]

  \pause
  \[
    \set{(1, 2), (2, 2), (2, 3), (3, 1)}
  \]
\end{frame}
%%%%%%%%%%%%%%%

%%%%%%%%%%%%%%%
\begin{frame}{}
  \[
    \red{R \subseteq X \times X}
  \]

  \begin{definition}[Symmetric]
    \[
      \forall a, b \in X: a R b \implies b R a
    \]

    \fig{width = 0.25\textwidth}{figs/symmetric}
  \end{definition}

  \pause
  \begin{definition}[AntiSymmetric]
    \[
      \forall a, b \in X: (a R b \land b R a) \implies a = b
    \]
  \end{definition}

  \pause
  \[
    > \pause \text{\red{\it is} antisymmetric}.
  \]
\end{frame}
%%%%%%%%%%%%%%%

%%%%%%%%%%%%%%%
\begin{frame}{}
  \[
    \blue{A = \set{1, 2, 3}, R \subseteq A \times A}
  \]

  \[
    \set{(1, 1), (1, 2), (1, 3), (2, 1), (3, 1), (3, 3)}
  \]

  \pause
  \[
    \set{(1, 2), (2, 3), (2, 2), (3, 1)}
  \]

  \pause
  \[
    \set{(1, 1), (2, 2), (3, 3)}
  \]

  \pause
  \[
    \set{(1, 2), (2, 1), (2, 3)}
  \]
\end{frame}
%%%%%%%%%%%%%%%

%%%%%%%%%%%%%%%
\begin{frame}{}
  \[
    \red{R \subseteq X \times X}
  \]

  \begin{definition}[Transitive]
    \[
      \forall a, b, c \in X: a R b \land b R c \implies a R c
    \]

    \fig{width = 0.30\textwidth}{figs/transitive}
  \end{definition}
\end{frame}
%%%%%%%%%%%%%%%

%%%%%%%%%%%%%%%
\begin{frame}{}
  \[
    \blue{A = \set{1, 2, 3}, R \subseteq A \times A}
  \]

  \[
    \set{(1, 1), (1, 2), (1, 3), (2, 1), (2, 2), (2, 3)}
  \]

  \pause
  \[
    \set{(1, 2), (2, 3), (3, 1)}
  \]

  \pause
  \[
    \set{(1, 3)}
  \]

  \pause
  \[
    \emptyset
  \]
\end{frame}
%%%%%%%%%%%%%%%

%%%%%%%%%%%%%%%
\begin{frame}{}
  \[
    \red{R \subseteq X \times X}
  \]

  \begin{definition}[Connex]
    \[
      \forall a, b \in X: a R b \lor b R a
    \]
  \end{definition}

  \pause
  \vspace{0.60cm}
  \begin{definition}[Trichotomous]
    \[
      \forall a, b \in X:\; \text{\red{exactly one of }}\; a R b, b R a, \text{ or } a = b \text{ holds}
    \]
  \end{definition}
\end{frame}
%%%%%%%%%%%%%%%

%%%%%%%%%%%%%%%
\begin{frame}
  \begin{theorem}
    \[
      R \text{ is reflexive} \iff I \subseteq R
    \]
  \end{theorem}
  \[
    I = \set{(a, a) \in A \times A \mid a \in A}
  \]

  \pause
  \begin{theorem}
    \[
      R \text{ is symmetric} \iff R^{-1} = R
    \]
  \end{theorem}

  \pause
  \vspace{0.30cm}
  \begin{theorem}
    \[
      R \text{ is transitive} \iff R \circ R \subseteq R
    \]
  \end{theorem}

  \pause
  \[
    (1, 2), (2, 3), (1, 3), (4, 4)
  \]

  % \pause
  % \begin{theorem}
  %   \[
  %     R \text{ is symmetric and transitive} \iff R = R^{-1} \circ R
  %   \]
  % \end{theorem}
\end{frame}
%%%%%%%%%%%%%%%
