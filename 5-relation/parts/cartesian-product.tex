% cartesian-product.tex

%%%%%%%%%%%%%%%
\begin{frame}{}
  \begin{definition}[笛卡尔积 (Cartesian Products)]
    The \red{\it Cartesian product} $A \times B$ \blue{of $A$ and $B$}
    is defined as
    \[
      {A \times B \triangleq \set{(a, b) \mid a \in A \land b \in B}}
    \]
  \end{definition}

  \pause
  \[
    X^2 \triangleq X \times X
  \]

  \pause
  \fig{width = 0.50\textwidth}{figs/product-xyz-123}
  % \[
  %   A = \set{1, 2} \qquad B = \set{a, b}
  % \]
  % \[
  %   A \times B = \set{(1, a), (1, b), (2, a), (2, b)}
  % \]
  % \[
  %   B \times A = \set{(a, 1), (a, 2), (b, 1), (b, 2)}
  % \]
\end{frame}
%%%%%%%%%%%%%%%

%%%%%%%%%%%%%%%
\begin{frame}{}
  \fig{width = 0.50\textwidth}{figs/product-coord}
  \[
    \Z^{2} \triangleq \Z \times \Z
  \]
\end{frame}
%%%%%%%%%%%%%%%

%%%%%%%%%%%%%%%
\begin{frame}{}
  \[
    \text{Ranks} = \set{2, \dots, 10, J, Q, K, A}
  \]
  \fig{width = 0.70\textwidth}{figs/product-deck}
  \fig{width = 0.30\textwidth}{figs/suits}
  % \[
  %   \text{Suits} = \set{\ding{171}, {\color{red}\ding{170}},
  %     \ding{168}, {\color{red}\ding{169}}}
  %     % \spadesuit, \red{\varheartsuit}, \clubsuit, \red{\vardiamondsuit}
  % \]
\end{frame}
%%%%%%%%%%%%%%%

%%%%%%%%%%%%%%%
\begin{frame}{}
  \[
    X \times \emptyset = \emptyset \times X
  \]

  \pause
  \[
    X \times Y \;\red{\neq}\; Y \times X
  \]

  \pause
  \[
    (X \times Y) \times Z \;\red{\neq}\; X \times (Y \times Z)
  \]
  \pause
  \[
    \cyan{A = \set{1} \qquad (A \times A) \times A \;\red{\neq}\; A \times (A \times A)}
  \]
\end{frame}
%%%%%%%%%%%%%%%

%%%%%%%%%%%%%%%
\begin{frame}{}
  \begin{theorem}[分配律 (Distributivity)]
    \[
      A \times (B \cap C) = (A \times B) \cap (A \times C)
    \]
    \[
      A \times (B \cup C) = (A \times B) \cup (A \times C)
    \]
    \[
      A \times (B \setminus C) = (A \times B) \setminus (A \times C)
    \]
  \end{theorem}

  \pause
  \fig{width = 0.60\textwidth}{figs/product-illustration}
\end{frame}
%%%%%%%%%%%%%%%

%%%%%%%%%%%%%%%
\begin{frame}{}
  \[
    A \times (B \cap C) = (A \times B) \cap (A \times C)
  \]

  \pause
  \vspace{0.50cm}
  \red{对任意\blue{有序对} $(a, b)$,}
  \setcounter{equation}{0}
  \begin{align}
    &(a, b) \in A \times (B \cap C) \\[6pt]
    \uncover<3->{\iff& a \in A \land b \in (B \cap C) \\[6pt]}
    \uncover<4->{\iff& a \in A \land b \in B \land b \in C \\[6pt]}
    \uncover<5->{\iff& (\blue{a \in A} \land b \in B) \land (\blue{a \in A} \land b \in C) \\[6pt]}
    \uncover<6->{\iff& (a, b) \in A \times B \land (a, b) \in A \times C \\[6pt]}
    \uncover<7->{\iff& (a, b) \in (A \times B) \cap (A \times C)}
  \end{align}
\end{frame}
%%%%%%%%%%%%%%%

%%%%%%%%%%%%%%%
\begin{frame}{}
  \begin{definition}[$n$-元笛卡尔积 ($n$-ary Cartesian Product)]
    \[
      X_{1} \times X_{2} \times X_{3} \triangleq (X_{1} \times X_{2}) \times X_{3}
    \]

    \pause
    \[
      X_{1} \times X_{2} \times \dots \times X_{n}
        \triangleq (X_{1} \times X_{2} \times \dots \times X_{n-1}) \times X_{n}
    \]
  \end{definition}

  \pause
  \vspace{0.60cm}
  \[
    X^{n} \triangleq \underbrace{X \times \dots \times X}_{n}
  \]

  \pause
  \vspace{0.30cm}
  \begin{center}
    \blue{多数情况下, 我们仅处理``二元关系'', 因此也仅使用``二元笛卡尔积''}
  \end{center}
\end{frame}
%%%%%%%%%%%%%%%