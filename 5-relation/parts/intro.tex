% overview.tex

%%%%%%%%%%%%%%%
\begin{frame}{}
  \fig{width = 0.85\textwidth}{figs/set-theory-aspects}
\end{frame}
%%%%%%%%%%%%%%%

%%%%%%%%%%%%%%%
% \begin{frame}{}
%   \fig{width = 0.20\textwidth}{figs/lamport}
%
%   \begin{columns}
%     \column{0.45\textwidth}
%       \fig{width = 0.85\textwidth}{figs/lamport-ordering}
%     \column{0.45\textwidth}
%       \fig{width = 0.80\textwidth}{figs/lamport-std}
%   \end{columns}
% \end{frame}
%%%%%%%%%%%%%%%

%%%%%%%%%%%%%%%
\begin{frame}{}
  \begin{center}
    \blue{\large 我的工作日常 $\dots$}
  \end{center}
  \begin{columns}
    \column{0.45\textwidth}
      \uncover<2->{
        \fig{width = 0.50\textwidth}{figs/scream}
      }
      \fig{width = 0.90\textwidth}{figs/consistency-axioms}
    \column{0.50\textwidth}
      \fig{width = 0.95\textwidth}{figs/simulation}
  \end{columns}
\end{frame}
%%%%%%%%%%%%%%%

%%%%%%%%%%%%%%%
\begin{frame}{}
  \begin{center}
    \uncover<3->{\red{\large \bf 离散数学学得好不好, \\[5pt]
      一个重要的衡量标准就是是否完成了这种转变}}
  \end{center}
  \begin{columns}
    \column{0.50\textwidth}
      \fig{width = 0.70\textwidth}{figs/dont-be-scared}
    \column{0.50\textwidth}
      \pause
      \fig{width = 0.80\textwidth}{figs/excitement}
  \end{columns}
\end{frame}
%%%%%%%%%%%%%%%

%%%%%%%%%%%%%%%
\begin{frame}{}
  \begin{center}
    \textcolor{brown}{\LARGE The Relational Data Model}

    \vspace{0.20cm}
    \uncover<2->{\blue{\Large --- 如何靠\red{``关系''}赢得图灵奖?}}
  \end{center}

  \begin{columns}
    \column{0.45\textwidth}
      \fig{width = 0.70\textwidth}{figs/codd-relational-paper}
      {\vspace{-0.20cm}\centerline{Codd@CACM'1970}\centerline{\teal{(Turing Award'1981)}}}
    \column{0.45\textwidth}
      \fig{width = 0.60\textwidth}{figs/Codd}{\centerline{Edgar F. Codd (1923 -- 2003)}}
  \end{columns}
\end{frame}
%%%%%%%%%%%%%%%

%%%%%%%%%%%%%%%
\begin{frame}{}
  \[
    \R: \text{实数集}
  \]

  \pause
  \begin{center}
    ``Near''关系: $|a - b| < 1$
  \end{center}

  \pause
  \[
    R = \set{(a, b) \mid |a - b| < 1}
  \]
  \[
    (0, 0.618) \;\red{\in R}\; \qquad (-0.618, 0.618) \;\red{\notin}\; R
  \]

  \pause
  \[
    \forall a \in X.\; (a, a) \in R
      \pause\qquad \text{\red{(自反性)}}
  \]
  \pause
  \[
    \forall a, b \in X.\; ((a, b) \in R \to (b, a) \in R)
      \pause\qquad \text{\red{(对称性)}}
  \]
  \pause
  \[
    \forall a, b, c \in X.\; ((a, b) \in R \land (b, c) \in R \to (a, c) \in R)
      \pause\qquad \text{\textcolor{gray}{(传递性)}}
  \]
  \pause
  \begin{center}
    \fbox{自反性 + 反对称性 = \red{\bf 相容关系}}
  \end{center}
\end{frame}
%%%%%%%%%%%%%%%

%%%%%%%%%%%%%%%
\begin{frame}{}
  \[
    X = \set{1, 2, 3, 4, 5, 6, 10, 12, 15, 20, 30, 60}
  \]

  \pause
  \vspace{0.30cm}
  \begin{center}
    $X$ 上的整除关系
  \end{center}
  \[
    R = \pause \set{(1, 2), \dots, (4, 12), \dots, (12, 60), \dots, (4, 60), \dots, (60, 60)}
  \]

  \pause
  \[
    \forall a \in X.\; (a, a) \in R
      \pause\qquad \text{\red{(自反性)}}
  \]
  \pause
  \[
    \forall a, b \in X.\; ((a, b) \in R \land (b, a) \in R \to a = b)
      \pause\qquad \text{\red{(反对称性)}}
  \]
  \pause
  \[
    \forall a, b, c \in X.\; ((a, b) \in R \land (b, c) \in R \to (a, c) \in R)
      \pause\qquad \text{\red{(传递性)}}
  \]
  \pause
  \begin{center}
    \fbox{自反性 + 反对称性 + 传递性 = \red{\bf 偏序关系}}
  \end{center}
\end{frame}
%%%%%%%%%%%%%%%

%%%%%%%%%%%%%%%
\begin{frame}{}
  \[
    X = \set{1, 2, 3, 4, 5, 6, 10, 12, 15, 20, 30, 60}
  \]

  \fig{width = 0.70\textwidth}{figs/60-poset}

  \pause
  \begin{center}
    ``偏序''严格刻画了人类对于\red{``序''}的直观理解
  \end{center}
\end{frame}
%%%%%%%%%%%%%%%

%%%%%%%%%%%%%%%
\begin{frame}{}
  \[
    \N: \text{自然数集}
  \]
  \pause
  \[
    \le \;= \set{(a, b) \mid a \le b}
  \]
  \pause
  \[
    \forall a \in X.\; (a, a) \in R
      \pause\qquad \text{\red{(自反性)}}
  \]
  \pause
  \[
    \forall a, b \in X.\; ((a, b) \in R \land (b, a) \in R \to a = b)
      \pause\qquad \text{\red{(反对称性)}}
  \]
  \pause
  \[
    \forall a, b, c \in X.\; ((a, b) \in R \land (b, c) \in R \to (a, c) \in R)
      \pause\qquad \text{\red{(传递性)}}
  \]
  \pause
  \[
    \forall a, b \in X.\; ((a, b) \in R \lor (b, a) \in R)
      \pause\qquad \text{\red{(连接性)}}
  \]
  \pause
  \begin{center}
    \fbox{自反性 + 反对称性 + 传递性 + 连接性 = \red{\bf 全序关系}}
  \end{center}
\end{frame}
%%%%%%%%%%%%%%%

%%%%%%%%%%%%%%%
\begin{frame}{}
  \begin{center}
    考虑无向图中的\blue{顶点}集合 \\[8pt]
  \end{center}

  \fig{width = 0.45\textwidth}{figs/component}

  \pause
  \begin{center}
    顶点间的``可达(Reachability)关系'':
    $R = \set{(a, b) \mid a \leadsto b}$
  \end{center}
\end{frame}
%%%%%%%%%%%%%%%

%%%%%%%%%%%%%%%
\begin{frame}{}
  \fig{width = 0.30\textwidth}{figs/component}

  \pause
  \[
    \forall a \in X.\; (a, a) \in R
      \pause\qquad \text{\red{(自反性)}}
  \]
  \pause
  \[
    \forall a, b \in X.\; ((a, b) \in R \to (b, a) \in R)
      \pause\qquad \text{\red{(对称性)}}
  \]
  \pause
  \[
    \forall a, b, c \in X.\; ((a, b) \in R \land (b, c) \in R \to (a, c) \in R)
      \pause\qquad \text{\red{(传递性)}}
  \]
  \pause
  \begin{center}
    \fbox{自反性 + 对称性 + 传递性 = \red{\bf 等价关系}}

    \pause
    \vspace{0.30cm}
    ``可达关系''将顶点划分成相互独立的``连通分量''
  \end{center}
\end{frame}
%%%%%%%%%%%%%%%

%%%%%%%%%%%%%%%
\begin{frame}{}
  \fig{width = 0.60\textwidth}{figs/go}
\end{frame}
%%%%%%%%%%%%%%%

%%%%%%%%%%%%%%%
% \begin{frame}{}
%   \begin{definition}[关系 (Relations)]
%     A \red{\it relation} $R$ \blue{from $A$ to $B$} is a subset of $A \times B$:
%     \[
%       {R \subseteq A \times B}
%     \]
%   \end{definition}
%
%   \pause
%   \vspace{0.30cm}
%   \begin{definition}[Cartesian Products]
%     The \red{\it Cartesian product} $A \times B$ of $A$ and $B$ is defined as
%     \[
%       {A \times B \triangleq \set{(a, b) \mid a \in A \land b \in B}}
%     \]
%   \end{definition}
%
%   \pause
%   \vspace{0.30cm}
%   \begin{theorem}[Ordered Pairs]
%     \[
%       (a, b) = (c, d) \iff a = c \land b = d
%     \]
%   \end{theorem}
% \end{frame}
%%%%%%%%%%%%%%%