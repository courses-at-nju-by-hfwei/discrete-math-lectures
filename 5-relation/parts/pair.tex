% pair.tex

%%%%%%%%%%%%%%%
\begin{frame}{}
  \begin{definition}[有序对\red{\bf 公理} (Ordered Pairs)]
    \[
      (a, b) = (c, d) \iff a = c \land b = d
    \]
  \end{definition}
\end{frame}
%%%%%%%%%%%%%%%

%%%%%%%%%%%%%%%
\begin{frame}{}
  \begin{definition}[Ordered Pairs \teal{(Norbert Wiener; 1914)}]
    \[
      (a, b) \;\red{\triangleq}\; \Bset{\bset{\set{a}, \emptyset}, \bset{\set{b}}}
    \]
  \end{definition}

  \fig{width = 0.40\textwidth}{figs/wiener}

  \pause
  \begin{theorem}
    \[
      (a, b) = (c, d) \iff a = c \land b = d
    \]
  \end{theorem}
\end{frame}
%%%%%%%%%%%%%%%

%%%%%%%%%%%%%%%
\begin{frame}{}
  \begin{definition}[Ordered Pairs \teal{(Kazimierz Kuratowski; 1921)}]
    \[
      (a, b) \;\red{\triangleq}\; \bset{\set{a}, \set{a, b}}
    \]
  \end{definition}

  \fig{width = 0.25\textwidth}{figs/Kuratowski}

  \pause
  \begin{theorem}
    \[
      (a, b) = (c, d) \iff a = c \land b = d
    \]
  \end{theorem}
\end{frame}
%%%%%%%%%%%%%%%

%%%%%%%%%%%%%%%
\begin{frame}{}
  \begin{theorem}
    \[
      (a, b) = (c, d) \iff a = c \land b = d
    \]
  \end{theorem}

  \pause
  \[
    \Big(\bset{\set{a}, \set{a, b}} = \bset{\set{c}, \set{c, d}}\Big) \iff (a = c \land b = d)
  \]

  \pause

  \begin{align*}
    \onslide<3->{&\bset{\set{a}, \set{a, b}} = \bset{\set{c}, \set{c, d}} \\[6pt]}
    \onslide<4->{\iff& (\set{a} = \set{c} \lor \set{a} = \set{c, d}) \land (\set{a, b} = \set{c} \lor \set{a, b} = \set{c, d}) \\[6pt]}
    \onslide<5->{\iff& (\set{a} = \set{c} \land \set{a, b} = \set{c}) \;\lor \\[3pt]
                     & (\set{a} = \set{c} \land \set{a, b} = \set{c, d}) \;\lor \\[3pt]
                     & (\set{a} = \set{c, d} \land \set{a, b} = \set{c}) \;\lor \\[3pt]
                     & (\set{a} = \set{c, d} \land \set{a, b} = \set{c, d})}
  \end{align*}

  % \begin{columns}
  %   \column{0.50\textwidth}
  %     \[
  %       \blue{\textsc{Case I}: a = b}
  %     \]
  %   \column{0.50\textwidth}
  %     \[
  %       \blue{\textsc{Case II}: a \neq b}
  %     \]
  % \end{columns}
\end{frame}
%%%%%%%%%%%%%%%

%%%%%%%%%%%%%%%
\begin{frame}{}
  \begin{definition}[$n$-元组 (n-ary tuples)]
    \[
      (x, y, z) \triangleq ((x, y), z)
    \]

    \pause
    \[
      (x_{1}, x_{2}, \dots, x_{n-1}, x_{n})
        \triangleq ((x_{1}, x_{2}, \dots, x_{n-1}), x_{n})
    \]
  \end{definition}

  \pause
  \vspace{0.60cm}
  \begin{theorem}
    \[
      (x_{1}, \dots, x_{n}) = (y_{1}, \dots, y_{n})
      \iff x_{1} = y_{1} \land \dots x_{n} = y_{n}
    \]
  \end{theorem}

  \pause
  \vspace{0.20cm}
  \begin{center}
    By mathematical induction.
  \end{center}

  \pause
  \vspace{0.30cm}
  \begin{center}
    \blue{多数情况下, 我们仅处理``二元关系'', 因此也仅使用``有序对''}
  \end{center}
\end{frame}
%%%%%%%%%%%%%%%