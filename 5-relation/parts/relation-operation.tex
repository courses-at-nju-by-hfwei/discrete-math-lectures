% relation-operation.tex

%%%%%%%%%%%%%%%
\begin{frame}{}
  \begin{center}
    \teal{\Large 5 Operations}
  \end{center}
\end{frame}
%%%%%%%%%%%%%%%

%%%%%%%%%%%%%%%
\begin{frame}{}
  \begin{definition}[Inverse]
    The {\it inverse} of $R$ is the \purple{relation}
    \[
      R^{-1} = \set{(a, b) \mid (b, a) \in R}
    \]
  \end{definition}

  \pause
  \vspace{0.50cm}
  \begin{theorem}
    \[
      (R^{-1})^{-1} = R
    \]
  \end{theorem}
  
  \pause
  \vspace{0.50cm}
  \begin{definition}[Restriction]
    The {\it restriction} of $R$ to $X$ is the \purple{relation}
    \[
      R|_{X} = \set{(a, b) \in R \mid \red{a \in X}}
    \]
  \end{definition}
\end{frame}
%%%%%%%%%%%%%%%

%%%%%%%%%%%%%%%
\begin{frame}{}
  \begin{definition}[Image]
    The {\it image} of $X$ under $R$ is the \teal{set}
    \[
      {R[X] = \set{\red{b \in \text{ran}(R)} \mid \exists a \in X: (a, b) \in R} \pause = \cyan{\text{ran}(R|_{X})}}
    \]
  \end{definition}

  \pause
  \vspace{0.50cm}
  \begin{definition}[Inverse Image]
    The {\it inverse image} of $Y$ under $R$ is the \teal{set}
    \[
      {R^{-1}[Y] = \set{\red{b \in \text{dom}(R)} \mid \exists b \in Y: (a, b) \in R} \pause = \cyan{\text{ran}(R^{-1}|_{Y})}}
    \]
  \end{definition}
\end{frame}
%%%%%%%%%%%%%%%

%%%%%%%%%%%%%%%
\begin{frame}{}
  \[
    R \subseteq A \times B \qquad X \subseteq A \qquad Y \subseteq B
  \]

  \pause
  \[
    R^{-1}[R[X]] \;\red{?}\; X
  \]

  \[
    R[R^{-1}[Y]] \;\red{?}\; Y
  \]

  \pause
  \vspace{0.60cm}
  \fig{width = 0.20\textwidth}{figs/be-careful}
\end{frame}
%%%%%%%%%%%%%%%

%%%%%%%%%%%%%%%
\begin{frame}{}
  \begin{theorem}
    \[
      R[X_1 \cup X_2] = R[X_1] \cup R[X_2]
    \]

    \[
      R[X_1 \cap X_2] \subseteq R[X_1] \cap R[X_2]
    \]

    \[
      R[X_1 \setminus X_2] \supseteq R[X_1] \setminus R[X_2]
    \]
  \end{theorem}

  \begin{align*}
    \onslide<2->{&\textcolor{white}{\iff}\; b \in R[X_1 \cup X_2] \\}
    \onslide<3->{&\iff \exists a \in X_1 \cup X_2: (a, b) \in R \\}
    \onslide<4->{&\iff \exists a \in X_1: (a, b) \in R \lor \exists a \in X_2: (a, b) \in R \\}
    \onslide<5->{&\iff b \in R[X_1] \lor b \in R[X_2] \\}
  \end{align*}
\end{frame}
%%%%%%%%%%%%%%%

%%%%%%%%%%%%%%%
\begin{frame}{}
  \begin{definition}[Composition]
    The {\it composition} of relations $R$ and $S$ is the \purple{relation}
    \[
      R \circ S = \set{(a, c) \mid \exists b: (a, b) \in S \land (b, c) \in R}
    \]
  \end{definition}

  \pause
  \[
    R = \set{(0, 1), (0, 2), (0, 3), (1, 1), (1, 2), (1,3), (2, 3)}
  \]

  \pause
  \vspace{-0.60cm}
  \[
    R \circ R = \set{\cdots}
  \]

  \pause
  \[
    \le \circ \le \;\red{=}\; \pause \blue{\le}
  \]

  \pause
  \vspace{-0.60cm}
  \[
    \le \circ \ge \;\red{=}\; \pause \blue{\mathbb{R} \times \mathbb{R}}
  \]
\end{frame}
%%%%%%%%%%%%%%%

%%%%%%%%%%%%%%%
\begin{frame}{}
  \begin{theorem}
    \[
      (R \circ S)^{-1} = S^{-1} \circ R^{-1}
    \]
  \end{theorem}

  \pause
  \[
    \red{(a, b) \in (R \circ S)^{-1}} \iff \cdots
  \]
  % \fig{width = 0.40\textwidth}{figs/}
\end{frame}
%%%%%%%%%%%%%%%

%%%%%%%%%%%%%%%
\begin{frame}{}
  \begin{theorem}
    \[
      (R \circ S) \circ T = R \circ (S \circ T)
    \]
  \end{theorem}

  \pause
  \[
    \red{(a, b) \in (R \circ S) \circ T} \iff \cdots
  \]
\end{frame}
%%%%%%%%%%%%%%%

%%%%%%%%%%%%%%%
\begin{frame}{}
  \begin{align*}
    &\textcolor{white}{\iff}\; \red{(a, b)} \in (R \circ S) \circ T \\
    \onslide<2->{&\iff \exists c: (a, c) \in T \land (c, b) \in R \circ S \\}
    \onslide<3->{&\iff \exists c: (a, c) \in T \land (\exists d: (c, d) \in S \land (d, b) \in R) \\}
    \onslide<4->{&\iff \exists d: \exists c: (a, c) \in T \land (c, d) \in S \land (d, b) \in R \\}
    \onslide<5->{&\iff \exists d: (\exists c: (a, c) \in T \land (c, d) \in S) \land (d, b) \in R \\}
    \onslide<6->{&\iff \exists d: (a, d) \in S \circ T \land (d, b) \in R \\}
    \onslide<7->{&\iff (a, b) \in R \circ (S \circ T)}
  \end{align*}
\end{frame}
%%%%%%%%%%%%%%%

%%%%%%%%%%%%%%%
\begin{frame}{}
  \fig{width = 0.60\textwidth}{figs/wulin-relation}

  \begin{center}
    {燕小六: ``帮我照顾好我七\red{舅姥爷}和我外甥女''}
  \end{center}
\end{frame}
%%%%%%%%%%%%%%%

%%%%%%%%%%%%%%%
\begin{frame}{}
  \begin{center}
    ``舅姥爷'': 姥姥的兄弟
  \end{center}

  \pause
  \[
    G = \set{(a,b): a \text{ 是}\; b \text{ 的舅姥爷}} 
  \]

  \pause
  \[
    M = \set{(a, b) \mid a \text{ is the mother of } b}
  \]
  \[
    B = \set{(a, b) \mid a \text{ is the brother of } b}
  \]

  \pause
  \[
    \blue{G = B \circ (M \circ M)}
  \]

  \pause
  \[
    \red{G = B \circ (M \circ M) = (B \circ M) \circ M}
  \]
\end{frame}
%%%%%%%%%%%%%%%
