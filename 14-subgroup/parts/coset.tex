% coset.tex

%%%%%%%%%%%%%%%
\begin{frame}
  \begin{definition}[Coset (陪集))]
    Suppose that $H \le G$.
    For $a \in G$,
    \[
      aH = \set{ah \mid h \in H}, \quad Ha = \set{ha \mid h \in H},
    \]
    is called the \red{left coset (左陪集)}
    and \red{right coset} of $H$ in $G$, respectively.
  \end{definition}
\end{frame}
%%%%%%%%%%%%%%%

%%%%%%%%%%%%%%%
\begin{frame}
  \[
    S_{3} = \set{(1), (1\; 2), (1\; 3), (2\; 3), (1\; 2\; 3), (1\; 3\; 2)}
  \]

  \[
    H = \set{(1), (1\; 2)} \le S_{3}
  \]

  \pause
  \begin{gather*}
    (1) H = H = (1\; 2) H \le S_{3} \\[6pt]
    (1\; 3) H = \set{(1\; 3), (1\; 2\; 3)} = (1\; 3)H \\[6pt]
    (2\; 3) H = \set{(2\; 3), (1\; 3\; 2)} = (2\; 3)H
  \end{gather*}
\end{frame}
%%%%%%%%%%%%%%%

%%%%%%%%%%%%%%%
\begin{frame}
  \[
    S_{3} = \set{(1), (1\; 2), (1\; 3), (2\; 3), (1\; 2\; 3), (1\; 3\; 2)}
  \]

  \[
    H = \set{(1), (1\; 2\; 3), (1\; 3\; 2)} \le S_{3}
  \]

  \pause
  \begin{gather*}
    (1)H = (1\;2\;3)H = (1\;3\;2)H = H \le S_{3} \\[6pt]
    (1\;2) H = (1\;3)H = (2\;3)H = \set{(1\;2), (1\;3), (2\;3)}
  \end{gather*}
\end{frame}
%%%%%%%%%%%%%%%

%%%%%%%%%%%%%%%
\begin{frame}{}
  \begin{theorem}
    Suppose that $H \le G$, $a, b \in G$.
    \begin{enumerate}[(1)]
      \item
        \[
          |aH| = |H| = |bH|
        \]
      \item
        \[
          a \in aH
        \]
      \item
        \[
          aH = H \iff a \in H \iff aH \le G
        \]
      \item
        \[
          aH = bH \iff a^{-1}b \in H
        \]
      \item
        \[
          \forall a, b \in G.\; (aH = bH) \lor (aH \cap bH = \emptyset)
        \]
    \end{enumerate}
  \end{theorem}
\end{frame}
%%%%%%%%%%%%%%%

%%%%%%%%%%%%%%%
\begin{frame}
  \[
    aH = bH \iff a^{-1}b \in H
  \]

  \pause
  \[
    \boxed{a^{-1}b \in H \iff a^{-1}b H = H}
  \]

  \pause
  \[
    aH = bH \implies a^{-1}aH = a^{-1}bH \implies a^{-1}bH = H \implies a^{-1}b \in H
  \]

  \pause
  \[
    a^{-1}b H = H \implies a(a^{-1}bH) = aH \implies bH = aH
  \]
\end{frame}
%%%%%%%%%%%%%%%

%%%%%%%%%%%%%%%
\begin{frame}
  \[
    \forall a, b \in G.\; (aH = bH) \lor (aH \cap bH = \emptyset)
  \]

  \pause
  \[
    \forall a, b \in G.\; (aH \cap bH \neq \emptyset \to aH = bH)
  \]

  \pause
  \vspace{0.30cm}
  \begin{center}
    Take any $g \in aH \cap bH$.
  \end{center}

  \pause
  \[
    \blue{\exists h_{1}, h_{2} \in H.\;
      (ah_{1} = g = ah_{2}) \land (h_{1}H = H = h_{2}H)}
  \]

  \pause
  \[
    aH = a(h_{1}H) = (ah_{1})H = (bh_{2})H = b(h_{2}H) = bH
  \]
\end{frame}
%%%%%%%%%%%%%%%

%%%%%%%%%%%%%%%
\begin{frame}
  \begin{center}
    A \cyan{balanced} \red{partition} of $G$ by its subgraph $H$
  \end{center}

  \fig{width = 0.50\textwidth}{figs/partition}
\end{frame}
%%%%%%%%%%%%%%%

%%%%%%%%%%%%%%%
\begin{frame}
  \begin{theorem}[Lagrange's Theorem]
    Suppose that $H \le G$. Then
    \[
      |G| = [G : H] \cdot |H|
    \]
  \end{theorem}

  \vspace{0.50cm}
  \begin{definition}[Index (指标)]
    \[
      G/H = \set{gH \mid g \in G}
    \]

    \[
      [G : H] \triangleq |G/H|
    \]
  \end{definition}
\end{frame}
%%%%%%%%%%%%%%%

%%%%%%%%%%%%%%%
\begin{frame}
  \[
    H \le G \implies |H| \mid |G|
  \]

  \pause
  \vspace{0.50cm}
  \begin{center}
    There are \red{\it no} subgraphs of order 5, 7, or 8 of a group of order 12.
  \end{center}

  \pause
  \vspace{0.50cm}
  \begin{theorem}
    \begin{itemize}
      \item There are only 2 groups of order 4.
      \item There are only 2 groups of order 6.
    \end{itemize}
  \end{theorem}
\end{frame}
%%%%%%%%%%%%%%%