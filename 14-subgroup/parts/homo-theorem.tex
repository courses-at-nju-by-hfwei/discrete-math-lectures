% homo-theorem.tex

%%%%%%%%%%%%%%%
\begin{frame}
  \begin{definition}[Homomorphism (同态)]
    Let $(G, \cdot)$ and $(G', \ast)$ be two groups.
    Let $\phi$ be a function such that
    \[
      \forall a, b \in G.\; \phi(ab) = \phi(a)\phi(b).
    \]
    Then $\phi$ is a \red{homomorphism} from $G$ to $G'$.
  \end{definition}

  \pause
  \vspace{0.50cm}
  \begin{center}
    If $\phi$ is a bijection, then $G$ and $G'$ are called \red{isomorphic}.
    \[
      \phi: G \cong G'
    \]
  \end{center}
\end{frame}
%%%%%%%%%%%%%%%

%%%%%%%%%%%%%%%
% \begin{frame}{}
%   \begin{exampleblock}{}
%     \[
%       (\R, +) \cong (\R^{+}, \ast)
%     \]
%
%     \[
%       \phi(x) = e^{x}
%     \]
%   \end{exampleblock}
% \end{frame}
%%%%%%%%%%%%%%%

%%%%%%%%%%%%%%%
% \begin{frame}{}
%   \begin{exampleblock}{Klein Four-group (四元群; $K_{4}$)}
%     \fig{width = 0.30\textwidth}{figs/4group}
%     \[
%       a^2 = b^2 = c^2 = (ab)^{2} = e
%     \]
%     \[
%       ab = c = ba \quad ac = b = ca \quad bc = a = cb
%     \]
%
%     \pause
%     \[
%       U(8) = \set{1, 3, 5, 7}
%     \]
%   \end{exampleblock}
% \end{frame}
%%%%%%%%%%%%%%%

%%%%%%%%%%%%%%%
\begin{frame}{}
  \begin{align*}
    \phi:\; & \Z \to \R^{\ast} \\[6pt]
            & n \mapsto (-1)^{n}
  \end{align*}

  \pause
  \[
    \phi(m + n) = (-1)^{m+n} = \phi(m) \phi(n)
  \]
\end{frame}
%%%%%%%%%%%%%%%

%%%%%%%%%%%%%%%
\begin{frame}{}
  \begin{align*}
    \phi:\; & \Z \to \Z_{6} \\
            & a \mapsto [a]_{6}
  \end{align*}

  \[
    \phi(a + b) = [a + b]_{6} = \phi(a) + \phi(b)
  \]
\end{frame}
%%%%%%%%%%%%%%%

%%%%%%%%%%%%%%%
\begin{frame}{}
  \begin{align*}
    \phi:\; & \R[x] \to \R[x] \\
            & f(x) \mapsto f'(x)
  \end{align*}

  \vspace{0.30cm}
  \begin{center}
    $\R[x]:$ 全体实系数多项式关于多项式的加法构成的群
  \end{center}

  \[
    \phi(f(x) + g(x)) = (f(x) + g(x))' = \phi(f(x)) + \phi(g(x))
  \]
\end{frame}
%%%%%%%%%%%%%%%

%%%%%%%%%%%%%%%
\begin{frame}{}
  \begin{theorem}
    Suppose that \blue{$\phi$ is a homomorphism from $G$ to $G'$}. \\
    Let $e$ and $e'$ are identities of $G$ and $G'$, respectively.

    \begin{enumerate}[(1)]
      \setlength{\itemsep}{6pt}
      \item $\phi(e) = e'$
      \item \teal{$\phi(a^{-1}) = (\phi(a))^{-1}$}
    \end{enumerate}
  \end{theorem}

  \pause
  \[
    e' \phi(e) = \phi(e) = \phi(e e) = \phi(e) \phi(e) \implies \phi(e) = e'
  \]

  \pause
  \[
    \phi(a)\phi(a^{-1}) = \phi(aa^{-1}) = \phi(e) = e' = \phi(a)(\phi(a))^{-1}
  \]
\end{frame}
%%%%%%%%%%%%%%%

%%%%%%%%%%%%%%%
\begin{frame}
  \begin{theorem}
    Suppose that \blue{$\phi$ is a homomorphism from $G$ to $G'$}. \\
    \begin{enumerate}[(1)]
      \item
        \[
          H \le G \implies \phi(H) \le G'
        \]
      \item
        \[
          H \triangleleft G \implies \phi(H) \triangleleft G'
        \]
      \pause
      \item
        \[
          K \le G' \implies \phi^{-1}(K) \le G
        \]
      \item
        \[
          K \triangleleft G' \implies \phi^{-1}(K) \triangleleft G
        \]
    \end{enumerate}
  \end{theorem}
\end{frame}
%%%%%%%%%%%%%%%

%%%%%%%%%%%%%%%
\begin{frame}
  \begin{definition}[核 (Kernel)]
    Suppose that \blue{$\phi$ is a homomorphism from $G$ to $G'$}. \\
    Let $e'$ be the identity of $G'$.
    \[
      \phi^{-1}(\set{e'}) = \set{a \in G \mid \phi(a) = e'}
    \]
    is the \red{kernel} of $\phi$, denoted $\text{Ker}\; \phi$.
  \end{definition}

  \pause
  \[
    \text{Ker}\; \phi \triangleleft G
  \]
\end{frame}
%%%%%%%%%%%%%%%

%%%%%%%%%%%%%%%
\begin{frame}{}
  \begin{align*}
    \phi:\; & \Z \to \R^{\ast} \\[6pt]
            & n \mapsto (-1)^{n}
  \end{align*}

  \[
    \text{Ker}\; \phi = \pause 2\Z
  \]
\end{frame}
%%%%%%%%%%%%%%%

%%%%%%%%%%%%%%%
\begin{frame}{}
  \begin{align*}
    \phi:\; & \Z \to \Z_{6} \\[6pt]
            & a \mapsto [a]_{6}
  \end{align*}

  \[
    \text{Ker}\; \phi = \pause 6\Z
  \]
\end{frame}
%%%%%%%%%%%%%%%

%%%%%%%%%%%%%%%
\begin{frame}{}
  \begin{align*}
    \phi:\; & \R[x] \to \R[x] \\
            & f(x) \mapsto f'(x)
  \end{align*}

  \vspace{0.30cm}
  \begin{center}
    $\R[x]:$ 全体实系数多项式关于多项式的加法构成的群
  \end{center}

  \[
    \text{Ker}\; \phi = \pause \R
  \]
\end{frame}
%%%%%%%%%%%%%%%

%%%%%%%%%%%%%%%
\begin{frame}
  \begin{theorem}[Fundamental Homomorphism Theorem]
    Suppose that \blue{$\phi$ is a homomorphism from $G$ to $G'$}. Then
    \[
      G/\text{Ker}\; \phi \cong \phi(G).
    \]
  \end{theorem}

  \pause
  \begin{center}
    \red{同态核可以看作群 $G$ 与其同态像 $\phi(G)$ 之间相似程度的一种度量}
  \end{center}

  \pause
  \fig{width = 0.35\textwidth}{figs/homo-theorem}
  \[
    N = \text{Ker}\; \phi
  \]
\end{frame}
%%%%%%%%%%%%%%%

%%%%%%%%%%%%%%%
\begin{frame}
  \[
    G/(N \triangleq \text{Ker}\; h) \cong (h(G) \triangleq \text{im}\; h)
  \]

  \fig{width = 0.70\textwidth}{figs/homo-theorem-proof}

  \vspace{-0.80cm}
  \[
    aN \mapsto h(a)
  \]
\end{frame}
%%%%%%%%%%%%%%%

%%%%%%%%%%%%%%%
\begin{frame}{}
  \begin{align*}
    \phi:\; & \Z \to \R^{\ast} \\[6pt]
            & n \mapsto (-1)^{n}
  \end{align*}

  \[
    \text{Ker}\; \phi = 2\Z
  \]

  \pause
  \[
    \Z/(2\Z) = (2\Z, 2\Z + 1) \cong \phi(\Z) = (-1, 1)
  \]
\end{frame}
%%%%%%%%%%%%%%%

%%%%%%%%%%%%%%%
\begin{frame}{}
  \begin{align*}
    \phi:\; & \Z \to \Z_{6} \\[6pt]
            & a \mapsto [a]_{6}
  \end{align*}

  \[
    \text{Ker}\; \phi = 6\Z
  \]

  \pause
  \[
    \Z/(6\Z) = \set{0 + H, 1 + H, \dots, 5 + H} \cong \phi(\Z) = \Z_{6}
  \]
\end{frame}
%%%%%%%%%%%%%%%

%%%%%%%%%%%%%%%
\begin{frame}{}
  \begin{align*}
    \phi:\; & \R[x] \to \R[x] \\
            & f(x) \mapsto f'(x)
  \end{align*}

  \vspace{0.30cm}
  \begin{center}
    $\R[x]:$ 全体实系数多项式关于多项式的加法构成的群
  \end{center}

  \[
    \text{Ker}\; \phi = \R
  \]

  \pause
  \[
    \R[x]/\R \cong \phi(\R[x]) = \R[x]
  \]
\end{frame}
%%%%%%%%%%%%%%%