% variants.tex

%%%%%%%%%%%%%%%%%%%%%%%%%%%%%%
\begin{frame}{}
  \begin{theorem}
    对于任何自然数 $n$, $13^{n}$ 都可以写成两个自然数的平方之和。
  \end{theorem}

  \pause
  \begin{eqnarray*}
    13^{n+1} &=& 13 \cdot \red{13^n} \\
    &=& (2^2+3^2)(\red{a^2 + b^2}) \\
    &=& \blue{(\underbrace{2a+3b}_x)^2 + (\underbrace{3a-2b}_y)^2} \\
    &=& x^2+y^2
  \end{eqnarray*}
\end{frame}
%%%%%%%%%%%%%%%%%%%%%%%%%%%%%%

%%%%%%%%%%%%%%%%%%%%%%%%%%%%%%
\begin{frame}{}
  \begin{theorem}
    对于任何自然数 $n$, $13^{n}$ 都可以写成两个自然数的平方之和。
  \end{theorem}

  \pause
  \[
    13^0 = 1^2 + 0^2
  \]
  \pause
  \[
    13^1 = 2^2 + 3^2
  \]
  \pause
  \begin{eqnarray*}
    13^{n+2} &=& 13^2\cdot \red{13^n} \\
    &=& 13^2(\red{a^2 + b^2}) \\
    &=& (\underbrace{13a}_x)^2 + (\underbrace{13b}_y)^2 \\
    &=& x^2+y^2
  \end{eqnarray*}
\end{frame}
%%%%%%%%%%%%%%%%%%%%%%%%%%%%%%

%%%%%%%%%%%%%%%%%%%%%%%%%%%%%%
% \begin{frame}{}
%   \begin{exampleblock}{}
%     \begin{align*}
%       f(1, 1) &= 2 \\
%       f(m+1, n) &= f(m,n) + 2(m+n) \\
%       f(m, n+1) &= f(m,n) + 2(m+n-1)
%     \end{align*}
%     请证明,
%     \[
%       \forall m, n \in \N^{+}.\; f(m,n) = (m+n)^{2} - (m+n) - 2n + 2
%     \]
%   \end{exampleblock}
%
%   \pause
%   \begin{center}
%     \red{Double Induction (Induciton Twice)}
%
%     \pause
%     \[
%       f(1, 1)
%     \]
%     \pause
%     \[
%       f(k, 1) \to f(k+1, 1)
%     \]
%     \pause
%     \[
%       f(\red{h}, k) \to f(\red{h}, k+1) \text{ for any } h
%     \]
%   \end{center}
% \end{frame}
%%%%%%%%%%%%%%%%%%%%%%%%%%%%%%

%%%%%%%%%%%%%%%%%%%%%%%%%%%%%%
% \begin{frame}{}
%   \begin{exampleblock}{}
%     \begin{align*}
%       f(1, 1) &= 2 \\
%       f(m+1, n) &= f(m,n) + 2(m+n) \\
%       f(m, n+1) &= f(m,n) + 2(m+n-1)
%     \end{align*}
%     请证明,
%     \[
%       \forall m, n \in \N^{+}.\; f(m,n) = (m+n)^{2} - (m+n) - 2n + 2
%     \]
%   \end{exampleblock}
%
%   \pause
%   \begin{center}
%     \red{对 $m + n$ 作归纳}
%   \end{center}
% \end{frame}
%%%%%%%%%%%%%%%%%%%%%%%%%%%%%%

%%%%%%%%%%%%%%%%%%%%%%%%%%%%%%
% \begin{frame}{}
%   \begin{exampleblock}{Josephus Problem}
%     \fig{width = 0.50\textwidth}{figs/Josephus}
%
%     \[
%       J(12) = 9
%     \]
%   \end{exampleblock}
% \end{frame}
%%%%%%%%%%%%%%%%%%%%%%%%%%%%%%

%%%%%%%%%%%%%%%%%%%%%%%%%%%%%%
% \begin{frame}{}
%   \begin{columns}
%     \column{0.50\textwidth}
%       \begin{center}
%         $2n$ 个人
%       \end{center}
%       \fig{width = 0.80\textwidth}{figs/2n-people}
%
%       \pause
%       \[
%         J(2n) = 2J(n) - 1, n \ge 1
%       \]
%     \column{0.50\textwidth}
%       \pause
%       \begin{center}
%         $2n + 1$ 个人
%       \end{center}
%       \fig{width = 0.80\textwidth}{figs/2n+1-people}
%
%       \pause
%       \[
%         J(2n + 1) = 2J(n) + 1, n \ge 1
%       \]
%   \end{columns}
% \end{frame}
%%%%%%%%%%%%%%%%%%%%%%%%%%%%%%

%%%%%%%%%%%%%%%%%%%%%%%%%%%%%%
% \begin{frame}{}
%   \begin{align*}
%     J(1) &= 1 \\[8pt]
%     J(2n) &= 2J(n) - 1, n \ge 1 \\[8pt]
%     J(2n + 1) &= 2J(n) + 1, n \ge 1
%   \end{align*}
%
%   \pause
%   \begin{columns}
%     \column{0.50\textwidth}
%       \fig{width = 0.55\textwidth}{figs/concrete-mathematics-book}
%     \column{0.50\textwidth}
%       \fig{width = 0.70\textwidth}{figs/knuth}
%   \end{columns}
% \end{frame}
%%%%%%%%%%%%%%%%%%%%%%%%%%%%%%

%%%%%%%%%%%%%%%%%%%%%%%%%%%%%%
\begin{frame}{}
  \[
    m \in \N \qquad n \in \N
  \]
  \[
    a(m, n) = \begin{cases}
      0, & m = n = 0 \\[5pt]
      a(m-1, n) + 1, & n = 0 \land m > 0 \\[5pt]
      a(m, n-1) + n. & n > 0
    \end{cases}
  \]

  \[
    a(m, n) = m + n (n + 1)/2
  \]

  \pause
  \quad \red{对 $n$ 作归纳。}
  \pause
  \begin{description}
    \setlength{\itemsep}{6pt}
    \item[基础步骤:] \uncover<4->{$n = 0$。要证 $\forall m.\; a(m, 0) = m$。} \\[6pt]
      \uncover<5->{\blue{对 $m$ 作归纳。} (\purple{\bf Double Induction})}
    \item[归纳假设:] \uncover<6->{
      假设 $n = k$ 时, $\forall m.\; a(m, k) = m + k(k+1)/2$。}
    \item[归纳步骤:] \uncover<7->{
      考虑 $n = k + 1$。\\[6pt]
      要证 $\forall m.\; a(m, k + 1) = m + (k+1)(k+2)/2$。}
  \end{description}
\end{frame}
%%%%%%%%%%%%%%%%%%%%%%%%%%%%%%