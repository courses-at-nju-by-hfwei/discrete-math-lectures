% variants.tex

%%%%%%%%%%%%%%%%%%%%%%%%%%%%%%
\begin{frame}{}
  \begin{theorem}
    对于任何自然数 $n$, $13^{n}$ 都可以写成两个自然数的平方之和。
  \end{theorem}

  \begin{eqnarray*}
    13^{n+1} &=& 13 \cdot \red{13^n} \\
    &=& (2^2+3^2)(\red{a^2 + b^2}) \\
    &=& \blue{(\underbrace{2a+3b}_x)^2 + (\underbrace{3a-2b}_y)^2} \\
    &=& x^2+y^2
  \end{eqnarray*}
\end{frame}
%%%%%%%%%%%%%%%%%%%%%%%%%%%%%%

%%%%%%%%%%%%%%%%%%%%%%%%%%%%%%
\begin{frame}{}
  \begin{theorem}
    对于任何自然数 $n$, $13^{n}$ 都可以写成两个自然数的平方之和。
  \end{theorem}

  \pause
  \[
    13^0 = 1^2 + 0^2
  \]
  \pause
  \[
    13^1 = 2^2 + 3^2
  \]
  \pause
  \begin{eqnarray*}
    13^{n+2} &=& 13^2\cdot \red{13^n} \\
    &=& 13^2(\red{a^2 + b^2}) \\
    &=& (\underbrace{13a}_x)^2 + (\underbrace{13b}_y)^2 \\
    &=& x^2+y^2
  \end{eqnarray*}
\end{frame}
%%%%%%%%%%%%%%%%%%%%%%%%%%%%%%

%%%%%%%%%%%%%%%%%%%%%%%%%%%%%%
\begin{frame}{}
  \begin{exampleblock}{Josephus Problem}
    \fig{width = 0.50\textwidth}{figs/Josephus}
  \end{exampleblock}
\end{frame}
%%%%%%%%%%%%%%%%%%%%%%%%%%%%%%

%%%%%%%%%%%%%%%%%%%%%%%%%%%%%%
\begin{frame}{}
  \begin{exampleblock}{}
    \begin{align*}
      f(1, 1) &= 2 \\
      f(m+1, n) &= f(m,n) + 2(m+n) \\
      f(m, n+1) &= f(m,n) + 2(m+n-1)
    \end{align*}
    请证明,
    \[
      \forall m, n \in \N^{+}.\; f(m,n) = (m+n)^{2} - (m+n) - 2n + 2
    \]
  \end{exampleblock}

  \pause
  \begin{center}
    \red{Double Induction (Induciton Twice)}

    \pause
    \[
      f(1, 1)
    \]
    \pause
    \[
      f(k, 1) \to f(k+1, 1)
    \]
    \pause
    \[
      f(\red{h}, k) \to f(\red{h}, k+1) \text{ for any } h
    \]
  \end{center}
\end{frame}
%%%%%%%%%%%%%%%%%%%%%%%%%%%%%%

%%%%%%%%%%%%%%%%%%%%%%%%%%%%%%
\begin{frame}{}
  \begin{exampleblock}{}
    \begin{align*}
      f(1, 1) &= 2 \\
      f(m+1, n) &= f(m,n) + 2(m+n) \\
      f(m, n+1) &= f(m,n) + 2(m+n-1)
    \end{align*}
    请证明,
    \[
      \forall m, n \in \N^{+}.\; f(m,n) = (m+n)^{2} - (m+n) - 2n + 2
    \]
  \end{exampleblock}

  \pause
  \begin{center}
    \red{对 $m + n$ 作归纳}
  \end{center}
\end{frame}
%%%%%%%%%%%%%%%%%%%%%%%%%%%%%%

%%%%%%%%%%%%%%%%%%%%%%%%%%%%%%
\begin{frame}{}
  gcd
\end{frame}
%%%%%%%%%%%%%%%%%%%%%%%%%%%%%%