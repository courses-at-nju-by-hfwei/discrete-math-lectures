% induction.tex

%%%%%%%%%%%%%%%%%%%%%%%%%%%%%%
\begin{frame}{}
  \begin{theorem}[第一数学归纳法 (The First Mathematical Induction)]
    设 $P(n)$ 是关于自然数的一个性质。
    如果
    \begin{enumerate}[(i)]
      \setlength{\itemsep}{8pt}
      \item $P(0)$ 成立;
      \item 对任意自然数 $n$, 如果 $P(n)$ 成立, 则 $P(n+1)$ 成立。
    \end{enumerate}
    \vspace{0.20cm}
    那么, $P(n)$ 对所有自然数 $n$ 都成立。
  \end{theorem}

  \pause
  \[
    \nd{P(0) \qquad \forall n \in \N.\; \Big(P(n) \to P(n+1) \Big)}{
      \forall n \in \N.\; P(n)}{\text{第一数学归纳法}}
  \]

  \pause
  \[
    \biggl(P(0) \land \forall n \in \N.\; \Big(P(n) \to P(n+1) \Big) \biggr)
	  \to \forall n \in \N.\; P(n).
  \]
\end{frame}
%%%%%%%%%%%%%%%%%%%%%%%%%%%%%%

%%%%%%%%%%%%%%%%%%%%%%%%%%%%%%
\begin{frame}{}
  \begin{theorem}[第二数学归纳法 (The Second Mathematical Induction)]
    设 $Q(n)$ 是关于自然数的一个性质。
    如果
    \begin{enumerate}[(i)]
      \setlength{\itemsep}{8pt}
      \item $Q(0)$ 成立;
      \item 对任意自然数 $n$, 如果 $Q(0), Q(1), \dots, Q(n)$ 都成立, \\ 则 $Q(n+1)$ 成立。
    \end{enumerate}
    \vspace{0.20cm}
    那么, $Q(n)$ 对所有自然数 $n$ 都成立。
  \end{theorem}

  \pause
  \[
    \nd{Q(0) \;\; \forall n \in \N.\; \Big(\big(Q(0) \land \dots \land Q(n)\big) \to Q(n+1) \Big)}{
      \forall n \in \N.\; Q(n)}{\text{第二数学归纳法}}
  \]

  \pause
  \[
    \biggl(Q(0) \land \forall n \in \N.\; \Big(\big(Q(0) \land \cdots \land Q(n)\big) \to Q(n+1) \Big) \biggr)
      \to \forall n \in \N.\; Q(n).
  \]
\end{frame}
%%%%%%%%%%%%%%%%%%%%%%%%%%%%%%

%%%%%%%%%%%%%%%%%%%%%%%%%%%%%%
\begin{frame}{}
  \begin{theorem}[数学归纳法]
    第一数学归纳法与第二数学归纳法等价。
  \end{theorem}

  \pause
  \vspace{0.50cm}
  \begin{center}
    {\red{$Q:$} 第二数学归纳法也被称为\red{\bf ``强'' (Strong)} 数学归纳法, 它强在何处?}
  \end{center}
\end{frame}
%%%%%%%%%%%%%%%%%%%%%%%%%%%%%%

%%%%%%%%%%%%%%%%%%%%%%%%%%%%%%
\begin{frame}{}
  \begin{lemma}
    第二数学归纳法蕴含第一数学归纳法。
  \end{lemma}

  \pause
  \[
    Q(n) \triangleq P(n)
  \]
\end{frame}
%%%%%%%%%%%%%%%%%%%%%%%%%%%%%%

%%%%%%%%%%%%%%%%%%%%%%%%%%%%%%
\begin{frame}{}
  \begin{lemma}
    第一数学归纳法蕴含第二数学归纳法。
  \end{lemma}

  \pause
  \[
    P(n) \triangleq Q(0) \land \cdots \land Q(n)
  \]
\end{frame}
%%%%%%%%%%%%%%%%%%%%%%%%%%%%%%

%%%%%%%%%%%%%%%%%%%%%%%%%%%%%%
\begin{frame}{}
  \begin{center}
    \red{\bf \large 数学归纳法为何成立?}

    \fig{width = 0.40\textwidth}{figs/why}
  \end{center}
\end{frame}
%%%%%%%%%%%%%%%%%%%%%%%%%%%%%%

%%%%%%%%%%%%%%%%%%%%%%%%%%%%%%
\begin{frame}{}
  \begin{center}
    Peano 公理体系刻画了\red{\bf 自然数的递归结构}
  \end{center}

  \begin{definition}[Peano Axioms]
    \begin{enumerate}[(1)]
      \setlength{\itemsep}{6pt}
      \item 0 是自然数;
      \item 如果 $n$ 是自然数, 则它的后继 ${\bf S}n$ 也是自然数;
      \item 0 不是任何自然数的后继;
      \item 两个自然数相等当且仅当它们的后继相等;
      \item \red{\bf 数学归纳原理:} 如果
        \begin{enumerate}[(i)]
          \setlength{\itemsep}{8pt}
          \item $P(0)$ 成立;
          \item 对任意自然数 $n$, 如果 $P(n)$ 成立, 则 $P(n+1)$ 成立。
        \end{enumerate}
        那么, $P(n)$ 对所有自然数 $n$ 都成立。
    \end{enumerate}
  \end{definition}
\end{frame}
%%%%%%%%%%%%%%%%%%%%%%%%%%%%%%

%%%%%%%%%%%%%%%%%%%%%%%%%%%%%%
\begin{frame}{}
  \begin{definition}[良序原理 (The Well-Ordering Principle)]
    \red{自然数集}的任意\blue{非空}子集都有一个最小元。
  \end{definition}

  \pause
  \vspace{0.60cm}
  \begin{theorem}{}
    良序原理与(第一)数学归纳法等价。
  \end{theorem}
\end{frame}
%%%%%%%%%%%%%%%%%%%%%%%%%%%%%%

%%%%%%%%%%%%%%%%%%%%%%%%%%%%%%
\begin{frame}{}
  \begin{lemma}
    (第一)数学归纳法蕴含良序原理。
  \end{lemma}

  \pause
  \begin{proof}
    \begin{center}
      \red{By mathematical induction on the size $n$ of non-empty subsets of $\mathbb{N}$.}

      \vspace{-0.30cm}
      \[
        P(n): \text{All subsets of size $n$ contain a minimum.}
      \]

      \pause
      \begin{description}[Inductive Hypothesis:]
        \item[Basis Step:] $P(1)$
        \item[\textcolor{cyan}{Inductive Hypothesis:}] $P(n)$
        \item[Inductive Step:] $P(n) \to P(n+1)$
          \only<4>{
            \begin{itemize}
              \item $A' \gets A \setminus {a}$
              \item $x \gets \min A'$
              \item Compare $x$ with $a$
            \end{itemize}
          }
      \end{description}

      \only<5>{
        \vspace{-0.50cm}
        \[
          \red{\forall n \in \mathbb{N}: P(n) \quad \text{\it vs. } \quad P(\infty)}
        \]
      }
    \end{center}
  \end{proof}
\end{frame}
%%%%%%%%%%%%%%%%%%%%%%%%%%%%%%

%%%%%%%%%%%%%%%%%%%%%%%%%%%%%%
\begin{frame}{}
  \begin{lemma}
    (第一)数学归纳法蕴含良序原理。
  \end{lemma}

  \begin{center}
    \red{$P(n):$ 任何一个\blue{含有 $\le n$ 的某个自然数}的自然数子集都有最小元}

    \pause
    \vspace{0.20cm}
    \[
      P(0)
    \]
    \pause
    \[
      P(n) \to P(n + 1)
    \]
    \pause
    \[
      \exists x.\; x \le n \qquad \forall x.\; x \ge n + 1
    \]
  \end{center}
\end{frame}
%%%%%%%%%%%%%%%%%%%%%%%%%%%%%%

%%%%%%%%%%%%%%%%%%%%%%%%%%%%%%
\begin{frame}{}
  \begin{lemma}
    良序原理蕴含(第一)数学归纳法。
  \end{lemma}

  \begin{center}
    \red{\bf 反证法} \\[6pt]

    \pause
    设 $P(0)$ 成立且$\forall n \in \N.\; P(n) \to P(n+1)$ 成立, \\
    但 $\exists n \in \N.\; P(n)$ 不成立 \\[6pt]

    \pause
    \[
      A = \set{k \in \N \mid \lnot P(k)} \neq \emptyset
    \]
    \pause
    \[
      m \triangleq \min A \quad \text{(by 良序原理)} \pause \qquad \red{\lnot P(m)}
    \]
    \pause
    \[
      m \neq 0 \pause \qquad m \ge 1
    \]
    \pause
    \[
      n \triangleq m - 1
        \pause \blue{\qquad P(n)}
        \pause \qquad P(n + 1)
        \pause \qquad \red{P(m)}
    \]
  \end{center}
\end{frame}
%%%%%%%%%%%%%%%%%%%%%%%%%%%%%%

%%%%%%%%%%%%%%%%%%%%%%%%%%%%%%
\begin{frame}{}
  \fig{width = 0.70\textwidth}{figs/learn-by-examples}
\end{frame}
%%%%%%%%%%%%%%%%%%%%%%%%%%%%%%