% mip.tex

%%%%%%%%%%%%%%%%%%%%%%%%%%%%%%
\begin{frame}{}
  \begin{exampleblock}{Tiling Puzzle}
    任何一个\blue{缺失了一格的} $2^{n} \times 2^{n}$ 的网格都可以被 $L$ 型填满。
  \end{exampleblock}

  \fig{width = 0.75\textwidth}{figs/L-tiling-case-2}

  \pause
  \vspace{-1.50cm}
  \begin{center}
    \red{对自然数 $n$ 作归纳}
  \end{center}
\end{frame}
%%%%%%%%%%%%%%%%%%%%%%%%%%%%%%

%%%%%%%%%%%%%%%%%%%%%%%%%%%%%%
\begin{frame}{}
  \begin{definition}[Line Map]
    \begin{itemize}
      \item A blank circle is a line map;
      \item A line map with a chord (弦) is a line map.
    \end{itemize}
  \end{definition}

  \begin{columns}
    \column{0.50\textwidth}
      \fig{width = 0.80\textwidth}{figs/n=4-color}
    \column{0.50\textwidth}
      \uncover<3->{
        \fig{width = 0.80\textwidth}{figs/n=5-color}
      }
  \end{columns}

  \pause
  \vspace{-0.30cm}
  \begin{theorem}
    Any line map can be two-colored.
  \end{theorem}
\end{frame}
%%%%%%%%%%%%%%%%%%%%%%%%%%%%%%

%%%%%%%%%%%%%%%%%%%%%%%%%%%%%%
\begin{frame}{}
  \begin{exampleblock}{The Tower of Hanoi}
    \fig{width = 0.50\textwidth}{figs/hanoi}

    \vspace{0.30cm}
    \begin{center}
      $\textsc{Hanoi}(n, A, B, C):$ 借助于$B$柱, 将$n$个盘子从$A$柱移到$C$柱
    \end{center}
  \end{exampleblock}

  \pause
  \[
    T_{n}: \text{the \red{\bf minimum} number of moves for $n$ disks}
  \]
\end{frame}
%%%%%%%%%%%%%%%%%%%%%%%%%%%%%%

%%%%%%%%%%%%%%%%%%%%%%%%%%%%%%
\begin{frame}{}
  \fig{width = 0.45\textwidth}{figs/hanoi-solution}

  \pause
  \vspace{-0.80cm}
  \[
    T(n) \;\red{\le}\; 2T(n-1) + 1 \qquad (n \ge 1)
  \]
\end{frame}
%%%%%%%%%%%%%%%%%%%%%%%%%%%%%%

%%%%%%%%%%%%%%%%%%%%%%%%%%%%%%
\begin{frame}{}
  \begin{center}
    考虑\green{\bf 第一次}以及\yellow{\bf 最后一次}移动\purple{\bf 最大盘}时的情况 \\
    \pause
    \red{\bf 另外 $(n-1)$ 个盘子一定在同一个柱子上}
  \end{center}

  \fig{width = 0.40\textwidth}{figs/hanoi-lower-bound}

  \pause
  \vspace{-0.80cm}
  \[
    T(n) \;\red{\ge}\; 2T(n-1) + 1 \qquad (n \ge 1)
  \]
\end{frame}
%%%%%%%%%%%%%%%%%%%%%%%%%%%%%%

%%%%%%%%%%%%%%%%%%%%%%%%%%%%%%
\begin{frame}{}
  \begin{align*}
    T(0) &= 0, \\[6pt]
    T(n) &\;\red{=}\; 2T(n-1) + 1, \quad n \ge 1
  \end{align*}

  \pause
  \[
    T(n) = 2^{n} - 1, \quad n \ge 0
  \]
\end{frame}
%%%%%%%%%%%%%%%%%%%%%%%%%%%%%%

%%%%%%%%%%%%%%%%%%%%%%%%%%%%%%
\begin{frame}{}
  \begin{theorem}[Fermat's Little Theorem]
    对于任意自然数 $a$ 与任意素数 $p$,
    \[
      a^{p} \equiv a \;(\mathrm{mod}\; p).
    \]
  \end{theorem}

  \pause
  \vspace{0.30cm}
  \begin{center}
    \red{对自然数 $a$ 作归纳} \pause \blue{(对于任意素数 $p$)}
  \end{center}

  \pause
  \[
    (a+1)^p = \purple{a^p} + \binom{p}{1} a^{p-1} + \binom{p}{2} a^{p-2}
      + \dots + \binom{p}{p-1}a + \purple{1}
  \]

  \pause
  \vspace{0.30cm}
  \[
    \binom{p}{k} = \frac{\blue{p} (p-1) \dots (p-k+1)}{k!} \equiv 0 \;(\mathrm{mod}\; p)
      \quad (1 \le k \le p-1)
  \]

  \pause
  \vspace{0.30cm}
  \[
    (a + 1)^{p} \equiv a + 1\; (\mathrm{mod}\; p)
  \]
\end{frame}
%%%%%%%%%%%%%%%%%%%%%%%%%%%%%%

%%%%%%%%%%%%%%%%%%%%%%%%%%%%%%
\begin{frame}{}
  \begin{exampleblock}{}
    \[
      \binom{n}{k} = \frac{n!}{k! (n-k)!} \in \N \quad (0 \le k \le n)
    \]
  \end{exampleblock}

  \pause
  \vspace{0.30cm}
  \begin{center}
    \red{对自然数 $n$ 作归纳} \blue{(对于任意的 $0 \le k \le n$)}
  \end{center}

  \pause
  \[
    \binom{n+1}{k} = \binom{n}{k} + \binom{n}{k-1}
  \]

  \pause
  \vspace{0.30cm}
  \[
    k = 0 \qquad k = n + 1 \qquad 1 \le k \le n
  \]
\end{frame}
%%%%%%%%%%%%%%%%%%%%%%%%%%%%%%

%%%%%%%%%%%%%%%%%%%%%%%%%%%%%%
\begin{frame}{}
  \begin{exampleblock}{Horse Paradox}
    所有马的颜色都相同。
  \end{exampleblock}

  \pause
  \vspace{0.30cm}
  \begin{center}
    \red{对马的数目 $n \ge 1$ 作归纳}

    \pause
    \fig{width = 0.50\textwidth}{figs/induction-horses}
  \end{center}

  \pause
  \[
    \blue{n = 1 \centernot\implies n = 2}
  \]
\end{frame}
%%%%%%%%%%%%%%%%%%%%%%%%%%%%%%