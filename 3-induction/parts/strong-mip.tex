% strong-mip.tex

%%%%%%%%%%%%%%%%%%%%%%%%%%%%%%
\begin{frame}{}
  \begin{exampleblock}{}
    \[
      F(n) \;\text{是偶数, 当且仅当}\; F(n+3) \;\text{是偶数。}
    \]
  \end{exampleblock}
\end{frame}
%%%%%%%%%%%%%%%%%%%%%%%%%%%%%%

%%%%%%%%%%%%%%%%%%%%%%%%%%%%%%
\begin{frame}{}
  \begin{exampleblock}{}
    \begin{center}
      设 $\ast$ 是一个满足结合律的二元运算符。 请证明,
      \[
        a_{1} \ast a_{2} \ast \dots \ast a_{n}
      \]
      的值与括号的使用方式无关。
    \end{center}
  \end{exampleblock}
\end{frame}
%%%%%%%%%%%%%%%%%%%%%%%%%%%%%%

%%%%%%%%%%%%%%%%%%%%%%%%%%%%%%
\begin{frame}{}
  \begin{exampleblock}{算术基本定理}
    Prove that every integer greater than 2 can be written as product of primes.
  \end{exampleblock}
\end{frame}
%%%%%%%%%%%%%%%%%%%%%%%%%%%%%%

%%%%%%%%%%%%%%%%%%%%%%%%%%%%%%
\begin{frame}{}
  \begin{exampleblock}{}
    请证明, 只用4分与5分邮票, 就可以组成12分及以上的每种邮资。 \\[8pt]

    (每个不小于12的整数都可以写成若干个4或5的和。)
  \end{exampleblock}
\end{frame}
%%%%%%%%%%%%%%%%%%%%%%%%%%%%%%

%%%%%%%%%%%%%%%%%%%%%%%%%%%%%%
\begin{frame}{}
  \begin{exampleblock}{堆盒子游戏}
    现有 $n$ 个盒子堆在一起。
    你可以移动这些盒子, 每次移动只能将一堆盒子分成不为空的两堆盒子,
    最后得到 $n$ 堆盒子, 即每堆只有一个盒子时, 游戏结束。 \\[8pt]
    每次移动盒子时, 如果将高度为 $a + b$ 的盒子堆拆分成高度为 $a$ 和 $b$ 的两堆,
    玩家可以得 $ab$ 分。 \\[8pt]
    玩家的总得分是每次移动盒子得分的总和。
    请问, 如何才能得到最高分?
  \end{exampleblock}

  +fig
\end{frame}
%%%%%%%%%%%%%%%%%%%%%%%%%%%%%%

%%%%%%%%%%%%%%%%%%%%%%%%%%%%%%
\begin{frame}{}
  \begin{lemma}{}
    任何一种平铺$n$个盒子的方法, 得分都是 $\frac{n(n-1)}{2}$。
  \end{lemma}
\end{frame}
%%%%%%%%%%%%%%%%%%%%%%%%%%%%%%

%%%%%%%%%%%%%%%%%%%%%%%%%%%%%%
\begin{frame}{}
  \begin{exampleblock}{}
    只用以下三种图示拼出 $2 \times n$ 的形状, 有几种不同的拼法?
  \end{exampleblock}

  \[
    T(n) = ? T(n-1) + ?? T(n-2) + \dots
  \]
\end{frame}
%%%%%%%%%%%%%%%%%%%%%%%%%%%%%%