% strong-mip.tex

%%%%%%%%%%%%%%%%%%%%%%%%%%%%%%
\begin{frame}{}
  \begin{exampleblock}{算术基本定理 (The Fundamental Theorem of Arithmetic)}
    任何一个$\ge 2$的自然数都可以\blue{(唯一)}写为若干素数的乘积。
  \end{exampleblock}

  \pause
  \vspace{0.30cm}
  \begin{center}
    \red{对自然数 $n$ 作强数学归纳}
  \end{center}
\end{frame}
%%%%%%%%%%%%%%%%%%%%%%%%%%%%%%

%%%%%%%%%%%%%%%%%%%%%%%%%%%%%%
\begin{frame}{}
  \begin{exampleblock}{}
    \begin{center}
      设 $\ast$ 是一个满足结合律的二元运算符, 即
      \[
        (a \ast b) \ast c = a \ast (b \ast c).
      \]
      请证明, $a_{1} \ast a_{2} \ast \dots \ast a_{n}\; (n \ge 3)$
      的值与括号的使用方式无关。
    \end{center}
  \end{exampleblock}

  \pause
  \vspace{0.30cm}
  \begin{center}
    \red{对 $n$ 作强数学归纳}
  \end{center}

  \pause
  \[
    (\dots) \;\red{\ast}\; (\dots)
  \]
\end{frame}
%%%%%%%%%%%%%%%%%%%%%%%%%%%%%%

%%%%%%%%%%%%%%%%%%%%%%%%%%%%%%
\begin{frame}{}
  \begin{exampleblock}{}
    \begin{align*}
      F_{0} &= 0, \qquad F_{1} = 1, \\[6pt]
      F_{n} &= F(n-1) + F(n-2) \quad (n \ge 2)
    \end{align*}
    \[
      \text{请证明:}\; F(n) \;\text{是偶数当且仅当}\; F(n+3) \;\text{是偶数。}
    \]
  \end{exampleblock}

  \pause
  \begin{center}
    \red{对 $n$ 作归纳}

    \pause
    \vspace{0.30cm}
    \blue{基础步骤:} 命题对 $n = 0, 1$ 成立

    \pause
    \vspace{0.30cm}
    \blue{归纳假设:} $F(n)$是偶数当且仅当$F(n+3)$是偶数

    \pause
    \begin{align*}
      F(n+1) &= \cyan{F(n)} + \teal{F(n-1)} \\[6pt]
      F(n+4) &= \cyan{F(n+3)} + \teal{F(n+2)}
    \end{align*}
  \end{center}
\end{frame}
%%%%%%%%%%%%%%%%%%%%%%%%%%%%%%

%%%%%%%%%%%%%%%%%%%%%%%%%%%%%%
\begin{frame}{}
  \begin{exampleblock}{Tiling Puzzle}
    只用$1 \times 1$ 与 $1 \times 2$两种矩形,
    拼出 $1 \times n$ 的形状, 有几种不同的拼法?
    \fig{width = 0.50\textwidth}{figs/1xn}
  \end{exampleblock}

  \pause
  \vspace{0.30cm}
  \fig{width = 0.40\textwidth}{figs/1x1-tiling}

  \pause
  \fig{width = 0.40\textwidth}{figs/1x2-tiling}

  \pause
  \vspace{-0.30cm}
  \begin{align*}
    T_{0} &= 0, \qquad T_{1} = 1, \\[6pt]
    T_{n} &= \blue{T(n-1)} + \red{T(n-2)} \quad (n \ge 2)
  \end{align*}

  \pause
  \[
    \cyan{\boxed{F_{n} = T_{n}}}
  \]
\end{frame}
%%%%%%%%%%%%%%%%%%%%%%%%%%%%%%

%%%%%%%%%%%%%%%%%%%%%%%%%%%%%%
\begin{frame}{}
  \[
    \cyan{\boxed{F_{n} = T_{n}}}
  \]

  \pause
  \vspace{0.50cm}
  \[
    F_{2n} = \teal{(F_{n})^{2}} + \violet{(F_{n-1})^{2}}
  \]

  \pause
  \vspace{0.50cm}
  \fig{width = 0.70\textwidth}{figs/f2n-proof}
\end{frame}
%%%%%%%%%%%%%%%%%%%%%%%%%%%%%%