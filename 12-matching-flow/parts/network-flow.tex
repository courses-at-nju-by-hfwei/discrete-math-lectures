% network-flow.tex

%%%%%%%%%%%%%%%g
\begin{frame}{}
  \begin{definition}[Network (网络)]
    A \red{network} is a \purple{digraph} with
    \begin{itemize}
      \setlength{\itemsep}{5pt}
      \item a distinguished \blue{source vertex} $s$,
      \item a distinguished \blue{sink vertex} $t$,
      \item a \blue{capacity} $c(e) \ge 0$ on each edge $e$
    \end{itemize}
  \end{definition}

  \fig{width = 0.60\textwidth}{figs/flow-value-1}
\end{frame}
%%%%%%%%%%%%%%%g

%%%%%%%%%%%%%%%g
\begin{frame}{}
  \begin{definition}[Flow (流)]
    A \red{flow} $f$ is a \cyan{function} that assigns a value $f(e)$ to each edge $e$.
  \end{definition}

  \fig{width = 0.60\textwidth}{figs/flow-value-1}

  \pause
  \[
    f^{+}(v) = \sum_{vw \;\in\; E} f(vw)
    \pause \qquad
    f^{-}(v) = \sum_{uv \;\in\; E} f(uv)
  \]
\end{frame}
%%%%%%%%%%%%%%%g

%%%%%%%%%%%%%%%g
\begin{frame}{}
  \begin{definition}[Feasible]
    A \blue{flow} $f$ is \red{feasible} if it satisfies
    \begin{description}
      \item[Capacity Constraints:]
        \[
          \forall e \in E(G).\; 0 \le f(e) \le c(e)
        \]
      \item[Conservation Constraints:]
        \[
          \forall v \in V(G) - \set{s, t}.\; f^{+}(v) = f^{-}(v)
        \]
    \end{description}
  \end{definition}

  \fig{width = 0.60\textwidth}{figs/flow-value-1}
\end{frame}
%%%%%%%%%%%%%%%g

%%%%%%%%%%%%%%%g
\begin{frame}{}
  \begin{definition}[Value (值)]
    The \red{value} $\val(f)$ of a \blue{flow} $f$ is
    \[
      \val(f) = f^{-}(t) = f^{+}(s).
    \]
  \end{definition}

  \vspace{0.30cm}
  \begin{columns}
    \column{0.50\textwidth}
      \fig{width = 0.90\textwidth}{figs/flow-value-1}
    \column{0.50\textwidth}
      \pause
      \fig{width = 0.90\textwidth}{figs/flow-value-2}
  \end{columns}

  \pause
  \vspace{0.50cm}
  \begin{definition}[Maximum Flow (最大流)]
    A \red{maximum flow} is a \blue{feasible flow} of maximum \blue{value}.
  \end{definition}
\end{frame}
%%%%%%%%%%%%%%%g

%%%%%%%%%%%%%%%g
\begin{frame}{}
  \begin{columns}
    \column{0.50\textwidth}
      \fig{width = 0.90\textwidth}{figs/flow-value-1}
    \column{0.50\textwidth}
      \fig{width = 0.90\textwidth}{figs/flow-value-2}
  \end{columns}

  \pause
  \vspace{0.30cm}
  \[
    s \to x \;\red{\to}\; v \to t
  \]

  \pause
  \begin{definition}[$f$-augmenting Paths (增广路径)]
    \[
      \min_{e \in E(P)} \epsilon(e)
    \]
  \end{definition}
\end{frame}
%%%%%%%%%%%%%%%g

%%%%%%%%%%%%%%%g
\begin{frame}{}
  \begin{definition}[Source/Sink Cut (割)]
    In a network, a \red{source/sink cut} $[S, T]$
    consists of the edges \blue{from} a \cyan{source set} $S$
    \blue{to} a \cyan{sink set} $T$, where
    \[
      V = S \;\red{\uplus}\; T \land s \in S \land t \in T
    \]
  \end{definition}

  \vspace{0.50cm}
  \fig{width = 0.50\textwidth}{figs/ST-cut}
\end{frame}
%%%%%%%%%%%%%%%g

%%%%%%%%%%%%%%%g
\begin{frame}{}
  \fig{width = 0.50\textwidth}{figs/ST-cut}

  \begin{definition}[Capacity of Cut (割的容量)]
    \[
      \capacity(S, T) = \sum_{u \in S, v \in T, uv \in E} c(uv)
    \]
  \end{definition}
\end{frame}
%%%%%%%%%%%%%%%g

%%%%%%%%%%%%%%%g
\begin{frame}{}
  \fig{width = 0.50\textwidth}{figs/flow-value-1}

  \vspace{0.50cm}
  \begin{definition}[Minimum Cut (最小割)]
    A \red{minimum cut} is a \blue{cut} of minimum value.
  \end{definition}
\end{frame}
%%%%%%%%%%%%%%%g

%%%%%%%%%%%%%%%g
\begin{frame}{}
  \begin{theorem}[Weak Duality (弱对偶定理)]
    Let $f$ be any feasible \red{flow} and $[S, T]$ be any source/sink \red{cut}.
    \[
      \val(f) \le \capacity(S, T).
    \]
  \end{theorem}

  \fig{width = 0.50\textwidth}{figs/ST-cut}

  \pause
  \[
    \val(f) = f^{+}(S) - f^{-}(S) \pause \le f^{+}(S) \pause \le \capacity(S, T)
  \]
\end{frame}
%%%%%%%%%%%%%%%g

%%%%%%%%%%%%%%%g
\begin{frame}{}
  \begin{lemma}
    \[
      \max_{f} \val(f) \le \min_{[S, T]} \capacity(S, T)
    \]
  \end{lemma}

  \fig{width = 0.50\textwidth}{figs/flow-value-2}
\end{frame}
%%%%%%%%%%%%%%%g

%%%%%%%%%%%%%%%g
\begin{frame}{}
  \begin{center}
    \red{What if $\val(f) = \capacity(S, T)$ for some flow $f$ and some cut $[S, T]$?}
  \end{center}

  \pause
  \[
    \val(f) = f^{+}(S) - f^{-}(S) \;\red{=}\; f^{+}(S) \;\red{=}\; \capacity(S, T)
  \]

  \pause
  \[
    f^{-1}(S) = 0 \land f^{+}(S) = \capacity(S, T)
  \]

  \pause
  \fig{width = 0.50\textwidth}{figs/flow-value-2}
\end{frame}
%%%%%%%%%%%%%%%g

%%%%%%%%%%%%%%%g
\begin{frame}{}
  \begin{theorem}[Max-flow Min-cut Theorem (Ford and Fulkerson; 1956)]
    \[
      \max_{f} \val(f) = \min_{[S, T]} \capacity(S, T)
    \]
    \begin{center}
      \red{(Strong Duality)}
    \end{center}
  \end{theorem}

  \begin{columns}
    \column{0.50\textwidth}
      \fig{width = 0.50\textwidth}{figs/Ford}
      \begin{center}
        \teal{L. R. Ford Jr. ($1927 \sim 2017$)}
      \end{center}
    \column{0.50\textwidth}
      \fig{width = 0.50\textwidth}{figs/Fulkerson}
      \begin{center}
        \teal{D. R. Fulkerson ($1924 \sim 1976$)}
      \end{center}
  \end{columns}
\end{frame}
%%%%%%%%%%%%%%%g

%%%%%%%%%%%%%%%g
\begin{frame}{}
  \fig{width = 0.60\textwidth}{figs/maxflow-mincut}
\end{frame}
%%%%%%%%%%%%%%%g

%%%%%%%%%%%%%%%g
\begin{frame}{}
  
\end{frame}
%%%%%%%%%%%%%%%g

%%%%%%%%%%%%%%%g
\begin{frame}{}
\end{frame}
%%%%%%%%%%%%%%%g

%%%%%%%%%%%%%%%g
\begin{frame}{}
\end{frame}
%%%%%%%%%%%%%%%g

%%%%%%%%%%%%%%%g
\begin{frame}{}
\end{frame}
%%%%%%%%%%%%%%%g