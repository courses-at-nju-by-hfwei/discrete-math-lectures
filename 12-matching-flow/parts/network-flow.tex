% network-flow.tex

%%%%%%%%%%%%%%%g
\begin{frame}{}
  \begin{definition}[Network (网络)]
    A \red{network} is a \purple{digraph} with
    \begin{itemize}
      \setlength{\itemsep}{5pt}
      \item a distinguished \blue{source vertex} $s$,
      \item a distinguished \blue{sink vertex} $t$,
      \item a \blue{capacity} $c(e) \ge 0$ on each edge $e$
    \end{itemize}
  \end{definition}

  \vspace{0.30cm}
  \fig{width = 0.60\textwidth}{figs/West-network-example}
\end{frame}
%%%%%%%%%%%%%%%g

%%%%%%%%%%%%%%%g
\begin{frame}{}
  \begin{definition}[Flow (流)]
    A \red{flow} $f$ is a \cyan{function} that assigns a value $f(e)$ to each edge $e$.
  \end{definition}

  \vspace{0.30cm}
  \fig{width = 0.60\textwidth}{figs/flow-value-1}
\end{frame}
%%%%%%%%%%%%%%%g

%%%%%%%%%%%%%%%g
\begin{frame}{}
  \begin{definition}[Feasible Flow (可行流)]
    A \blue{flow} $f$ is \red{feasible} if it satisfies
    \begin{description}[Capacity Constraints:]
      \item[Capacity Constraints:]
        \[
          \forall e \in E.\; 0 \le f(e) \le c(e)
        \]
      \item[Flow Conservation:]
        \[
          \forall v \in V.\; f^{+}(v) = f^{-}(v)
        \]
    \end{description}
  \end{definition}

  \fig{width = 0.60\textwidth}{figs/flow-value-1}
\end{frame}
%%%%%%%%%%%%%%%g

%%%%%%%%%%%%%%%g
\begin{frame}{}
  \[
    f^{+}(v) = \sum_{(v, w) \;\in\; E} f(v, w) \qquad
    f^{-}(v) = \sum_{(u, v) \;\in\; E} f(u, v)
  \]

  \fig{width = 0.60\textwidth}{figs/flow-value-1}

  \pause
  \[
    f^{+}(U) = \sum_{u \in U, v \in \overline{U}, (u, v) \in E} f(u, v) \pause\qquad
    f^{-}(U) = \sum_{v \in \overline{U}, u \in U, (v, u) \in E} f(v, u)
  \]
\end{frame}
%%%%%%%%%%%%%%%g

%%%%%%%%%%%%%%%g
\begin{frame}{}
  \fig{width = 0.60\textwidth}{figs/flow-value-1}

  \pause
  \[
    \forall U \subseteq (V - \set{s, t}).\; f^{+}(U) = f^{-}(U)
  \]

  \pause
  \[
    s \in U \land t \notin U \implies f^{+}(U) - f^{-}(U) = \pause f^{+}(s)
  \]
\end{frame}
%%%%%%%%%%%%%%%g

%%%%%%%%%%%%%%%g
\begin{frame}{}
  \begin{definition}[Value (值)]
    The \red{value} $\val(f)$ of a \blue{flow} $f$ is
    \[
      \val(f) = f^{-}(t) = f^{+}(s).
    \]
  \end{definition}

  \vspace{0.30cm}
  \begin{columns}
    \column{0.50\textwidth}
      \fig{width = 0.90\textwidth}{figs/flow-value-1}
    \column{0.50\textwidth}
      \pause
      \fig{width = 0.90\textwidth}{figs/flow-value-2}
  \end{columns}

  \pause
  \vspace{0.50cm}
  \begin{definition}[Maximum Flow (最大流)]
    A \red{maximum flow} is a \blue{feasible flow} of maximum \blue{value}.
  \end{definition}
\end{frame}
%%%%%%%%%%%%%%%g

%%%%%%%%%%%%%%%g
\begin{frame}{}
  \[
    s - x \;\red{-}\; v - t
  \]
  \vspace{-0.30cm}
  \begin{columns}
    \column{0.50\textwidth}
      \fig{width = 0.90\textwidth}{figs/flow-value-1}
    \column{0.50\textwidth}
      \fig{width = 0.90\textwidth}{figs/flow-value-2}
  \end{columns}

  \pause
  \vspace{0.30cm}
  \begin{definition}[$f$-augmenting Paths (增广路径)]
    When $f$ is a feasible flow, an \red{$f$-augmenting path} is a $s \sim t$ path $P$ \\
    \blue{in the underlying graph} such that for each edge $e \in E(P)$,
    \begin{enumerate}[(a)]
      \item if $P$ follows $e$ in the forward direction, then $f(e) < c(e)$;
      \item if $P$ follows $e$ in the backward direction, then $f(e) > 0$.
    \end{enumerate}
  \end{definition}
\end{frame}
%%%%%%%%%%%%%%%g

%%%%%%%%%%%%%%%g
\begin{frame}{}
  \begin{definition}[$f$-augmenting Paths]
    Let $P$ be an $f$-augmenting path.

    \[
      \epsilon(e) = \begin{cases}
        c(e) - f(e) & \text{ if } e \text{ is forward on } P \\
        f(e)        & \text{ if } e \text{ is backward on } P
      \end{cases}
    \]
  \end{definition}

  \begin{columns}
    \column{0.50\textwidth}
      \fig{width = 0.90\textwidth}{figs/flow-value-1}
    \column{0.50\textwidth}
      \fig{width = 0.90\textwidth}{figs/flow-value-2}
  \end{columns}

  \pause
  \vspace{0.30cm}
  \begin{center}
    An $f$-augmenting path leads to a flow with \red{larger} value.
    \[
      \min_{e \;\in\; E(P)} \epsilon(e)
    \]
  \end{center}
\end{frame}
%%%%%%%%%%%%%%%g

%%%%%%%%%%%%%%%g
\begin{frame}{}
  \fig{width = 0.60\textwidth}{figs/CLRS-flow-19}

  \pause
  \vspace{0.30cm}
  \fig{width = 0.60\textwidth}{figs/CLRS-flow-19}
\end{frame}
%%%%%%%%%%%%%%%g

%%%%%%%%%%%%%%%g
\begin{frame}{}
  \begin{definition}[Source/Sink Cut (割)]
    In a network, a \red{source/sink cut} $[S, T]$
    consists of the edges \blue{from} a \cyan{source set} $S$
    \blue{to} a \cyan{sink set} $T$, where
    \[
      (T = V - S) \land (s \in S) \land (t \in T)
    \]
  \end{definition}

  \vspace{0.50cm}
  \fig{width = 0.50\textwidth}{figs/ST-cut}
\end{frame}
%%%%%%%%%%%%%%%g

%%%%%%%%%%%%%%%g
\begin{frame}{}
  \fig{width = 0.50\textwidth}{figs/ST-cut}

  \begin{definition}[Capacity of Cut (割的容量)]
    \[
      \capacity(S, T) = \sum_{u \in S, v \in T, uv \in E} c(u, v)
    \]
  \end{definition}
\end{frame}
%%%%%%%%%%%%%%%g

%%%%%%%%%%%%%%%g
\begin{frame}{}
  \fig{width = 0.50\textwidth}{figs/West-network-example}

  \pause
  \begin{definition}[Minimum Cut (最小割)]
    A \red{minimum cut} is a \blue{cut} of minimum value.
  \end{definition}

  \pause
  \vspace{0.50cm}
  \fig{width = 0.50\textwidth}{figs/CLRS-network-example}
\end{frame}
%%%%%%%%%%%%%%%g

%%%%%%%%%%%%%%%g
\begin{frame}{}
  \begin{theorem}[Weak Duality (弱对偶定理)]
    Let $f$ be any feasible \red{flow} and $[S, T]$ be any source/sink \red{cut}.
    \[
      \val(f) \le \capacity(S, T).
    \]
  \end{theorem}

  \fig{width = 0.50\textwidth}{figs/ST-cut}

  \pause
  \[
    \val(f) = f^{+}(S) - f^{-}(S) \pause \le f^{+}(S) \pause \le \capacity(S, T)
  \]
\end{frame}
%%%%%%%%%%%%%%%g

%%%%%%%%%%%%%%%g
\begin{frame}{}
  \begin{lemma}
    \[
      \max_{f} \val(f) \le \min_{[S, T]} \capacity(S, T)
    \]
  \end{lemma}

  \pause
  \vspace{0.60cm}
  \begin{center}
    \red{What if $\val(f) = \capacity(S, T)$ for some flow $f$ and some cut $[S, T]$?}

    \pause
    \vspace{0.50cm}
    \blue{$f$ is maximum and $[S, T]$ is minimum}
  \end{center}
\end{frame}
%%%%%%%%%%%%%%%g

%%%%%%%%%%%%%%%g
\begin{frame}{}
  \[
    \val(f) = f^{+}(S) - f^{-}(S) \;\red{=}\; f^{+}(S) \;\red{=}\; \capacity(S, T)
  \]

  \pause
  \[
    \blue{\boxed{f^{-1}(S) = 0 \land f^{+}(S) = \capacity(S, T)}}
  \]

  \only<3>{\fig{width = 0.60\textwidth}{figs/flow-value-2}}
  \only<4->{\fig{width = 0.60\textwidth}{figs/ClRS-flow-23}}
\end{frame}
%%%%%%%%%%%%%%%g

%%%%%%%%%%%%%%%g
\begin{frame}{}
  \begin{theorem}[Max-flow Min-cut Theorem (Ford and Fulkerson; 1956)]
    \[
      \max_{f} \val(f) = \min_{[S, T]} \capacity(S, T)
    \]
    \begin{center}
      \red{(Strong Duality)}
    \end{center}
  \end{theorem}

  \begin{columns}
    \column{0.50\textwidth}
      \fig{width = 0.50\textwidth}{figs/Ford}
      \begin{center}
        \teal{L. R. Ford Jr. ($1927 \sim 2017$)}
      \end{center}
    \column{0.50\textwidth}
      \fig{width = 0.50\textwidth}{figs/Fulkerson}
      \begin{center}
        \teal{D. R. Fulkerson ($1924 \sim 1976$)}
      \end{center}
  \end{columns}
\end{frame}
%%%%%%%%%%%%%%%g

%%%%%%%%%%%%%%%g
\begin{frame}{}
  \fig{width = 0.60\textwidth}{figs/CLRS-flow-23}

  \pause
  \vspace{0.30cm}
  \fig{width = 0.60\textwidth}{figs/maxflow-mincut}
\end{frame}
%%%%%%%%%%%%%%%g

%%%%%%%%%%%%%%%g
\begin{frame}{}
  \begin{theorem}
    A feasible flow $f$ is maximum iff there are no $f$-augmenting paths.
  \end{theorem}

  \pause
  \vspace{0.30cm}
  \begin{center}
    \red{We construct a cut $[S, T]$ with $\val(f) = \capacity(S, T)$.}

    \pause
    \fig{width = 0.60\textwidth}{figs/maxflow-mincut}
    \[
      S \triangleq \set{\text{the vertices reachable from $s$
      along \red{partial} $f$-augmenting paths}}
    \]
  \end{center}
\end{frame}
%%%%%%%%%%%%%%%g

%%%%%%%%%%%%%%%g
\begin{frame}{}
  \begin{center}
    \[
      S \triangleq \set{\text{the vertices reachable from $s$
      along \red{partial} $f$-augmenting paths}}
    \]
    \[
      T \triangleq V - S
    \]
    \fig{width = 0.60\textwidth}{figs/maxflow-mincut}
    \pause
    \[
      f^{-1}(S) = 0 \land f^{+}(S) = \capacity(S, T)
    \]
    \pause
    \[
      \val(f) = f^{+}(S) - f^{-}(S) \;\red{=}\; f^{+}(S) \;\red{=}\; \capacity(S, T)
    \]
  \end{center}
\end{frame}
%%%%%%%%%%%%%%%g

%%%%%%%%%%%%%%%g
\begin{frame}{}
  \begin{exampleblock}{The Ford-Fulkerson Method}
    Repeatedly \red{finding} \blue{$f$}-augmenting paths
    until \cyan{no more ones exist}.
  \end{exampleblock}

  \vspace{0.50cm}
  \only<2>{\fig{width = 0.70\textwidth}{figs/West-network-example}}
  \only<3>{\fig{width = 0.75\textwidth}{figs/CLRS-network-example}}
\end{frame}
%%%%%%%%%%%%%%%g

%%%%%%%%%%%%%%%g
\begin{frame}{}
  \begin{exampleblock}{The Edmonds-Karp Algorithm}
    Using \red{BFS} (Breadth-first Search) to find $f$-augmenting paths.
  \end{exampleblock}

  \vspace{0.50cm}
  \only<2>{\fig{width = 0.70\textwidth}{figs/West-network-example}}
  \only<3>{\fig{width = 0.75\textwidth}{figs/CLRS-network-example}}
\end{frame}
%%%%%%%%%%%%%%%g

%%%%%%%%%%%%%%%g
\begin{frame}{}
  \begin{theorem}[Hall Theorem; 1935]
    There is an \red{$X$-perfect matching} of $G$ \purple{iff}
    \[
      \forall W \subseteq X.\; \card{W} \le \card{N_{G}(W)}
    \]
  \end{theorem}

  \pause
  \fig{width = 0.50\textwidth}{figs/Flow-Hall}
  \[
    \forall x \in X.\; c(s, x) = 1 \quad \forall y \in Y.\; c(y, t) = 1
    \quad \forall x \in X, y \in Y.\; c(x, y) = \infty
  \]

  \pause
  \begin{center}
    \blue{We need to show that $\max\limits_{f} \val(f) = \card{X}$.}
  \end{center}
\end{frame}
%%%%%%%%%%%%%%%g

%%%%%%%%%%%%%%%g
\begin{frame}{}
  \begin{center}
    We need to show that
    $\min\limits_{[S, \overline{S}]} \capacity(S, \overline{S}) = \card{X}.$
  \end{center}
  \fig{width = 0.50\textwidth}{figs/Flow-Hall}

  \pause
  \[
    \min\limits_{[S, \overline{S}]} \capacity(S, \overline{S}) \le \card{X}
  \]

  \pause
  \begin{center}
    \blue{Therefore, we need to show that
      $\min\limits_{[S, \overline{S}]} \capacity(S, \overline{S}) \ge \card{X}.$
    }
  \end{center}
\end{frame}
%%%%%%%%%%%%%%%g

%%%%%%%%%%%%%%%g
\begin{frame}{}
  \begin{center}
    Let $[S, \overline{S}]$ be a minimum cut.
    We need to show that $\capacity(S, \overline{S}) = \card{X}$.

    \vspace{0.30cm}
    \only<1-2>{\fig{width = 0.50\textwidth}{figs/Flow-Hall}}
    \only<3->{\fig{width = 0.50\textwidth}{figs/Flow-Hall-Cut}}

    \uncover<2->{
    \[
      \red{\boxed{N(S \cap X) \subseteq (S \cap Y)}}
    \]}
  \end{center}
\end{frame}
%%%%%%%%%%%%%%%g

%%%%%%%%%%%%%%%g
\begin{frame}{}
  \fig{width = 0.50\textwidth}{figs/Flow-Hall-Cut}

  \pause
  \vspace{-0.30cm}
  \begin{align*}
    \capacity(S, \overline{S}) &= \sum_{u \in S, v \in \overline{S}} c(x, y) \\
      \uncover<3->{&= \sum_{v \in \overline{S} \cap X} c(\red{s}, v) +
         \sum_{u \in S \cap Y} c(u, \red{t}) \\}
      \uncover<4->{&= \card{X} - \card{S \cap X} + \card{S \cap Y}} \\
      \uncover<5->{&\;\red{\ge}\; \card{X} - \card{S \cap X} + \;\red{\card{N(S \cap X)}}} \\
      \uncover<6->{&\;\blue{\ge}\; \card{X}}
  \end{align*}
\end{frame}
%%%%%%%%%%%%%%%g

%%%%%%%%%%%%%%%g
\begin{frame}{}
  \begin{theorem}[K\"{o}nig (1931), Egerv\'{a}ry (1931)]
    If $G$ is a bipartite graph, then
    the maximum size of a mathching in $G$ equals
    the minimum size of a vertex cover of $G$
  \end{theorem}

  \pause
  \vspace{0.50cm}
  \fig{width = 0.35\textwidth}{figs/keep-calm-and-try-yourself}
\end{frame}
%%%%%%%%%%%%%%%g