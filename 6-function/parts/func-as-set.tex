% func-as-set.tex

%%%%%%%%%%%%%%%
\begin{frame}{}
  \begin{center}
    \teal{\LARGE Functions as Sets}
  \end{center}
\end{frame}
%%%%%%%%%%%%%%%

%%%%%%%%%%%%%%%
\begin{frame}{}
  \begin{theorem}[函数的外延性原理 (The Principle of Functional Extensionality)]
    \[
      f, g \text{ are functions}:
    \]
    \[
      f = g \iff \dom(f) = \dom(g)
        \land \big(\forall x \in \dom(f).\; f(x) = g(x) \big)
    \]
  \end{theorem}

  \pause
  \[
    f = g \iff \forall (a, b).\; ((a, b) \in f \leftrightarrow (a, b) \in g).
  \]

  \pause
  \[
    \red{\text{It may be that } \cod(f) \neq \cod(g).}
  \]
\end{frame}
%%%%%%%%%%%%%%%

%%%%%%%%%%%%%%%
\begin{frame}{}
  \[
    f: A \to B \qquad g: C \to D
  \]

  \begin{center}
    \red{$Q:$ Is $f \cap g$ a function?}
  \end{center}

  \pause
  \begin{theorem}[Intersection of Functions]
    \[
      f \cap g: (A \cap C) \to (B \cap D)
    \]
  \end{theorem}
\end{frame}
%%%%%%%%%%%%%%%

%%%%%%%%%%%%%%%
\begin{frame}{}
  \[
    f: A \to B \qquad g: C \to D
  \]

  \begin{center}
    \red{$Q:$ Is $f \cup g$ a function?}
  \end{center}

  \pause
  \begin{theorem}[Union of Functions]
    {\[
      f \cup g: (A \cup C) \to (B \cup D) \iff
      \forall x \in \dom(f) \cap \dom(g).\; f(x) = g(x)
    \]}
  \end{theorem}

  \pause
  \begin{exampleblock}{}
    \[
      f: \mathbb{Q} \to \R
    \]

    \[
      f(x) = \left\{\begin{array}{ll}
        x + 1, & \text{if } x \in 2\Z{} \\
        x - 1, & \text{if } x \in 3\Z{} \\
        2,     & \text{otherwise}
      \end{array}\right.
    \]
  \end{exampleblock}
\end{frame}
%%%%%%%%%%%%%%%

%%%%%%%%%%%%%%%
\begin{frame}{}
  \begin{exampleblock}{}
    \[
      f: \ps{\R} \to \Z
    \]

    \[
      f(A) = \left\{\begin{array}{ll}
        \min(A \cap \N) & \text{if } A \cap \N \neq \emptyset \\
          -1 & \text{if } A \cap \N = \emptyset
      \end{array}\right.
    \]
  \end{exampleblock}

  \pause
  \vspace{0.50cm}
  \begin{center}
    \[
      \dom(f) \cap \dom(g) = \emptyset
    \]
    By the \red{\it Well-Ordering Principle} of $\N$
  \end{center}
\end{frame}
%%%%%%%%%%%%%%%

%%%%%%%%%%%%%%%
\begin{frame}{}
  \[
    D: \R \to \R
  \]

  \[
    D(x) = \left\{\begin{array}{ll}
      1 & \text{if } x \in \Q \\
      0 & \text{if } x \in \R \setminus \Q
    \end{array}\right.
  \]

  \vspace{0.60cm}
  \begin{center}
    {Dirichlet Function} \\[5pt]
    \teal{``处处不连续''}
  \end{center}
\end{frame}
%%%%%%%%%%%%%%%