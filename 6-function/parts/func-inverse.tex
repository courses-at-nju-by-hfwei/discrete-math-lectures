% func-inverse.tex

%%%%%%%%%%%%%%%
\begin{frame}{}
  \begin{center}
    \teal{\LARGE Inverse Functions}
  \end{center}

  \fig{width = 0.40\textwidth}{figs/inverse-function}
\end{frame}
%%%%%%%%%%%%%%%

%%%%%%%%%%%%%%%
\begin{frame}{}
  \begin{definition}[Inverse]
    Let $f: A \to B$ be a \red{bijective} function. \\[5pt]

    The {\it inverse} of $f$ is the \teal{function} \purple{$f^{-1}$}: $B \to A$ defined by
    \[
      f^{-1}(b) = a \iff f(a) = b.
    \]
  \end{definition}

  \pause
  \fig{width = 0.30\textwidth}{figs/info}
\end{frame}
%%%%%%%%%%%%%%%

%%%%%%%%%%%%%%%
\begin{frame}{}
  \begin{definition}[Invertible]
    $f: X \to Y$ is {\it invertible} if there exists $g: Y \to X$ such that

    \[
      f(x) = y \iff g(y) = x.
    \]
  \end{definition}

  \pause
  \begin{theorem}
    $f$ is invertible $\iff$ $f$ is bijective.
  \end{theorem}

  \begin{columns}
    \column{0.50\textwidth}
      \pause
      \begin{center}
	\blue{$f$ is invertible $\implies$ $f$ is bijective} \pause \\[10pt]
	$g$ is a function $\implies$ $f$ is injective \\[5pt]
	$\dom{g} = Y \implies f \text{ is surjective}$
      \end{center}
    \column{0.50\textwidth}
      \pause
      \begin{center}
	\blue{$f$ is bijective $\implies$ $f$ is invertible} \pause \\[10pt]
	To show that $g$ defined above is indeed a function from $Y$ to $X$.
      \end{center}
  \end{columns}
\end{frame}
%%%%%%%%%%%%%%%

%%%%%%%%%%%%%%%
\begin{frame}{}
  \begin{definition}[Invertible]
    $f: X \to Y$ is {\it invertible} if there exists $g: Y \to X$ such that

    \[
      f(x) = y \iff g(y) = x.
    \]
  \end{definition}

  \pause
  \vspace{0.30cm}
  \begin{theorem}
    $g: Y \to X$ is unique.
  \end{theorem}

  \pause
  \[
    \text{\teal{By Contradiction}}
  \]

  \pause
  \[
    \blue{f^{-1} \triangleq g}
  \]

  \pause
  \[
    \red{\boxed{f(x) = y \iff f^{-1}(y) = x}}
  \]
\end{frame}
%%%%%%%%%%%%%%%

%%%%%%%%%%%%%%%
\begin{frame}{}
  \begin{theorem}[UD Theorem 16.4]
    \[
      \red{f: A \to B \text{ is bijective}}
    \]

    \begin{enumerate}[(i)]
      \item $f \circ f^{-1} = I_B$
      \item $f^{-1} \circ f = I_A$
      \vspace{0.30cm}
      \item $f^{-1} \text{ is bijective}.$
      \vspace{0.30cm}
      \item $g: B \to A \land f \circ g = I_B \implies g = f^{-1}$
      \item $g: B \to A \land g \circ f = I_A \implies g = f^{-1}$
    \end{enumerate}
  \end{theorem}

  \pause
  \begin{center}
    \teal{The ways to find/check $f^{-1}$.}
  \end{center}

  \pause
  \vspace{-0.30cm}
  \[
    g = \purple{f^{-1} \circ (f \circ g) = f^{-1} \circ I_B} = f^{-1}
  \]
\end{frame}
%%%%%%%%%%%%%%%

%%%%%%%%%%%%%%%
% \begin{frame}{}
%   \begin{alertblock}{Bijective $\implies$ Inverse:}
%     \[
%       f: A \to B \text{ is bijective }
%     \]
%     \[ 
%       \implies 
%     \]
%     \[
%       \exists g: B \to A\; \Big( f \circ g = i_B \land g \circ f = i_A \Big) \pause \red{\;\land\; g = f^{-1}}
%     \]
%   \end{alertblock}
% 
%   \pause
%   \vspace{0.50cm}
%   \begin{theorem}[Inverse $\implies$ Bijective (UD Theorem $15.8$ (iii))]
%     \[
%       \exists g: B \to A\; \Big( g \circ f = i_A \land f \circ g = i_B \Big) 
%     \]
%     \[ 
%       \implies 
%     \]
%     \[
%       f: A \to B \text{ is bijective} \pause \red{\;\land\; g = f^{-1}}
%     \]
%   \end{theorem}
% \end{frame}
%%%%%%%%%%%%%%%

%%%%%%%%%%%%%%%
\begin{frame}{}
  \begin{theorem}[Inverse of Composition (UD Theorem $16.6$)]
    \[
      f: A \to B \quad g: B \to C \text{ are bijective}
    \]

    \begin{enumerate}[(i)]
      \item $g \circ f \text{ is bijective}$
      \item $(g \circ f)^{-1} = f^{-1} \circ g^{-1}$
    \end{enumerate}
  \end{theorem}

  \begin{proof}[Proof for (ii)]
    \begin{center}
      \teal{It suffices to check \red{either} one of the following identities:}
    \end{center}

    \[
      (f^{-1} \circ g^{-1}) \circ (g \circ f) = I_A
    \]

    \[
      (g \circ f) \circ (f^{-1} \circ g^{-1}) = I_C
    \]
  \end{proof}
\end{frame}
%%%%%%%%%%%%%%%

%%%%%%%%%%%%%%%
\begin{frame}{}
  \begin{theorem}[UD Theorem $16.8$]
    \[
      f: A \to B \quad g: B \to A
    \]

    \begin{enumerate}[(i)]
      \setcounter{enumi}{2}
      \item $f \circ g = I_B \land g \circ f = I_A \implies g = f^{-1}$
    \end{enumerate}
  \end{theorem}

  \pause
  \begin{center}
    \teal{You need to check \red{both} identities.}
  \end{center}

  \pause
  \begin{theorem}[UD Theorem 16.8]
    \[
      f: A \to B \qquad g: B \to C
    \]

    \begin{enumerate}[(i)]
      \item If $g \circ f$ is surjective, then $g$ is surjective.
      \item If $g \circ f$ is injective, then $f$ is injective.
    \end{enumerate}
  \end{theorem}

  \pause
  \begin{center}
    \teal{First show that $f$ is bijective, and then use Theorem 16.4.}
  \end{center}
\end{frame}
%%%%%%%%%%%%%%%
