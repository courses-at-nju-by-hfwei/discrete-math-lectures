% func-inverse.tex

%%%%%%%%%%%%%%%
\begin{frame}{}
  \begin{center}
    \teal{\LARGE Inverse Functions}
  \end{center}

  \fig{width = 0.20\textwidth}{figs/inverse-function-wiki}
\end{frame}
%%%%%%%%%%%%%%%

%%%%%%%%%%%%%%%
\begin{frame}{}
  \[
    f: A \to B
  \]

  \vspace{0.60cm}
  \begin{center}
    \red{What is the condition for $f^{-1}$ to be a function from $B$ to $A$?}
    \[
      f^{-1}: B \to A
    \]
  \end{center}
\end{frame}
%%%%%%%%%%%%%%%

%%%%%%%%%%%%%%%
\begin{frame}{}
  \begin{definition}[反函数 (Inverse Function)]
    Let $f: A \to B$ be a function. \\[5pt]

    The \red{\it inverse} of $f$ is a \teal{function}
    from $B$ to $A$, denoted \blue{$f^{-1}: B \to A$} \\[3pt]
    if $f$ is bijective. \\[8pt]

    \pause
    We call $f^{-1}$ \blue{the} \red{\it inverse function} of $f$.
  \end{definition}

  \pause
  \vspace{0.50cm}
  \[
    f \text{ is bijective}: \red{\boxed{f(x) = y \iff f^{-1}(y) = x}}
  \]
\end{frame}
%%%%%%%%%%%%%%%

%%%%%%%%%%%%%%%
% \begin{frame}{}
%   \begin{definition}[Invertible]
%     $f: X \to Y$ is {\it invertible} if there exists $g: Y \to X$ such that
%
%     \[
%       f(x) = y \iff g(y) = x.
%     \]
%   \end{definition}
%
%   \pause
%   \begin{theorem}
%     $f$ is invertible $\iff$ $f$ is bijective.
%   \end{theorem}
%
%   \begin{columns}
%     \column{0.50\textwidth}
%       \pause
%       \begin{center}
% 	\blue{$f$ is invertible $\implies$ $f$ is bijective} \pause \\[10pt]
% 	$g$ is a function $\implies$ $f$ is injective \\[5pt]
% 	$\dom{g} = Y \implies f \text{ is surjective}$
%       \end{center}
%     \column{0.50\textwidth}
%       \pause
%       \begin{center}
% 	\blue{$f$ is bijective $\implies$ $f$ is invertible} \pause \\[10pt]
% 	To show that $g$ defined above is indeed a function from $Y$ to $X$.
%       \end{center}
%   \end{columns}
% \end{frame}
%%%%%%%%%%%%%%%

%%%%%%%%%%%%%%%
\begin{frame}{}
  \begin{theorem}
    Suppose that $f: A \to B$ is bijective.
    Then, its inverse function $f^{-1}: B \to A$ is unique.
  \end{theorem}

  \pause
  \[
    \text{\red{By Contradiction}}
  \]

  \pause
  \[
    f(x) = y \iff f^{-1}(y) = x
  \]
  \[
    f(x) = y \iff g(y) = x
  \]

  \pause
  \[
    \forall y \in B.\; f^{-1}(y) = g(y)
  \]
\end{frame}
%%%%%%%%%%%%%%%

%%%%%%%%%%%%%%%
\begin{frame}{}
  examples
\end{frame}
%%%%%%%%%%%%%%%

%%%%%%%%%%%%%%%
\begin{frame}{}
  \begin{theorem}[]
    \[
      \red{\boxed{f: A \to B \text{ is bijective}}}
    \]

    \begin{enumerate}[(i)]
      \item $f \circ f^{-1} = I_B$
      \item $f^{-1} \circ f = I_A$
      \vspace{0.30cm}
      \item \teal{$f^{-1} \text{ is bijective}$}
      \vspace{0.30cm}
      \item $g: B \to A \land f \circ g = I_B \implies g = f^{-1}$
      \item $g: B \to A \land g \circ f = I_A \implies g = f^{-1}$
    \end{enumerate}
  \end{theorem}

  \pause
  \vspace{0.30cm}
  \begin{center}
    \blue{The ways to find/check $f^{-1}$.}
  \end{center}
\end{frame}
%%%%%%%%%%%%%%%

%%%%%%%%%%%%%%%
\begin{frame}{}
  \begin{theorem}
    \[
      \red{f: A \to B \text{ is bijective}}
    \]
    \[
      f \circ f^{-1} = I_{B}
    \]
  \end{theorem}

  \pause
  \vspace{0.30cm}
  \red{对任意 $b \in B$,}
  \[
    (f \circ f^{-1})(b) = f(f^{-1}(b))
  \]
  \pause
  \begin{center}
    Suppose that $a = f^{-1}(b)$
    \[
      \red{\boxed{a = f^{-1}(b) \iff f(a) = b}}
    \]
  \end{center}

  \pause
  \[
    (f \circ f^{-1})(b) = f(f^{-1}(b)) = \red{f(a)} = b
  \]
\end{frame}
%%%%%%%%%%%%%%%

%%%%%%%%%%%%%%%
\begin{frame}{}
  \begin{theorem}
    \[
      \red{f: A \to B \text{ is bijective}}
    \]
    \[
      g: B \to A \land f \circ g = I_B \implies g = f^{-1}
    \]
  \end{theorem}

  \pause
  \[
    g = \purple{(f^{-1} \circ f) \circ g = f^{-1} \circ (f \circ g) = f^{-1} \circ I_B} = f^{-1}
  \]
\end{frame}
%%%%%%%%%%%%%%%

%%%%%%%%%%%%%%%
\begin{frame}{}
  \begin{theorem}[Inverse of Composition]
    \[
      \text{Both } f: A \to B \text{ and } g: B \to C \text{ are bijective}
    \]

    \begin{enumerate}[(i)]
      \item $g \circ f \text{ is bijective}$
      \item $(g \circ f)^{-1} = f^{-1} \circ g^{-1}$
    \end{enumerate}
  \end{theorem}

  \pause
  \begin{proof}[Proof for (ii)]
    \pause
    \begin{center}
      \blue{It suffices to check \red{either} one of the following identities:}
    \end{center}

    \[
      (f^{-1} \circ g^{-1}) \circ (g \circ f) = I_A
    \]
    \[
      (g \circ f) \circ (f^{-1} \circ g^{-1}) = I_C
    \]
  \end{proof}
\end{frame}
%%%%%%%%%%%%%%%

%%%%%%%%%%%%%%%
\begin{frame}{}
  \begin{theorem}[]
    \[
      f: A \to B \quad g: B \to A
    \]

    \begin{enumerate}[(i)]
      \setcounter{enumi}{2}
      \item \teal{$f \circ g = I_B \land g \circ f = I_A \implies g = f^{-1}$}
    \end{enumerate}
  \end{theorem}

  \pause
  \begin{center}
    \red{You do {\it not} know that $f$ is bijective.} \\[5pt]
    \blue{You need to check \red{both} identities.}
  \end{center}

  \pause
  \vspace{0.20cm}
  \begin{center}
    \fig{width = 0.40\textwidth}{figs/recall}
    Use two previous theorems
  \end{center}
\end{frame}
%%%%%%%%%%%%%%%