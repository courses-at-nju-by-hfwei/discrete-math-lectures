% special-func.tex

%%%%%%%%%%%%%%%
\begin{frame}{}
  \begin{center}
    \teal{\LARGE Special Functions (\Large {\it -jectivity})}
  \end{center}
\end{frame}
%%%%%%%%%%%%%%%

%%%%%%%%%%%%%%%
\begin{frame}{}
  \begin{definition}[Injective (one-to-one; 1-1) 单射函数]
    \[
      f: A \to B \qquad \uncover<2->{f: A \red{\;\rightarrowtail\;} B}
    \]

    \vspace{-0.60cm}
    \[
      \forall a_1, a_2 \in A.\; a_1 \neq a_2 \to f(a_1) \neq f(a_2)
    \]
  \end{definition}

  \vspace{0.50cm}
  \uncover<3->{
    \begin{alertblock}{For Proof:}
      \begin{itemize}
        \item To prove that $f$ \blue{\it is} 1-1:
          \[
            \forall a_1, a_2 \in A.\; f(a_1) = f(a_2) \to a_1 = a_2
          \]
        \item<4-> To show that $f$ \blue{\it is not} 1-1:
          \[
            \red{\exists} a_1, a_2 \in A.\; a_1 \neq a_2 \land f(a_1) = f(a_2)
          \]
      \end{itemize}
    \end{alertblock}
  }
\end{frame}
%%%%%%%%%%%%%%%

%%%%%%%%%%%%%%%
\begin{frame}{}
  \begin{definition}[Surjective (onto) 满射函数]
    \[
      f: A \to B \qquad \uncover<2->{f: A \red{\;\twoheadrightarrow\;} B}
    \]

    \vspace{-0.60cm}
    \[
      \ran(f) = B
    \]
  \end{definition}

  \vspace{0.50cm}
  \uncover<3->{
    \begin{alertblock}{For Proof:}
      \begin{itemize}
        \item To prove that $f$ \blue{\it is} onto:
          \[
            \forall b \in B. \; \big(\red{\exists} a \in A.\; f(a) = b \big)
          \]
        \item<4-> To show that $f$ \blue{\it is not} onto:
          \[
            \red{\exists} b \in B.\; \big(\red{\forall} a \in A.\; f(a) \neq b \big)
          \]
        \end{itemize}
    \end{alertblock}
  }
\end{frame}
%%%%%%%%%%%%%%%

%%%%%%%%%%%%%%%
\begin{frame}{}
  \begin{definition}[Bijective (one-to-one correspondence) 双射; 一一对应]
    \[
      f: A \to B \qquad \uncover<2->{f: A \red{\;\xleftrightarrow[onto]{1-1}\;} B}
    \]

    \begin{center}
      {1-1 \& onto}
    \end{center}
  \end{definition}
\end{frame}
%%%%%%%%%%%%%%%

%%%%%%%%%%%%%%%
\begin{frame}{}
  \[
    f: \Z \to \N, \qquad f(x) = x^2 + 1
  \]

  \pause
  \[
    f: \N \to \Q, \qquad f(x) = \frac{1}{x}
  \]

  \pause
  \[
    f: \N \to \N, \qquad f(x) = 2^{x}
  \]

  \pause
  \[
    f: \Z \times \N \to \Q, \qquad f(z, n) = \frac{z}{n+1}
  \]

  \pause
  \[
    f: \R^{2} \to \R^{2}, \qquad f(x,y) = (x+1, y+1)
  \]
\end{frame}
%%%%%%%%%%%%%%%

%%%%%%%%%%%%%%%
% cantor-theorem.tex

%%%%%%%%%%%%%%%
\begin{frame}{}
  \begin{theorem}[Cantor Theorem]
    If $f: A \to 2^{A}$, then $f$ is \red{not} onto.
  \end{theorem}

  \pause
  \vspace{0.60cm}
  \fig{width = 0.70\textwidth}{figs/cantor-theorem-proof}
\end{frame}
%%%%%%%%%%%%%%%

%%%%%%%%%%%%%%%
\begin{frame}{}
  \begin{theorem}[Cantor Theorem]
    If $f: A \to 2^{A}$, then $f$ is \red{not} onto.
  \end{theorem}

  \vspace{0.60cm}
  \begin{columns}
    \column{0.28\textwidth}
      \fig{width = 1.00\textwidth}{figs/talking-about}
      \pause
      \column{0.25\textwidth}
        \fig{width = 0.90\textwidth}{figs/interesting}
      \pause
      \column{0.25\textwidth}
        \fig{width = 0.80\textwidth}{figs/genius}
      \pause
      \column{0.25\textwidth}
        \fig{width = 1.00\textwidth}{figs/stupid}
    \end{columns}
\end{frame}
%%%%%%%%%%%%%%%

%%%%%%%%%%%%%%%
\begin{frame}{}
  \begin{theorem}[Cantor Theorem]
    If $f: A \to 2^{A}$, then $f$ is \red{not} onto.
  \end{theorem}

  \vspace{0.30cm}
  \purple{Understanding this problem:}
  \[
    A = \set{1,2,3}
  \]
  \pause
  \[
    2^{A} = \ps{A} = \Big\{\emptyset, \set{1}, \set{2}, \set{3}, \set{1,2}, \set{1,3}, \set{2,3}, \set{1,2,3}\Big\}
  \]

  \pause
  \begin{description}
    \item[Onto]
      \[
        \forall B \in 2^{A}.\; \Big(\exists a \in A.\; f(a) = B\Big)
      \]
    \pause
    \item[Not Onto]
      \[
        \red{\exists} B \in 2^{A}.\; \Big(\red{\forall} a \in A.\; f(a) \neq B\Big)
      \]
  \end{description}
\end{frame}
%%%%%%%%%%%%%%%

%%%%%%%%%%%%%%%
\begin{frame}{}
  \begin{align*}
    f(1) &= \set{1, 2} \\
    f(2) &= \set{1, 3} \\
    f(3) &= \emptyset
  \end{align*}

  \pause
  \[
    B = \set{2, 3}
  \]

  \pause
  \[
    B = \set{x \in \set{1, 2, 3} \mid x \notin f(x)} = \set{2, 3}
  \]
\end{frame}
%%%%%%%%%%%%%%%

%%%%%%%%%%%%%%%
\begin{frame}{}
  \begin{theorem}[Cantor Theorem]
    If $f: A \to 2^{A}$, then $f$ is \red{not} onto.
  \end{theorem}

  \[
    \red{\exists} B \in 2^{A}.\; \Big(\red{\forall} a \in A.\; f(a) \neq B\Big)
  \]

  \pause
  \begin{columns}[t]
    \column{0.50\textwidth}
      \begin{itemize}
        \item<2-> Constructive proof (\red{$\exists$}):
          \[
            B = \set{a \in A \mid a \notin f(a)}
          \]
        \item<4-> By contradiction (\red{$\forall$}):
          \[
            \exists a \in A.\; f(a) = B.
          \]
      \end{itemize}
    \column{0.40\textwidth}
      \uncover<3->{\fig{width = 0.80\textwidth}{figs/what-is-this}}
  \end{columns}

  \vspace{-0.30cm}
  \uncover<5->{
    \[
      \red{Q: a \in B{\it ?}}
    \]
  }

  \vspace{-0.60cm}
  \uncover<6->{
    \[
      \teal{a \in B \iff a \notin B}
    \]
  }
\end{frame}
%%%%%%%%%%%%%%%

%%%%%%%%%%%%%%%
\begin{frame}{}
  \begin{theorem}[Cantor Theorem]
    If $f: A \to 2^{A}$, then $f$ is \red{not} onto.
  \end{theorem}

  \begin{proof}[对角线论证 (Cantor's diagonal argument) \only<5->{\footnotesize \purple{(以下仅适用于可数集合 $A$)}}]
    \pause
    \begin{table}[]
      \centering
      $\begin{tabu}{|c||c|c|c|c|c|c|}
        \hline
        a      & \multicolumn{6}{c|}{f(a)} \\ \hline
              & 1      & 2      & 3      & 4      & 5      & \cdots \\ \hline \hline
        1      & \redoverlay{1}{3-}      & 1      & 0      & 0      & 1      & \cdots \\ \hline
        2      & 0      & \redoverlay{0}{3-}      & 0      & 0      & 0      & \cdots \\ \hline
        3      & 1      & 0      & \redoverlay{0}{3-}      & 1      & 0      & \cdots \\ \hline
        4      & 1      & 1      & 1      & \redoverlay{1}{3-}      & 1      & \cdots \\ \hline
        5      & 0      & 1      & 0      & 1      & \redoverlay{0}{3-}      & \cdots \\ \hline
        \vdots & \vdots & \vdots & \vdots & \vdots & \vdots & \cdots \\ \hline
      \end{tabu}$
    \end{table}

    \uncover<4->{
      \[
        B = \blue{\set{0, 1, 1, 0, 1}}
      \]
    }
  \end{proof}
\end{frame}
%%%%%%%%%%%%%%%
%%%%%%%%%%%%%%%

%%%%%%%%%%%%%%%
% \begin{frame}{}
%   \begin{exampleblock}{UD Problem $14.12$}
%     \[
%       a, b, c, d \in \real{},\; a < b,\; c < d
%     \]
%
%     \vspace{0.30cm}
%     \centerline{Define a bijective function:}
%     \vspace{0.10cm}
%
%     \[
%       f: [a,b] \xleftrightarrow[onto]{1-1} [c,d]
%     \]
%     \[
%       f: (a,b) \xleftrightarrow[onto]{1-1} (c,d)
%     \]
%   \end{exampleblock}
% 
%   \pause
%   \begin{proof}[Answer]
%     \[
%       f(x) = c + \frac{d-c}{b-a}(x-a)
%     \]
%   \end{proof}
% \end{frame}
%%%%%%%%%%%%%%%
