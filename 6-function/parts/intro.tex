% intro.tex

%%%%%%%%%%%%%%%
\begin{frame}{}
  \fig{width = 0.85\textwidth}{figs/set-theory-aspects}
\end{frame}
%%%%%%%%%%%%%%%

%%%%%%%%%%%%%%%
\begin{frame}{}
  \begin{center}
    从``关系''的角度理解``函数''
  \end{center}

  \fig{width = 0.50\textwidth}{figs/relation-function}
\end{frame}
%%%%%%%%%%%%%%%

%%%%%%%%%%%%%%%
\begin{frame}{}
  \[
    f(x) = 2x + 1
  \]
  \fig{width = 0.50\textwidth}{figs/f(x)}
  \begin{center}
    ``函数''也是``关系''
  \end{center}
  \[
    \set{\dots, (-2, -3), (-1, -1), (0, 1), (1, 3), \dots}
  \]
\end{frame}
%%%%%%%%%%%%%%%

%%%%%%%%%%%%%%%
\begin{frame}{}
  \[
    (x^2 + y^2)^2 + 4 a x (x^2 + y^2) - 4a^2 y^2 = 0
  \]

  \fig{width = 0.50\textwidth}{figs/heart-graph}

  \begin{center}
    ``函数''\red{不允许}``一对多''
  \end{center}
\end{frame}
%%%%%%%%%%%%%%%

%%%%%%%%%%%%%%%
\begin{frame}{}
  \begin{center}
    \teal{\LARGE Functions}

    \pause
    \fig{width = 0.25\textwidth}{figs/aims}

    \begin{center}
      \red{\Large PROOF!}
    \end{center}
  \end{center}
\end{frame}
%%%%%%%%%%%%%%%