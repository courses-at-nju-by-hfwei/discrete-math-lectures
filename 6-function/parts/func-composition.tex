% func-composition.tex

%%%%%%%%%%%%%%%
\begin{frame}{}
  \begin{center}
    \teal{\LARGE Function Composition}
  \end{center}

  \fig{width = 0.40\textwidth}{figs/function-composition}
\end{frame}
%%%%%%%%%%%%%%%

%%%%%%%%%%%%%%%
\begin{frame}{}
  \begin{definition}[Composition]
    \[
      f: A \to B \qquad g: C \to D
    \]
    \[
      \red{\ran(f) \subseteq C}
    \]

    The \red{\it composite function} \blue{$g \circ f: A \to D$} is defined as

    \[
      (g \circ f) (x) = g(f(x))
    \]
  \end{definition}

  \begin{center}
    \uncover<3->{\red{Why not ``$\exists b$'' as below?}}
  \end{center}

  \pause
  \vspace{-0.30cm}
  \begin{definition}[Composition]
    The {\it composition} of relations $R$ and $S$ is the relation
    \[
      R \circ S = \set{(a, c) \mid \red{\exists b}: (a, b) \in S \land (b, c) \in R}
    \]
  \end{definition}
\end{frame}
%%%%%%%%%%%%%%%

%%%%%%%%%%%%%%%
\begin{frame}{}
  \begin{theorem}[Associative Property for Composition]
    \[
      f: A \to B \quad g: B \to C \quad h: C \to D
    \]

    \[
      h \circ (g \circ f) = (h \circ g) \circ f
    \]
  \end{theorem}

  \pause
  \vspace{0.60cm}
  \begin{proof}
    \pause
    \begin{enumerate}[(i)]
      \item
        \[
          \dom(h \circ (g \circ f)) = \dom((h \circ g) \circ f)
        \]
      \item
        \[
          \forall x \in A.\; (h \circ (g \circ f))(x) = ((h \circ g) \circ f)(x)
        \]
    \end{enumerate}
  \end{proof}
\end{frame}
%%%%%%%%%%%%%%%

%%%%%%%%%%%%%%%
\begin{frame}{}
  \[
    (h \circ (g \circ f))(x) = ((h \circ g) \circ f)(x)
  \]

  \vspace{0.50cm}
  \begin{columns}
    \column{0.50\textwidth}
      \pause
      \setcounter{equation}{0}
      \begin{align}
        &(h \circ (g \circ f))(x) \\[6pt]
        =\; &h((g \circ f) (x)) \\[6pt]
        =\; &h(g(f(x)))
      \end{align}
    \column{0.50\textwidth}
      \pause
      \setcounter{equation}{0}
      \begin{align}
        &((h \circ g) \circ f)(x) \\[6pt]
        =\; &((h \circ g) (f(x))) \\[6pt]
        =\; &h(g(f(x)))
      \end{align}
  \end{columns}
\end{frame}
%%%%%%%%%%%%%%%

%%%%%%%%%%%%%%%
\begin{frame}{}
  \begin{theorem}[]
    \[
      f: A \to B \qquad g: B \to C
    \]

    \begin{enumerate}[(i)]
      \setlength{\itemsep}{6pt}
      \item If $f, g$ are injective, then $g \circ f$ is injective.
      \item \teal{If $f, g$ are surjective, then $g \circ f$ is surjective.}
      \item If $f, g$ are bijective, then $g \circ f$ is bijective.
    \end{enumerate}
  \end{theorem}
\end{frame}
%%%%%%%%%%%%%%%

%%%%%%%%%%%%%%%
\begin{frame}{}
  \begin{theorem}
    \[
      f: A \to B \qquad g: B \to C
    \]
    \begin{center}
      If $f, g$ are injective, then $g \circ f$ is injective.
    \end{center}
  \end{theorem}

  \pause
  \[
    \forall a_1, a_2 \in A.\;
      \Big( (g \circ f)(a_1) = (g \circ f)(a_2) \to a_1 = a_2 \Big)
  \]

  \pause
  \vspace{-0.30cm}
  \setcounter{equation}{0}
  \begin{align}
    &(g \circ f)(a_1) = (g \circ f)(a_2) \\[6pt]
    \uncover<3->{\iff &g(f(a_{1})) = g(f(a_{2})) \\[6pt]}
    \uncover<4->{\implies &f(a_{1}) = f(a_{2}) \\[6pt]}
    \uncover<5->{\implies &a_{1} = a_{2}}
  \end{align}
\end{frame}
%%%%%%%%%%%%%%%

%%%%%%%%%%%%%%%
% \begin{frame}{}
%   \[
%     f: A \to B \qquad g: B \to C
%   \]
%   \begin{center}
%     If $f, g$ are surjective, then $g \circ f$ is surjective.
%   \end{center}
%
%   \pause
%   \[
%     \forall c \in C.\; \Big( \exists a \in A.\; (g \circ f)(a) = c \Big)
%   \]
% \end{frame}
%%%%%%%%%%%%%%%

%%%%%%%%%%%%%%%
\begin{frame}{}
  \begin{theorem}[]
    \[
      f: A \to B \qquad g: B \to C
    \]

    \begin{enumerate}[(i)]
      \item \teal{If $g \circ f$ is injective, then $f$ is injective.}
      \item If $g \circ f$ is surjective, then $g$ is surjective.
    \end{enumerate}
  \end{theorem}
\end{frame}
%%%%%%%%%%%%%%%

%%%%%%%%%%%%%%%
% \begin{frame}{}
%   \begin{theorem}[]
%     \[
%       f: A \to B \qquad g: B \to C
%     \]
%
%     \begin{center}
%       \teal{If $g \circ f$ is injective, then $f$ is injective.}
%     \end{center}
%   \end{theorem}
%
%   \pause
%   \vspace{0.30cm}
%   \red{对任意 $a_{1}, a_{2}$,}
%   \setcounter{equation}{0}
%   \begin{align}
%     &f(a_{1}) = f(a_{2}) \\[6pt]
%     \uncover<3->{\implies & g(f(a_{1})) = g(f(a_{2})) \\[6pt]}
%     \uncover<4->{\implies & (g \circ f) (a_{1}) = (g \circ f) (a_{2}) \\[6pt]}
%     \uncover<5->{\implies & a_{1} = a_{2}}
%   \end{align}
% \end{frame}
%%%%%%%%%%%%%%%

%%%%%%%%%%%%%%%
\begin{frame}{}
  \begin{theorem}[]
    \[
      f: A \to B \qquad g: B \to C
    \]

    \begin{center}
      If $g \circ f$ is surjective, then $g$ is surjective.
    \end{center}
  \end{theorem}

  \pause
  \vspace{0.30cm}
  \red{对任意 $a_{1}, a_{2}$,}
  \setcounter{equation}{0}
  \begin{align}
    & g \circ f \text{ is surjective}\\[6pt]
    \uncover<3->{\iff & \forall c \in C.\; \exists a \in A.\; (g \circ f)(a) = c \\[6pt]}
    \uncover<4->{\iff & \forall c \in C.\; \exists a \in A.\; g(f(a)) = c \\[6pt]}
    \uncover<5->{\red{\implies} & \forall c \in C.\; \exists b \in B.\; g(b) = c \\[6pt]}
    \uncover<6->{\iff & g \text{ is surjective}}
  \end{align}
\end{frame}
%%%%%%%%%%%%%%%