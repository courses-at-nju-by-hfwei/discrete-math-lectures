% bridg-it.tex

%%%%%%%%%%%%%%%%%%%%
\begin{frame}
  \begin{exampleblock}{Bridg-It Game (David Gale, 1958)}
    \fig{width = 0.45\textwidth}{figs/bridg-it}
    \[
      \red{5 \times 6} \;\text{\it vs.}\; \blue{6 \times 5}
    \]
  \end{exampleblock}
\end{frame}
%%%%%%%%%%%%%%%%%%%%

%%%%%%%%%%%%%%%%%%%%
\begin{frame}
  \begin{center}
    \fig{width = 0.50\textwidth}{figs/red-win}
    \vspace{-0.60cm}
    \[
      {\red{5 \times 6} \;\text{\it vs.}\; \blue{6 \times 5}}
    \]
  \end{center}
\end{frame}
%%%%%%%%%%%%%%%%%%%%

%%%%%%%%%%%%%%%%%%%%
\begin{frame}
  \begin{center}
    \href{https://ludii.games/index.php}{\cyan{\bf \Large Let's Play with it!}}
  \end{center}
\end{frame}
%%%%%%%%%%%%%%%%%%%%

%%%%%%%%%%%%%%%%%%%%
\begin{frame}
  \begin{center}
    {\Large Let's Analyze it!}
  \end{center}
\end{frame}
%%%%%%%%%%%%%%%%%%%%

%%%%%%%%%%%%%%%%%%%%
\begin{frame}
  \begin{center}
    Will Bridg-It \red{\bf end in a tie}?

    \pause
    \vspace{0.80cm}
    No! By \blue{\bf contradiction}.
  \end{center}
\end{frame}
%%%%%%%%%%%%%%%%%%%%

%%%%%%%%%%%%%%%%%%%%
\begin{frame}{}
  \begin{center}
    Does \red{\bf Player 2} have a \red{\bf winning strategy}?

    \pause
    \vspace{0.80cm}
    No! By the \blue{\bf strategy stealing argument}.
  \end{center}
\end{frame}
%%%%%%%%%%%%%%%%%%%%

%%%%%%%%%%%%%%%%%%%%
\begin{frame}{}
  \begin{center}
    Does \red{\bf Player 1} have a \red{\bf winning strategy}?

    \pause
    \vspace{0.80cm}
    Yes! It uses \blue{\bf spanning trees} in \blue{\bf graph theory}.

    \fig{width = 0.50\textwidth}{figs/stay-tuned}
  \end{center}
\end{frame}
%%%%%%%%%%%%%%%%%%%%