% axiom.tex

%%%%%%%%%%%%%%%
\begin{frame}{}
\end{frame}
%%%%%%%%%%%%%%%

%%%%%%%%%%%%%%%
\begin{frame}{}
\end{frame}
%%%%%%%%%%%%%%%

%%%%%%%%%%%%%%%
% \begin{frame}{}
%   \begin{center}
%     {\Large Axiomatic Set Theory \blue{(ZFC)}}
%   \end{center}
%
%   \vspace{0.80cm}
%   \begin{columns}
%     \column{0.45\textwidth}
%       \fig{width = 0.50\textwidth}{figs/Zermelo}{\centerline{Ernst Zermelo (1871--1953)}}
%     \column{0.45\textwidth}
%       \fig{width = 0.48\textwidth}{figs/Fraenkel}{\centerline{Abraham Fraenkel (1891--1965)}}
%   \end{columns}
% \end{frame}
%%%%%%%%%%%%%%%

%%%%%%%%%%%%%%%
\begin{frame}{}
  \fig{width = 0.30\textwidth}{figs/elements-ch}

  \begin{enumerate}[(1)]
    \item To draw a straight line from any point to any point.
    \item To extend a finite straight line continuously in a straight line.
    \item To describe a circle with any center and radius.
    \item That all right angles are equal to one another.
    \item The parallel postulate.
  \end{enumerate}
\end{frame}
%%%%%%%%%%%%%%%

%%%%%%%%%%%%%%%%%%%%
\begin{frame}
\end{frame}
%%%%%%%%%%%%%%%%%%%%