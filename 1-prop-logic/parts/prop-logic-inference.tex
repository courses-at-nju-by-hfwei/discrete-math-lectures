% prop-logic-inference.tex

%%%%%%%%%%%%%%%%%%%%
\begin{frame}{}
  \begin{center}
    {\Large 命题逻辑的自然推理 (演绎; 推演) 系统}
  \end{center}
\end{frame}
%%%%%%%%%%%%%%%%%%%%

%%%%%%%%%%%%%%%%%%%%
\begin{frame}{}
  \begin{center}
    从上往下看: 展示证明过程

    \[
      \nd{\alpha \qquad \beta \qquad \dots \quad\text{(前提)}}{\gamma \quad\text{(结论)}}{\text{规则名称}}
    \]

    \[
      \red{\set{\alpha, \beta, \dots} \vdash \gamma}
    \]

    \vspace{0.80cm}
    从下往上看: 提供证明思路
  \end{center}
\end{frame}
%%%%%%%%%%%%%%%%%%%%

%%%%%%%%%%%%%%%%%%%%
\begin{frame}
  \[
    \nd{}{[x : P]}{\text{assum}}
  \]

  \vspace{0.50cm}
  \begin{center}
    所有引入的假设最终必须被``\red{\bf 释放}''(discharged)
  \end{center}
\end{frame}
%%%%%%%%%%%%%%%%%%%%

%%%%%%%%%%%%%%%%%%%%
\begin{frame}{$\land$}
  \begin{gather*}
    \nd{P \qquad Q}{P \land Q}{\land\text{-intro}} \\[35pt]
    \nd{P \land Q}{P}{\land\text{-elim-left}} \\[35pt]
    \nd{P \land Q}{Q}{\land\text{-elim-right}}
  \end{gather*}
\end{frame}
%%%%%%%%%%%%%%%%%%%%

%%%%%%%%%%%%%%%%%%%%
\begin{frame}{}
  \begin{exampleblock}{``$\land$''推理规则的应用}
    \[
      \set{p \land q, r} \vdash q \land r
    \]
  \end{exampleblock}

  \pause
  \vspace{0.50cm}
  \begin{proof}
    \begin{align}
      &p \land q &\text{前提} \label{eqn:pq} \\
      &r &\text{前提} \label{eqn:r} \\
      &q &\land{\text{-elim-right}\; (\ref{eqn:pq}}) \label{eqn:q} \\
      &q \land r &\land{\text{-intro}}\; (\ref{eqn:q}), (\ref{eqn:r})
    \end{align}
  \end{proof}
\end{frame}
%%%%%%%%%%%%%%%%%%%%

%%%%%%%%%%%%%%%%%%%%
\begin{frame}{$\lnot\lnot$}
  \[
    \nd{\alpha}{\lnot \lnot \alpha}{\lnot\lnot\text{-intro}}
  \]

  \vspace{0.60cm}
  \[
    \nd{\lnot \lnot \alpha}{\alpha}{\lnot\lnot\text{-elem}}
  \]
\end{frame}
%%%%%%%%%%%%%%%%%%%%

%%%%%%%%%%%%%%%%%%%%
\begin{frame}{}
  \begin{exampleblock}{``$\lnot\lnot$''推理规则的应用}
    \[
      \set{p, \lnot\lnot(q \land r)} \vdash \lnot\lnot p \land r
    \]
  \end{exampleblock}
\end{frame}
%%%%%%%%%%%%%%%%%%%%

%%%%%%%%%%%%%%%%%%%%
\begin{frame}{$\to$}
  \[
    \nd{\alpha \to \beta \qquad \alpha}{\beta}{\to\text{-elim (modus ponens)}}
  \]

  \vspace{0.60cm}
  \[
    \nd{\alpha \to \beta \qquad \lnot \beta}{\lnot \alpha}{\text{modus tollens}}
  \]
\end{frame}
%%%%%%%%%%%%%%%%%%%%

%%%%%%%%%%%%%%%%%%%%
\begin{frame}
  \begin{exampleblock}{``$\to$''推理规则的应用}
    \[
      \set{p \to (q \to r), p, \lnot r} \vdash \lnot q
    \]
  \end{exampleblock}
\end{frame}
%%%%%%%%%%%%%%%%%%%%

%%%%%%%%%%%%%%%%%%%%
\begin{frame}{$\to$}
  \begin{gather*}
    \nd{\stackrel{[x \;:\; \alpha]}{\stackrel{\vdots}{\beta}}}{\alpha \to \beta}{\to\text{-intro}/x} \\[10pt]
    \text{\blue{Assumption $x$ is discharged}}
  \end{gather*}
\end{frame}
%%%%%%%%%%%%%%%%%%%%

%%%%%%%%%%%%%%%%%%%%
\begin{frame}{}
  \begin{exampleblock}{``$\to$''推理规则的应用}
    \[
      \vdash p \to p
    \]
  \end{exampleblock}

  \pause
  \vspace{0.80cm}
  \begin{exampleblock}{``$\to$''推理规则的应用}
    \[
      \set{\lnot q \to \lnot p} \vdash p \to \lnot\lnot q
    \]
  \end{exampleblock}
\end{frame}
%%%%%%%%%%%%%%%%%%%%

%%%%%%%%%%%%%%%%%%%%
\begin{frame}{}
  \begin{exampleblock}{``$\to$''推理规则的应用}
    \[
      \set{p \land q \to r} \vdash p \to (q \to r)
    \]
  \end{exampleblock}
\end{frame}
%%%%%%%%%%%%%%%%%%%%

%%%%%%%%%%%%%%%%%%%%
\begin{frame}{$\lor$}
  \[
    \nd{\alpha}{\alpha \lor \beta}{\lor\text{-intro-left}}
  \]

  \vspace{0.60cm}
  \[
    \nd{\beta}{\alpha \lor \beta}{\lor\text{-intro-right}}
  \]

  \pause
  \vspace{0.60cm}
  \[
    \nd{\alpha \lor \beta \qquad \alpha \to \gamma \qquad \beta \to \gamma}{\gamma}{\lor\text{-elim; (分情况分析)}}
  \]
\end{frame}
%%%%%%%%%%%%%%%%%%%%

%%%%%%%%%%%%%%%%%%%%
\begin{frame}{}
  \begin{exampleblock}{``$\lor$''推理规则的应用}
    \[
      p \lor q \vdash q \lor p
    \]
  \end{exampleblock}

  \pause
  \vspace{0.50cm}
  \[
    \ndnoname{p \lor q \qquad p \to q \lor p \qquad q \to q \lor p}{q \lor p}
  \]
\end{frame}
%%%%%%%%%%%%%%%%%%%%

%%%%%%%%%%%%%%%%%%%%
\begin{frame}{}
  \begin{exampleblock}{``$\lor$''-推理规则的应用}
    \[
      q \to r \vdash (p \lor q) \to (p \lor r)
    \]
  \end{exampleblock}
\end{frame}
%%%%%%%%%%%%%%%%%%%%

%%%%%%%%%%%%%%%%%%%%
\begin{frame}{}
  \begin{exampleblock}{$\lor$-推理规则的应用}
    \[
      p \land (q \lor r) \vdash (p \land q) \lor (p \land r)
    \]
  \end{exampleblock}

  \vspace{0.60cm}
  \begin{exampleblock}{$\lor$-推理规则的应用 (请自行练习)}
    \[
      (p \land q) \lor (p \land r) \vdash p \land (q \lor r)
    \]
  \end{exampleblock}
\end{frame}
%%%%%%%%%%%%%%%%%%%%

%%%%%%%%%%%%%%%%%%%%
\begin{frame}{$\bot$}
  \[
    \nd{\alpha \qquad \lnot \alpha}{\bot}{\bot\text{-intro}}
  \]

  \[
    \nd{\bot}{\alpha}{\bot\text{-elim}; \text{EFQ, ex falso quodlibet (Principle of Explosion)}}
  \]
\end{frame}
%%%%%%%%%%%%%%%%%%%%

%%%%%%%%%%%%%%%%%%%%
\begin{frame}{}
  \begin{exampleblock}{``$\bot$''-推理规则的应用}
    \[
      \lnot p \lor q \vdash p \to q
    \]
  \end{exampleblock}

  % \pause
  % \vspace{0.60cm}
  % \begin{exampleblock}{``$\bot$''-推理规则的应用}
  %   \[
  %     p \to q \vdash \lnot p \lor q
  %   \]
  % \end{exampleblock}
\end{frame}
%%%%%%%%%%%%%%%%%%%%

%%%%%%%%%%%%%%%%%%%%
\begin{frame}{$\lnot$}
  \begin{gather*}
    \nd{\alpha \to \bot}{\lnot \alpha}{\lnot\text{-intro}} \\[35pt]
    \nd{\lnot \alpha}{\alpha \to \bot}{\lnot\text{-elim}}
  \end{gather*}
\end{frame}
%%%%%%%%%%%%%%%%%%%%

%%%%%%%%%%%%%%%%%%%%
\begin{frame}{}
  \begin{exampleblock}{``$\lnot$''-推理规则的应用}
    \[
      \set{p \to q, p \to \lnot q} \vdash \lnot p
    \]
  \end{exampleblock}

  % \pause
  % \vspace{0.60cm}
  % \begin{exampleblock}{``$\lnot$''-推理规则的应用}
  %   \[
  %     \set{(p \land \lnot q) \to r, \lnot r, p} \vdash q
  %   \]
  % \end{exampleblock}
  % \begin{center}
  %   (提示: $\lnot \lnot q \vdash q$)
  % \end{center}
\end{frame}
%%%%%%%%%%%%%%%%%%%%

%%%%%%%%%%%%%%%%%%%%
\begin{frame}{}
  \begin{exampleblock}{``$\lnot$''-推理规则的应用}
    \[
      \nd{\lnot p \to \bot}{p}{
        \text{RAA}/x \text{, reductio ad absurdum (反证法)}}
    \]
  \end{exampleblock}

  \begin{center}
    (提示: $\lnot \lnot q \vdash q$)
  \end{center}
\end{frame}
%%%%%%%%%%%%%%%%%%%%

%%%%%%%%%%%%%%%%%%%%
\begin{frame}{}
  \begin{exampleblock}{}
    \[
      \vdash p \lor \lnot p \qquad \text{(排中律; Law of the Excluded Middle (LEM))}
    \]
  \end{exampleblock}

  \setcounter{equation}{0}
  \begin{align}
    \blue{\lnot (p \lor \lnot p)} & \qquad (\text{引入假设}) \\
    \red{[p]}                     & \qquad (\text{引入假设}) \\
    \red{p \lor \lnot p}          & \qquad (\lor\text{-intro-left}) \\
    \red{\bot}                    & \qquad (\bot\text{-intro}) \\
    \purple{\lnot p}              & \qquad (\lnot\text{-intro}) \\
    \blue{p \lor \lnot p}         & \qquad (\lor\text{-intro-right}) \\
    \blue{\bot}                   & \qquad (\bot\text{-intro}) \\
    \lnot\lnot (p \lor \lnot p)   & \qquad (\lnot\text{-intro}) \\
    p \lor \lnot p                & \qquad (\lnot\lnot\text{-elim})
  \end{align}
\end{frame}
%%%%%%%%%%%%%%%%%%%%

%%%%%%%%%%%%%%%%%%%%
\begin{frame}{}
  \begin{center}
    \red{\bf \large 连这些推理规则也一并忘却吧!!!}

    \vspace{0.30cm}
    \fig{width = 0.80\textwidth}{figs/li}
  \end{center}
\end{frame}
%%%%%%%%%%%%%%%%%%%%

%%%%%%%%%%%%%%%%%%%%
\begin{frame}{}
  \begin{exampleblock}{使用命题逻辑进行推理}
    某公司要从赵、钱、孙、李、吴5名员工中选派某些人出国考察。\\
    由于某些不可描述的原因, 选派要求如下:

    \begin{columns}
      \column{0.50\textwidth}
        \begin{enumerate}[(1)]
          \item 若赵去, 钱也去;
          \item 李、吴两人中必有一人去;
          \item 钱、孙两人中去且仅去一人;
          \item 孙、李两人同去或同不去;
          \item 若吴去, 则赵、钱也去;
          \item 只有孙去, 赵才会去。
        \end{enumerate}
      \column{0.50\textwidth}
        \begin{enumerate}[(1)]
          \item $\teal{Z \to Q}$;
          \item $\teal{L \lor W}$;
          \item $\teal{(Q \land \lnot S) \lor (S \land \lnot Q)}$;
          \item $\teal{(S \land L) \lor (\lnot S \land \lnot L)}$;
          \item $\teal{W \to Z \land Q}$;
          \item $\teal{Z \to S}$。
        \end{enumerate}
    \end{columns}

    \vspace{0.50cm}
    \blue{请使用形式化推理的方法帮该公司判断应选哪些人出国考察。}
  \end{exampleblock}
\end{frame}
%%%%%%%%%%%%%%%%%%%%

%%%%%%%%%%%%%%%%%%%%
\begin{frame}{}
  \begin{theorem}[命题逻辑的可靠性 (Soundness)]
    如果 $\Sigma \vdash \alpha$, 则 $\Sigma \models \alpha$。
  \end{theorem}

  \vspace{0.30cm}
  \fig{width = 0.50\textwidth}{figs/syntax-semantics}
  \vspace{0.30cm}

  \begin{theorem}[命题逻辑的完备性 (Completeness)]
    如果 $\Sigma \models \alpha$, 则 $\Sigma \vdash \alpha$。
  \end{theorem}
\end{frame}
%%%%%%%%%%%%%%%%%%%%