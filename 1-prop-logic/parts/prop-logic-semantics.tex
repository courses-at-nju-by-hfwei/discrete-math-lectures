% prop-logic-semantics.tex

%%%%%%%%%%%%%%%%%%%%
\begin{frame}{}
  \fig{width = 0.60\textwidth}{figs/syntax-semantics}
  \begin{center}
    命题逻辑的语义
  \end{center}
\end{frame}
%%%%%%%%%%%%%%%%%%%%

%%%%%%%%%%%%%%%%%%%%
\begin{frame}{}
  \begin{center}
    命题逻辑公式的语义就是该公式的\red{``真 ($T/1$)''、``假 ($F/0$)''}值。

    \[
      (((B \to (A \to C)) \leftrightarrow ((B \land A) \to C)))
    \]

    \pause
    \vspace{0.60cm}
    某个公式$\alpha$的真假值取决于
    \vspace{0.30cm}
    \begin{columns}
      \column{0.25\textwidth}
      \column{0.50\textwidth}
        \begin{itemize}
          \setlength{\itemsep}{6pt}
          \item $\alpha$ 中所含命题符号的真假值; 与
          \item 逻辑连接符号的含义
        \end{itemize}
      \column{0.25\textwidth}
    \end{columns}
  \end{center}
\end{frame}
%%%%%%%%%%%%%%%%%%%%

%%%%%%%%%%%%%%%%%%%%
\begin{frame}{}
  \begin{definition}[真值指派 ($v$)]
    令 $S$ 为一个命题符号的集合。
    $S$ 上的一个\red{\bf 真值指派} $v$ 是一个从 $S$ 到真假值的映射
    \[
      v: S \to \set{T, F}.
    \]
  \end{definition}

  \pause
  \vspace{0.30cm}
  \[
    \alpha: (((B \to (A \to C)) \leftrightarrow ((B \land A) \to C)))
  \]

  \[
    v(A) = v(B) = T, \quad v(C) = F
  \]
\end{frame}
%%%%%%%%%%%%%%%%%%%%

%%%%%%%%%%%%%%%%%%%%
\begin{frame}{}
  \begin{center}
    使用\red{\bf 真值表}定义\blue{\bf 逻辑联接符的语义}
  \end{center}

  \begin{definition}[真值表 (Truth Table)]
    % truth-table.tex

\begin{table}
  \centering
  \begin{tabular}{|c|c|c|c|c|c|c|}
    \hline
    $\alpha$ & $\beta$ & $\lnot \alpha$ &
    $\alpha \land \beta$ & $\alpha \lor \beta$ &
    $\alpha \to \beta$ & $\alpha \leftrightarrow \beta$ \\
    \hline \hline
    $T$ & $T$ & $F$ & \red{$T$} & $T$ & $T$ & \cyan{$T$} \\
    \hline
    $T$ & $F$ & $F$ & $F$ & $T$ & \violet{$F$} & $F$ \\
    \hline
    $F$ & $T$ & $T$ & $F$ & $T$ & $T$ & $F$ \\
    \hline
    $F$ & $F$ & $T$ & $F$ & \purple{$F$} & $T$ & \cyan{$T$} \\
    \hline
  \end{tabular}
\end{table}
  \end{definition}

  ``如果中国足球队夺冠, 我就好好学习''
\end{frame}
%%%%%%%%%%%%%%%%%%%%

%%%%%%%%%%%%%%%%%%%%
\begin{frame}{}
  \begin{definition}[真值指派的扩张 ($\bar{v}$)]
    令 $\bar{S}$ 为只含有 $S$ 中命题符号的公式集。
    $S$ 上的真值指派 $v$ 的扩张是一个从 $\bar{S}$ 到真假值的映射
    \[
      \bar{v}: \bar{S} \to \set{T, F}.
    \]
  \end{definition}

  \pause
  \[
    \alpha: (((B \to (A \to C)) \leftrightarrow ((B \land A) \to C)))
  \]
  \[
    v(A) = v(B) = T, \quad v(C) = F
  \]

  \pause
  \[
    \bar{v}(\alpha) = \pause T
  \]
\end{frame}
%%%%%%%%%%%%%%%%%%%%

%%%%%%%%%%%%%%%%%%%%
\begin{frame}{}
  \begin{exampleblock}{构造下列公式的真值表}
    \[
      \lnot P \lor Q
    \]
    \[
      (P \land Q) \lor (\lnot P \land \lnot Q)
    \]
    \[
      (P \land Q) \land \lnot P
    \]
    \[
      (((B \to (A \to C)) \leftrightarrow ((B \land A) \to C)))
    \]
  \end{exampleblock}
\end{frame}
%%%%%%%%%%%%%%%%%%%%

%%%%%%%%%%%%%%%%%%%%
\begin{frame}{}
  \begin{definition}[满足 (Satisfy)]
    如果 $\bar{v}(\alpha) = T$, 则称真值指派 $v$ \red{\bf 满足}公式 $\alpha$。
  \end{definition}

  \pause
  \[
    \alpha: (((B \to (A \to C)) \leftrightarrow ((B \land A) \to C)))
  \]
  \[
    v(A) = v(B) = T, \quad v(C) = F
  \]

  \[
    \text{真值指派}\; v \text{ 满足 }\; \alpha
  \]
\end{frame}
%%%%%%%%%%%%%%%%%%%%

%%%%%%%%%%%%%%%%%%%%
\begin{frame}{}
  \begin{definition}[可满足的 (Satisfiable)]
  \end{definition}

  \pause
  \vspace{0.60cm}
  \begin{definition}[可满足性问题 (Satisfiability)]
  \end{definition}
\end{frame}
%%%%%%%%%%%%%%%%%%%%

%%%%%%%%%%%%%%%%%%%%
\begin{frame}{}
  \begin{definition}[重言蕴含 (Tautologically Implies)]
    \begin{center}
      设 $\Sigma$ 为一个公式集。\\[10pt]

      $\Sigma$ \red{\bf 重言蕴含} 公式 $\alpha$,
      记为 $\Sigma \models \alpha$, \\[5pt]

      如果\blue{每个}满足 $\Sigma$ 中\blue{所有}公式的真值指派都满足 $\alpha$。
    \end{center}
  \end{definition}

  \vspace{0.30cm}
  \begin{center}
    $\Sigma:$ 前提  \qquad $\alpha:$ 结论
  \end{center}

  \pause
  \[
    \set{A, \lnot A} \models B
  \]
  \pause
  \[
    \set{\alpha \land \beta} \models \alpha
  \]
\end{frame}
%%%%%%%%%%%%%%%%%%%%

%%%%%%%%%%%%%%%%%%%%
\begin{frame}{}
  \begin{theorem}
    请证明: $\Sigma \cup \set{a} \models \beta$
    \blue{当且仅当} $\Sigma \models (\alpha \to \beta)$。
  \end{theorem}
\end{frame}
%%%%%%%%%%%%%%%%%%%%

%%%%%%%%%%%%%%%%%%%%
\begin{frame}{}
  \[
    \Sigma = \emptyset
  \]
  \[
    \emptyset \models \alpha \text{ 简记为 } \models \alpha
  \]

  \pause
  \vspace{0.50cm}
  \[
    \Sigma = \set{\alpha} \text{ 只含有一个公式}
  \]
  \[
    \set{\alpha} \models \beta \text{ 简记为 } \;\alpha \models \beta
  \]
\end{frame}
%%%%%%%%%%%%%%%%%%%%

%%%%%%%%%%%%%%%%%%%%
\begin{frame}{}
  \begin{definition}[重言式/永真式 (Tautology)]
    如果 $\emptyset \models \alpha$, 则称 $\alpha$ 为\red{\bf 重言式},
    记为 $\models \alpha$。
  \end{definition}

  \begin{center}
    重言式在所有真值指派下均为真。
  \end{center}

  \pause
  \vspace{0.60cm}
  \begin{definition}[矛盾式/永假式 (Contradiction)]
    若公式 $\alpha$ 在所有真值指派下均为假, 则称 $\alpha$ 为\red{\bf 矛盾式}。
  \end{definition}
\end{frame}
%%%%%%%%%%%%%%%%%%%%

%%%%%%%%%%%%%%%%%%%%
\begin{frame}{}
  \begin{definition}[重言等价]
    如果 $\alpha \models \beta$ 且 $\beta \models \alpha$,
    则称 $\alpha$ 与 $\beta$ 重言等价, 记为 $\alpha \equiv \beta$。
  \end{definition}
\end{frame}
%%%%%%%%%%%%%%%%%%%%

%%%%%%%%%%%%%%%%%%%%
\begin{frame}{}
  \begin{description}[<+->][分配律:]
    \item[交换律:]
      \[
        (A \land B) \leftrightarrow (B \land A)
      \]
      \[
        (A \lor B) \leftrightarrow (B \lor A)
      \]
    \item[结合律:]
      \[
        ((A \land B) \land C) \leftrightarrow ((A \land B) \land C)
      \]
      \[
        ((A \lor B) \lor C) \leftrightarrow ((A \lor B) \lor C)
      \]
    \item[分配律:]
      \[
        (A \land (B \lor C)) \leftrightarrow ((A \land B) \lor (A \land C))
      \]
      \[
        (A \lor (B \land C)) \leftrightarrow ((A \lor B) \land (A \lor C))
      \]
    \item[德摩根 (De Morgan) 律:]
      \[
        \lnot (A \land B) \leftrightarrow (\lnot A \land \lnot B)
      \]
      \[
        \lnot (A \lor B) \leftrightarrow (\lnot A \lor \lnot B)
      \]
  \end{description}
\end{frame}
%%%%%%%%%%%%%%%%%%%%

%%%%%%%%%%%%%%%%%%%%
\begin{frame}{}
  \begin{description}[<+->][双重否定律:]
    \setlength{\itemsep}{8pt}
    \item[双重否定律:]
      \[
        ((\lnot (\lnot A)) \leftrightarrow A)
      \]
    \item[排中律:]
      \[
        A \lor (\lnot A)
      \]
    \item[矛盾律:]
      \[
        \lnot (A \land \lnot A)
      \]
    \item[逆否命题:]
      \[
        (A \to B) \leftrightarrow (\lnot B \to \lnot A)
      \]
  \end{description}
\end{frame}
%%%%%%%%%%%%%%%%%%%%

%%%%%%%%%%%%%%%%%%%%
\begin{frame}{}
  \begin{enumerate}[(1)]
    \item
      \[
        \alpha \to (\beta \to \alpha)
      \]
    \item
      \[
        (\alpha \to \beta) \leftrightarrow (\lnot \alpha \lor \beta)
      \]
    \item
      \[
        (\alpha \to (\beta \to \gamma)) \leftrightarrow ((\alpha \land \beta) \to \gamma)
      \]
    \item
      \[
        (\alpha \to (\beta \to \gamma)) \to ((\alpha \to \beta) \to (\alpha \to \gamma))
      \]
  \end{enumerate}
\end{frame}
%%%%%%%%%%%%%%%%%%%%

%%%%%%%%%%%%%%%%%%%%
\begin{frame}{}
  \begin{definition}[合取范式 (Conjunctive Normal Form)]
    我们称公式 $\alpha$ 是\red{\bf 合取范式}, 如果它形如
    \[
      \alpha = \beta_{1} \land \beta_{2} \land \dots \land \beta_{k},
    \]
    其中, 每个 $\beta_{i}$ 都形如
    \[
      \beta_{i} = \beta_{i1} \lor \beta_{i2} \lor \dots \lor \beta_{in},
    \]
    并且 $\beta_{ij}$ 或是一个命题符号, 或者命题符号的否定。
  \end{definition}

  \[
    (P \lor \lnot Q \lor R) \;\red{\land}\; (\lnot P \lor Q) \;\red{\land}\; \lnot Q
  \]
\end{frame}
%%%%%%%%%%%%%%%%%%%%

%%%%%%%%%%%%%%%%%%%%
\begin{frame}{}
  \begin{exampleblock}{求下列公式的合取范式}
    \[
      (P \land (Q \to R)) \to S
    \]

    \[
      \lnot (P \lor Q) \leftrightarrow (P \land Q)
    \]
  \end{exampleblock}
\end{frame}
%%%%%%%%%%%%%%%%%%%%

%%%%%%%%%%%%%%%%%%%%
\begin{frame}{}
  \begin{definition}[析取范式 (Disjunctive Normal Form)]
    我们称公式 $\alpha$ 是\red{\bf 析取范式}, 如果它形如
    \[
      \alpha = \beta_{1} \lor \beta_{2} \lor \dots \lor \beta_{k},
    \]
    其中, 每个 $\beta_{i}$ 都形如
    \[
      \beta_{i} = \beta_{i1} \land \beta_{i2} \land \dots \land \beta_{in},
    \]
    并且 $\beta_{ij}$ 或是一个命题符号, 或者命题符号的否定。
  \end{definition}

  \[
    (P \land \lnot Q \land R) \;\red{\lor}\; (\lnot P \land Q) \;\red{\lor}\; \lnot Q
  \]
\end{frame}
%%%%%%%%%%%%%%%%%%%%

%%%%%%%%%%%%%%%%%%%%
\begin{frame}{}
  \begin{exampleblock}{求下列公式的析取范式}
    \[
      (P \land (Q \to R)) \to S
    \]

    \[
      \lnot (P \lor Q) \leftrightarrow (P \land Q)
    \]
  \end{exampleblock}
\end{frame}
%%%%%%%%%%%%%%%%%%%%

%%%%%%%%%%%%%%%%%%%%
\begin{frame}{}
  \begin{exampleblock}{求合取范式与析取范式的方法}
    \begin{enumerate}[(1)]
      \setlength{\itemsep}{8pt}
      \item 先将公式中的联结符号化归成 $\lnot$, $\land$ 与 $\lor$;
      \item 再使用 De Morgan 律将 $\lnot$ 移到各个命题变元之前 (\blue{``否定深入''});
      \item 最后使用结合律、分配律将公式化归成合取范式或析取范式。
    \end{enumerate}
  \end{exampleblock}
\end{frame}
%%%%%%%%%%%%%%%%%%%%