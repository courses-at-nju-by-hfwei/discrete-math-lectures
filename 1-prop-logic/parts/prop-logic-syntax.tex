% prop-logic-syntax.tex

%%%%%%%%%%%%%%%%%%%%
\begin{frame}{}
  \begin{definition}[Proposition (命题)]
    A {\it proposition} is a sentence that is either true or false,
    (but not both).
  \end{definition}

  \vspace{0.30cm}
  \pause
  \begin{exampleblock}{Which are propositions?}
    \begin{enumerate}
      \item $1 + 1 = 2$
      \item $X + 6 = 0$
      \item $X = X$
      \item 哥德巴赫猜想。
      \item Today is rainy.
      \item Tomorrow is Friday.
      \item This sentence is false.
    \end{enumerate}
  \end{exampleblock}
\end{frame}
%%%%%%%%%%%%%%%%%%%%

%%%%%%%%%%%%%%%%%%%%
\begin{frame}{}
  \begin{center}
    \red{Forget ``Propositions''!!!}

    \vspace{0.50cm}
    Mathematical logic is \red{\it NOT} about propositions,
    but about \blue{the }
  \end{center}
\end{frame}
%%%%%%%%%%%%%%%%%%%%

%%%%%%%%%%%%%%%%%%%%
\begin{frame}{}
  \begin{definition}[命题逻辑的语言]
      
  \end{definition}
\end{frame}
%%%%%%%%%%%%%%%%%%%%

%%%%%%%%%%%%%%%%%%%%
\begin{frame}{}
  \begin{definition}[公式 (Formula)]
    \begin{enumerate}[(1)]
      \setlength{\itemsep}{8pt}
      \item 每个命题符号 $A_{i}$ 都是公式;
      \item 如果 $\alpha$ 和 $\beta$都是公式,
        则 $(\lnot \alpha)$, $(\alpha \land \beta)$, $(\alpha \lor \beta)$,
        $(\alpha \to \beta)$ 和 $(\alpha \leftrightarrow \beta)$ 也是公式;
      \item \red{除此之外, 别无其它。}
    \end{enumerate}
  \end{definition}

  % connectives.tex

\begin{table}
  \centering
  % \resizebox{0.90\textwidth}{!}{%
  \begin{tabular}{|c||c|c|c|c|}
    \hline
    符号& 名称 & 英文读法 & 中文读法 & \LaTeX \\
    \hline \hline
    $\lnot$ & \incell{negation}{(否定)} & not & 非 & \verb|\lnot| \\
    \hline
    $\land$ & \incell{conjunction}{(合取)} & and & 与 & \verb|\land| \\
    \hline
    $\lor$ & \incell{disjunction}{(析取)} & or & 或 & \verb|\lor| \\
    \hline
    $\to$ & conditional & \incell{implies}{(if then)}
      & \incell{蕴含}{(如果, 那么)} & \verb|\to| \\
    \hline
    $\leftrightarrow$ & biconditional & if and only if
      & 当且仅当 & \verb|\leftrightarrow| \\
    \hline
    \end{tabular}%
  %}
\end{table}
\end{frame}
%%%%%%%%%%%%%%%%%%%%

%%%%%%%%%%%%%%%%%%%%
\begin{frame}{}
  \begin{lemma}[公式的括号匹配性质]
    每个公式中左右括号的数目相同。
  \end{lemma}

  \pause
  \vspace{0.60cm}
  \begin{center}
    对公式的\red{结构}作归纳。
  \end{center}
\end{frame}
%%%%%%%%%%%%%%%%%%%%

%%%%%%%%%%%%%%%%%%%%
\begin{frame}{}
  \begin{theorem}[归纳原理]
    令 $P(\alpha)$ 为一个关于公式的性质。假设
    \begin{enumerate}[(1)]
      \setlength{\itemsep}{8pt}
      \item 对所有的命题符号 $A_{i}$, 性质 $P(A_{i})$ 成立; 并且
      \item 对所有的公式$\alpha$和$\beta$, 如果 $P(\alpha)$和$P(\beta)$成立,
        则 $P((\lnot \alpha))$, $P(\alpha \;\red{\ast}\; \beta)$ 也成立,
    \end{enumerate}
    那么 $P(\alpha)$ 对所有的公式 $\alpha$ 都成立。
  \end{theorem}
\end{frame}
%%%%%%%%%%%%%%%%%%%%

%%%%%%%%%%%%%%%%%%%%
\begin{frame}{}
  \begin{definition}[公式的长度]
    公式$\alpha$的长度 $|\alpha|$ 定义如下:
    \begin{enumerate}[(1)]
      \setlength{\itemsep}{8pt}
      \item 如果 $\alpha$ 是命题符号, 则 $|\alpha| = 1$;
      \item 如果 $\alpha = (\lnot \beta)$, 则 $|\alpha| = 1 + |\beta|$;
      \item 如果 $\alpha = (\beta \ast \gamma)$, 则 $|\alpha| = 1 + |\beta| + |\gamma|$。
    \end{enumerate}
  \end{definition}

  \pause
  \vspace{0.30cm}
  \begin{exampleblock}{作业题}
    假设公式 $\alpha$ 不含 ``$\lnot$'' 符号。

    请证明, $\alpha$ 中超过四分之一的符号是命题符号。
  \end{exampleblock}
\end{frame}
%%%%%%%%%%%%%%%%%%%%

%%%%%%%%%%%%%%%%%%%%
\begin{frame}{}
  \begin{center}
    关于``公式''的\blue{约定}

    \vspace{0.60cm}
    \begin{columns}
      \column{0.20\textwidth}
      \column{0.60\textwidth}
        \begin{itemize}
          \setlength{\itemsep}{6pt}
          \item 最外层的括号可以省略
          \item 优先级: $\lnot$, $\land$, $\lor$, $\to$, $\leftrightarrow$
          \item 结合性: 右结合 ($\alpha \land \beta \land \gamma$,
            $\alpha \to \beta \to \gamma$)
        \end{itemize}
      \column{0.20\textwidth}
    \end{columns}

    \pause
    \vspace{1.00cm}
    \red{不要过分依赖这些约定; 尽情地使用括号吧}

    \[
      \xout{P \land Q \to R}
    \]
    \[
      (P \land Q) \to R
    \]
  \end{center}
\end{frame}
%%%%%%%%%%%%%%%%%%%%

%%%%%%%%%%%%%%%%%%%%
\begin{frame}{}
  \begin{exampleblock}{用符号形式表示下列命题}
    \begin{enumerate}[<+->][(1)]
      \setlength{\itemsep}{8pt}
      \item 或者你没有给我写信, 或者它在途中丢失了
      \item 我们不能既写作业又打游戏
      \item 如果中国足球队夺冠, 我就好好学习
      \item 如果张三和李四都不去, 王五就去
      \item 如果周六不下雨, 我就去看电影, 否则就去图书馆
      \item 我今天进城, 除非下雨
      \item 如果你来了, 那么他是否唱歌将取决于你是否伴奏
    \end{enumerate}
  \end{exampleblock}
\end{frame}
%%%%%%%%%%%%%%%%%%%%