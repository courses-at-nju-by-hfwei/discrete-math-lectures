%%%%%%%%%%%%%%%
\begin{frame}{}
  \fig{width = 0.40\textwidth}{figs/less-logo}
\end{frame}
%%%%%%%%%%%%%%%

%%%%%%%%%%%%%%%
\begin{frame}{}
  \begin{definition}[$|A| \le |B|$]
    $|A| \le |B|$ if there exists an \red{\it one-to-one} function $f$ from $A$ into $B$.
  \end{definition}

  % \pause
  % \vspace{0.80cm}
  % \begin{center}
  %   {\red{\it $Q:$ What about onto function $f: A \to B$?}} \\[8pt] \pause
  %   {\blue{$|B| \le |A|$ \pause {\footnotesize (Axiom of Choice)}}}
  % \end{center}
\end{frame}
%%%%%%%%%%%%%%%

%%%%%%%%%%%%%%%
\begin{frame}{}
  \begin{definition}[$|A| < |B|$]
    $|A| < |B| \iff |A| \le |B| \land |A| \neq |B|$
  \end{definition}

  \pause
  \[
    |\N| < |\R|
  \]

  \[
    |X| < |2^{X}|
  \]

  \[
    |\N| < |2^{\N}|
  \]
\end{frame}
%%%%%%%%%%%%%%%

%%%%%%%%%%%%%%%
\begin{frame}{}
  \begin{definition}[Countable Revisited]
    $X$ is countable:
    \[
      (\exists n \in \N: |X| = n) \red{\;\lor\;} |X| = |\N|
    \]
  \end{definition}

  \pause
  \vspace{0.60cm}
  \begin{theorem}[Proof for Countable]
    $X$ is countable iff
    \[
      \blue{|X| \le |\N|.}
    \]
    \pause
    $X$ is countable iff there exists a \red{\it one-to-one} function
    \[
      f: X \to \N.
    \]
  \end{theorem}
\end{frame}
%%%%%%%%%%%%%%%

%%%%%%%%%%%%%%%
\begin{frame}{}
  \begin{exampleblock}{Subsets of Countable Set}
    Every \blue{subset} $B$ of a \red{countable} set $A$ is countable.
  \end{exampleblock}

  \pause
  \vspace{0.60cm}
  \[
    \exists \red{f: A \xrightarrow{1-1} \N}
  \]

  \pause
  \[
    \blue{g = f|_{B}}
  \]
\end{frame}
%%%%%%%%%%%%%%%

%%%%%%%%%%%%%%%
% \begin{frame}{}
%   \begin{exampleblock}{Set Union}
%     Give an example, if possible, of
%     \begin{enumerate}[(a)]
%       \setcounter{enumi}{2}
%       \item a countably infinite collection of \blue{\it pairwise disjoint} nonempty sets whose union is finite.
%       \setcounter{enumi}{1}
%       \pause
%       \item a countably infinite collection of nonempty sets whose union is finite.
%     \end{enumerate}
%   \end{exampleblock}
%
%   \pause
%   \[
%     \Big(\set{A_i: i \in R} \quad A_{i} = \set{1}\Big) \;\red{= \set{\set{1}}}
%   \]
%
%   \pause
%   \[
%     |A| = n \implies |2^{A}| = 2^n
%   \]
% \end{frame}
%%%%%%%%%%%%%%%

%%%%%%%%%%%%%%%
\begin{frame}{}
  \begin{exampleblock}{Slope}
    \begin{enumerate}[(a)]
      \item The set of all lines with rational slopes
    \end{enumerate}
  \end{exampleblock}

  \pause
  \[
    (\Q, \;\red{\R})
  \]

  \pause
  \[
    \teal{|\R| \le \;\red{|\Q \times \R|}\; \le |\R \times \R| = |\R|}
  \]
\end{frame}
%%%%%%%%%%%%%%%

%%%%%%%%%%%%%%%
\begin{frame}{}
  \begin{center}
    {\large \red{\it $Q:$ Is ``$\le$'' a partial order?}}
  \end{center}

  \pause
  \vspace{0.30cm}
  \begin{theorem}[Cantor-Schr\"{o}der–Bernstein (1887)]
    \[
      |X| \le |Y| \land |Y| \le |X| \implies |X| = |Y|
    \]
    \pause
    \[
      \exists\; f: X \xrightarrow{1-1} Y \land g: Y \xrightarrow{1-1} X
      \implies \exists\; h: X \xleftrightarrow[\text{onto}]{1-1} Y
    \]
  \end{theorem}

  \pause
  \begin{columns}
    \column{0.30\textwidth}
      \fig{width = 0.50\textwidth}{figs/proof-cantor-bernstein-book}
    \column{0.30\textwidth}
      \pause
      \fig{width = 0.50\textwidth}{figs/fudan-set-theory-book}
    \column{0.30\textwidth}
      \pause
      \fig{width = 0.50\textwidth}{figs/qrcode-cantor-bernstein-wiki}
      \vspace{-0.60cm}
      \begin{center}
        {\href{https://en.wikipedia.org/wiki/Schr\%C3\%B6der\%E2\%80\%93Bernstein\_theorem}{\teal{\footnotesize Schr\"{o}der–Bernstein theorem @ wiki}}}
      \end{center}
  \end{columns}
\end{frame}
%%%%%%%%%%%%%%%

%%%%%%%%%%%%%%%
\begin{frame}{}
  \begin{center}
    {\large \red{\it $Q:$ Is ``$\le$'' a total order?}}
  \end{center}

  \pause
  \vspace{0.50cm}
  \begin{theorem}[PCC]
    \begin{center}
      Principle of Cardinal Comparability (PCC) $\iff$ Axiom of Choice
    \end{center}
  \end{theorem}
\end{frame}
%%%%%%%%%%%%%%%

%%%%%%%%%%%%%%%
\begin{frame}{}
  \begin{theorem}[]
    \[
      |\R| = |\ps{\N}| = |\ps{\Q}|
    \]
  \end{theorem}

  \pause
  \[
    |\R| \le |\ps{\Q}|  \qquad |\ps{\Q}| \le |\R|
  \]
  \pause
  \begin{center}
    \teal{\url{https://en.wikipedia.org/wiki/Cardinality\_of\_the\_continuum\#Cardinal\_equalities}}
  \end{center}

  \pause
  \[
    \red{\mathfrak{c}} \triangleq |\R| = |\ps{\N}| = |2^{\N}|
      \triangleq \red{2^{\aleph_{0}}}
  \]

  \pause
  \fig{width = 0.30\textwidth}{figs/cantor-monumento}
\end{frame}
%%%%%%%%%%%%%%%