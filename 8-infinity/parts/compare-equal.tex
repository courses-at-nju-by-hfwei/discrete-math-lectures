% compare-equal.tex

%%%%%%%%%%%%%%%
\begin{frame}{}
  \fig{width = 0.35\textwidth}{figs/equal-logo}
\end{frame}
%%%%%%%%%%%%%%%

%%%%%%%%%%%%%%%
\begin{frame}{}
  \begin{definition}[$|A| = |B| \;(A \approx B)$ (1878)]
    $A$ and $B$ are \red{\it equipotent (等势)}
    if there exists a \purple{\it bijection} from $A$ to $B$.
  \end{definition}

  \pause
  \[
    \overline{\overline{A}} \quad \text{(two \red{abstractions})}
  \]

  \pause
  \[
    \text{\blue{\small Abstract from elements: }} \set{1,2,3} \quad \text{\it vs.} \quad \set{a,b,c}
  \]

  \pause
  \[
    \text{\blue{\small Abstract from order: }} \set{1,2,3,\cdots} \quad \text{\it vs.} \quad \set{1,3,5,\cdots, 2,4,6,\cdots}
  \]
\end{frame}
%%%%%%%%%%%%%%%

%%%%%%%%%%%%%%%
\begin{frame}{}
  \begin{definition}[$|A| = |B| \;(A \approx B)$ (1878)]
    $A$ and $B$ are \red{\it equipotent} if there exists a \purple{\it bijection} from $A$ to $B$.
  \end{definition}

  \vspace{0.30cm}
  \begin{center}
    \red{$Q:$ Is ``$\approx$'' an equivalence relation?}
  \end{center}

  \pause
  \begin{theorem}[\uncover<3->{\purple{The ``Equivalence Concept'' of Equipotent}}]
    For any sets $A$, $B$, $C$:
    \begin{enumerate}[(a)]
      \item $A \approx B$
      \item $A \approx B \implies B \approx A$
      \item $A \approx B \land B \approx C \implies A \approx C$
    \end{enumerate}
  \end{theorem}

  \vspace{0.30cm}
  \uncover<4->{
    \begin{center}
      \cyan{Do not care too much.}
    \end{center}
  }
\end{frame}
%%%%%%%%%%%%%%%

%%%%%%%%%%%%%%%
\begin{frame}{}
  \begin{definition}[Finite]
    $X$ is finite if
    \[
      \exists n \in \N: |X| = |\red{n}| = |\red{\set{0, 1, \dots, n-1}}|.
    \]
  \end{definition}

  \pause
  \vspace{0.20cm}
  \[
    \text{简记为 }\; |X| = n
  \]

  \pause
  \vspace{0.30cm}
  \begin{center}
    集合 $X$ 是有穷的当且仅当它与某个自然数等势。
  \end{center}

  % \pause
  % \begin{theorem}[]
  %   Let $A$ be a finite set.
  %   There is a \red{\it unique} $n \in \mathbb{N}$ such that $A \approx \set{0, 1, \cdots, n-1}$.
  % \end{theorem}

  % \pause
  % \vspace{0.30cm}
  % \begin{center}
  %   {\red{\it $Q:$ How to prove that a set is infinite?}} \\[8pt]
  %   \pause
  %   {\blue{By contradiction.}}
  % \end{center}
\end{frame}
%%%%%%%%%%%%%%%

%%%%%%%%%%%%%%%
\begin{frame}{}
  \begin{definition}[Infinite]
    $X$ is infinite if it is not finite:
    \[
      \forall n \in \N: |X| \neq n.
    \]
  \end{definition}

  \pause
  \begin{theorem}[\red{Existence of Infinite Sets!}]
    $\N$ is infinite. \uncover<9->{\cyan{(So are $\Z$, $\Q$, $\R$.)}}
  \end{theorem}

  \pause
  \begin{center}
    \purple{\it By Contradiction.}

    \pause
    \vspace{-0.30cm}
    \[
      \exists n \in \N: |\N| = n.
    \]

    \pause
    \vspace{-0.30cm}
    \[
      \exists f: \N \xleftrightarrow[onto]{1-1} \set{0, 1, \cdots, n-1}
    \]

    \pause
    \vspace{-0.30cm}
    \[
      g \triangleq \purple{f|_{\set{0, 1, \ldots, n}}}: \set{0, 1, \cdots, n} \to \set{0, 1, \cdots, n-1}
    \]

    \pause
    \vspace{-0.30cm}
    \[
      \text{\teal{By the Pigeonhole Principle}}: g \text{ is not 1-1} \pause \implies f \text{ is not 1-1}
    \]
  \end{center}
\end{frame}
%%%%%%%%%%%%%%%

%%%%%%%%%%%%%%%
\begin{frame}{}
  \begin{definition}[Infinite]
    For any set $X$,
    \begin{description}[<+->][Countably Infinite]
      \item[Countably Infinite]
        \[
          |X| = |\N| \red{\;\triangleq \aleph_{0}}
        \]
      \item[Countable]
        \[
          \text{(finite $\lor$ countably infinite)}
        \]
      \item[Uncountable]
        \[
          \text{($\lnot$ countable)}
        \]
        \[
          \text{(infinite) $\land$ \Big($\lnot$ (countably infinite)\Big)}
        \]
    \end{description}
  \end{definition}
\end{frame}
%%%%%%%%%%%%%%%

%%%%%%%%%%%%%%%
\begin{frame}{}
  \fig{width = 0.40\textwidth}{figs/aleph0}
\end{frame}
%%%%%%%%%%%%%%%

%%%%%%%%%%%%%%%
\begin{frame}{}
  \begin{theorem}[$\Z$ is Countable.]
    \[
      |\Z| = |\N|
    \]
  \end{theorem}

  \fig{width = 0.60\textwidth}{figs/integers}

  \pause
  \[
    0 \quad 1 \quad -1 \quad 2 \quad -2 \quad \cdots
  \]
\end{frame}
%%%%%%%%%%%%%%%

%%%%%%%%%%%%%%%
\begin{frame}{}
  \begin{theorem}[$\Q$ is Countable. (Cantor 1873-11; Published in 1874)]
    \[
      |\Q| = |\N|
    \]
  \end{theorem}

  \pause
  \begin{exampleblock}{$|\Q| = |\N|$}
    \[
      q \in \red{\Q^{+}}: a / b \; (a,b \in \N^{+})
    \]

    \pause
    \vspace{-0.30cm}
    \begin{columns}
      \column{0.50\textwidth}
        \fig{width = 0.55\textwidth}{figs/cantor-monumento}
      \column{0.50\textwidth}
        \fig{width = 0.65\textwidth}{figs/counting-Q}
    \end{columns}
  \end{exampleblock}
\end{frame}
%%%%%%%%%%%%%%%

%%%%%%%%%%%%%%%
\begin{frame}{}
  \begin{theorem}[$\N \times \N$ is Countable.]
    \[
      |\N \times \N| = |\N|
    \]
  \end{theorem}

  \begin{columns}
    \column{0.50\textwidth}
      \fig{width = 0.70\textwidth}{figs/n-n-countable}
    \column{0.50\textwidth}
      \pause
      \[
        \pi: \N \times \N \to \N
      \]

      \pause
      \[
        \pi(k_1, k_2) = \frac{1}{2} (k_1 + k_2)(k_1 + k_2 + 1) + k_2
      \]
      \[
        \pi(2, 1) = 7
      \]
  \end{columns}

  \pause
  \begin{center}
    \red{Cantor Pairing Function}
  \end{center}
\end{frame}
%%%%%%%%%%%%%%%

%%%%%%%%%%%%%%%
\begin{frame}{}
  \begin{theorem}[$\N^{n}$ is Countable.]
    \[
      |\N^{n}| = |\N|
    \]
  \end{theorem}

  \pause
  \begin{theorem}
    The Cartesian product of \red{finitely many} countable sets is countable.
  \end{theorem}

  \pause
  \[
    \red{\N^{n} \quad \text{\it vs.} \quad \N^{\N}}
  \]

  \pause
  \[
    \pi^{(n)}:\N^n \to \N
  \]

  \pause
  \[
    \pi^{2}(k_1, k_2) = \frac{1}{2} (k_1 + k_2)(k_1 + k_2 + 1) + k_2
  \]
  \pause
  \[
    \pi^{(n)}(k_1, \ldots, k_{n-1}, k_n) = \pi (\red{\pi^{(n-1)}}(k_1, \ldots, k_{n-1}) , k_n)
    \quad (n \ge 3)
  \]
\end{frame}
%%%%%%%%%%%%%%%

%%%%%%%%%%%%%%%
\begin{frame}{}
  \begin{theorem}
    Any \red{finite} union of countable sets is countable.
  \end{theorem}

  \pause
  \[
    A = \set{a_n \mid n \in \N} \quad B = \set{b_n \mid n \in \N} \quad C = \set{c_n \mid n \in \N}
  \]

  \pause
  \[
    a_0 \quad b_0 \quad c_0 \quad a_1 \quad b_1 \quad c_1 \cdots
  \]
\end{frame}
%%%%%%%%%%%%%%%

%%%%%%%%%%%%%%%
\begin{frame}{}
  \begin{theorem}
    The union of \red{countably many} countable sets is countable.
  \end{theorem}

  \pause
  \fig{width = 0.40\textwidth}{figs/countable-union-countable}
  \begin{center}
    \teal{Counting by Diagonals} % \\[15pt] \pause
    % \red{We need Axiom of (Countable) Choice!}
  \end{center}
\end{frame}
%%%%%%%%%%%%%%%

%%%%%%%%%%%%%%%
\begin{frame}{}
  \begin{center}
    \teal{\Huge Beyond}
  \end{center}

  \fig{width = 0.30\textwidth}{figs/aleph0}
\end{frame}
%%%%%%%%%%%%%%%

%%%%%%%%%%%%%%%
\begin{frame}{}
  \begin{theorem}[$\R$ is Uncountable. (Cantor 1873-12; Published in 1874)]
    \[
      |\R| \neq |\N|
    \]
  \end{theorem}

  \pause
  \vspace{0.50cm}
  \fig{width = 0.30\textwidth}{figs/very-important}

  \pause
  \begin{center}
    Different ``Sizes'' of Infinity \\[10pt] \pause
    Cantor's Diagonal Argument \teal{(1890)}
  \end{center}
\end{frame}
%%%%%%%%%%%%%%%

%%%%%%%%%%%%%%%
\begin{frame}{}
  \begin{theorem}[$\R$ is Uncountable. (Cantor 1873-12; Published in 1874)]
    \[
      |\R| \neq |\N|
    \]
  \end{theorem}

  \pause
  \vspace{0.20cm}
  \begin{center}
    \red{By Contradiction.}
    \pause
    \[
      f: \R \xleftrightarrow[onto]{1-1} \N
    \]

    \pause
    \fig{width = 0.16\textwidth}{figs/r-uncountable}
  \end{center}
\end{frame}
%%%%%%%%%%%%%%%

%%%%%%%%%%%%%%%
% \begin{frame}{}
    % 实数不能表示为有理数的有序对。
% \end{frame}
%%%%%%%%%%%%%%%

% %%%%%%%%%%%%%%%
% \begin{frame}{}
%   \begin{exampleblock}{Infinite Sequences of $0$'s and $1$'s}
%     Is the set of all infinite sequences of $0$'s and $1$'s finite,
%     countably infinite, or uncountable?
%   \end{exampleblock}
%
%   \pause
%   \fig{width = 0.35\textwidth}{figs/diagonal-argument-01}
%   \begin{center}
%     \red{By Diagonal Argument.}
%   \end{center}
% \end{frame}
% %%%%%%%%%%%%%%%
%
% %%%%%%%%%%%%%%%
% \begin{frame}{}
%   \begin{exampleblock}{Infinite Sequences of $0$'s and $1$'s}
%     Is the set of all infinite sequences of $0$'s and $1$'s finite,
%     countably infinite, or uncountable?
%   \end{exampleblock}
%
%   \[
%     f: \set{\set{0,1}^{\ast}} \to \N
%   \]
%
%   \pause
%   \[
%     \purple{f(x_0 x_1 \cdots) = \sum_{i=0}^{\infty} x_i 2^{i}}
%   \]
%
%   \pause
%   \fig{width = 0.20\textwidth}{figs/wrong}
% \end{frame}
% %%%%%%%%%%%%%%%

%%%%%%%%%%%%%%%
\begin{frame}{}
  \[
    \scalebox{5.5}{$\mathfrak{c} \triangleq |\R|$}
  \]

  \pause
  \vspace{0.50cm}
  \begin{center}
    \teal{$\R$ 是一个连续统 (Continuum)} \\[8pt]
    \cyan{\url{https://en.wikipedia.org/wiki/Continuum\_(set\_theory)}}
  \end{center}
\end{frame}
%%%%%%%%%%%%%%%

%%%%%%%%%%%%%%%
\begin{frame}{}
  \begin{theorem}[$|\R|$ (Cantor 1877)]
    \[
      |(0,1)| = \red{|\R| = |\R \times \R|} \blue{\;= |\R^{n \in \N}|}
    \]
  \end{theorem}

  \pause
  \vspace{0.50cm}
  \[
    f(x) = \tan \frac{(2x-1)\pi}{2}
  \]
  \[
    |(0,1)| = |(-\frac{\pi}{2}, \frac{\pi}{2})| = |\R|
  \]
\end{frame}
%%%%%%%%%%%%%%%

%%%%%%%%%%%%%%%
\begin{frame}{}
  \[
    \R \times \R \approx \R
  \]

  \pause
  \[
    (x = 0.\blue{a_1 a_2 a_3 \cdots}, y = 0.\red{b_1 b_2 b_3 \cdots}) \pause \mapsto 0.\blue{a_1}\red{b_1}\blue{a_2}\red{b_2}\blue{a_3}\red{b_3}\cdots
  \]

  \pause
  \fig{width = 0.25\textwidth}{figs/flawed}

  \begin{center}
    \href{https://www.maa.org/sites/default/files/pdf/pubs/AMM-March11\_Cantor.pdf}{\teal{Was Cantor Surprised?}}
  \end{center}
\end{frame}
%%%%%%%%%%%%%%%

%%%%%%%%%%%%%%%
\begin{frame}{}
  \begin{theorem}[$|\R|$ (Cantor 1877)]
    \[
      |(0,1)| = \red{|\R| = |\R \times \R|} \blue{\;= |\R^{n}|}
    \]
  \end{theorem}

  \vspace{0.60cm}
  \begin{quote}
    \begin{center}
      ``Je le vois, mais je ne le crois pas !'' \\[4pt]
      \red{``I see it, but I don't believe it !''} \\[5pt]
      \hfill --- Cantor's letter to Dedekind (1877).
    \end{center}
  \end{quote}

  \pause
  \begin{center}
    {\purple{\it $Q:$ Then, what is ``dimension''?}}
  \end{center}

  \pause
  \begin{theorem}[Brouwer (Topological Invariance of Dimension)]
    There is no \red{continuous} bijections between $\R^m$ and $\R^n$ for $m \neq n$.
  \end{theorem}
\end{frame}
%%%%%%%%%%%%%%%

% cantor-theorem.tex

%%%%%%%%%%%%%%%
\begin{frame}{}
  \begin{center}
    \teal{\Huge Beyond}
  \end{center}

  \[
    \scalebox{8.5}{$\mathfrak{c}$}
  \]
\end{frame}
%%%%%%%%%%%%%%%

%%%%%%%%%%%%%%%
\begin{frame}{}
  \begin{theorem}[Cantor's Theorem (1891)]
    \[
      |A| \neq |\ps{A}|
    \]
  \end{theorem}

  \pause
  \vspace{0.80cm}
  \begin{theorem}[Cantor Theorem]
    If $f: A \to \ps{A}$, then $f$ is not onto.
  \end{theorem}
\end{frame}
%%%%%%%%%%%%%%%

% %%%%%%%%%%%%%%%
% \begin{frame}{}
  % \begin{theorem}[Cantor Theorem]
    % If $f: A \to \ps{A}$, then $f$ is not onto.
  % \end{theorem}

  % \vspace{0.60cm}
  % \begin{columns}
    % \column{0.28\textwidth}
      % \fig{width = 1.00\textwidth}{figs/talking-about}
    % \pause
    % \column{0.25\textwidth}
      % \fig{width = 0.90\textwidth}{figs/interesting}
    % \pause
    % \column{0.25\textwidth}
      % \fig{width = 0.80\textwidth}{figs/genius}
    % \pause
    % \column{0.25\textwidth}
      % \fig{width = 1.00\textwidth}{figs/stupid}
  % \end{columns}
% \end{frame}
% %%%%%%%%%%%%%%%

% %%%%%%%%%%%%%%%
% \begin{frame}{}
  % \begin{theorem}[Cantor Theorem]
    % If $f: A \to \ps{A}$, then $f$ is not onto.
  % \end{theorem}

  % \vspace{0.30cm}
  % \purple{Understanding this problem:}
  % \[
    % A = \set{1,2,3}
  % \]
  % \pause
  % \[
    % \ps{A} = \Big\{\emptyset, \set{1}, \set{2}, \set{3}, \set{1,2}, \set{1,3}, \set{2,3}, \set{1,2,3}\Big\}
  % \]

  % \pause
  % \begin{description}
    % \item[Onto]
      % \[
	% \forall B \in \ps{A}: \Big(\exists a \in A: f(a) = B\Big)
      % \]
    % \pause
    % \item[Not Onto]
      % \[
	% \red{\exists} B \in \ps{A}: \Big(\red{\forall} a \in A: f(a) \neq B\Big)
      % \]
  % \end{description}
% \end{frame}
% %%%%%%%%%%%%%%%

% %%%%%%%%%%%%%%%
% \begin{frame}{}
  % \begin{theorem}[Cantor Theorem]
    % If $f: A \to \ps{A}$, then $f$ is not onto.
  % \end{theorem}

  % \[
    % \red{\exists} B \in \ps{A}: \Big(\red{\forall} a \in A: f(a) \neq B\Big)
  % \]

  % \pause
  % \begin{columns}[t]
    % \column{0.50\textwidth}
      % \begin{itemize}
	% \item<2-> Constructive proof (\red{$\exists$}):
	  % \[
	    % B = \set{a \in A \mid a \notin f(a)}
	  % \]
	% \item<4-> By contradiction (\red{$\forall$}):
	  % \[
	    % \exists a \in A: f(a) = B.
	  % \]
      % \end{itemize}
    % \column{0.40\textwidth}
      % \uncover<3->{\fig{width = 0.80\textwidth}{figs/what-is-this}}
  % \end{columns}

  % \vspace{-0.30cm}
  % \uncover<5->{
    % \[
      % \red{Q: a \in B\emph{?}}
    % \]
  % }

  % \vspace{-0.60cm}
  % \uncover<6->{
    % \[
      % \teal{a \in B \iff a \notin B}
    % \]
  % }
% \end{frame}
% %%%%%%%%%%%%%%%

% %%%%%%%%%%%%%%%
% \begin{frame}{}
  % \begin{theorem}[Cantor Theorem]
    % If $f: A \to \ps{A}$, then $f$ is not onto.
  % \end{theorem}

  % \begin{proof}[Diagonal Argument \only<5->{\footnotesize \purple{(以下仅适用于可数集合 $A$)}}]
    % \pause
    % \begin{table}[]
      % \centering
      % $\begin{tabu}{|c||c|c|c|c|c|c|}
	% \hline
	% a      & \multicolumn{6}{c|}{f(a)} \\ \hline
	       % & 1      & 2      & 3      & 4      & 5      & \cdots \\ \hline \hline
	% 1      & \redoverlay{1}{3-}      & 1      & 0      & 0      & 1      & \cdots \\ \hline
	% 2      & 0      & \redoverlay{0}{3-}      & 0      & 0      & 0      & \cdots \\ \hline
	% 3      & 1      & 0      & \redoverlay{0}{3-}      & 1      & 0      & \cdots \\ \hline
	% 4      & 1      & 1      & 1      & \redoverlay{1}{3-}      & 1      & \cdots \\ \hline
	% 5      & 0      & 1      & 0      & 1      & \redoverlay{0}{3-}      & \cdots \\ \hline
	% \vdots & \vdots & \vdots & \vdots & \vdots & \vdots & \cdots \\ \hline
      % \end{tabu}$
    % \end{table}

    % \uncover<4->{
      % \[
	% B = \blue{\set{0, 1, 1, 0, 1}}
      % \]
    % }
  % \end{proof}
% \end{frame}
% %%%%%%%%%%%%%%%

%%%%%%%%%%%%%%%
\begin{frame}{}
  \begin{theorem}[Cantor Theorem]
    \[
      |A| \;\red{<}\; |\ps{A}|
    \]
  \end{theorem}

  \pause
  \[
    A \qquad \ps{A} \qquad \ps{\ps{A}} \qquad \ldots
  \]

  \pause
  \vspace{0.60cm}
  \begin{center}
    \red{\large There is no largest infinity.}
  \end{center}
\end{frame}
%%%%%%%%%%%%%%%
