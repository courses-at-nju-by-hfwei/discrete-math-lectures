% finite.tex

%%%%%%%%%%%%%%%
\begin{frame}{}
  \begin{center}
    {\LARGE Finite Sets}
  \end{center}

  \fig{width = 0.40\textwidth}{figs/confusion}

  \pause
  \begin{quote}
    \begin{center}
      {\large \red{``关于有穷, 我原以为我是懂的''}}
    \end{center}
  \end{quote}
\end{frame}
%%%%%%%%%%%%%%%

%%%%%%%%%%%%%%%
% \begin{frame}{}
%   \begin{exampleblock}{学生反馈 (改编版)}
%     \begin{quote}
%       \begin{center}
% 	``明明很显然的事情,为什么要那么繁琐的证明?\\[6pt]
% 	依靠直觉不可以吗?''
%       \end{center}
%     \end{quote}
%   \end{exampleblock}
% \end{frame}
%%%%%%%%%%%%%%%

%%%%%%%%%%%%%%%
\begin{frame}{}
  \begin{definition}[Finite]
    $X$ is finite if
    \[
      \exists n \in \N: |X| = n
    \]
  \end{definition}

  \pause
  \vspace{0.60cm}
  \begin{theorem}[Pigeonhole Principle]
    \[
      f: \set{1, \cdots, m} \to \set{1, \cdots, n} \; (m,n \in \N^{+}, m > n)
    \]

    $f$ is not one-to-one.
  \end{theorem}
\end{frame}
%%%%%%%%%%%%%%%

%%%%%%%%%%%%%%%
\begin{frame}{}
  \begin{exampleblock}{$A \setminus \set{a}$}
    Let $A$ be a nonempty finite set with $|A| = n$ and let $a \in A$.

    Prove that $A \setminus \set{a}$ is finite and $|A \setminus \set{a}| = n - 1$.
  \end{exampleblock}

  \pause
  \[
    f: A \xleftrightarrow[onto]{1-1} \set{1, \cdots, n}
  \]

  \pause
  \[
    f|_{A \setminus \set{a}}: A \setminus \set{a} \xleftrightarrow[onto]{1-1} \set{1, \cdots, n} \setminus \set{f(a)} \;
    \only<4->{\red{\xleftrightarrow[onto]{1-1} \set{1, \cdots, n - 1}}}
  \]
\end{frame}
%%%%%%%%%%%%%%%

%%%%%%%%%%%%%%%
\begin{frame}{}
  \begin{exampleblock}{$|A| \le |B|$}
    $A$ and $B$ are finite sets and $f: A \to B$ is one-to-one.

    Show that $|A| \le |B|$.
  \end{exampleblock}

  \pause
  \fig{width = 0.35\textwidth}{figs/gaoshenme}

  \pause
  \vspace{0.30cm}
  \centerline{By contradiction and the pigeonhole principle.}
\end{frame}
%%%%%%%%%%%%%%%

%%%%%%%%%%%%%%%
\begin{frame}{}
  \begin{exampleblock}{}
    \begin{enumerate}[(a)]
      \item $A$ is a finite set and $B \subseteq A$. \textcolor{gray}{We showed that $B$ is finite (Corollary $20.11$)}. Show that $|B| \le |A|$.
	\pause
	\[
	  \teal{\text{one-to-one} \;f: B \to A}
	\]
      \item \pause $A$ is a finite set and $B \subseteq A$. Show that if $B \neq A$, then $|B| < |A|$.
	% \pause
	% \[
	%   \red{\sout{B \neq A, |B| \le |A| \implies |B| < |A|}}
	% \]
	\pause
	% \vspace{-0.80cm}
	\[
	  \teal{\exists a: a \in A \land a \notin B \qquad f: B \to A \setminus \set{a} \qquad |B| \le |A \setminus \set{a}|}
	\]
      \item \pause If two finite sets $A$ and $B$ satisfy $B \subseteq A$ and $|A| \le |B|$, then $A = B$.
	\pause
	\[
	  \teal{\text{By contradiction and (b).}}
	\]
    \end{enumerate}
  \end{exampleblock}
\end{frame}
%%%%%%%%%%%%%%%

%%%%%%%%%%%%%%%
\begin{frame}{}
  \begin{exampleblock}{Cardinality of $|\text{ran}(f)|$}
    Let $A$ and $B$ be sets with $A$ finite.

    \[
      f: A \to B
    \]

    Prove that $|\text{ran}(f)| \le |A|$.
  \end{exampleblock}

  \pause
  \vspace{0.50cm}
  \[
    \teal{\text{one-to-one} \;g: \text{ran}(f) \to A}
  \]
  \pause
  \centerline{\red{\footnotesize (No Axiom of Choice Here)}}
\end{frame}
%%%%%%%%%%%%%%%

%%%%%%%%%%%%%%%
\begin{frame}{}
  \begin{exampleblock}{$f: A \to A$}
    Let $A$ be a finite set.
    \[
      f: A \to A
    \]
    Prove that
    \[
      f \text{ is one-to-one } \iff f \text{ is onto}.
    \]
  \end{exampleblock}

  \pause
  \vspace{0.60cm}
  \begin{columns}
    \column{0.45\textwidth}
      \[
	\implies
      \]
      \centerline{By contradiction.}
      \pause
      \[
	f': A \to A \setminus \set{a}
      \]
      \pause
    \column{0.45\textwidth}
      \[
	\Longleftarrow
      \]
      \pause \vspace{-0.30cm}
      \[
	\forall y \in A\; \exists x \in A: y = f(x)
      \]
      \pause \vspace{-0.30cm}
      \[
	\teal{\forall y, \text{choose $x$}: (g: g(y) = x)}
      \]
      \pause \vspace{-0.30cm}
      \[
	g \text{ is bijective.}
      \]
      \pause \vspace{-0.30cm}
      \[
	\teal{f(g(y)) = f(x) = y \implies f = g^{-1}}
      \]
  \end{columns}
\end{frame}
%%%%%%%%%%%%%%%
