% before-cantor.tex

%%%%%%%%%%%%%%%
\begin{frame}{}
  \begin{center}
    \teal{\Large Before Cantor}
  \end{center}

  \fig{width = 0.30\textwidth}{figs/cantor-infinity}
\end{frame}
%%%%%%%%%%%%%%%

%%%%%%%%%%%%%%%
\begin{frame}{}
  \begin{columns}
    \column{0.50\textwidth}
      \fig{width = 0.60\textwidth}{figs/Euclid}
    \column{0.50\textwidth}
      \fig{width = 0.60\textwidth}{figs/elements-ch}
  \end{columns}

  \pause
  \vspace{0.60cm}
  \begin{center}
    \red{\Large 公理: ``整体大于部分''}
  \end{center}
\end{frame}
%%%%%%%%%%%%%%%

%%%%%%%%%%%%%%%
\begin{frame}{}
  \begin{columns}
    \column{0.50\textwidth}
      \fig{width = 0.50\textwidth}{figs/galileo}{\centerline{Galileo Galilei (1564 -- 1642)}}
    \column{0.50\textwidth}
      \fig{width = 0.45\textwidth}{figs/galileo-book}{\centerline{``关于两门新科学的对话'' (1638)}}
  \end{columns}

  \pause
  \vspace{0.50cm}
  \begin{center}
    \teal{\Large ``用我们有限的心智来讨论无限 $\cdots$''}
  \end{center}
\end{frame}
%%%%%%%%%%%%%%%

%%%%%%%%%%%%%%%
\begin{frame}{}
  \[
    S_1 = \set{1, 2, 3, \cdots, n, \cdots}
  \]
  \[
    S_2 = \set{1, 4, 9, \cdots, n^2, \cdots}
  \]

  \begin{columns}
    \column{0.50\textwidth}
      \pause
      \[
	\blue{|S_1| = |S_2| \qquad S_2 \subset S_1}
      \]

      \pause
      \begin{center}
	{\red{\large ``部分等于全体''}}
      \end{center}
    \column{0.50\textwidth}
      \pause
      \fig{width = 0.35\textwidth}{figs/chijing}
  \end{columns}

  \pause
  \vspace{0.80cm}
  \begin{quote}
    说到底,``等于''、``大于''和``小于''诸性质不能用于无限,而只能用于有限的数量。 \hfill --- Galileo Galilei
  \end{quote}

  \pause
  \vspace{0.20cm}
  \begin{quote}
    无穷数是不可能的。 \hfill --- Gottfried Wilhelm Leibniz
  \end{quote}
\end{frame}
%%%%%%%%%%%%%%%

%%%%%%%%%%%%%%%
\begin{frame}{}
  \begin{quote}
    这些证明一开始就期望那些数要具有有穷数的一切性质,
    或者甚至于\blue{把有穷数的性质强加于无穷}。\pause \\[15pt]

    相反,这些无穷数,如果它们能够以任何形式被理解的话,
    倒是由于它们与有穷数的对应,\red{它们必须具有完全新的数量特征}。\pause \\[15pt]

    \teal{这些性质完全依赖于事物的本性},$\cdots$而并非来自我们的主观任意性
    或我们的偏见。

    \hfill --- Georg Cantor (1885)
  \end{quote}
\end{frame}
%%%%%%%%%%%%%%%

%%%%%%%%%%%%%%%
\begin{frame}{}
  \begin{definition}[Dedekind-infinite \& Dedekind-finite (Dedekind, 1888)]
    A set $A$ is \purple{\it Dedekind-infinite}
    if there is a bijective function from $A$ to some \red{proper} subset $B$ of $A$. \\[8pt]

    A set is \purple{\it Dedekind-finite} if it is not Dedekind-infinite.
  \end{definition}

  \pause
  \fig{width = 0.20\textwidth}{figs/there-is-another-way}

  \begin{center}
    This is a \red{theorem} in our theory of infinity.
  \end{center}
\end{frame}
%%%%%%%%%%%%%%%

%%%%%%%%%%%%%%%
\begin{frame}{}
  \fig{width = 0.30\textwidth}{figs/warning}

  \begin{center}
    \red{\Large We have not defined ``finite'' and ``infinite''!}
  \end{center}
\end{frame}
%%%%%%%%%%%%%%%