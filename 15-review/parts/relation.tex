% relation.tex

%%%%%%%%%%%%%%%
\begin{frame}{}
  \[
    R \subseteq A \times A
  \]

  \[
    \begin{cases}
      R^{0} = I_{A} \\[6pt]
      R^{n+1} = R \circ R^{n}
    \end{cases}
  \]
\end{frame}
%%%%%%%%%%%%%%%

%%%%%%%%%%%%%%%
\begin{frame}{}
  \begin{center}
    Representing Relations as \gray{Matrices}/Digraphs
  \end{center}

  \[
    A = \set{1, 2, 3, 4}
  \]
  \[
    R = \set{(1, 1), (1, 2), (2, 1), (2, 2), (2, 3), (2, 4), (3, 4), (4, 1)}
  \]

  \pause
  \[
    R^{2} \qquad R^{3}
  \]
  \pause
  \[
    \red{R^{+}} = \bigcup_{i=1}^{\infty} R \qquad
    \blue{R^{\ast}} = \bigcup_{i=0}^{\infty} R
  \]
\end{frame}
%%%%%%%%%%%%%%%

%%%%%%%%%%%%%%%
\begin{frame}{}
  \begin{definition}[Reflexive Closure (自反闭包)]
    The \red{reflexive closure} $\textsf{cl}_{\text{ref}}(R)$
    of a relation $R \subseteq X \times X$ is
    the \red{smallest} reflexive relation on $X$ that contains $R$.
  \end{definition}

  \pause
  \vspace{0.50cm}
  \[
    \textsf{cl}_{\text{ref}}(R) = R \cup I_{X}
  \]
\end{frame}
%%%%%%%%%%%%%%%

%%%%%%%%%%%%%%%
\begin{frame}{}
  \begin{definition}[Symmetric Closure (对称闭包)]
    The \red{symmetric closure} $\textsf{cl}_{\text{sym}}(R)$
    of a relation $R \subseteq X \times X$
    is the \red{smallest} symmetric relation on $X$ that contains $R$.
  \end{definition}

  \pause
  \vspace{0.50cm}
  \[
    \textsf{cl}_{\text{sym}}(R) = R \cup R^{-1}
  \]
\end{frame}
%%%%%%%%%%%%%%%

%%%%%%%%%%%%%%%
\begin{frame}{}
  \begin{definition}[Transitive Closure (传递闭包)]
    The \red{transitive closure} $\textsf{cl}_{\text{trn}}(R)$
    of a relation $R \subseteq X \times X$
    is the \red{smallest} transitive relation on $X$ that contains $R$.
  \end{definition}

  \pause
  \[
    \textsf{cl}_{\text{trn}}(R) = R^{+}
  \]

  \pause
  \begin{columns}
    \column{0.35\textwidth}
    \column{0.30\textwidth}
      \begin{itemize}
        \item $R^{+}$ contains $R$
        \item $R^{+}$ is transitive
        \item $R^{+}$ is minimal
      \end{itemize}
    \column{0.35\textwidth}
  \end{columns}

  \pause
  \vspace{0.50cm}
  \begin{center}
    If $T$ is any transitive relation containing $R$, then $R^{+} \subset T$.

    \pause
    \vspace{0.20cm}
    \red{By induction on $i$, we can show that $R^{i} \subseteq T$.}
  \end{center}
\end{frame}
%%%%%%%%%%%%%%%