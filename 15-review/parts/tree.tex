% tree.tex

%%%%%%%%%%%%%%%
\begin{frame}{}
  \begin{definition}[Rooted Tree (有根树)]
    A \red{rooted tree} is a \blue{tree}
    where one vertex has been \purple{designated the root}.
  \end{definition}

  \fig{width = 0.60\textwidth}{figs/rooted-tree}

  \pause
  \begin{definition}[Directed Rooted Tree (有向有根树)]
    A \red{directed rooted tree} is a \blue{rooted tree}
    where all edges directed \\ \cyan{away from} or \cyan{towards} the root.
  \end{definition}
\end{frame}
%%%%%%%%%%%%%%%

%%%%%%%%%%%%%%%
\begin{frame}{}
  \begin{definition}
    Parent, Child; \quad Sibling; \quad Ancestor, Descendant
  \end{definition}

  \pause
  \vspace{0.50cm}
  \begin{definition}[$k$-ary Trees ($k$-叉树)]
    A \red{$k$-ary tree} is a rooted tree
    in which each vertex has \blue{$\le k$} children.

    \vspace{0.20cm}
    $2$-ary trees are often called \red{binary trees}.
  \end{definition}

  \pause
  \vspace{0.50cm}
  \begin{definition}[Complete $k$-Tree (完全 $k$-叉树)]
    A \red{complete $k$-tree} is a $k$-ary tree in which each vertex, \\
    other than leaves, has \blue{$= k$} children.
  \end{definition}
\end{frame}
%%%%%%%%%%%%%%%

%%%%%%%%%%%%%%%
\begin{frame}{}
  \fig{width = 0.60\textwidth}{figs/dfs-binary-tree}

  \begin{center}
    Depth-First Search (DFS)
  \end{center}
\end{frame}
%%%%%%%%%%%%%%%

%%%%%%%%%%%%%%%
\begin{frame}{}
  \fig{width = 0.60\textwidth}{figs/dfs-binary-tree}

  \begin{description}
    \centering
    \item[\red{Pre-order} (前序) Traversal:] $F, B, A, D, C, E, G, I, H$
  \end{description}
\end{frame}
%%%%%%%%%%%%%%%

%%%%%%%%%%%%%%%
\begin{frame}{}
  \fig{width = 0.60\textwidth}{figs/dfs-binary-tree}

  \begin{description}
    \centering
    \item[\green{In-order} (中序) Traversal:] $A, B, C, D, E, F, G, H, I$
  \end{description}
\end{frame}
%%%%%%%%%%%%%%%

%%%%%%%%%%%%%%%
\begin{frame}{}
  \fig{width = 0.60\textwidth}{figs/dfs-binary-tree}

  \begin{description}
    \centering
    \item[\blue{Post-order} (后序) Traversal:] $A, C, E, D, B, H, I, G, F$
  \end{description}
\end{frame}
%%%%%%%%%%%%%%%

%%%%%%%%%%%%%%%
\begin{frame}{}
  \fig{width = 0.60\textwidth}{figs/tree-expression}

  \begin{columns}
    \column{0.10\textwidth}
    \column{0.80\textwidth}
      \begin{description}[<+->][Postfix Expression (后缀表达式):]
        \item[Prefix Expression (前缀表达式):]
          $+ * A - B C + D E$
        \item[Infix Expression (中缀表达式):]
          $A * \gray{(}B - C\gray{)} + \gray{(}D + E\gray{)}$
        \item[Postfix Expression (后缀表达式):]
          $A B C - * D E + +$
      \end{description}
    \column{0.10\textwidth}
  \end{columns}
\end{frame}
%%%%%%%%%%%%%%%

%%%%%%%%%%%%%%%
\begin{frame}{}
  \fig{width = 0.50\textwidth}{figs/bfs-binary-tree}

  \begin{description}
    \centering
    \item[Breadth-First Search (BFS):] $F, B, G, A, D, I, C, E, H$
  \end{description}
\end{frame}
%%%%%%%%%%%%%%%

%%%%%%%%%%%%%%%
\begin{frame}{}
  \fig{width = 0.35\textwidth}{figs/Huffman}

  \begin{center}
    \teal{David A. Huffman ($1925 \sim 1999$)}
  \end{center}
\end{frame}
%%%%%%%%%%%%%%%

%%%%%%%%%%%%%%%
\begin{frame}{}
  \begin{center}
    \begin{table}
      \centering
      \renewcommand*{\arraystretch}{1.3}
      \begin{tabular}{|c|c|c|c|c|c|c|}
	\hline
        $C[1 \dots n]$ & $a$ & $b$ & $c$ & $d$ & $e$ & $f$ \\ \hline
	$F[1 \dots n]$ & 45 & 13 & 12 & 16 & 9 & 5 \\ \hline
	Fixed Length Code & 000 & 001 & 010 & 011 & 100 & 101 \\ \hline
	Variable Length Code & 0 & 101 & 100 & 111 & 1101 & 1100 \\ \hline
      \end{tabular}
    \end{table}

    \vspace{0.80cm}
    Prefix code (前缀码): \red{No} code is a \red{prefix} of some other code
  \end{center}
\end{frame}
%%%%%%%%%%%%%%%

%%%%%%%%%%%%%%%
\begin{frame}{}
  \begin{exampleblock}{The Encoding Problem}
    To find the \red{optimal} binary prefix code for $C$ and $F$.
  \end{exampleblock}

  \vspace{0.50cm}
  \begin{center}
    Let $E$ be a binary prefix code for $C$ and $F$.
    The length $L(E)$ is
  \end{center}
  \[
    L(E) = \sum_{c \in C} f_{c} \cdot l_{E}(c)
  \]
\end{frame}
%%%%%%%%%%%%%%%

%%%%%%%%%%%%%%%
\begin{frame}{}
  \begin{table}
    \centering
    \renewcommand*{\arraystretch}{1.3}
    \begin{tabular}{|c|c|c|c|c|c|c|}
      \hline
      $C[1 \dots n]$ & $a$ & $b$ & $c$ & $d$ & $e$ & $f$ \\ \hline
      $F[1 \dots n]$ & 45 & 13 & 12 & 16 & 9 & 5 \\ \hline
    \end{tabular}
  \end{table}

  \fig{width = 0.40\textwidth}{figs/Huffman-CLRS}
\end{frame}
%%%%%%%%%%%%%%%