% permutation.tex

%%%%%%%%%%%%%%%
\begin{frame}{}
  \begin{center}
    \red{Cyclic Notation} (轮换表示法) \& \blue{Transposition} (对换)
  \end{center}
  \[
    \sigma = \begin{pmatrix}
      1 & 2 & 3 & 4 & 5 & 6 \\
      4 & 3 & 6 & 1 & 5 & 2
    \end{pmatrix}
  \]
  \vspace{-0.50cm}
  \begin{align*}
    \sigma &= (1\; 4) (2\; 3\; 6) (5) \\[6pt]
      &= \red{(1\; 4) (2\; 3\; 6)} \\[6pt]
      &= (2\; 3\; 6) (1\; 4) \\[6pt]
      &= (2\; 3\; 6) (4\; 1) \\[6pt]
      &= (3\; 6\; 2) (4\; 1) \\[6pt]
      &= \blue{(3\; 6) (6\; 2) (4\; 1)}
  \end{align*}

  \pause
  \vspace{-0.50cm}
  \[
    (i_{1}\; i_{2}\; \dots\; i_{r}) =
      (i_{1}\; i_{2})(i_{2}\; i_{3}) \dots (i_{r-2}\; i_{r-1}) (i_{r-1}\; i_{r})
  \]
  \begin{center}
    \violet{By induction on the length $r$.}
  \end{center}
\end{frame}
%%%%%%%%%%%%%%%

%%%%%%%%%%%%%%%
\begin{frame}{}
  \begin{align*}
    \sigma = \begin{pmatrix}
      1 & 2 & 3 & 4 & 5 & 6 & 7 \\
      7 & 3 & 6 & 2 & 5 & 4 & 1
    \end{pmatrix} &=
    (1\;7) (2\;3) (3\;6) (6\;4) \\
    &= (1\;7) (3\;6) (2\;5) (6\;4) (4\;5) (2\;5)
  \end{align*}

  \pause
  \vspace{0.30cm}
  \begin{theorem}[Parity (奇偶性) of Permutations]
    将一个置换表示成若干对换的乘积, 所用对换个数的奇偶性是唯一的。
  \end{theorem}
\end{frame}
%%%%%%%%%%%%%%%

%%%%%%%%%%%%%%%
\begin{frame}{}
  \begin{definition}[Even/Odd Permutations (偶置换/奇置换)]
    可表示为偶数个对换的乘积的置换称为偶置换; 否则, 称为奇置换。
  \end{definition}

  \pause
  \vspace{0.80cm}
  \begin{definition}[Alternating Group (交错群; $A_{n}$)]
    由 $S_{n}$ 的\blue{全体偶置换}构成的\cyan{子群}称为 $n$ 次\red{交错群}。
  \end{definition}

  \pause
  \[
    A_{3} = \set{(1), (1\;2\;3), (1\;3\;2)}
  \]
\end{frame}
%%%%%%%%%%%%%%%

%%%%%%%%%%%%%%%
\begin{frame}{}
  \[
    \text{sgn}: S_{n} \to \set{1, -1}
  \]

  \pause
  \[
    \text{sgn}(\sigma_{1}\sigma_{2}) = \text{sgn}(\sigma_{1})\text{sgn}(\sigma_{2})
  \]

  \pause
  \[
    S_{n}/A_{n} \cong \set{1, -1}
  \]
\end{frame}
%%%%%%%%%%%%%%%

%%%%%%%%%%%%%%%
\begin{frame}{}
\end{frame}
%%%%%%%%%%%%%%%