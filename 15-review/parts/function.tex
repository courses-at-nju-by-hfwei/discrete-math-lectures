% function.tex

%%%%%%%%%%%%%%%
\begin{frame}{}
  \fig{width = 0.50\textwidth}{figs/fx}
\end{frame}
%%%%%%%%%%%%%%%

%%%%%%%%%%%%%%%
\begin{frame}{}
  \begin{center}
    Injection (one-to-one; 1-1)

    \vspace{0.60cm}
    Surjection

    \vspace{0.60cm}
    Bijection (one-to-one correspondence)
  \end{center}
\end{frame}
%%%%%%%%%%%%%%%

%%%%%%%%%%%%%%%
\begin{frame}{}
  \begin{definition}[Characteristic Function (特征函数) of a Subset]
    For a given subset $A \subseteq X$,
    \[
      \chi_{A}: X \to \set{0, 1}
    \]
    \[
      \chi_{A}(x) = 1 \iff x \in A.
    \]
  \end{definition}

  \pause
  \vspace{0.30cm}
  \fig{width = 0.50\textwidth}{figs/powset}
  \[
    \chi_{A}: X \to \set{0, 1} \quad\text{\it vs.}\quad \ps{X}
  \]
\end{frame}
%%%%%%%%%%%%%%%

%%%%%%%%%%%%%%%
\begin{frame}{}
  \begin{definition}[Natural Function]
    Let $R \subseteq A \times A$ be an equivalence relation.
    The following function $f$
    \[
      f: A \to A/R
    \]
    \[
      f: a \mapsto [a]_{R}
    \]
    is called the \red{natural function} on $A$.
  \end{definition}

  \fig{width = 0.30\textwidth}{figs/set-partition}
\end{frame}
%%%%%%%%%%%%%%%

%%%%%%%%%%%%%%%
\begin{frame}{}
  \begin{center}
    Asymptotic Growth Rates of Functions
  \end{center}

  \begin{columns}
    \column{0.50\textwidth}
      \fig{width = 0.80\textwidth}{figs/growth-rate}
    \column{0.50\textwidth}
      \fig{width = 0.70\textwidth}{figs/qrcode-efficiency}
  \end{columns}

  \vspace{0.30cm}
  \begin{center}
    \teal{\small \url{https://www.bilibili.com/video/BV175411T7ph?share_source=copy_web}}
  \end{center}
\end{frame}
%%%%%%%%%%%%%%%