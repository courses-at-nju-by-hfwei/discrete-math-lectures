% planarity.tex

%%%%%%%%%%%%%%%
\begin{frame}{}
  \begin{definition}[Planar Graph (平面图)]
    A \red{planar graph} is a graph that
    \purple{can} be drawn in the plane without \cyan{edge crossings}.
  \end{definition}

  \pause
  \fig{width = 0.60\textwidth}{figs/K4-allinone}

  \pause
  \vspace{0.60cm}
  \begin{theorem}[K. Wagner (1936); I. F\'{a}ry (1948)]
    Every \blue{simple} planar graph can be drawn with \red{straight lines}.
  \end{theorem}
\end{frame}
%%%%%%%%%%%%%%%

%%%%%%%%%%%%%%%
\begin{frame}{}
  \begin{theorem}[\teal{Kazimierz Kuratowski, 1930}]
    The \blue{utility graph} $K_{3, 3}$ is non-planar.
    \fig{width = 0.25\textwidth}{figs/K-3-3-Bipartite}
  \end{theorem}

  \begin{columns}
    \column{0.50\textwidth}
      \pause
      \fig{width = 0.70\textwidth}{figs/3-utilities-problem}
    \column{0.50\textwidth}
      \pause
      \fig{width = 0.70\textwidth}{figs/3-utilities-problem-plane}
  \end{columns}
\end{frame}
%%%%%%%%%%%%%%%

%%%%%%%%%%%%%%%
\begin{frame}{}
  \fig{width = 0.50\textwidth}{figs/K-3-3-Proof-without-Words}
\end{frame}
%%%%%%%%%%%%%%%

%%%%%%%%%%%%%%%
\begin{frame}{}
  \fig{width = 0.50\textwidth}{figs/K-3-3-cycle}

  \pause
  \fig{width = 0.80\textwidth}{figs/K-3-3-Nonplanar-Proof}

  \pause
  \vspace{-0.60cm}
  \[
    \red{\text{cr}}(K_{3, 3}) = 1
  \]
\end{frame}
%%%%%%%%%%%%%%%

%%%%%%%%%%%%%%%
\begin{frame}{}
  \begin{theorem}
    $K_{5}$ is non-planar.
    \fig{width = 0.25\textwidth}{figs/K5}
  \end{theorem}

  \pause
  \fig{width = 0.70\textwidth}{figs/K5-Nonplanar-Proof}

  \pause
  \vspace{-0.60cm}
  \[
    \red{\text{cr}}(K_{5}) = 1
  \]
\end{frame}
%%%%%%%%%%%%%%%

%%%%%%%%%%%%%%%
\begin{frame}{}
  \begin{theorem}
    The Petersen graph is non-planar.
  \end{theorem}

  \begin{columns}
    \column{0.50\textwidth}
      \fig{width = 0.80\textwidth}{figs/Petersen-Graph}
    \column{0.50\textwidth}
      \pause
      \fig{width = 0.80\textwidth}{figs/Petersen-Graph-Crossing}
  \end{columns}

  \pause
  \[
    \red{\text{cr}}(\text{Petersen Graph}) = 2
  \]
\end{frame}
%%%%%%%%%%%%%%%

%%%%%%%%%%%%%%%
\begin{frame}{}
  \begin{theorem}[Kazimierz Kuratowski, 1930]
    A graph is planar iff it contains no \blue{subgraph}
    \red{homeomorphic} to \cyan{$K_{5}$} or \cyan{$K_{3, 3}$}.
  \end{theorem}

  \pause
  \fig{width = 0.30\textwidth}{figs/Kuratowski}

  \pause
  \begin{quote}
    \begin{center}
      ``The $K$ in $K_{5}$ stands for Kazimierz,\\
      and the $K$ in $K_{3, 3}$ stands for Kuratowski.''
    \end{center}
  \end{quote}
\end{frame}
%%%%%%%%%%%%%%%

%%%%%%%%%%%%%%%
\begin{frame}{}
  \begin{theorem}[Kazimierz Kuratowski, 1930]
    A graph is planar iff it contains no \blue{subgraph}
    \red{homeomorphic} to \cyan{$K_{5}$} or \cyan{$K_{3, 3}$}.
  \end{theorem}

  \pause
  \begin{columns}
    \column{0.50\textwidth}
      \fig{width = 0.60\textwidth}{figs/homeomorphism-example-1}
    \column{0.50\textwidth}
      \fig{width = 0.60\textwidth}{figs/homeomorphism-example-2}
  \end{columns}

  \begin{definition}[Homeomorphic]
    Two graphs are \red{homeomorphic} if
    one can be obtained from another by \blue{inserting or contracting}
    vertices of \cyan{degree 2}.
  \end{definition}
\end{frame}
%%%%%%%%%%%%%%%

%%%%%%%%%%%%%%%
\begin{frame}{}
  \begin{theorem}
    The Petersen graph is non-planar.
  \end{theorem}

  \pause
  \begin{center}
    \transduration<0-83>{0}
    \multiinclude[<+->][format = png, graphics = {width=0.60\textwidth}]{figs/Kuratowski-Petersen}
  \end{center}
\end{frame}
%%%%%%%%%%%%%%%

%%%%%%%%%%%%%%%
\begin{frame}{}
  \begin{center}
    \red{\Large A planar graph should not has too many edges.}
  \end{center}
\end{frame}
%%%%%%%%%%%%%%%

%%%%%%%%%%%%%%%
\begin{frame}{}
  \begin{theorem}[Euler's Formula, 1750]
    Let $G$ be a \blue{plane drawing} of a \cyan{connected} planar graph,
    and let $n$, $m$, and \red{$f$} denote respectively the number of vertices,
    edges, and \red{faces} of $G$.
    \[
      n - m + \red{f} = 2
    \]
  \end{theorem}

  \pause
  \fig{width = 0.40\textwidth}{figs/euler-formula-example}
  \pause
  \[
    n - m + \red{f} = 20 - 30 + \red{12} = 2
  \]
\end{frame}
%%%%%%%%%%%%%%%

%%%%%%%%%%%%%%%
\begin{frame}{}
  \begin{theorem}[Euler's Formula, 1750]
    Let $G$ be a \blue{plane drawing} of a \cyan{connected} planar graph,
    and let $n$, $m$, and \red{$f$} denote respectively the number of vertices,
    edges, and \red{faces} of $G$.
    \[
      n - m + \red{f} = 2
    \]
  \end{theorem}

  \pause
  \fig{width = 0.40\textwidth}{figs/euler-formula-tree}
  \pause
  \[
    n - m + \red{f} = n - (n - 1) + \red{1} = 2
  \]
\end{frame}
%%%%%%%%%%%%%%%

%%%%%%%%%%%%%%%
\begin{frame}{}
  \begin{center}
    \red{By induction on the number of \blue{edges} of $G$.}

    \pause
    \vspace{0.30cm}
    \begin{description}[<+->][Induction Hypothesis:]
      \setlength{\itemsep}{8pt}
      \item[Basis Step:] $m = 0$. We have $n = 1$ and $f = 1$.
      \item[Induction Hypothesis:]
        It holds for \cyan{plane graphs} with $m$ edges.
      \item[Induction Step:] Consider a plane graph $G$ with $m+1$ edges. \\[5pt]
        \uncover<5->{If $G$ is a tree, we are done. \\[5pt]}
        \uncover<6->{Otherwise, $G$ contains a cycle. \\[5pt]}
        \uncover<7->{Let $e$ be an edge in some cycle of $G$. \\[5pt]}
        \uncover<8->{Consider $G' = G - e$.}
        \uncover<9->{
          \[
            n - (m - 1) + \red{(f - 1)} = 2
          \]
        }
        \uncover<10->{Therefore,
          \[
            n - m + \red{f} = 2
          \]
        }
    \end{description}
  \end{center}
\end{frame}
%%%%%%%%%%%%%%%

%%%%%%%%%%%%%%%
\begin{frame}{}
  \begin{theorem}
    Let $G$ be a simple connected planar graph with $n \ge 3$ vertices
    and $m$ edges. Then
    \[
      m \le 3n -6.
    \]
  \end{theorem}

  \pause
  \[
    n - m + f = 2
  \]
  \pause
  \[
    3f \le 2m
  \]
  \pause
  \begin{center}
    \red{Double Counting:} \\[3pt]
    each face is bounded by $\ge 3$ edges; \\
    each edge bounds $2$ faces
  \end{center}
\end{frame}
%%%%%%%%%%%%%%%

%%%%%%%%%%%%%%%
\begin{frame}{}
  \begin{theorem}
    $K_{5}$ is non-planar.
  \end{theorem}

  \pause
  \[
    m \le 3n -6
  \]

  \pause
  \[
    10 \le 3 \times 5 - 6
  \]
\end{frame}
%%%%%%%%%%%%%%%

%%%%%%%%%%%%%%%
\begin{frame}{}
  \begin{theorem}
    $K_{3, 3}$ is non-planar.
  \end{theorem}

  \pause
  \[
    m \le 3n -6
  \]

  \pause
  \[
    9 \le 3 \times 6 - 6
  \]

  \pause
  \fig{width = 0.30\textwidth}{figs/failed}
\end{frame}
%%%%%%%%%%%%%%%

%%%%%%%%%%%%%%%
\begin{frame}{}
  \begin{theorem}
    Let $G$ be a simple connected planar graph with $n \ge 3$ vertices
    and $m$ edges. \red{If $G$ has no triangles}, then
    \[
      m \le 2n - 4.
    \]
  \end{theorem}

  \pause
  \[
    n - m + f = 2
  \]

  \pause
  \[
    4f \le 2m
  \]
\end{frame}
%%%%%%%%%%%%%%%

%%%%%%%%%%%%%%%
\begin{frame}{}
  \begin{theorem}
    $K_{3, 3}$ is non-planar.
  \end{theorem}

  \pause
  \[
    m \le 2n -4
  \]

  \pause
  \[
    9 \le 2 \times 6 - 4
  \]
\end{frame}
%%%%%%%%%%%%%%%

%%%%%%%%%%%%%%%
\begin{frame}{}
  \begin{theorem}
    Every simple planar graph contains a vertex of degree $\le 5$.
  \end{theorem}

  \pause
  \[
    m \le 3n - 6
  \]

  \pause
  \begin{center}
    Suppose that, \red{by contradiction}, $\delta(G) \ge 6$.
  \end{center}

  \pause
  \[
    6n \le 2m
  \]

  \pause
  \[
    3n \le m \le 3n - 6
  \]
\end{frame}
%%%%%%%%%%%%%%%