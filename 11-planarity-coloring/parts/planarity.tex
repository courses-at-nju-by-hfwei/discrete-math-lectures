% planarity.tex

%%%%%%%%%%%%%%%
\begin{frame}{}
  \begin{definition}[Planar Graph (平面图)]
    A \red{planar graph} is a graph that
    \purple{can} be drawn in the plane without \cyan{edge crossings}.
  \end{definition}

  \pause
  \fig{width = 0.60\textwidth}{figs/K4-allinone}

  \pause
  \vspace{0.60cm}
  \begin{theorem}[K. Wagner (1936); I. F\'{a}ry (1948)]
    Every \blue{simple} planar graph can be drawn with \red{straight lines}.
  \end{theorem}
\end{frame}
%%%%%%%%%%%%%%%

%%%%%%%%%%%%%%%
\begin{frame}{}
  \begin{theorem}[\teal{Kazimierz Kuratowski, 1930}]
    The \blue{utility graph} $K_{3, 3}$ is non-planar.
    \fig{width = 0.25\textwidth}{figs/K-3-3-Bipartite}
  \end{theorem}

  \begin{columns}
    \column{0.50\textwidth}
      \pause
      \fig{width = 0.70\textwidth}{figs/3-utilities-problem}
    \column{0.50\textwidth}
      \pause
      \fig{width = 0.70\textwidth}{figs/3-utilities-problem-plane}
  \end{columns}
\end{frame}
%%%%%%%%%%%%%%%

%%%%%%%%%%%%%%%
\begin{frame}{}
  \fig{width = 0.50\textwidth}{figs/K-3-3-Proof-without-Words}
\end{frame}
%%%%%%%%%%%%%%%

%%%%%%%%%%%%%%%
\begin{frame}{}
  \fig{width = 0.50\textwidth}{figs/K-3-3-cycle}

  \pause
  \fig{width = 0.80\textwidth}{figs/K-3-3-Nonplanar-Proof}

  \pause
  \vspace{-0.60cm}
  \[
    \red{\text{cr}}(K_{3, 3}) = 1
  \]
\end{frame}
%%%%%%%%%%%%%%%

%%%%%%%%%%%%%%%
\begin{frame}{}
  \begin{theorem}
    $K_{5}$ is non-planar.
    \fig{width = 0.25\textwidth}{figs/K5}
  \end{theorem}

  \pause
  \fig{width = 0.70\textwidth}{figs/K5-Nonplanar-Proof}

  \pause
  \vspace{-0.60cm}
  \[
    \red{\text{cr}}(K_{5}) = 1
  \]
\end{frame}
%%%%%%%%%%%%%%%

%%%%%%%%%%%%%%%
\begin{frame}{}
  \begin{theorem}
    The Petersen graph is non-planar.
  \end{theorem}

  \begin{columns}
    \column{0.50\textwidth}
      \fig{width = 0.80\textwidth}{figs/Petersen-Graph}
    \column{0.50\textwidth}
      \pause
      \fig{width = 0.80\textwidth}{figs/Petersen-Graph-Crossing}
  \end{columns}

  \pause
  \[
    \red{\text{cr}}(\text{Petersen Graph}) = 2
  \]
\end{frame}
%%%%%%%%%%%%%%%

%%%%%%%%%%%%%%%
\begin{frame}{}
  \begin{theorem}[Kazimierz Kuratowski, 1930]
    A graph is planar iff it contains no \blue{subgraph}
    \red{homeomorphic} to \cyan{$K_{5}$} or \cyan{$K_{3, 3}$}.
  \end{theorem}

  \pause
  \fig{width = 0.30\textwidth}{figs/Kuratowski}

  \pause
  \begin{quote}
    \begin{center}
      ``The $K$ in $K_{5}$ stands for Kazimierz,\\
      and the $K$ in $K_{3, 3}$ stands for Kuratowski.''
    \end{center}
  \end{quote}
\end{frame}
%%%%%%%%%%%%%%%

%%%%%%%%%%%%%%%
\begin{frame}{}
  \begin{theorem}[Kazimierz Kuratowski, 1930]
    A graph is planar iff it contains no \blue{subgraph}
    \red{homeomorphic} to \cyan{$K_{5}$} or \cyan{$K_{3, 3}$}.
  \end{theorem}

  \pause
  \begin{columns}
    \column{0.50\textwidth}
      \fig{width = 0.60\textwidth}{figs/homeomorphism-example-1}
    \column{0.50\textwidth}
      \fig{width = 0.60\textwidth}{figs/homeomorphism-example-2}
  \end{columns}

  \begin{definition}[Homeomorphic]
    Two graphs are \red{homeomorphic} if
    one can be obtained from another by \blue{inserting or contracting}
    vertices of \cyan{degree 2}.
  \end{definition}
\end{frame}
%%%%%%%%%%%%%%%

%%%%%%%%%%%%%%%
\begin{frame}{}
  \begin{theorem}
    The Petersen graph is non-planar.
  \end{theorem}

  \pause
  \begin{center}
    \transduration<0-83>{0}
    \multiinclude[<+->][format = png, graphics = {width=0.60\textwidth}]{figs/Kuratowski-Petersen}
  \end{center}
\end{frame}
%%%%%%%%%%%%%%%

%%%%%%%%%%%%%%%
\begin{frame}{}
\end{frame}
%%%%%%%%%%%%%%%

%%%%%%%%%%%%%%%
\begin{frame}{}
\end{frame}
%%%%%%%%%%%%%%%