% overview.tex

%%%%%%%%%%%%%%%
\begin{frame}
  \fig{width = 0.50\textwidth}{figs/4-colored-map-wiki}
\end{frame}
%%%%%%%%%%%%%%%

%%%%%%%%%%%%%%%
\begin{frame}{}
  \begin{theorem}[Four Color (Map) Theorem \teal{(informal)}]
    Every \red{map} can be colored with only \blue{four} colors such that \\
    no two \purple{adjacent} \cyan{regions} share the same color.
  \end{theorem}

  \pause
  \fig{width = 0.35\textwidth}{figs/4-colored-map-wiki}
\end{frame}
%%%%%%%%%%%%%%%

%%%%%%%%%%%%%%%
\begin{frame}{}
  \fig{width = 0.50\textwidth}{figs/4-color-example}
\end{frame}
%%%%%%%%%%%%%%%

%%%%%%%%%%%%%%%
\begin{frame}{}
  \fig{width = 0.50\textwidth}{figs/must-4-color}

  \pause
  \begin{center}
    Every region is adjacent to the other $3$ regions.
  \end{center}
\end{frame}
%%%%%%%%%%%%%%%

%%%%%%%%%%%%%%%
% \begin{frame}{}
%   \begin{theorem}[Four Color (Map) Theorem \teal{(informal)}]
%     Every \red{map} can be colored with only \blue{four} colors such that \\
%     no two \purple{adjacent} \cyan{regions} share the same color.
%   \end{theorem}
%
%   \begin{columns}
%     \column{0.50\textwidth}
%       \pause
%       \fig{width = 0.80\textwidth}{figs/non-contiguous-region}
%         \begin{center}
%           {Regions should be \teal{contiguous}.}
%         \end{center}
%     \column{0.50\textwidth}
%       \pause
%       \fig{width = 1.00\textwidth}{figs/triangulation}
%       \begin{center}
%         \purple{Adjacent} regions share a segment.
%       \end{center}
%   \end{columns}
% \end{frame}
%%%%%%%%%%%%%%%

%%%%%%%%%%%%%%%
\begin{frame}{}
  \begin{center}
    What if we have a map which contains \blue{$5$} regions so that \\[5pt]
    every region is adjacent to the other \red{$4$} regions?
  \end{center}

  \pause
  \fig{width = 0.40\textwidth}{figs/impossible}
\end{frame}
%%%%%%%%%%%%%%%

%%%%%%%%%%%%%%%
\begin{frame}{}
  \begin{center}
    What does \blue{Four Color Theorem} to do with \red{Graph Theory}?
  \end{center}

  \fig{width = 0.40\textwidth}{figs/must-4-color}
\end{frame}
%%%%%%%%%%%%%%%

%%%%%%%%%%%%%%%
\begin{frame}{}
  \fig{width = 0.75\textwidth}{figs/4CT-Graph-Coloring}

  \pause
  \vspace{0.40cm}
  \begin{center}
    Every map produces a \red{planar} graph.
  \end{center}
\end{frame}
%%%%%%%%%%%%%%%

%%%%%%%%%%%%%%%
\begin{frame}{}
  \begin{theorem}[Four Color Theorem \teal{\small (Kenneth Appel, Wolfgang Haken; 1976)}]
    Every \cyan{simple} \red{planar} graph is \blue{4-colorable}.
  \end{theorem}

  \fig{width = 0.30\textwidth}{figs/4CT-People}

  \pause
  \begin{center}
    I will \red{\it not} show its proof (which I don't understand either)!
  \end{center}
\end{frame}
%%%%%%%%%%%%%%%

%%%%%%%%%%%%%%%
\begin{frame}{}
  \begin{theorem}
    Every \cyan{simple} \red{planar} graph is \blue{6-colorable}.
  \end{theorem}

  \pause
  \vspace{1.50cm}
  \begin{theorem}[Percy John Heawood (1890)]
    Every \cyan{simple} \red{planar} graph is \blue{5-colorable}.
  \end{theorem}
\end{frame}
%%%%%%%%%%%%%%%