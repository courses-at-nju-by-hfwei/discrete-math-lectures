% overview.tex

%%%%%%%%%%%%%%%
\begin{frame}
  \fig{width = 0.50\textwidth}{figs/4-colored-map-wiki}
\end{frame}
%%%%%%%%%%%%%%%

%%%%%%%%%%%%%%%
\begin{frame}{}
  \begin{theorem}[Four Color (Map) Theorem \teal{(informal)}]
    Every \red{map} can be colored with only \blue{four} colors such that \\
    no two \purple{adjacent} \cyan{regions} share the same color.
  \end{theorem}

  \pause
  \fig{width = 0.35\textwidth}{figs/4-colored-map-wiki}
\end{frame}
%%%%%%%%%%%%%%%

%%%%%%%%%%%%%%%
\begin{frame}{}
  \begin{theorem}[Four Color (Map) Theorem \teal{(informal)}]
    Every \red{map} can be colored with only \blue{four} colors such that \\
    no two \purple{adjacent} \cyan{regions} share the same color.
  \end{theorem}

  \begin{columns}
    \column{0.50\textwidth}
      \pause
      \fig{width = 0.80\textwidth}{figs/non-contiguous-region}
        \begin{center}
          {Regions should be \teal{contiguous}.}
        \end{center}
    \column{0.50\textwidth}
      \pause
      \fig{width = 1.00\textwidth}{figs/triangulation}
      \begin{center}
        \purple{Adjacent} regions share a segment.
      \end{center}
  \end{columns}
\end{frame}
%%%%%%%%%%%%%%%

%%%%%%%%%%%%%%%
\begin{frame}{}
  \begin{theorem}[Four Color (Map) Theorem \teal{(informal)}]
    Every \red{map} can be colored with only \blue{four} colors such that \\
    no two \purple{adjacent} \cyan{regions} share the same color.
  \end{theorem}

  \pause
  \fig{width = 0.30\textwidth}{figs/do-you-believe}

  \pause
  \begin{center}
    What if we have a map in which \\
    every region is adjacent to $\ge 5$ other regions?
  \end{center}
\end{frame}
%%%%%%%%%%%%%%%

%%%%%%%%%%%%%%%
\begin{frame}{}
  \begin{theorem}[Four Color (Map) Theorem \teal{(informal)}]
    Every \red{map} can be colored with only \blue{four} colors such that \\
    no two \purple{adjacent} \cyan{regions} share the same color.
  \end{theorem}

  \pause
  \vspace{0.50cm}
  \begin{center}
    What does it to do with \violet{\textsc{\LARGE graph theory}}?
  \end{center}
\end{frame}
%%%%%%%%%%%%%%%

%%%%%%%%%%%%%%%
\begin{frame}{}
  \begin{theorem}[Four Color (Map) Theorem \teal{(informal)}]
    Every \red{map} can be colored with only \blue{four} colors such that \\
    no two \purple{adjacent} \cyan{regions} share the same color.
  \end{theorem}

  \pause
  \fig{width = 0.80\textwidth}{figs/map2graph}

  \pause
  \begin{theorem}[Four Color (Map) Theorem]
    Every \red{planar} graph is \blue{four-colorable}.
  \end{theorem}
\end{frame}
%%%%%%%%%%%%%%%