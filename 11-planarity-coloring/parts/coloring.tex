% coloring.tex

%%%%%%%%%%%%%%%
\begin{frame}{}
  \begin{definition}[$k$-Colorable ($k$-可着色的)]
    If $G$ is a \gray{connected} undirected graph \teal{without loops}, \\
    then $G$ is \red{$k$-colorable} if its vertices can be colored in \purple{$\le k$} colors
    \\ so that adjacent vertices have different colors.
  \end{definition}

  \pause
  \fig{width = 0.40\textwidth}{figs/Petersen-3coloring}
  \begin{center}
    The Petersen graph is \purple{$\ge 3$}-colorable.
  \end{center}
\end{frame}
%%%%%%%%%%%%%%%

%%%%%%%%%%%%%%%
\begin{frame}{}
  \begin{definition}[$k$-Chromatic ($k$-色数的)]
    If $G$ is $k$-colorable, but is not $(k-1)$-colorable,
    then $G$ is $k$-chromatic.

    \[
      \chi(G) = k
    \]
  \end{definition}

  \pause
  \fig{width = 0.40\textwidth}{figs/Petersen-3coloring}
  \begin{center}
    The Petersen graph is \purple{3}-chromatic.
  \end{center}
\end{frame}
%%%%%%%%%%%%%%%

%%%%%%%%%%%%%%%
\begin{frame}{}
  \begin{lemma}
    The empty graph (null graph) is $1$-chromatic.
  \end{lemma}

  \pause
  \begin{columns}
    \column{0.50\textwidth}
      \fig{width = 0.90\textwidth}{figs/empty-graph}
    \column{0.50\textwidth}
      \uncover<4->{\fig{width = 0.90\textwidth}{figs/complete-graphs}}
  \end{columns}

  \pause
  \begin{lemma}
    $K_{n}$ is $n$-chromatic.
  \end{lemma}
\end{frame}
%%%%%%%%%%%%%%%

%%%%%%%%%%%%%%%
\begin{frame}{}
  \begin{theorem}
    A graph is 2-colorable iff \pause it is \red{bipartite}.
  \end{theorem}
\end{frame}
%%%%%%%%%%%%%%%

%%%%%%%%%%%%%%%
\begin{frame}{}
  \begin{theorem}[Characterization of Bipartite Graphs]
    A graph is 2-colorable iff it is \red{bipartite}.

    \pause
    \vspace{0.30cm}
    A graph is \red{bipartite} iff it does not contain any \purple{odd} cycles.
  \end{theorem}

  \begin{columns}
    \column{0.50\textwidth}
      \pause
      \fig{width = 0.80\textwidth}{figs/K-5-3}
      \[
        K_{5, 3}
      \]
    \column{0.50\textwidth}
      \pause
      \fig{width = 0.70\textwidth}{figs/bipartite-graph-no-cycles}
  \end{columns}

  \pause
  \begin{lemma}
    Every tree is bipartite and is thus $2$-colorable.
  \end{lemma}
\end{frame}
%%%%%%%%%%%%%%%

%%%%%%%%%%%%%%%
\begin{frame}{}
  \begin{Lemma}[Characterization of Bipartite Graphs (\!$\implies$\!)]
    If a graph is \red{bipartite},
    then it does not contain any \purple{odd} cycles.
  \end{Lemma}
\end{frame}
%%%%%%%%%%%%%%%

%%%%%%%%%%%%%%%
\begin{frame}{}
  \begin{Lemma}[Characterization of Bipartite Graphs ($\Longleftarrow$)]
    If a graph does not contain any \purple{odd} cycles,
    then it is \red{bipartite}.
  \end{Lemma}
\end{frame}
%%%%%%%%%%%%%%%

%%%%%%%%%%%%%%%
\begin{frame}{}
alg
\end{frame}
%%%%%%%%%%%%%%%

%%%%%%%%%%%%%%%
\begin{frame}{}
  \begin{theorem}{}
    The $3$-coloring problem (i.e., testing whether a graph is $3$-colorable or not)
    is \textsf{NP-complete}.
  \end{theorem}

  \pause
  \begin{columns}
    \column{0.50\textwidth}
      \fig{width = 1.00\textwidth}{figs/odd-cycles}
    \column{0.50\textwidth}
      \fig{width = 0.70\textwidth}{figs/Petersen-3coloring}
  \end{columns}

  \pause
  \begin{theorem}{}
    The $4$-coloring problem is also \textsf{NP-complete}.
  \end{theorem}
\end{frame}
%%%%%%%%%%%%%%%

%%%%%%%%%%%%%%%
\begin{frame}{}
  \begin{theorem}
    Let $G$ be a \red{simple} connected graph. Then,
    \[
      \chi(G) \le \Delta(G) + 1.
    \]
  \end{theorem}

  \pause
  \begin{center}
    \red{By induction on the number of vertices of $G$.}

    \pause
    \vspace{0.20cm}
    \begin{description}[<+->][Induction Hypothesis:]
      \setlength{\itemsep}{5pt}
      \item[Basis Step:] $n = 1$. $\chi(G) = 1$, $\Delta(G) = 0$.
      \item[Induction Hypothesis:] Suppose that for any simple connected graph $G$
        with $n$ vertices,
        \[
          \chi(G) \le \Delta(G) + 1.
        \]
      \item[Induction Step:] Consider a simple connected graph $G$ with $n+1$ vertices.
        \uncover<7->{\qquad\qquad \red{$\deg(v) \le \Delta(G)$}}
        \uncover<6->{\fig{width = 0.50\textwidth}{figs/Delta+1-induction}}
    \end{description}
  \end{center}
\end{frame}
%%%%%%%%%%%%%%%

%%%%%%%%%%%%%%%
\begin{frame}{}
  alg
\end{frame}
%%%%%%%%%%%%%%%

%%%%%%%%%%%%%%%
\begin{frame}{}
  \begin{theorem}[Brooks's Theorem \teal{(R. Leonard Brooks; 1941)}]
    Let $G$ be a \red{simple} connected graph other than a \blue{complete graph}
    or an \blue{odd cycle}. Then
    \[
      \chi(G) \le \Delta(G).
    \]
  \end{theorem}

  \begin{columns}
    \column{0.50\textwidth}
      \pause
      \fig{width = 0.80\textwidth}{figs/K6-coloring}
    \column{0.50\textwidth}
      \pause
      \fig{width = 0.80\textwidth}{figs/Petersen-3coloring}
  \end{columns}
\end{frame}
%%%%%%%%%%%%%%%

%%%%%%%%%%%%%%%
\begin{frame}{}
  \begin{theorem}[Brooks's Theorem \teal{(R. Leonard Brooks; 1941)}]
    Let $G$ be a \red{simple} connected graph other than a \blue{complete graph}
    or an \blue{odd cycle}. Then
    \[
      \chi(G) \le \Delta(G).
    \]
  \end{theorem}

  \pause
  \fig{width = 0.25\textwidth}{figs/Book-West}
  \begin{center}
    \teal{Theorem 5.1.22}
  \end{center}
\end{frame}
%%%%%%%%%%%%%%%