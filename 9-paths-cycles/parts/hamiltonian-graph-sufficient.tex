% hamiltonian-graph-sufficient.tex

%%%%%%%%%%%%%%%
\begin{frame}{}
  \fig{width = 0.50\textwidth}{figs/sufficient}

  \begin{center}
    ``If $G$ has enough edges, then $G$ is Hamiltonian.''
  \end{center}
\end{frame}
%%%%%%%%%%%%%%%

%%%%%%%%%%%%%%%
\begin{frame}{}
  \begin{theorem}[Ore's Theorem, 1960]
    Let $G$ be a \blue{simple} graph with \blue{$n \ge 3$} vertices. If
    \[
      \degree(u) + \degree(v) \ge n
    \]
    for \cyan{each pair} of \purple{non-adjacent} vertices $u$ and $v$,
    then $G$ is \red{Hamiltonian}.
  \end{theorem}

  \pause
  \fig{width = 0.40\textwidth}{figs/Ore-example}
\end{frame}
%%%%%%%%%%%%%%%

%%%%%%%%%%%%%%%
\begin{frame}{}
  \begin{center}
    \red{\it By Contradiction.}

    \pause
    \vspace{0.30cm}
    Let $G$ be a \cyan{\it non-Hamiltonian} (simple) graph with $n \ge 3$ vertices.

    \pause
    \vspace{0.30cm}
    Suppose that $G$ meets the \blue{Ore's Condition}. \\[3pt]
    We need to derive a contradiction.

    \pause
    \vspace{0.80cm}
    \red{\it By Extremality.} \\[3pt] \pause
    Adding edges cannot violate the \blue{Ore's Condition}. \\[3pt] \pause
    Thus we may consider only \red{\it maximal} non-Hamiltonian graphs: \\[3pt]
    \cyan{adding any edge gives a Hamiltonian graph}.
  \end{center}
\end{frame}
%%%%%%%%%%%%%%%

%%%%%%%%%%%%%%%
\begin{frame}{}
  \begin{center}
    By its ``maximality'', $G$ contains a \blue{Hamiltonian path}
    \[
      v_{1} \to v_{2} \to \dots \to v_{n}
    \]

    \pause
    \vspace{-0.60cm}
    \[
      v_{1} \text{ and } v_{n} \text{ are \red{non-adjacent}}
    \]

    \pause
    \vspace{-0.60cm}
    \[
      \degree(v_{1}) + \degree(v_{2}) \ge n
    \]

    \pause
    \fig{width = 0.50\textwidth}{figs/Ore-proof}
    There must be some vertex \purple{$v_{i}$ adjacent to $v_{1}$} \\[3pt]
    such that \purple{$v_{i-1}$ is adjacent to $v_{n}$}.
  \end{center}
\end{frame}
%%%%%%%%%%%%%%%

%%%%%%%%%%%%%%%
\begin{frame}{}
  \begin{theorem}[Dirac's Theorem (1952; Gabriel Andrew Dirac)]
    A \blue{simple} graph $G = (V, E)$
    with \blue{$n \ge 3$} vertices is \red{Hamiltonian}
    \[
      \forall v \in V.\; \degree(v) \ge n/2.
    \]
  \end{theorem}

  \pause
  \[
    \delta(G) \triangleq \min_{v \in V} \degree(v)
  \]
  \[
    \cyan{\delta(G) \ge n/2}
  \]

  \pause
  \fig{width = 0.85\textwidth}{figs/Ore-family}
\end{frame}
%%%%%%%%%%%%%%%

%%%%%%%%%%%%%%%
\begin{frame}{}
  \begin{theorem}[Dirac's Theorem (1952)]
    A \blue{simple} graph $G = (V, E)$
    with \blue{$n \ge 3$} vertices is \red{Hamiltonian}
    \[
      \cyan{\delta(G) \ge n/2}
    \]
  \end{theorem}

  \pause
  \[
    \delta(G) = \lfloor (n-1)/2 \rfloor
  \]

  \pause
  \vspace{0.30cm}
  \begin{center}
    \red{Counterexample:}
    $C_{\lfloor (n + 1)/ 2 \rfloor}$ and $C_{\lceil (n+1)/2 \rceil}$
    sharing a vertex
  \end{center}
\end{frame}
%%%%%%%%%%%%%%%