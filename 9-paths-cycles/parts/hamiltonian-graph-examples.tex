% hamiltonian-graph-examples.tex

%%%%%%%%%%%%%%%
\begin{frame}{}
  \fig{width = 0.50\textwidth}{figs/examples}
\end{frame}
%%%%%%%%%%%%%%%

%%%%%%%%%%%%%%%
\begin{frame}{}
  \begin{exampleblock}{}
    \begin{itemize}
      \item Every \red{cycle} is Hamiltonian
    \end{itemize}
  \end{exampleblock}

  \vspace{0.60cm}
  \fig{width = 0.40\textwidth}{figs/C6}
  \[
    C_{6}
  \]
\end{frame}
%%%%%%%%%%%%%%%

%%%%%%%%%%%%%%%
\begin{frame}{}
  \begin{exampleblock}{}
    \begin{itemize}
      \item A \red{complete} graph (完全图) with $|V| > 2$ is Hamiltonian.
    \end{itemize}
  \end{exampleblock}

  \pause
  \vspace{0.60cm}
  \begin{columns}
    \column{0.25\textwidth}
      \fig{width = 0.80\textwidth}{figs/K1}
      \[
        K_{1}
      \]
    \column{0.25\textwidth}
      \fig{width = 0.80\textwidth}{figs/K2}
      \[
        K_{2}
      \]
    \column{0.25\textwidth}
      \fig{width = 0.80\textwidth}{figs/K3}
      \[
        K_{3}
      \]
    \column{0.25\textwidth}
      \fig{width = 0.80\textwidth}{figs/K5}
      \[
        K_{5}
      \]
  \end{columns}
\end{frame}
%%%%%%%%%%%%%%%

%%%%%%%%%%%%%%%
\begin{frame}{}
  \begin{exampleblock}{}
    \begin{itemize}
      \item A \red{complete bipartite} graph $K_{m, n}$ is Hamiltonian
        \blue{iff} $m = n \ge 2$.
    \end{itemize}
  \end{exampleblock}
\end{frame}
%%%%%%%%%%%%%%%

%%%%%%%%%%%%%%%
\begin{frame}{}
  \begin{definition}[Bipartite Graph (Bigraph; 二部图)]
    A \red{bipartite graph} $G = (U, V, E)$ is a graph
    whose \blue{vertices} \purple{can} be divided into
    two disjoint sets $U$ and $V$ such that
    every edge connects a vertex in $U$ to one in $V$.
  \end{definition}

  \begin{columns}
    \column{0.50\textwidth}
      \fig{width = 0.70\textwidth}{figs/bipartite-graph}
    \column{0.50\textwidth}
      \pause
      \fig{width = 0.60\textwidth}{figs/K-3-3-plain}
  \end{columns}
\end{frame}
%%%%%%%%%%%%%%%

%%%%%%%%%%%%%%%
\begin{frame}{}
  \begin{definition}[Complete Bipartite Graph (Biclique; 完全二部图)]
    A \red{complete bipartite graph} $G = (U, V, E)$ is \blue{bipartite graph}
    where every vertex of $U$ is connected to every vertex of $V$.
    \[
      \blue{K_{m, n}}: m = |U|, n = |V|
    \]
  \end{definition}

  \pause
  \begin{columns}
    \column{0.33\textwidth}
      \fig{width = 0.75\textwidth}{figs/K-1-7}
      \[
        K_{1, 5} \quad (\text{star})
      \]
    \column{0.33\textwidth}
      \fig{width = 0.80\textwidth}{figs/K-3-3}
      \[
        K_{3, 3} \quad (\text{utility graph})
      \]
    \column{0.33\textwidth}
      \fig{width = 0.80\textwidth}{figs/K-4-7}
      \[
        K_{4, 7}
      \]
  \end{columns}
\end{frame}
%%%%%%%%%%%%%%%

%%%%%%%%%%%%%%%
\begin{frame}{}
  \begin{exampleblock}{}
    \begin{itemize}
      \item A \red{complete bipartite} graph $K_{m, n}$ is Hamiltonian
        \blue{iff} $m = n \ge 2$.
    \end{itemize}
  \end{exampleblock}

  \pause
  \vspace{0.60cm}
  \begin{columns}
    \column{0.50\textwidth}
      \fig{width = 0.60\textwidth}{figs/K-3-3}
    \column{0.50\textwidth}
      \fig{width = 0.70\textwidth}{figs/K-5-3}
  \end{columns}
\end{frame}
%%%%%%%%%%%%%%%

%%%%%%%%%%%%%%%
\begin{frame}{}
  \begin{exampleblock}{}
    \begin{itemize}
      \item Every \red{platonic solid} (正多面体), considered as a graph, is Hamiltonian.
    \end{itemize}
  \end{exampleblock}

  \pause
  \vspace{0.60cm}
  \begin{columns}
    \column{0.20\textwidth}
      \fig{width = 0.90\textwidth}{figs/T4}
      \begin{center}
        \teal{Tetrahedron}
      \end{center}
    \column{0.20\textwidth}
      \fig{width = 0.90\textwidth}{figs/Cube6}
      \begin{center}
        \teal{Cube}
      \end{center}
    \column{0.20\textwidth}
      \fig{width = 0.90\textwidth}{figs/O8}
      \begin{center}
        \teal{Octahedron}
      \end{center}
    \column{0.20\textwidth}
      \fig{width = 0.90\textwidth}{figs/D12}
      \begin{center}
        \teal{Dodecahedron}
      \end{center}
    \column{0.20\textwidth}
      \fig{width = 0.90\textwidth}{figs/I20}
      \begin{center}
        \teal{Icosahedron}
      \end{center}
  \end{columns}
\end{frame}
%%%%%%%%%%%%%%%

%%%%%%%%%%%%%%%
\begin{frame}{}
  \fig{width = 0.26\textwidth}{figs/Hamiltonian-platonic-graphs}
\end{frame}
%%%%%%%%%%%%%%%

%%%%%%%%%%%%%%%
\begin{frame}{}
  \begin{theorem}{}
    \begin{itemize}
      \item \red{Petersen graph} is \blue{\it not} Hamiltonian.
        \fig{width = 0.20\textwidth}{figs/Petersen}
        \begin{center}
          \teal{Julius Petersen ($1839 \sim 1910$)}
        \end{center}
    \end{itemize}
  \end{theorem}

  \pause
  \begin{columns}
    \column{0.25\textwidth}
      \fig{width = 0.80\textwidth}{figs/Petersen-graph}
    \column{0.25\textwidth}
      \fig{width = 0.80\textwidth}{figs/Petersen-graph-unit}
    \column{0.25\textwidth}
      \fig{width = 0.80\textwidth}{figs/Petersen-graph-crossing}
    \column{0.25\textwidth}
      \fig{width = 0.80\textwidth}{figs/Petersen-graph-hypo-Hamiltonian}
  \end{columns}
\end{frame}
%%%%%%%%%%%%%%%