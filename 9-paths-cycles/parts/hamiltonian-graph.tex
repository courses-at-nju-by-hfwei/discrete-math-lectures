% hamiltonian-graph.tex

%%%%%%%%%%%%%%%
\begin{frame}{}
  \begin{center}
    Dodecahedron: \blue{12 faces}, 20 vertices, and 30 edges

    \fig{width = 0.40\textwidth}{figs/dodecahedron}

    \pause
    \red{Is there a \red{cycle} that visits each \red{vertex} exactly once?}
  \end{center}
\end{frame}
%%%%%%%%%%%%%%%

%%%%%%%%%%%%%%%
\begin{frame}{}
  \fig{width = 0.60\textwidth}{figs/Dodecahedron-Diagram}
  \vspace{-0.50cm}
  \begin{center}
    \red{Is there a \red{cycle} that visits each \red{vertex} exactly once?}
  \end{center}
\end{frame}
%%%%%%%%%%%%%%%

%%%%%%%%%%%%%%%
\begin{frame}{}
  \begin{definition}[Hamiltonian Path]
    A \red{Hamiltonian path} is a \blue{path}
    that visits each \blue{vertex} \purple{exactly once}.
  \end{definition}

  \fig{width = 0.50\textwidth}{figs/Hamiltonian-Path-Cycle}

  \begin{definition}[Hamiltonian Cycle]
    A \red{Hamiltonian cycle} is a \blue{Hamiltonian path} that is a \purple{cycle}.
  \end{definition}
\end{frame}
%%%%%%%%%%%%%%%

%%%%%%%%%%%%%%%
\begin{frame}{}
  \begin{definition}[Hamiltonian Graph]
    A graph is a \red{Hamiltonian graph} if it has a \blue{Hamiltonian cycle}.
  \end{definition}

  \pause
  \vspace{0.30cm}
  \begin{definition}[Semi-Hamiltonian Graph]
    A \blue{non-Hamiltonian} graph is \red{semi-Hamiltonian}
    if it has a \blue{Hamiltonian path}.
  \end{definition}
\end{frame}
%%%%%%%%%%%%%%%

%%%%%%%%%%%%%%%
\begin{frame}{}
  \begin{columns}
    \column{0.50\textwidth}
      \fig{width = 0.70\textwidth}{figs/Hamilton}
      \begin{center}
        \teal{William Rowan Hamilton $(1805 \sim 1865)$}
      \end{center}
    \column{0.50\textwidth}
      \fig{width = 0.80\textwidth}{figs/Quaternion-Bridge}
      \begin{center}
        \teal{(October 16, 1843)}
      \end{center}
      \[
        i^2 = j^{2} = k^{2} = ijk = -1
      \]
  \end{columns}
\end{frame}
%%%%%%%%%%%%%%%

%%%%%%%%%%%%%%%
\begin{frame}{}
  \begin{columns}
    \column{0.50\textwidth}
      \fig{width = 0.80\textwidth}{figs/Hamiltonian-path-solution}
    \column{0.50\textwidth}
      \pause
      \fig{width = 0.80\textwidth}{figs/Hamiltonian-path-solution-3d}
  \end{columns}
\end{frame}
%%%%%%%%%%%%%%%

%%%%%%%%%%%%%%%
\begin{frame}{}
  \begin{center}
    \red{What is ``THE'' theorem for finding a Hamiltonian path/cycle
        \\ or determining its existence?}

    \pause
    \vspace{0.80cm}
    I do not know.

    \pause
    \vspace{0.80cm}
    Nobody knows.

    \pause
    \vspace{0.80cm}
    We will probably never know it.
  \end{center}
\end{frame}
%%%%%%%%%%%%%%%

%%%%%%%%%%%%%%%
\begin{frame}{}
  \fig{width = 0.50\textwidth}{figs/millionaire}

  \begin{theorem}
    The Hamiltonian Path/Cycle problem is NP-complete.
  \end{theorem}
\end{frame}
%%%%%%%%%%%%%%%

%%%%%%%%%%%%%%%
\begin{frame}{}
  \begin{center}
    Typical (Positive/Negative) Graph Examples

    \vspace{0.80cm}
    Sufficient Conditions

    \vspace{0.80cm}
    Necessary Conditions
  \end{center}
\end{frame}
%%%%%%%%%%%%%%%

%%%%%%%%%%%%%%%
\begin{frame}{}
  \begin{exampleblock}{}
    \begin{itemize}
      \item A \red{complete} graph (完全图) with $|V| > 2$ is Hamiltonian.
    \end{itemize}
  \end{exampleblock}

  \pause
  \vspace{0.60cm}
  \begin{columns}
    \column{0.25\textwidth}
      \fig{width = 0.80\textwidth}{figs/K1}
      \[
        K_{1}
      \]
    \column{0.25\textwidth}
      \fig{width = 0.80\textwidth}{figs/K2}
      \[
        K_{2}
      \]
    \column{0.25\textwidth}
      \fig{width = 0.80\textwidth}{figs/K3}
      \[
        K_{3}
      \]
    \column{0.25\textwidth}
      \fig{width = 0.80\textwidth}{figs/K5}
      \[
        K_{5}
      \]
  \end{columns}
\end{frame}
%%%%%%%%%%%%%%%

%%%%%%%%%%%%%%%
\begin{frame}{}
  \begin{exampleblock}{}
    \begin{itemize}
      \item Every \red{cycle} is Hamiltonian
    \end{itemize}
  \end{exampleblock}

  \vspace{0.60cm}
  \fig{width = 0.40\textwidth}{figs/C6}
  \[
    C_{6}
  \]
\end{frame}
%%%%%%%%%%%%%%%

%%%%%%%%%%%%%%%
\begin{frame}{}
  \begin{exampleblock}{}
    \begin{itemize}
      \item Every \red{platonic solid} (正多面体), considered as a graph, is Hamiltonian.
    \end{itemize}
  \end{exampleblock}

  \pause
  \vspace{0.60cm}
  \begin{columns}
    \column{0.20\textwidth}
      \fig{width = 0.90\textwidth}{figs/T4}
      \begin{center}
        \teal{Tetrahedron}
      \end{center}
    \column{0.20\textwidth}
      \fig{width = 0.90\textwidth}{figs/Cube6}
      \begin{center}
        \teal{Cube}
      \end{center}
    \column{0.20\textwidth}
      \fig{width = 0.90\textwidth}{figs/O8}
      \begin{center}
        \teal{Octahedron}
      \end{center}
    \column{0.20\textwidth}
      \fig{width = 0.90\textwidth}{figs/D12}
      \begin{center}
        \teal{Dodecahedron}
      \end{center}
    \column{0.20\textwidth}
      \fig{width = 0.90\textwidth}{figs/I20}
      \begin{center}
        \teal{Icosahedron}
      \end{center}
  \end{columns}
\end{frame}
%%%%%%%%%%%%%%%

%%%%%%%%%%%%%%%
\begin{frame}{}
  \fig{width = 0.26\textwidth}{figs/Hamiltonian-platonic-graphs}
\end{frame}
%%%%%%%%%%%%%%%

%%%%%%%%%%%%%%%
\begin{frame}{}
  \begin{exampleblock}{}
    \begin{itemize}
      \item \red{Petersen graph} is \blue{\it not} Hamiltonian.
        \fig{width = 0.20\textwidth}{figs/Petersen}
        \begin{center}
          \teal{Julius Petersen ($1839 \sim 1910$)}
        \end{center}
    \end{itemize}
  \end{exampleblock}

  \pause
  \begin{columns}
    \column{0.25\textwidth}
      \fig{width = 0.80\textwidth}{figs/Petersen-graph}
    \column{0.25\textwidth}
      \fig{width = 0.80\textwidth}{figs/Petersen-graph-unit}
    \column{0.25\textwidth}
      \fig{width = 0.80\textwidth}{figs/Petersen-graph-crossing}
    \column{0.25\textwidth}
      \fig{width = 0.80\textwidth}{figs/Petersen-graph-hypo-Hamiltonian}
  \end{columns}
\end{frame}
%%%%%%%%%%%%%%%

%%%%%%%%%%%%%%%
\begin{frame}{}
  \fig{width = 0.50\textwidth}{figs/sufficient}

  \begin{center}
    ``If $G$ has enough edges, then $G$ is Hamiltonian.''
  \end{center}
\end{frame}
%%%%%%%%%%%%%%%

%%%%%%%%%%%%%%%
\begin{frame}{}
  \begin{theorem}[Ore's Theorem, 1960]
    Let $G$ be a \blue{simple} graph with \blue{$n \ge 3$} vertices. If
    \[
      \degree(u) + \degree(v) \ge n
    \]
    for \cyan{each pair} of \purple{non-adjacent} vertices $u$ and $v$,
    then $G$ is \red{Hamiltonian}.
  \end{theorem}

  \pause
  \fig{width = 0.40\textwidth}{figs/Ore-example}
\end{frame}
%%%%%%%%%%%%%%%

%%%%%%%%%%%%%%%
\begin{frame}{}
  \begin{center}
    \red{\it By Contradiction.}

    \pause
    \vspace{0.30cm}
    Let $G$ be a \cyan{\it non-Hamiltonian} (simple) graph with $n \ge 3$ vertices.

    \pause
    \vspace{0.30cm}
    Suppose that $G$ meets the \blue{Ore's Condition}. \\[3pt]
    We need to derive a contradiction.

    \pause
    \vspace{0.80cm}
    \red{\it By Extremality.} \\[3pt] \pause
    Adding edges cannot violate the \blue{Ore's Condition}. \\[3pt] \pause
    Thus we may consider only \red{\it maximal} non-Hamiltonian graphs: \\[3pt]
    \cyan{adding any edge gives a Hamiltonian graph}.
  \end{center}
\end{frame}
%%%%%%%%%%%%%%%

%%%%%%%%%%%%%%%
\begin{frame}{}
  \begin{center}
    By its ``maximality'', $G$ contains a \blue{Hamiltonian path}
    \[
      v_{1} \to v_{2} \to \dots \to v_{n}
    \]

    \pause
    \vspace{-0.60cm}
    \[
      v_{1} \text{ and } v_{n} \text{ are \red{non-adjacent}}
    \]

    \pause
    \vspace{-0.60cm}
    \[
      \degree(v_{1}) + \degree(v_{2}) \ge n
    \]

    \pause
    \fig{width = 0.50\textwidth}{figs/Ore-proof}
    There must be some vertex \purple{$v_{i}$ adjacent to $v_{1}$} \\[3pt]
    such that \purple{$v_{i-1}$ is adjacent to $v_{n}$}.
  \end{center}
\end{frame}
%%%%%%%%%%%%%%%

%%%%%%%%%%%%%%%
\begin{frame}{}
  \begin{theorem}[Dirac's Theorem (1952; Gabriel Andrew Dirac)]
    A \blue{simple} graph $G = (V, E)$
    with \blue{$n \ge 3$} vertices is \red{Hamiltonian}
    \[
      \forall v \in V.\; \degree(v) \ge n/2.
    \]
  \end{theorem}

  \pause
  \[
    \delta(G) \triangleq \min_{v \in V} \degree(v)
  \]
  \[
    \cyan{\delta(G) \ge n/2}
  \]

  \pause
  \fig{width = 0.80\textwidth}{figs/Ore-family}
\end{frame}
%%%%%%%%%%%%%%%

%%%%%%%%%%%%%%%
\begin{frame}{}
  \begin{theorem}[Dirac's Theorem (1952)]
    A \blue{simple} graph $G = (V, E)$
    with \blue{$n \ge 3$} vertices is \red{Hamiltonian}
    \[
      \cyan{\delta(G) \ge n/2}
    \]
  \end{theorem}

  \pause
  \[
    \delta(G) = \lfloor (n-1)/2 \rfloor
  \]

  \pause
  \vspace{0.30cm}
  \begin{center}
    \red{Counterexample:}
    $C_{\lfloor (n + 1)/ 2 \rfloor}$ and $C_{\lceil (n+1)/2 \rceil}$
    sharing a vertex
  \end{center}
\end{frame}
%%%%%%%%%%%%%%%