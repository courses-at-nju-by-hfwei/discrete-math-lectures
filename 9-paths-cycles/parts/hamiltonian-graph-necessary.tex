% hamiltonian-graph-necessary.tex

%%%%%%%%%%%%%%%
\begin{frame}{}
  \begin{center}
    ``If $G$ is Hamiltonian, then $G$ must be somewhat connected.''

    \fig{width = 0.60\textwidth}{figs/necessary}

    \pause
    ``If $G$ is not so connected, then $G$ is non-Hamiltonian.''
  \end{center}
\end{frame}
%%%%%%%%%%%%%%%

%%%%%%%%%%%%%%%
\begin{frame}{}
  \begin{theorem}{}
    If $G = (V, E)$ is Hamiltonian,
    then for each nonempty set $S \subset V$, \\[3pt]
    the graph $G - S$ has \red{$\le |S|$} \blue{components}.
  \end{theorem}
\end{frame}
%%%%%%%%%%%%%%%

%%%%%%%%%%%%%%%
\begin{frame}{}
  \begin{definition}[Components (连通分支)]
    A \red{component} of an \blue{undirected graph}
    is an \cyan{subgraph} in which any two vertices are connected to each other by paths,
    and which is connected to no additional vertices in the rest of the graph.
  \end{definition}

  \vspace{0.20cm}
  \fig{width = 0.40\textwidth}{figs/components}

  \pause
  \vspace{-0.20cm}
  \[
    \red{c(G)} = 3
  \]
\end{frame}
%%%%%%%%%%%%%%%

%%%%%%%%%%%%%%%
\begin{frame}{}
  \begin{theorem}{}
    If $G = (V, E)$ is Hamiltonian, then
    \[
      \forall S \subset V.\; c(G - S) \le |S|.
    \]
  \end{theorem}

  \pause
  \vspace{0.30cm}
  \[
    S \neq V
  \]

  \pause
  \vspace{0.30cm}
  \begin{center}
    $C_{\lfloor (n + 1)/ 2 \rfloor}$ and $C_{\lceil (n+1)/2 \rceil}$
    sharing a vertex
  \end{center}
\end{frame}
%%%%%%%%%%%%%%%

%%%%%%%%%%%%%%%
\begin{frame}{}
  \begin{exampleblock}{}
    \begin{itemize}
      \item A \red{complete bipartite} graph $K_{m, n}$ is Hamiltonian
        \blue{iff} $m = n \ge 2$.
    \end{itemize}
  \end{exampleblock}

  \pause
  \vspace{0.50cm}
  \fig{width = 0.40\textwidth}{figs/bipartite-non-hamiltonian}

  \begin{center}
    A \cyan{non-Hamiltonian} bipartite graph with $m = n = 10$
  \end{center}
\end{frame}
%%%%%%%%%%%%%%%

%%%%%%%%%%%%%%%
\begin{frame}{}
  \begin{theorem}{}
    If $G = (V, E)$ is Hamiltonian,
    \[
      \forall S \subset V.\; c(G - S) \le |S|.
    \]
  \end{theorem}

  \pause
  \vspace{0.50cm}
  \fig{width = 0.35\textwidth}{figs/Hamiltonian-necessary}

  \pause
  \begin{center}
    ``When a Hamiltonian cycle \red{leaves a component} of $G - S$, \\[3pt]
    it can go only to a \blue{distinct vertex in $S$}.''
  \end{center}
\end{frame}
%%%%%%%%%%%%%%%

%%%%%%%%%%%%%%%
\begin{frame}{}
  \begin{theorem}{}
    If $G = (V, E)$ is Hamiltonian,
    \[
      \forall S \subset V.\; c(G - S) \le |S|.
    \]
  \end{theorem}

  \pause
  \vspace{0.30cm}
  \begin{center}
    The condition is \red{\it not} sufficient.

    \pause
    \fig{width = 0.30\textwidth}{figs/necessary-not-sufficient}

    \pause
    \begin{center}
      \blue{All edges incident to vertices of \cyan{degree 2} must be used.}
    \end{center}
  \end{center}
\end{frame}
%%%%%%%%%%%%%%%

%%%%%%%%%%%%%%%
\begin{frame}{}
  \begin{exampleblock}{Chessboard Problem (``马踏棋盘''问题)}
    Is it possible for a ``knight'' to visit every field of
    a $4 \times 4$ or \teal{$5 \times 5$} chessboard exactly once
    and return to the starting point?
  \end{exampleblock}

  \vspace{0.50cm}
  \fig{width = 0.60\textwidth}{figs/chessboard}
\end{frame}
%%%%%%%%%%%%%%%

%%%%%%%%%%%%%%%
\begin{frame}{}
  \fig{width = 0.40\textwidth}{figs/chessboard-5-5}

  \[
    G = (U, V, E): |U| = 12, |V| = 13
  \]
\end{frame}
%%%%%%%%%%%%%%%

%%%%%%%%%%%%%%%
\begin{frame}{}
  \fig{width = 0.40\textwidth}{figs/chessboard-4-4}

  \pause
  \vspace{0.20cm}
  \begin{center}
    Removing the middle \blue{4 squares} leaves \blue{$\ge 5$} components.
  \end{center}
\end{frame}
%%%%%%%%%%%%%%%

%%%%%%%%%%%%%%%
\begin{frame}{}
  \begin{exampleblock}{Chessboard Problem}
    \fig{width = 0.65\textwidth}{figs/chessboard-4-n}
    \[
      \teal{4 \times n}
    \]
  \end{exampleblock}
\end{frame}
%%%%%%%%%%%%%%%

%%%%%%%%%%%%%%%
\begin{frame}{}
  \begin{definition}[Travelling Salesman/Salesperson Problem (TSP; 旅行商问题)]
    Given a list of cities and the distances between each pair of cities, \\[3pt]
    what is the \red{shortest} possible route \\[3pt]
    that visits each city \blue{exactly once} and returns to the origin city?
  \end{definition}

  \fig{width = 0.30\textwidth}{figs/TSP}
\end{frame}
%%%%%%%%%%%%%%%

%%%%%%%%%%%%%%%
\begin{frame}{}
  \fig{width = 0.35\textwidth}{figs/TSP-book}
\end{frame}
%%%%%%%%%%%%%%%