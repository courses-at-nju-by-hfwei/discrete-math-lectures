% eulerian-graph.tex

%%%%%%%%%%%%%%%
\begin{frame}{}
  \fig{width = 0.70\textwidth}{figs/Konigsberg}

  \pause
  \begin{center}
    {\it ``to devise a walk through the \red{city} \\
    that would cross each of those \blue{bridges} \violet{once and only once}''}
  \end{center}
\end{frame}
%%%%%%%%%%%%%%%

%%%%%%%%%%%%%%%
\begin{frame}{}
  \begin{columns}
    \column{0.50\textwidth}
      \fig{width = 0.80\textwidth}{figs/Konigsberg}
    \column{0.50\textwidth}
      \fig{width = 0.80\textwidth}{figs/bridges}
  \end{columns}
\end{frame}
%%%%%%%%%%%%%%%

%%%%%%%%%%%%%%%
\begin{frame}{}
  \begin{columns}
    \column{0.50\textwidth}
      \fig{width = 0.80\textwidth}{figs/bridges}
    \column{0.50\textwidth}
      \fig{width = 0.80\textwidth}{figs/bridges-graph}
  \end{columns}

  \pause
  \vspace{0.60cm}
  \begin{center}
    {\it ``to devise a walk through the \red{graph} \\
    that would cross each of those \blue{edges} \violet{once and only once}''}
  \end{center}
\end{frame}
%%%%%%%%%%%%%%%

%%%%%%%%%%%%%%%
\begin{frame}{}
  \begin{definition}[Graph (图)]
    An \red{(undirected simple) graph} is a \blue{pair} $G = (V, E)$ where \\[6pt]
    \begin{itemize}
      \setlength{\itemsep}{5pt}
      \item $V$ is a set of \purple{vertices} (顶点);
      \item $E \subseteq \set{\violet{\set{x, y}} \mid x, y \in V \land \cyan{x \neq y}}$
        is a set of \purple{edges}
    \end{itemize}
  \end{definition}

  \vspace{0.30cm}
  \fig{width = 0.35\textwidth}{figs/simple-graph}
\end{frame}
%%%%%%%%%%%%%%%

%%%%%%%%%%%%%%%
\begin{frame}{}
  \fig{width = 0.60\textwidth}{figs/bridges-graph}

  \begin{center}
    {Undirected \red{Multigraph}}
  \end{center}
\end{frame}
%%%%%%%%%%%%%%%

%%%%%%%%%%%%%%%
\begin{frame}{}
  \fig{width = 0.60\textwidth}{figs/multigraph-loop}

  \vspace{-0.50cm}
  \begin{center}
    {Undirected Multigraph Permitting \red{Loops}}
  \end{center}
\end{frame}
%%%%%%%%%%%%%%%

%%%%%%%%%%%%%%%
\begin{frame}{}
  \begin{definition}[Walk (道路)]
    Given a graph $G$, a (finite) \red{walk} in $G$ is a sequence of edges of the form
    \[
      \set{v_{0}, v_{1}}, \set{v_{1}, v_{2}}, \dots, \set{v_{m-1}, v_{m}}.
    \]
    \[
      (v_{0} \to v_{1} \to v_{2} \to \dots \to v_{m})
    \]
    \pause
    \vspace{-0.80cm}
    \begin{center}
      It is a \red{walk} from the \blue{initial vertex} $v_{0}$ to the \blue{final vertex} $v_{m}$.
    \end{center}
  \end{definition}

  \pause
  \[
    D \to \purple{E} \to B \to A \to \purple{E} \to F
  \]
  \fig{width = 0.50\textwidth}{figs/walk}
  \pause
  \[
    D \to \purple{E} \;\cyan{\to}\; B \;\cyan{\to}\; \purple{E} \to F
  \]
\end{frame}
%%%%%%%%%%%%%%%

%%%%%%%%%%%%%%%
\begin{frame}{}
  \begin{definition}[Trail (迹)]
    A \red{trail} is a \blue{walk} in which all the \cyan{edges} are distinct.
  \end{definition}

  \pause
  \[
    D \to \purple{E} \to B \to A \to \purple{E} \to F
  \]
  \fig{width = 0.50\textwidth}{figs/walk}
  \[
    D \to \purple{E} \;\cyan{\to}\; B \;\cyan{\to}\; \purple{E} \to F
  \]
\end{frame}
%%%%%%%%%%%%%%%

%%%%%%%%%%%%%%%
\begin{frame}{}
  \begin{definition}[Path (路径)]
    A \red{path} is a \blue{trial} in which all \purple{vertices} are distinct.
  \end{definition}

  \pause
  \[
    D \to \purple{E} \to B \to A \to \purple{E} \to F
  \]
  \fig{width = 0.50\textwidth}{figs/walk}
  \[
    D \to \purple{E} \to F
  \]
\end{frame}
%%%%%%%%%%%%%%%

%%%%%%%%%%%%%%%
\begin{frame}{}
  \begin{definition}[Closed Walk/Trail/Path]
    A walk, trail, or path is \red{closed} if $v_{0} = v_{m}$.
  \end{definition}

  \pause
  \fig{width = 0.50\textwidth}{figs/walk}

  \pause
  \begin{definition}[Cycle]
    A \red{cycle} is a \blue{closed path} with at least one edge.
  \end{definition}
\end{frame}
%%%%%%%%%%%%%%%

%%%%%%%%%%%%%%%
\begin{frame}{}
  \begin{center}
    {\it ``to devise a walk through the \red{graph} \\
    that would cross each of those \blue{edges} \violet{once and only once}''}

    \fig{width = 0.50\textwidth}{figs/bridges-graph}

    \pause
    {\it to find a \violet{trail} that contains all \blue{edges} of the \red{graph}}
  \end{center}
\end{frame}
%%%%%%%%%%%%%%%

%%%%%%%%%%%%%%%
\begin{frame}{}
  \[
    v_{0} \to v_{1} \to \dots \to \red{v_{i}} \to \dots \to v_{m}
  \]

  \fig{width = 0.50\textwidth}{figs/bridges-graph}

  \pause
  \vspace{-0.50cm}
  \[
    v_{i} \notin \set{v_{0}, v_{m}} \implies
    \red{\degree(v_{i})} \text{ is even}
  \]
\end{frame}
%%%%%%%%%%%%%%%

%%%%%%%%%%%%%%%
\begin{frame}{}
  \begin{lemma}[Necessary Condition for Eulerian Trails]
    If a graph has \red{Eulerian trails},
    then \blue{zero or two} vertices have an \blue{odd} degree.
  \end{lemma}

  \pause
  \fig{width = 0.40\textwidth}{figs/bridges-graph}

  \begin{center}
    4 vertices of odd degree $\implies$ has no Eulerian trails
  \end{center}
\end{frame}
%%%%%%%%%%%%%%%

%%%%%%%%%%%%%%%
\begin{frame}{}
  \begin{theorem}[Euler's Theorem]
    A graph has \red{Eulerian trails} iff
    it contains \blue{zero or two} vertices that have an \blue{odd} degree.
  \end{theorem}

  \pause
  \vspace{0.60cm}
  \begin{center}
    Euler stated but did \red{\it not}
    prove the ``$\Longleftarrow$ ({\it \red{if}})'' direction.

    \pause
    \vspace{0.60cm}
    Carl Hierholzer \teal{$(1840 \sim 1871)$} gave the first complete proof in 1873.

    \pause
    \vspace{0.60cm}
    Pierre-Henry Fleury gave another proof in 1883.
  \end{center}
\end{frame}
%%%%%%%%%%%%%%%

%%%%%%%%%%%%%%%
\begin{frame}{}
  \begin{theorem}[Euler's Theorem (Carl Hierholzer)]
    A \purple{connected} graph has \red{Eulerian cycles} iff
    \blue{every vertex} has \blue{even} degree.
    \begin{center}
      \teal{(Such graphs are called Eulerian graphs.)}
    \end{center}
  \end{theorem}

  \pause
  \vspace{0.30cm}
  \fig{width = 0.50\textwidth}{figs/Hierholzer}
  \begin{center}
    cycle + cycle + cycle + $\dots$
  \end{center}
\end{frame}
%%%%%%%%%%%%%%%

%%%%%%%%%%%%%%%
\begin{frame}{}
  \fig{width = 0.50\textwidth}{figs/blank}
  \begin{center}
    Can you repeat the Carl Hierholzer's algorithm?
  \end{center}
\end{frame}
%%%%%%%%%%%%%%%

%%%%%%%%%%%%%%%
\begin{frame}{}
  \begin{lemma}
    If every vertex of a graph $G$ has degree $\ge 2$,
    then $G$ contains a cycle.
  \end{lemma}

  \pause
  \vspace{0.30cm}
  \begin{center}
    \red{\it By Extremality.}
    \pause
    \vspace{0.50cm}

    Let $P: u \to \dots$ be a \blue{\it maximal} path in $G$.

    \fig{width = 0.50\textwidth}{figs/path-cycle}
    \pause
    \[
      \degree(u) \ge 2
    \]
  \end{center}
\end{frame}
%%%%%%%%%%%%%%%

%%%%%%%%%%%%%%%
\begin{frame}{}
  \begin{theorem}[Euler's Theorem (Carl Hierholzer)]
    A \purple{connected} graph has \red{Eulerian cycles} iff
    \blue{every vertex} has \blue{even} degree.
  \end{theorem}

  \pause
  \vspace{0.30cm}
  \begin{center}
    \red{\it By induction on the number of \cyan{edges} $m$.}

    \pause
    \vspace{0.30cm}
    \fig{width = 0.50\textwidth}{figs/euler-proof}
    \[
      G' = G - C
    \]
  \end{center}
\end{frame}
%%%%%%%%%%%%%%%

%%%%%%%%%%%%%%%
\begin{frame}{}
  \begin{theorem}{}
    Every Eulerian graph decomposes into cycles.
  \end{theorem}

  \begin{columns}
    \column{0.50\textwidth}
      \fig{width = 0.80\textwidth}{figs/euler-proof}
    \column{0.50\textwidth}
      \fig{width = 0.80\textwidth}{figs/Hierholzer}
  \end{columns}
\end{frame}
%%%%%%%%%%%%%%%

%%%%%%%%%%%%%%%
% \begin{frame}{}
%   directed graphs
% \end{frame}
%%%%%%%%%%%%%%%