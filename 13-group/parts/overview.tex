% overview.tex

%%%%%%%%%%%%%%%
\begin{frame}{}
  \begin{quote}
    \begin{center}
      \red{\large ``这里需要补充说明, 可是我没有时间了!''}
    \end{center}
  \end{quote}

  \begin{columns}
    \column{0.50\textwidth}
      \fig{width = 0.70\textwidth}{figs/Galois}
      \begin{center}
        \teal{Évariste Galois \\ (伽罗瓦; $1811 \sim 1832$)}
      \end{center}
    \column{0.50\textwidth}
      \fig{width = 0.70\textwidth}{figs/final-page}
      \begin{center}
        \blue{May 29, 1832}
      \end{center}
  \end{columns}
\end{frame}
%%%%%%%%%%%%%%%

%%%%%%%%%%%%%%%
\begin{frame}{}
  \begin{center}
    \blue{\large ``论五次方程的代数解法问题'' (1929)}
  \end{center}

  \begin{columns}
    \column{0.33\textwidth}
      \pause
      \fig{width = 0.70\textwidth}{figs/Cauchy}
      \begin{center}
        \teal{Augustin-Louis Cauchy ($1789 \sim 1857$)}
      \end{center}
    \column{0.33\textwidth}
      \pause
      \fig{width = 0.80\textwidth}{figs/Fourier}
      \begin{center}
        \teal{Joseph Fourier ($1768 \sim 1830$)}
      \end{center}
    \column{0.33\textwidth}
      \pause
      \fig{width = 0.70\textwidth}{figs/Poisson}
      \begin{center}
        \teal{Siméon Denis Poisson ($1781 \sim 1840$)}
      \end{center}
  \end{columns}
\end{frame}
%%%%%%%%%%%%%%%

%%%%%%%%%%%%%%%
\begin{frame}{}
  \begin{center}
    ``Ask \red{Jacobi} or \red{Gauss} publicly to give their opinion, \\[6pt]
    not as to the \blue{truth}, but as to the \blue{importance} of these theorems.''
  \end{center}
\end{frame}
%%%%%%%%%%%%%%%

%%%%%%%%%%%%%%%
\begin{frame}{}
  \begin{center}
    ``Is there a \purple{formula} for the roots of a \red{$\ge 5$ degree}
    polynomial equation \\[8pt]
    in terms of its \blue{coefficients},
    using only $+, -, \times, \div, \red{\sqrt[r]{\;\;}}$?
  \end{center}
\end{frame}
%%%%%%%%%%%%%%%

%%%%%%%%%%%%%%%
\begin{frame}{}
  \[
    x^3 + px + q = 0
  \]

  \begin{columns}
    \column{0.50\textwidth}
      \fig{width = 0.95\textwidth}{figs/3-root}
    \column{0.50\textwidth}
      \fig{width = 0.60\textwidth}{figs/Cardano}
      \begin{center}
        \teal{Girolamo Cardano ($1501 \sim 21/09/1576$)}
      \end{center}
  \end{columns}
\end{frame}
%%%%%%%%%%%%%%%

%%%%%%%%%%%%%%%
\begin{frame}{}
  \fig{width = 0.65\textwidth}{figs/4-root}
\end{frame}
%%%%%%%%%%%%%%%

%%%%%%%%%%%%%%%
\begin{frame}{}
  \begin{theorem}[Abel-Ruffini Thoerem]
    There is \red{\it no} solution in \red{radicals} to
    polynomial equations of \red{$\ge 5$ degree}.
  \end{theorem}

  \pause
  \vspace{0.30cm}
  \fig{width = 0.35\textwidth}{figs/Abel}
  \begin{center}
    \teal{Niels Henrik Abel ($1802 \sim 1829$)}
  \end{center}
\end{frame}
%%%%%%%%%%%%%%%

%%%%%%%%%%%%%%%
\begin{frame}
  \begin{theorem}[Galois Theorem]
    An equation is \blue{solvable} in terms of radicals \red{iff}
    the \red{Galois group} of its splitting field is \blue{solvable}.
  \end{theorem}
\end{frame}
%%%%%%%%%%%%%%%

%%%%%%%%%%%%%%%
\begin{frame}
  \fig{width = 0.70\textwidth}{figs/nju-sun}

  \begin{center}
    \teal{\url{https://www.bilibili.com/video/BV1Ex411k7wk?share_source=copy_web}}
  \end{center}
\end{frame}
%%%%%%%%%%%%%%%

%%%%%%%%%%%%%%%
\begin{frame}
  \begin{quote}
  \begin{center}
    ``我看出了 Galois 用来证明这个美妙定理的方法是完全正确的。\\[8pt]
    在那个瞬间, 我体验到一种强烈的愉悦。'' \\[15pt]

    \hfill --- J. Liouville (刘维尔; 1846)
  \end{center}
\end{quote}
\end{frame}
%%%%%%%%%%%%%%%