% counting.tex

%%%%%%%%%%%%%%%
\begin{frame}{}
  \fig{width = 0.50\textwidth}{figs/trees-movie}
\end{frame}
%%%%%%%%%%%%%%%

%%%%%%%%%%%%%%%
\begin{frame}{}
  \begin{theorem}[Cayley's Formula]
    The number $T_{n}$ of \red{labeled} trees on $n \ge 2$ vertices is $n^{n-2}$.
  \end{theorem}

  \pause
  \fig{width = 0.40\textwidth}{figs/Cayley-Formula-4}
\end{frame}
%%%%%%%%%%%%%%%

%%%%%%%%%%%%%%%
\begin{frame}{}
  \fig{width = 0.40\textwidth}{figs/Arthur-Cayley}
  \begin{center}
    \teal{Arthur Cayley ($1821 \sim 1895$)}
  \end{center}
\end{frame}
%%%%%%%%%%%%%%%

%%%%%%%%%%%%%%%
\begin{frame}{}
  \fig{width = 0.45\textwidth}{figs/the-book-6}

  \begin{center}
    \blue{Chapter 33: Cayley's formula for the number of trees}
  \end{center}
\end{frame}
%%%%%%%%%%%%%%%

%%%%%%%%%%%%%%%
\begin{frame}{}
  \begin{columns}
    \column{0.20\textwidth}
    \column{0.60\textwidth}
        \begin{center}
          {\large By \red{Double Counting}}. \\[8pt]
          \hfill --- Jim Pitman
        \end{center}
    \column{0.20\textwidth}
  \end{columns}

  \pause
  \vspace{1.00cm}
  \begin{center}
    \teal{\url{https://en.wikipedia.org/wiki/Double\_counting\_(proof\_technique)\#Counting_trees}}

    \pause
    \vspace{0.80cm}
    How many ways are there of forming a \red{rooted tree} \\[3pt]
    from an \blue{empty graph} by adding \purple{directed edges} one by one?
  \end{center}
\end{frame}
%%%%%%%%%%%%%%%

%%%%%%%%%%%%%%%
\begin{frame}{}
  \begin{center}
    Choose one of the $\red{T_{n}}$ labeled trees on $n$ vertices.

    \pause
    \vspace{0.60cm}
    Choose one of its \purple{$n$} vertices as root.

    \pause
    \vspace{0.60cm}
    Choose one of the \blue{$(n-1)!$} possible sequences \\
    in which to add its $n-1$ directed edges.
  \end{center}

  \[
    \red{T_n} \purple{n}\blue{(n-1)}! = T_n n!
  \]
\end{frame}
%%%%%%%%%%%%%%%

%%%%%%%%%%%%%%%
\begin{frame}{}
  \begin{center}
    Suppose that we have added \cyan{$n-k$} directed edges.

    \pause
    \vspace{0.30cm}
    We obtain a rooted forest with \blue{$k$ trees}.

    \pause
    \vspace{0.30cm}
    There are $\red{n}\blue{(k-1)}$ choices for the next edge to add.
  \end{center}

  \pause
  \fig{width = 0.30\textwidth}{figs/Cayley-Formula-adding-one-edge}

  \pause
  \[
    \prod_{\purple{k=2}}^{\purple{n}} \red{n}\blue{(k-1)}
      = n^{n-1} (n-1)! = n^{n-2} n!
  \]
\end{frame}
%%%%%%%%%%%%%%%

%%%%%%%%%%%%%%%
\begin{frame}{}
  \[
    T_n n! = n^{n-2}n!
  \]

  \pause
  \[
    T_{n} = n^{n-2}
  \]

  \pause
  \fig{width = 0.30\textwidth}{figs/enjoy-it}
\end{frame}
%%%%%%%%%%%%%%%

%%%%%%%%%%%%%%%
\begin{frame}{}
  \begin{definition}[Irreducible Tree]
    An \red{irreducible tree} is a tree $T$ where
    \[
      \forall v \in V(T).\; \degree(v) \neq 2.
    \]
  \end{definition}

  \pause
  \vspace{0.30cm}
  \begin{columns}
    \column{0.50\textwidth}
      \fig{width = 0.80\textwidth}{figs/trees-10}
    \column{0.50\textwidth}
      \pause
      \fig{width = 0.80\textwidth}{figs/trees-movie}
  \end{columns}

  \pause
  \vspace{0.30cm}
  \begin{center}
    \blue{Homeomorphically} Irreducible Trees of size $n = 10$
  \end{center}
\end{frame}
%%%%%%%%%%%%%%%