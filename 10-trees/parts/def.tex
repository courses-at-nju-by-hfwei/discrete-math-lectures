% def.tex

%%%%%%%%%%%%%%%
\begin{frame}{}
  \begin{definition}[Tree (树)]
    A \red{tree} is a \violet{connected} \purple{acyclic} \blue{undirected} graph.
  \end{definition}

  \pause
  \begin{columns}
    \column{0.50\textwidth}
      \fig{width = 0.40\textwidth}{figs/tree-wiki}
    \column{0.50\textwidth}
      \fig{width = 0.90\textwidth}{figs/CaterpillarTree}
  \end{columns}

  \pause
  \vspace{0.30cm}
  \fig{width = 0.70\textwidth}{figs/trees}

  \pause
  \begin{definition}[Forest (森林)]
    A \red{forest} is a \purple{acyclic} \blue{undirected} graph.
  \end{definition}
\end{frame}
%%%%%%%%%%%%%%%

%%%%%%%%%%%%%%%
\begin{frame}{}
  \begin{definition}[Internal Vertex (内部顶点); Leaf (叶子)]
    In a tree $T$ with \blue{$\ge 2$} vertices, for a vertex $v$ in $T$, if
    \[
      \deg(v) = 1
    \]
    then $v$ is called a \red{leaf};
    otherwise, $v$ is an \red{internal vertex}.
  \end{definition}

  \pause
  \vspace{0.30cm}
  \fig{width = 0.50\textwidth}{figs/CaterpillarTree}

  \pause
  \begin{lemma}
    Any tree $T$ with \blue{$\ge 2$} vertices contains \red{$\ge 1$} leaf.
  \end{lemma}

  \pause
  \begin{center}
    Otherwise, $\forall v \in V.\; \degree(v) \ge 2 \implies T \text{ has cycles}$.
  \end{center}
\end{frame}
%%%%%%%%%%%%%%%

%%%%%%%%%%%%%%%
\begin{frame}{}
  \begin{lemma}
    Any tree $T$ with \blue{$\ge 2$} vertices contains \red{$\ge 2$} leaves.
  \end{lemma}

  \pause
  \vspace{0.30cm}
  \[
    \sum_{v \in V} \degree(v) = 2n - 2
  \]

  \pause
  \vspace{0.50cm}
  \begin{center}
    Consider the two endpoints of any \red{maximal} (nontrivial) path in $T$.
    \pause \\[5pt]
    They are leaves of $T$.
  \end{center}
\end{frame}
%%%%%%%%%%%%%%%

%%%%%%%%%%%%%%%
\begin{frame}{}
  \begin{lemma}
    Deleting a \red{leaf} from a tree $T$ with $n$ vertices
    produces a tree with $n-1$ vertices.
  \end{lemma}

  \pause
  \vspace{0.30cm}
  \begin{center}
    \fig{width = 0.40\textwidth}{figs/G-leaf}

    \vspace{0.20cm}
    $G' = G - v$ is \violet{connected} and \purple{acyclic}.

    \pause
    \vspace{0.50cm}
    \blue{A leaf does \red{\it not} belong to any paths connecting two other vertices.}
  \end{center}
\end{frame}
%%%%%%%%%%%%%%%

%%%%%%%%%%%%%%%
\begin{frame}{}
  \begin{definition}[Irreducible Tree]
    An \red{irreducible tree} is a tree $T$ where
    \[
      \forall v \in V(T).\; \degree(v) \neq 2.
    \]
  \end{definition}

  \pause
  \vspace{0.30cm}
  \begin{columns}
    \column{0.50\textwidth}
      \fig{width = 0.80\textwidth}{figs/trees-10}
    \column{0.50\textwidth}
      \pause
      \fig{width = 0.80\textwidth}{figs/trees-movie}
  \end{columns}

  \pause
  \vspace{0.30cm}
  \begin{center}
    \blue{Homeomorphically} Irreducible Trees of size $n = 10$
  \end{center}
\end{frame}
%%%%%%%%%%%%%%%

%%%%%%%%%%%%%%%
\begin{frame}{}
  \begin{theorem}[\cyan{(We call it)} Tree Theorem]
    Let $T$ be an undirected graph with $n$ vertices. \\[3pt]
    Then the following statements are \red{equivalent}:
    \begin{enumerate}[(1)]
      \setlength{\itemsep}{6pt}
      \item $T$ is a tree;
      \item $T$ is acyclic, and has $n-1$ edges;
      \item $T$ is connected, and has $n-1$ edges;
      \item $T$ is connected, and each edge is a \cyan{bridge};
      \item Any two vertices of $T$ are connected by exactly one path;
      \item $T$ is acyclic, but the addition of any edge creates exactly one cycle.
    \end{enumerate}
  \end{theorem}
\end{frame}
%%%%%%%%%%%%%%%

%%%%%%%%%%%%%%%
\begin{frame}{}
  \fig{width = 0.30\textwidth}{figs/keep-calm-prove-it}
\end{frame}
%%%%%%%%%%%%%%%

%%%%%%%%%%%%%%%
\begin{frame}{}
\end{frame}
%%%%%%%%%%%%%%%

%%%%%%%%%%%%%%%
\begin{frame}{}
\end{frame}
%%%%%%%%%%%%%%%

%%%%%%%%%%%%%%%
\begin{frame}{}
\end{frame}
%%%%%%%%%%%%%%%

%%%%%%%%%%%%%%%
\begin{frame}{}
\end{frame}
%%%%%%%%%%%%%%%

%%%%%%%%%%%%%%%
\begin{frame}{}
\end{frame}
%%%%%%%%%%%%%%%
