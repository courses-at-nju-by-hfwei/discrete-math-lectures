% def.tex

%%%%%%%%%%%%%%%
\begin{frame}{}
  \begin{definition}[Tree (树)]
    A \red{tree} is a \violet{connected} \purple{acyclic} \blue{undirected} graph.
  \end{definition}

  \pause
  \vspace{0.80cm}
  \begin{definition}[Forest (森林)]
    A \red{forest} is a \purple{acyclic} \blue{undirected} graph.
  \end{definition}
\end{frame}
%%%%%%%%%%%%%%%

%%%%%%%%%%%%%%%
\begin{frame}{}
  \begin{columns}
    \column{0.50\textwidth}
      \fig{width = 0.40\textwidth}{figs/tree-wiki}
    \column{0.50\textwidth}
      \fig{width = 0.90\textwidth}{figs/CaterpillarTree}
  \end{columns}

  \pause
  \vspace{0.80cm}
  \fig{width = 0.70\textwidth}{figs/trees}
\end{frame}
%%%%%%%%%%%%%%%

%%%%%%%%%%%%%%%
\begin{frame}{}
  \fig{width = 0.40\textwidth}{figs/Cayley-Graph}
  \begin{center}
    \teal{Cayley Graph} (4-regular tree)
  \end{center}
\end{frame}
%%%%%%%%%%%%%%%

%%%%%%%%%%%%%%%
\begin{frame}{}
  \begin{definition}[Internal Vertex (内部顶点); Leaf (叶子)]
    In a tree $T$ with \blue{$\ge 2$} vertices, for a vertex $v$ in $T$, if
    \[
      \deg(v) = 1
    \]
    then $v$ is called a \red{leaf};
    otherwise, $v$ is an \red{internal vertex}.
  \end{definition}

  \pause
  \vspace{0.30cm}
  \fig{width = 0.50\textwidth}{figs/CaterpillarTree}

  \pause
  \begin{lemma}
    Any tree $T$ with \blue{$\ge 2$} vertices contains \red{$\ge 1$} leaf.
  \end{lemma}

  \pause
  \begin{center}
    Otherwise, $\forall v \in V.\; \degree(v) \ge 2 \implies T \text{ has cycles}$.
  \end{center}
\end{frame}
%%%%%%%%%%%%%%%

%%%%%%%%%%%%%%%
\begin{frame}{}
  \begin{lemma}
    Any tree $T$ with \blue{$\ge 2$} vertices contains \red{$\ge 2$} leaves.
  \end{lemma}

  \pause
  \vspace{0.30cm}
  \[
    \sum_{v \in V} \degree(v) = 2n - 2
  \]

  \pause
  \vspace{0.50cm}
  \begin{center}
    Consider the two endpoints of any \red{maximal} (nontrivial) path in $T$.
    \pause \\[5pt]
    They are leaves of $T$.
  \end{center}
\end{frame}
%%%%%%%%%%%%%%%

%%%%%%%%%%%%%%%
\begin{frame}{}
  \begin{lemma}
    Deleting a \red{leaf} from a tree $T$ with $n$ vertices
    produces a tree with $n-1$ vertices.
  \end{lemma}

  \pause
  \vspace{0.30cm}
  \begin{center}
    \fig{width = 0.40\textwidth}{figs/G-leaf}

    \vspace{0.20cm}
    $G' = G - v$ is \violet{connected} and \purple{acyclic}.

    \pause
    \vspace{0.50cm}
    \blue{A leaf does \red{\it not} belong to any paths connecting two other vertices.}

    \pause
    \vspace{0.80cm}
    \red{This lemma can be used in induction for trees!}
  \end{center}
\end{frame}
%%%%%%%%%%%%%%%

%%%%%%%%%%%%%%%
\begin{frame}{}
  \begin{theorem}[\cyan{(We call it)} Characterization of Trees]
    Let $T$ be an undirected graph with $n$ vertices. \\[3pt]
    Then the following statements are \red{equivalent}:
    \begin{enumerate}[(1)]
      \setlength{\itemsep}{6pt}
      \item $T$ is a tree;
      \item $T$ is acyclic, and has $m = n-1$ edges;
      \item $T$ is connected, and has $m = n-1$ edges;
      \item $T$ is connected, and each edge is a \cyan{bridge};
      \item Any two vertices of $T$ are connected by exactly one path;
      \item $T$ is acyclic, but the addition of any edge creates exactly one cycle.
    \end{enumerate}
  \end{theorem}

  \pause
  \[
    \red{(1)} \implies (2) \implies (3) \implies (4) \implies (5) \implies (6) \implies \red{(1)}
  \]
\end{frame}
%%%%%%%%%%%%%%%

%%%%%%%%%%%%%%%
\begin{frame}{}
  \begin{theorem}[Characterization of Trees]
    \begin{enumerate}[(1)]
        \setlength{\itemsep}{6pt}
        \item $T$ is a tree;
        \item $T$ is acyclic, and has $m = n-1$ edges.
    \end{enumerate}
  \end{theorem}

  \pause
  \begin{center}
    \red{By induction on the number $n$ of vertices of trees.}
  \end{center}

  \begin{description}[Induction Hypothesis:]
    \setlength{\itemsep}{8pt}
    \item[Basis Step:]
      \uncover<3->{$n = 1$. $m = 0 = n - 1$.}
    \item[Induction Hypothesis:]
      \uncover<4->{Any trees with $n-1$ vertices has $n-2$ edges.}
    \item[Induction Step:]
      \uncover<5->{Consider a tree $T$ with $n \ge 2$ vertices. \\[5pt]}
      \uncover<6->{\blue{$T$ has a leaf $v$.} \\[5pt]}
      \uncover<7->{For \red{$T' = T - v$}, $m(T') = (n-1)-1 = n-2$.}
      \uncover<8->{
        \[
          m(T) = (n-2) + 1 = n - 1.
        \]
      }
  \end{description}
\end{frame}
%%%%%%%%%%%%%%%

%%%%%%%%%%%%%%%
\begin{frame}{}
  \begin{theorem}[Characterization of Trees]
    \begin{enumerate}[(1)]
        \setcounter{enumi}{1}
        \setlength{\itemsep}{6pt}
        \item $T$ is acyclic, and has $n-1$ edges;
        \item $T$ is connected, and has $n-1$ edges.
    \end{enumerate}
  \end{theorem}

  \pause
  \vspace{0.30cm}
  \begin{center}
    \red{By Contradiction.}

    \pause
    \vspace{0.30cm}
    Suppose that $T$ is {\it disconnected}.

    \pause
    \vspace{0.30cm}
    \blue{$T$ is a forest, consisting of $k \ge 2$ trees $T_{1}, T_{2}, \dots$.}

    \pause
    \vspace{0.30cm}
    \red{By (2), for each $T_{i}$, $m(T_{i}) = n(T_{i}) - 1$.}

    \pause
    \[
      m(T) = \sum_{i=1}^{k} m(T_{i}) = n - k \neq n - 1.
    \]
  \end{center}
\end{frame}
%%%%%%%%%%%%%%%

%%%%%%%%%%%%%%%
\begin{frame}{}
  \begin{theorem}[Characterization of Trees]
    \begin{enumerate}[(1)]
      \setcounter{enumi}{2}
      \setlength{\itemsep}{6pt}
      \item $T$ is connected, and has $n-1$ edges;
      \item $T$ is connected, and each edge is a \cyan{bridge}.
    \end{enumerate}
  \end{theorem}

  \pause
  \fig{width = 0.30\textwidth}{figs/bridge}

  \begin{definition}[Bridge (桥)]
    A \red{bridge} of a graph $G$ is an \blue{edge $e$} such that
    \[
      c(G - \blue{e}) > c(G).
    \]
  \end{definition}
\end{frame}
%%%%%%%%%%%%%%%

%%%%%%%%%%%%%%%
\begin{frame}{}
  \begin{theorem}[Characterization of Trees]
    \begin{enumerate}[(1)]
      \setcounter{enumi}{2}
      \setlength{\itemsep}{6pt}
      \item $T$ is connected, and has $n-1$ edges;
      \item $T$ is connected, and each edge is a \cyan{bridge}.
    \end{enumerate}
  \end{theorem}

  \pause
  \vspace{0.30cm}
  \begin{center}
    Consider any edge \blue{$e$} of $T$.
  \end{center}

  \pause
  \vspace{-0.20cm}
  \[
    m(T - \blue{e}) = (n - 1) - 1 = n - 2.
  \]

  \pause
  \vspace{-0.20cm}
  \begin{center}
    $T - \blue{e}$ must be disconnected.
  \end{center}
\end{frame}
%%%%%%%%%%%%%%%

%%%%%%%%%%%%%%%
\begin{frame}{}
  \begin{theorem}[Characterization of Trees]
    \begin{enumerate}[(1)]
      \setcounter{enumi}{3}
      \setlength{\itemsep}{6pt}
      \item $T$ is connected, and each edge is a \cyan{bridge};
      \item Any two vertices of $T$ are connected by exactly one path.
    \end{enumerate}
  \end{theorem}

  \pause
  \vspace{0.30cm}
  \begin{center}
    Consider any two vertices $u$ and $v$.

    \pause
    \vspace{0.30cm}
    $T$ is connected $\implies$ $u$ and $v$ are connected by $\ge 1$ path.

    \pause
    \vspace{0.60cm}
    \red{If $u$ and $v$ are connected by two paths}, \\[3pt]
    the edges on these two paths are not bridges.
  \end{center}
\end{frame}
%%%%%%%%%%%%%%%

%%%%%%%%%%%%%%%
\begin{frame}{}
  \begin{theorem}[Characterization of Trees]
    \begin{enumerate}[(1)]
      \setcounter{enumi}{4}
      \setlength{\itemsep}{6pt}
      \item Any two vertices of $T$ are connected by exactly one path;
      \item $T$ is acyclic, but the addition of any edge creates exactly one cycle.
    \end{enumerate}
  \end{theorem}

  \pause
  \vspace{0.30cm}
  \begin{center}
    \red{If $T$ has a cycle $C$}, \\[3pt]
    any two vertices in $C$ is connected by $\ge 2$ paths.

    \pause
    \vspace{0.60cm}
    Consider the addition of edge $\set{u, v}$ to $T$. \\[3pt]
    It creates a cycle, consisting of $\set{u, v}$ and the path from $u$ to $v$.

    \pause
    \begin{lemma}
      If two distinct cycles of a graph $G$ share \blue{a common edge $e$}, \\[3pt]
      then $G$ has a cycle that does \purple{not} contain $e$.
    \end{lemma}
  \end{center}
\end{frame}
%%%%%%%%%%%%%%%

%%%%%%%%%%%%%%%
\begin{frame}{}
  \begin{theorem}[Characterization of Trees]
    \begin{enumerate}[(1)]
      \setcounter{enumi}{5}
      \setlength{\itemsep}{6pt}
      \item $T$ is acyclic, but the addition of any edge creates exactly one cycle;
      \setcounter{enumi}{0}
      \item $T$ is a tree.
    \end{enumerate}
  \end{theorem}

  \pause
  \vspace{0.30cm}
  \begin{center}
    \red{Suppose that $T$ is disconnected.}

    \pause
    \vspace{0.60cm}
    $T$ is a forest, consisting of $\ge 2$ trees $T_{1}, T_{2}, \dots$

    \pause
    \vspace{0.30cm}
    Choose $u \in V(T_{1})$, $v \in V(T_{2})$.

    \vspace{0.30cm}
    $T + \set{u, v}$ does \red{not} create cycles.
  \end{center}
\end{frame}
%%%%%%%%%%%%%%%