% mst.tex

%%%%%%%%%%%%%%%
\begin{frame}{}
  \fig{width = 0.50\textwidth}{figs/st-grid}

  \begin{center}
    \red{Spanning Trees} (trees \cyan{in} graphs)
  \end{center}
\end{frame}
%%%%%%%%%%%%%%%

%%%%%%%%%%%%%%%
\begin{frame}{}
  \begin{definition}[Subgraph (子图)]
    A graph $S$ is a \red{subgraph} of $G$ if
    \[
      \blue{V(S) \subseteq V(G)} \land \cyan{E(S) \subseteq E(G)}
      \land \red{\bigcup E(S) \subseteq V(S)}
    \]
  \end{definition}

  \pause
  \fig{width = 0.45\textwidth}{figs/st-example}

  \pause
  \vspace{-0.30cm}
  \begin{definition}[Induced Subgraph (诱导子图)]
    A graph $S$ is an \red{induced subgraph} of $G$ if
    $S$ is a \blue{subgraph} of $G$ \\ such that
    \[
      \set{\set{u, v} \in E(G) \mid u \in V(S), v \in V(S)} \subseteq E(S).
    \]
  \end{definition}
\end{frame}
%%%%%%%%%%%%%%%

%%%%%%%%%%%%%%%
\begin{frame}{}
  \begin{definition}[Spanning Tree (生成树)]
    A \red{spanning tree} $T$ of an \cyan{undirected} graph $G$ is a \purple{subgraph} \\
    that is a \blue{tree} with all vertices of $G$.
  \end{definition}

  \pause
  \vspace{0.50cm}
  \begin{columns}
    \column{0.50\textwidth}
      \fig{width = 0.60\textwidth}{figs/st-grid}
    \column{0.50\textwidth}
      \pause
      \fig{width = 0.70\textwidth}{figs/st-4}
  \end{columns}
\end{frame}
%%%%%%%%%%%%%%%

%%%%%%%%%%%%%%%
\begin{frame}{}
  \begin{theorem}
    Every connected undirected graph $G$ \blue{admits} a spanning tree.
  \end{theorem}

  \pause
  \vspace{0.50cm}
  \begin{center}
    Repeatedly \blue{deleting edges in cycles} until the graph is \purple{acyclic}.

    \pause
    \vspace{0.30cm}
    The remaining graph is a spanning tree of $G$.
  \end{center}
\end{frame}
%%%%%%%%%%%%%%%

%%%%%%%%%%%%%%%
\begin{frame}{}
  \begin{definition}[Minimum Spanning Tree (MST; 最小生成树)]
    A \red{minimum spanning tree} $T$ of an \cyan{edge-weighted} undirected graph $G$ \\
    is a spanning tree with \purple{minimum} total weight of edges.
  \end{definition}

  \pause
  \vspace{0.50cm}
  \fig{width = 0.45\textwidth}{figs/mst-wiki}

  \pause
  \begin{center}
    \violet{\large Existence? \qquad Uniqueness? \qquad Algorithms?}
  \end{center}
\end{frame}
%%%%%%%%%%%%%%%

%%%%%%%%%%%%%%%
\begin{frame}{}
  \begin{theorem}
    Every weighted connected undirected graph $G$ \blue{admits} a minimum spanning tree.
  \end{theorem}

  \pause
  \fig{width = 0.30\textwidth}{figs/mst-not-unique}
\end{frame}
%%%%%%%%%%%%%%%

%%%%%%%%%%%%%%%
\begin{frame}{}
  \fig{width = 0.40\textwidth}{figs/Kruskal}

  \begin{center}
    \teal{Joseph Kruskal ($1928 \sim 2010$)}
  \end{center}
\end{frame}
%%%%%%%%%%%%%%%

%%%%%%%%%%%%%%%
\begin{frame}{}
  \begin{center}
    Repeatedly adding \red{the next lowest-weight} edge \\[5pt]
    that will \blue{not form a cycle}
    until $n-1$ edges are added.
  \end{center}

  \pause
  \fig{width = 0.60\textwidth}{figs/mst-example}
\end{frame}
%%%%%%%%%%%%%%%

%%%%%%%%%%%%%%%
\begin{frame}{}
  \fig{width = 0.40\textwidth}{figs/Prim}

  \begin{center}
    \teal{Robert C. Prim ($1921 \sim \ $)}
  \end{center}
\end{frame}
%%%%%%%%%%%%%%%

%%%%%%%%%%%%%%%
\begin{frame}{}
  \begin{center}
    Repeatedly adding the \red{cheapest} possible edge \\[5pt]
    from \blue{the partially built tree} to another vertex, \\[5pt]
    until $n-1$ edges are added.
  \end{center}

  \pause
  \fig{width = 0.60\textwidth}{figs/mst-example}
\end{frame}
%%%%%%%%%%%%%%%

%%%%%%%%%%%%%%%
% cut-property.tex

%%%%%%%%%%%%
\begin{frame}{}
  \begin{center}
    {\teal{\Large Cut Property}}
  \end{center}
\end{frame}
%%%%%%%%%%%%%

%%%%%%%%%%%%%
\begin{frame}{}
  \begin{exampleblock}{Cut Property (Version I)}
    \begin{description}
      \item[$X:$] A part of some MST $T_1$ of $G$
      \item[$(S, V \setminus S):$] A \teal{\it cut} such that $X$ does \teal{\it not} cross $(S, V \setminus S)$
    ­ \item[$e:$] \red{A} lightest edge across $(S, V \setminus S)$
    \end{description}

    \pause
    \vspace{0.30cm}
    \begin{center}
      {Then \blue{$X \cup \set{e}$} is a part of \red{some} MST $T_2$ of $G$.}
    \end{center}
  \end{exampleblock}

  \pause
  \vspace{0.60cm}
  \begin{center}
    {\red{\large Correctness of Prim's and Kruskal's algorithms.}}
  \end{center}
\end{frame}
%%%%%%%%%%%%%

%%%%%%%%%%%%%
\begin{frame}{}
  \begin{center}
    {\teal{\large By Exchange Argument.}}
  \end{center}

  \pause
  \vspace{0.30cm}
  \fig{width = 0.50\textwidth}{figs/cut-property}

  \pause
  \[
    T' = \underbrace{\underbrace{T}_{\blue{X \subseteq T}} + \set{e}}_{\red{\text{if } e \;\notin\; T}} - \set{e'}
  \]

  \pause
  \begin{center}
    {``a'' $\to$ ``the'' \red{$\implies$} ``some'' $\to$ ``all''}
  \end{center}
\end{frame}
%%%%%%%%%%%%%

%%%%%%%%%%%%%
\begin{frame}{}
  \begin{exampleblock}{Cut Property (Version II)}
    \begin{center}
      A cut $(S, V \setminus S)$ \\[6pt]
      Let $e = (u,v)$ be \red{\emph{a}} lightest edge across $(S, V \setminus S)$
      \[
        \red{\exists \text{ MST $T$ of } G: e \in T}
      \]
    \end{center}
  \end{exampleblock}

  \fig{width = 0.40\textwidth}{figs/cut-property-no-name}

  \pause
  \vspace{-0.50cm}
  \[
    T' = \underbrace{T + \set{e}}_{\red{\text{if } e \;\notin\; T}} - \set{e'}
  \]

  \pause
  \begin{center}
    {``a'' $\to$ ``the'' \red{$\implies$} ``$\exists$'' $\to$ ``$\forall$''}
  \end{center}
\end{frame}
%%%%%%%%%%%%%
% cycle-property.tex

%%%%%%%%%%%%%
\begin{frame}{}
  \begin{theorem}[Cycle Property]
    \begin{itemize}
      \item Let $C$ be any cycle in connected undirected $G$
      \item Let $e = (u,v)$ be \red{\it a} maximum-weight edge in $C$
    \end{itemize}

    \begin{center}
      {Then $\red{\exists} \text{ MST } T \text{ of } G: e \notin T$.}
    \end{center}
  \end{theorem}

  \pause
  \begin{center}
    Choose any MST $T$ of $G$. \pause \\[3pt]
    If $e \notin T$, we are done. Otherwise, construct $T'$.
  \end{center}

  \pause
  \fig{width = 0.50\textwidth}{figs/cycle-property}

  \pause
  \vspace{-0.30cm}
  \[
    T' = \underbrace{T - \set{e}}_{\red{\text{if } e \;\in\; T}} +\; \set{e'}
  \]
\end{frame}
%%%%%%%%%%%%%

%%%%%%%%%%%%%
\begin{frame}{}
  \begin{theorem}[Cycle Property]
    \begin{itemize}
      \item Let $C$ be any cycle in connected undirected $G$
      \item Let $e = (u,v)$ be \red{\it a} maximum-weight edge in $C$
    \end{itemize}

    \begin{center}
      {Then $\red{\exists} \text{ MST } T \text{ of } G: e \notin T$.}
    \end{center}
  \end{theorem}

  \pause
  \vspace{0.30cm}
  \begin{center}
    {``a'' $\to$ ``the'' \red{$\implies$} ``$\exists$'' $\to$ ``$\forall$''}
  \end{center}
\end{frame}
%%%%%%%%%%%%%

%%%%%%%%%%%%%
% \begin{frame}{}
%   \begin{proof}
%     Basic idea: pick any MST $T$ of $G$
%     \begin{itemize}
%       \item $e \notin T$
%       \item $e \in T \Rightarrow e \notin T'$
% 	\begin{itemize}
% 	  \item $T - \set{e}$ $\Rightarrow$ $(S, V \setminus S)$
% 	  \item $\exists e' = (u', v') \in C$ ($e' \in P_{u,v}$) across the cut
% 	  \item \textcolor{red}{$T' = T - \set{e} + \set{e'}$}: spanning tree
% 	  \item $w(e') \leq w(e) \Rightarrow w(T') \leq w(T) \Rightarrow w(T') = w(T)$
% 	\end{itemize}
%     \end{itemize}
%   \end{proof}
% \end{frame}
%%%%%%%%%%%%%
%%%%%%%%%%%%%%%

%%%%%%%%%%%%%%%
% \begin{frame}{}
%   \fig{width = 0.40\textwidth}{figs/Kruskal}
%
%   \begin{center}
%     \teal{Joseph Kruskal ($1928 \sim 2010$)}
%   \end{center}
% \end{frame}
%%%%%%%%%%%%%%%

%%%%%%%%%%%%%
% \begin{frame}{}
%   \begin{exampleblock}{Anti-Kruskal Algorithm}
%     \centerline{\href{https://en.wikipedia.org/wiki/Reverse-delete_algorithm}{Reverse-delete algorithm \teal{\small (wiki; clickable)}}}
%   \end{exampleblock}
%
%   \pause
%   \vspace{0.30cm}
%   \begin{center}
%     {\purple{Cycle Property}}
%   \end{center}
%
%   \[
%     \red{T \subseteq F \implies \exists\; T': T' \subseteq F - \set{e}}
%   \]
%
%   \pause
%   \vspace{0.20cm}
%   \begin{center}
%     {\it ``On the Shortest Spanning Subtree of a Graph \\
%     and the Traveling Salesman Problem''} \\
%     \hfill --- \red{Kruskal}, $1956$.
%   \end{center}
% \end{frame}
%%%%%%%%%%%%%

%%%%%%%%%%%%%%%
% \begin{frame}{}
%   \fig{width = 0.60\textwidth}{figs/mst-example}
% \end{frame}
%%%%%%%%%%%%%%%

%%%%%%%%%%%%%%%
% \begin{frame}{}
%   \fig{width = 0.40\textwidth}{figs/Boruvka}
%
%   \begin{center}
%     \teal{Otakar Borůvka ($1899 \sim 1995$)}
%   \end{center}
% \end{frame}
%%%%%%%%%%%%%%%

%%%%%%%%%%%%%%%
% \begin{frame}{}
%   \fig{width = 0.60\textwidth}{figs/mst-example}
% \end{frame}
%%%%%%%%%%%%%%%

%%%%%%%%%%%%%%%
% unique-mst.tex

%%%%%%%%%%%%
\begin{frame}{}
  \begin{theorem}[Uniqueness of MST]
    Let $G$ be an edge-weighted undirected graph. \\
    If each edge has a \purple{distinct} weight,
    then there is a \red{unique} {\it MST} of $G$.
  \end{theorem}

  \pause
  \vspace{0.50cm}
  \centerline{\red{By Contradiction.}}

  \pause
  \[
    \exists \text{ MSTs } T_1 \neq T_2
  \]

  \pause
  \[
    \Delta E = \set{e \mid e \in T_1 \setminus T_2 \lor e \in T_2 \setminus T_1}
  \]

  \pause
  \[
    e = \min \Delta E
  \]

  \pause
  \[
    \text{Suppose that } e \in T_1 \setminus T_2 \;\text{\small \it (w.l.o.g)}
  \]
\end{frame}
%%%%%%%%%%%%%

%%%%%%%%%%%%
\begin{frame}{}
  \fig{width = 0.50\textwidth}{figs/mst-unique}

  \pause
  \vspace{-0.30cm}
  \[
    T_2 + \set{e} \implies C
  \]

  \pause
  \vspace{-0.30cm}
  \[
    \exists (e' \in C) \red{\;\notin\; T_1} \pause \implies e' \in T_{2} \setminus T_{1} \implies e' \in \Delta E \pause \implies w(e') > w(e)
  \]

  \pause
  \vspace{-0.50cm}
  \[
    T' = T_{2} + \set{e} - \set{e'} \implies w(T') < w(T_{2})
  \]
\end{frame}
%%%%%%%%%%%%%

%%%%%%%%%%%%%
\begin{frame}{}
  \begin{exampleblock}{Condition for Uniqueness of MST}
    \centerline{Unique MST $\centernot\implies$ Distinct weights}
  \end{exampleblock}

  \pause
  \fig{width = 0.30\textwidth}{figs/unique-mst-partition}
\end{frame}
%%%%%%%%%%%%%
%%%%%%%%%%%%%%%