% mst.tex

%%%%%%%%%%%%%%%
\begin{frame}{}
  \begin{definition}[Spanning Tree (生成树)]
    A \red{spanning tree} $T$ of an \cyan{undirected} graph $G$
    is a \purple{subgraph} that is a \blue{tree} with all vertices of $G$.
  \end{definition}

  \pause
  \vspace{0.50cm}
  \begin{columns}
    \column{0.50\textwidth}
      \fig{width = 0.60\textwidth}{figs/st-grid}
    \column{0.50\textwidth}
      \fig{width = 0.70\textwidth}{figs/st-4}
  \end{columns}
\end{frame}
%%%%%%%%%%%%%%%

%%%%%%%%%%%%%%%
\begin{frame}{}
  \begin{definition}[Subgraph (子图)]
  \end{definition}

  \pause
  \vspace{0.50cm}
  \begin{definition}[Induced Subgraph (诱导子图)]
  \end{definition}
\end{frame}
%%%%%%%%%%%%%%%

%%%%%%%%%%%%%%%
\begin{frame}{}
  \begin{theorem}
    Every connected undirected graph $G$ admits a spanning tree.
  \end{theorem}

  \pause
  \vspace{0.50cm}
  \begin{center}
    Repeatedly \blue{deleting vertices in cycles} until the graph is \purple{acyclic}.
  \end{center}
\end{frame}
%%%%%%%%%%%%%%%

%%%%%%%%%%%%%%%
\begin{frame}{}
  \begin{definition}[Minimum Spanning Tree (MST; 最小生成树)]
    A \red{minimum spanning tree} $T$ of an \cyan{edge-weighted} undirected graph $G$ \\
    is a spanning tree with \purple{minimum} total weight of edges.
  \end{definition}

  \pause
  \vspace{0.50cm}
  \fig{width = 0.45\textwidth}{figs/mst-wiki}

  \pause
  \begin{center}
    Existence? \qquad Uniqueness? \qquad Algorithms?
  \end{center}
\end{frame}
%%%%%%%%%%%%%%%

%%%%%%%%%%%%%%%
\begin{frame}{}
  \begin{theorem}
    Every weighted connected undirected graph $G$ admits a minimum spanning tree.
  \end{theorem}

  \pause
  \fig{width = 0.30\textwidth}{figs/mst-not-unique}
\end{frame}
%%%%%%%%%%%%%%%

%%%%%%%%%%%%%%%
\begin{frame}{}
  \fig{width = 0.40\textwidth}{figs/Kruskal}

  \begin{center}
    \teal{Joseph Kruskal ($1928 \sim 2010$)}
  \end{center}
\end{frame}
%%%%%%%%%%%%%%%

%%%%%%%%%%%%%%%
\begin{frame}{}
\end{frame}
%%%%%%%%%%%%%%%

%%%%%%%%%%%%%%%
\begin{frame}{}
  \fig{width = 0.40\textwidth}{figs/Prim}

  \begin{center}
    \teal{Robert C. Prim ($1921 \sim \ $)}
  \end{center}
\end{frame}
%%%%%%%%%%%%%%%

%%%%%%%%%%%%%%%
\begin{frame}{}
\end{frame}
%%%%%%%%%%%%%%%

%%%%%%%%%%%%%%%
\begin{frame}{}
  \fig{width = 0.40\textwidth}{figs/Boruvka}

  \begin{center}
    \teal{Otakar Borůvka ($1899 \sim 1995$)}
  \end{center}
\end{frame}
%%%%%%%%%%%%%%%

%%%%%%%%%%%%%%%
\begin{frame}{}
\end{frame}
%%%%%%%%%%%%%%%

%%%%%%%%%%%%%%%
\begin{frame}{}
  \fig{width = 0.40\textwidth}{figs/Kruskal}

  \begin{center}
    \teal{Joseph Kruskal ($1928 \sim 2010$)}
  \end{center}
\end{frame}
%%%%%%%%%%%%%%%

%%%%%%%%%%%%%%%
\begin{frame}{}
\end{frame}
%%%%%%%%%%%%%%%

%%%%%%%%%%%%%%%
\begin{frame}{}
\end{frame}
%%%%%%%%%%%%%%%

%%%%%%%%%%%%%%%
\begin{frame}{}
\end{frame}
%%%%%%%%%%%%%%%

%%%%%%%%%%%%%%%
\begin{frame}{}
\end{frame}
%%%%%%%%%%%%%%%