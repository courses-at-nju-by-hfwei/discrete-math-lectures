% set-def.tex

%%%%%%%%%%%%%%%
\begin{frame}{}
  \begin{definition}[集合]
    \red{\bf 集合}就是任何一个\blue{有明确定义的}对象的\blue{整体}。
  \end{definition}

  \pause
  \vspace{0.30cm}
  \begin{columns}
    \column{0.50\textwidth}
      \fig{width = 0.45\textwidth}{figs/cantor-book}
    \column{0.50\textwidth}
      \fig{width = 0.40\textwidth}{figs/Cantor}
      {\centerline{\teal{Georg Cantor (1845--1918)}}}
  \end{columns}

  \pause
  \begin{definition}[集合]
    我们将\red{\bf 集合}理解为任何将\blue{我们思想中那些确定而彼此独立的对象}放在一起而形成的\blue{聚合}。
  \end{definition}
\end{frame}
%%%%%%%%%%%%%%%

%%%%%%%%%%%%%%%
\begin{frame}{}
  \begin{theorem}[概括原则]
    对于任意性质/谓词 $P(x)$, 都存在一个集合 $X$:
    \[
      X = \set{x \mid P(x)}
    \]
  \end{theorem}
\end{frame}
%%%%%%%%%%%%%%%

%%%%%%%%%%%%%%%
\begin{frame}{}
  \[
    A = \set{2, 3, 5, 7}
  \]

  \[
    B = \set{x \mid x < 10 \land \textrm{Prime}(x)}
  \]

  \[
    C = \set{x \mid x \;\text{是方程}\; x^{4} - 17x^3 + 101x^2 - 247x + 210 = 0 \;\text{的根}}
  \]
\end{frame}
%%%%%%%%%%%%%%%

%%%%%%%%%%%%%%%
\begin{frame}{}
  \[
    \set{2n \mid n \in \N}
  \]

  \pause
  \[
    \set{p/q \mid p, q \in \Z, q \neq 0}
  \]

  \pause
  \[
    \set{(t, 2t+1) \mid t \in \Z}
  \]
\end{frame}
%%%%%%%%%%%%%%%

%%%%%%%%%%%%%%%
\begin{frame}{}
  \begin{definition}[外延性原理 (Extensionality)]
    两个集合相等 $(A = B)$ 当且仅当它们包含相同的元素。
    % \[
    %   \forall A.\; \forall B.\;
    %     \Big(\big(\forall x.\; (x \in A \leftrightarrow x \in B)\big)
    %       \leftrightarrow A = B \Big)
    % \]
  \end{definition}

  \vspace{0.50cm}
  \begin{center}
    集合完全由它的元素决定
  \end{center}
\end{frame}
%%%%%%%%%%%%%%%

%%%%%%%%%%%%%%%
\begin{frame}{}
  \begin{definition}[子集]
    设 $A$、$B$ 是任意两个集合。\\[8pt]

    $A \subseteq B$ 表示 $A$ 是 $B$ 的\red{子集} (subset):
    \[
      A \subseteq B \iff \forall x \in A.\; (x \in A \to x \in B)
    \]

    $A \subset B$ 表示 $A$ 是 $B$ 的\red{真子集} (proper subset):
    \[
      A \subset B \iff A \subseteq B \land A \neq B
    \]
  \end{definition}

  \[
    \set{1, 2} \subseteq \set{1, 2, 3}
    \qquad \set{1, 2} \subset \set{1, 2, 3}
    \qquad \set{1, 4} \not\subseteq \set{1, 2, 3}
  \]
\end{frame}
%%%%%%%%%%%%%%%

%%%%%%%%%%%%%%%
\begin{frame}{}
  \begin{theorem}
    两个集合相等当且仅当它们互为子集。
    \[
      A = B \iff A \subseteq B \land B \subseteq A
    \]
  \end{theorem}

  \vspace{0.30cm}
  \begin{center}
    这是证明两个集合相等的基本方法
  \end{center}
\end{frame}
%%%%%%%%%%%%%%%