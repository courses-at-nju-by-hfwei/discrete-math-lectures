% ops-powerset.tex

%%%%%%%%%%%%%%%
\begin{frame}{}
  \begin{center}
    {\Large 集合的运算 (III)}
  \end{center}

  \[
    \ps{X}
  \]
\end{frame}
%%%%%%%%%%%%%%%

%%%%%%%%%%%%%%%
\begin{frame}{}
  \begin{definition}[幂集 (Powerset)]
    \[
      \ps{A} = \Bset{X \mid X \subseteq A}
    \]
  \end{definition}

  \[
    A = \set{1, 2, 3}
  \]
  \[
    \ps{A} = \set{\emptyset,
      \set{1}, \set{2}, \set{3},
      \set{1, 2}, \set{1, 3}, \set{2, 3},
      \set{1, 2, 3}}
  \]
\end{frame}
%%%%%%%%%%%%%%%

%%%%%%%%%%%%%%%
\begin{frame}{}
  \[
    |A| = n
  \]

  \pause
  \[
    |\ps{A}| = 2^{n}
  \]

  \pause
  \fig{width = 0.70\textwidth}{figs/powerset-2-x-notation}

  \pause
  \[
    \ps{A} \qquad 2^{A} \qquad \set{0,1}^{A}
  \]
\end{frame}
%%%%%%%%%%%%%%%

%%%%%%%%%%%%%%%
\begin{frame}{}
  \[
    \red{\boxed{S \in \ps{X} \iff S \subseteq X}}
  \]
\end{frame}
%%%%%%%%%%%%%%%

%%%%%%%%%%%%%%%
\begin{frame}{}
  \begin{exampleblock}{请证明}
    \[
      \set{\emptyset, \set{\emptyset}} \in \ps{\ps{\ps{S}}}
    \]
  \end{exampleblock}

  \pause
  \[
    \set{\emptyset, \set{\emptyset}} \in \ps{\ps{\ps{S}}} \iff \set{\emptyset, \set{\emptyset}} \subseteq \ps{\ps{S}}
  \]

  \pause
  \hrule
  \begin{columns}
    \column{0.50\textwidth}
      \[
        \red{\emptyset \in \ps{\ps{S}}}
      \]

      \uncover<4->{
        \[
          \iff \emptyset \subseteq \ps{S}
        \]
      }
    \column{0.50\textwidth}
      \[
        \red{\set{\emptyset} \in \ps{\ps{S}}}
      \]
      \uncover<5->{
        \[
          \iff \set{\emptyset} \subseteq \ps{S}
        \]
      }
      \uncover<6->{
        \[
          \iff \emptyset \in \ps{S}
        \]
      }
      \uncover<7->{
        \[
          \iff \emptyset \subseteq S
        \]
      }
  \end{columns}
\end{frame}
%%%%%%%%%%%%%%%

%%%%%%%%%%%%%%%
\begin{frame}{}
  \begin{exampleblock}{请证明}
    \[
      \ps{A} \cap \ps{B} = \ps{A \cap B}
    \]
  \end{exampleblock}

  \pause
  \vspace{0.30cm}
  对于任意 $x$,
  \begin{align*}
    \onslide<3->{&\textcolor{white}{\iff}\;\; x \in \ps{A} \cap \ps{B} \\[6pt]}
    \onslide<4->{&\iff x \in \ps{A} \land x \in \ps{B} \\[6pt]}
    \onslide<5->{&\iff x \subseteq A \land x \subseteq B \\[6pt]}
    \onslide<6->{&\iff x \subseteq A \cap B \\[6pt]}
    \onslide<7->{&\iff x \in \ps{A \cap B}}
  \end{align*}
\end{frame}
%%%%%%%%%%%%%%%

%%%%%%%%%%%%%%%
\begin{frame}{}
  \begin{exampleblock}{请证明}
    \[
      \bigcap_{\alpha \in I} \ps{A_{\alpha}} = \ps{\bigcap_{\alpha \in I} A_{\alpha}}
    \]
  \end{exampleblock}

  \pause
  \vspace{0.30cm}
  对于任意 $x$,
  \begin{align*}
    \onslide<3->{&\textcolor{white}{\iff}\;\; x \in \bigcap_{\alpha \in I} \ps{A_{\alpha}} \\[6pt]}
    \onslide<4->{&\iff \forall \alpha \in I.\; x \in \ps{A_{\alpha}} \\[6pt]}
    \onslide<5->{&\iff \forall \alpha \in I.\; x \subseteq A_{\alpha} \\[6pt]}
    \onslide<6->{&\iff x \subseteq \bigcap_{\alpha \in I} A_{\alpha} \\[6pt]}
    \onslide<7->{&\iff x \in \ps{\bigcap_{\alpha \in I} A_{\alpha}}}
  \end{align*}
\end{frame}
%%%%%%%%%%%%%%%