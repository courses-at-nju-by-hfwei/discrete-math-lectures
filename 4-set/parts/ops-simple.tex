% ops-simple.tex

%%%%%%%%%%%%%%%
\begin{frame}{}
  \begin{center}
    {\Large 集合的运算 (I)}
  \end{center}

  \[
    \text{\Large $\cup \qquad \cap \qquad \setminus \qquad \Delta$}
  \]
\end{frame}
%%%%%%%%%%%%%%%

%%%%%%%%%%%%%%%
\begin{frame}{}
  \begin{definition}[集合的并 (Union)]
    \[
      A \cup B \triangleq \set{x \mid x \in A \lor x \in B}
    \]
  \end{definition}

  \vspace{0.30cm}
  \fig{width = 0.40\textwidth}{figs/venn-union}
\end{frame}
%%%%%%%%%%%%%%%

%%%%%%%%%%%%%%%
\begin{frame}{}
  \[
    A \cup \emptyset = A
  \]

  \pause
  \[
    A \cup A = A
  \]

  \[
    A \cup B = B \cup A
  \]

  \[
    (A \cup B) \cup C = A \cup (B \cup C)
  \]

  \pause
  \[
    A \subseteq A \cup B
  \]

  \[
    B \subseteq A \cup B
  \]
\end{frame}
%%%%%%%%%%%%%%%

%%%%%%%%%%%%%%%
\begin{frame}{}
  \begin{definition}[集合的交 (Intersection)]
    \[
      A \cap B \triangleq \set{x \mid x \in A \land x \in B}
    \]
  \end{definition}

  \vspace{0.30cm}
  \fig{width = 0.40\textwidth}{figs/venn-intersection}
\end{frame}
%%%%%%%%%%%%%%%

%%%%%%%%%%%%%%%
\begin{frame}{}
  \[
    A \cap \emptyset = \emptyset
  \]

  \pause
  \[
    A \cap A = A
  \]

  \[
    A \cap B = B \cap A
  \]

  \[
    (A \cap B) \cap C = A \cap (B \cap C)
  \]

  \pause
  \[
    A \cap B \subseteq A
  \]

  \[
    A \cap B \subseteq B
  \]
\end{frame}
%%%%%%%%%%%%%%%

%%%%%%%%%%%%%%%
\begin{frame}{}
  \begin{theorem}[分配律 (Distributive Law)]
    \[
      A \cup (B \cap C) = (A \cup B) \cap (A \cup C)
    \]
    \[
      \teal{A \cap (B \cup C) = (A \cap B) \cup (A \cap C)}
    \]
  \end{theorem}

  \begin{columns}
    \column{0.50\textwidth}
      \pause
      \fig{width = 0.90\textwidth}{figs/distributive-proof}
    \column{0.50\textwidth}
      \pause
      \fig{width = 0.85\textwidth}{figs/kiss}
  \end{columns}
\end{frame}
%%%%%%%%%%%%%%%

%%%%%%%%%%%%%%%
\begin{frame}{}
  \begin{theorem}[分配律 (Distributive Law)]
    \[
      A \cup (B \cap C) = (A \cup B) \cap (A \cup C)
    \]
  \end{theorem}

  \pause
  \vspace{0.50cm}
  \begin{center}
    对于任意$x$,
  \end{center}
  \begin{align}
    \uncover<3->{&x \in A \cup (B \cap C) \\[6pt]}
    \uncover<4->{\iff & (x \in A) \lor (x \in B \land x \in C) \\[6pt]}
    \uncover<5->{\red{\iff} & (x \in A \lor x \in B) \land (x \in A \lor x \in C) \\[6pt]}
    \uncover<6->{\iff & (x \in A \cup B) \land (x \in A \cup C) \\[6pt]}
    \uncover<7->{\iff & x \in (A \cup B) \cap (A \cup C)}
  \end{align}
\end{frame}
%%%%%%%%%%%%%%%

%%%%%%%%%%%%%%%
\begin{frame}{}
  \begin{theorem}[吸收律 (Absorption Law)]
    \[
      A \cup (A \cap B) = A
    \]
    \[
      \teal{A \cap (A \cup B) = A}
    \]
  \end{theorem}

  \pause
  \vspace{0.50cm}
  \begin{center}
    对任意$x$,
  \end{center}
  \pause
  \setcounter{equation}{0}
  \begin{align}
    &x \in A \cup (A \cap B) \\[6pt]
    \iff & x \in A \lor (x \in A \land x \in B) \\[6pt]
    \red{\iff} & x \in A
  \end{align}
\end{frame}
%%%%%%%%%%%%%%%

%%%%%%%%%%%%%%%
\begin{frame}{}
  \begin{theorem}
    \[
      A \subseteq B \iff A \cup B = B \;\purple{\iff A \cap B = A}
    \]
  \end{theorem}

  \pause
  \vspace{0.50cm}
  \begin{columns}
    \column{0.50\textwidth}
      \[
        A \subseteq B \implies A \cup B \subseteq B
      \]
      \uncover<3->{
        \begin{center}
          对任意 $x$,
        \end{center}
        \setcounter{equation}{0}
        \begin{align}
          &x \in A \cup B \\[6pt]
          \implies & x \in A \lor x \in B \\[6pt]
          \red{\implies} & x \in B \lor x \in B \\[6pt]
          \implies & x \in B
        \end{align}
      }
    \column{0.50\textwidth}
      \[
        A \subseteq B \implies B \subseteq A \cup B
      \]
      \uncover<4->{
        \begin{center}
          对任意 $x$,
        \end{center}
        \setcounter{equation}{0}
        \begin{align}
          &x \in B \\[6pt]
          \red{\implies} &x \in A \lor x \in B \\[6pt]
          \implies &x \in A \cup B
        \end{align}
      }
  \end{columns}
  \vspace{-0.50cm}
  \uncover<5->{
    \[
      \purple{A \cap B = A \cap (A \cup B) = A}
    \]
  }
\end{frame}
%%%%%%%%%%%%%%%

%%%%%%%%%%%%%%%
\begin{frame}{}
  \begin{definition}[集合的差 (Set Difference); \blue{相对补} (Relative Complement)]
    \[
      A \setminus B = \set{x \mid x \in A \land x \notin B}
    \]
  \end{definition}

  \fig{width = 0.40\textwidth, angle = 180}{figs/venn-relative-complement}

  \pause
  \[
    A = \set{2, 5, 6} \quad B = \set{1, 2, 4, 7, 9}
  \]
  \[
    A \setminus B = \set{5, 6} \qquad
    B \setminus A = \set{1, 4, 7, 9}
  \]
\end{frame}
%%%%%%%%%%%%%%%

%%%%%%%%%%%%%%%
\begin{frame}{}
  \begin{definition}[绝对补 (Absolute Complement); \purple{$\overline{A}, A', A^{c}$}]
    \red{设全集为$U$。}
    \[
      \overline{A} = U \setminus A = \set{x \in U \mid x \notin A}
    \]
  \end{definition}

  \pause
  \vspace{0.30cm}
  \begin{center}
    \blue{全集 $U$ (Universe) 是当前正在考虑的所有元素构成的集合 \\[6pt]
    一般均默认存在}

    \pause
    \vspace{0.80cm}
    警告: 不存在``包罗万象''的全集
  \end{center}
\end{frame}
%%%%%%%%%%%%%%%

%%%%%%%%%%%%%%%
\begin{frame}{}
  \begin{center}
    \red{设全集为 $U$}
  \end{center}

  \[
    \overline{\overline{A}} = A
  \]

  \[
    \overline{U} = \emptyset
  \]

  \[
    \overline{\emptyset} = U
  \]

  \[
    A \cup \overline{A} = U
  \]

  \[
    A \cap \overline{A} = \emptyset
  \]
\end{frame}
%%%%%%%%%%%%%%%

%%%%%%%%%%%%%%%
\begin{frame}{}
  \begin{theorem}[``相对补''与``绝对补''之间的关系]
    \red{设全集为 $U$。}
    \[
      \red{\boxed{A \setminus B = A \cap \overline{B}}}
    \]
    % \[
    %   \teal{A \setminus B = A \setminus (A \cap B)}
    % \]
  \end{theorem}

  \pause
  \vspace{0.30cm}
  \begin{center}
    对任意 $x$,
  \end{center}
  \setcounter{equation}{0}
  \begin{align}
    &x \in A \setminus B \\[6pt]
    \iff & x \in A \land x \notin B \\[6pt]
    \red{\iff} & x \in A \land (x \in U \land x \notin B) \\[6pt]
    \iff & x \in A \land x \in \overline{B} \\[6pt]
    \iff & x \in A \cap \overline{B}
  \end{align}
\end{frame}
%%%%%%%%%%%%%%%

%%%%%%%%%%%%%%%
\begin{frame}{}
  \begin{theorem}[德摩根律 (绝对补)]
    \red{设全集为 $U$。}

    \[
      \overline{A \cup B} = \overline{A} \cap \overline{B}
    \]
    \[
      \teal{\overline{A \cap B} = \overline{A} \cup \overline{B}}
    \]
  \end{theorem}

  \pause
  \vspace{0.30cm}
  \begin{center}
    对任意 $x$,
  \end{center}
  \setcounter{equation}{0}
  \begin{align}
    &x \in \overline{A \cup B} \\[6pt]
    \iff & x \in U \land \lnot (x \in A \lor x \in B) \\[6pt]
    \red{\iff} & x \in U \land x \notin A \land x \notin B \\[6pt]
    \iff & (x \in U \land x \notin A) \land (x \in U \land x \notin B) \\[6pt]
    \iff & x \in \overline{A} \land x \in \overline{B} \\[6pt]
    \iff & x \in \overline{A} \cap \overline{B}
  \end{align}
\end{frame}
%%%%%%%%%%%%%%%

%%%%%%%%%%%%%%%
\begin{frame}{}
  \begin{theorem}[德摩根律 (相对补)]
    \[
      C \setminus (A \cup B) = (C \setminus A) \cap (C \setminus B)
    \]
    \[
      \teal{C \setminus (A \cap B) = (C \setminus A) \cup (C \setminus B)}
    \]
  \end{theorem}

  \pause
  \vspace{0.30cm}
  \setcounter{equation}{0}
  \begin{align}
    &C \setminus (A \cup B) \\[6pt]
    \red{\iff} & C \cap \overline{A \cup B} \\[6pt]
    \iff & C \cap (\overline{A} \cap \overline{B}) \\[6pt]
    \iff & (C \cap \overline{A}) \cap (C \cap \overline{B}) \\[6pt]
    \iff & (C \setminus A) \cap (C \setminus B)
  \end{align}
\end{frame}
%%%%%%%%%%%%%%%

%%%%%%%%%%%%%%%
\begin{frame}{}
  \begin{theorem}
    \[
      A \cap (B \setminus C) = (A \cap B) \setminus C
                             = (A \cap B) \setminus (A \cap C)
    \]
    \[
      A \setminus (B \setminus C) = (A \cap C) \cup (A \setminus B)
    \]
  \end{theorem}
\end{frame}
%%%%%%%%%%%%%%%

%%%%%%%%%%%%%%%
\begin{frame}{}
  \begin{theorem}
    \[
      A \subseteq B \implies \overline{B} \subseteq \overline{A}
    \]
    \[
      \teal{A \subseteq B \implies (B \setminus A) \cup A = B}
    \]
  \end{theorem}
\end{frame}
%%%%%%%%%%%%%%%

%%%%%%%%%%%%%%%
\begin{frame}{}
  \begin{definition}[对称差 (Symmetric Difference)]
    \[
      A \oplus B = (A \setminus B) \cup (B \setminus A)
        = (A \cap \overline{B}) \cup (B \cap \overline{A})
    \]
  \end{definition}

  \[
    A \oplus B = \set{x \mid (x \in A) \;\red{\oplus}\; (x \in B)}
  \]

  \pause
  \fig{width = 0.40\textwidth}{figs/venn-symmetric-difference}

  \pause
  \[
    \blue{\boxed{A \oplus B = (A \cup B) \setminus (A \cap B)
      = (A \cup B) \cap (\overline{A} \cup \overline{B})}}
  \]
\end{frame}
%%%%%%%%%%%%%%%

%%%%%%%%%%%%%%%
\begin{frame}{}
  \[
    A \oplus B \oplus C
  \]

  \fig{width = 0.40\textwidth}{figs/3set-venn-symmetric-difference}
\end{frame}
%%%%%%%%%%%%%%%

%%%%%%%%%%%%%%%
\begin{frame}{}
  \[
    A \oplus \emptyset = A
  \]

  \pause
  \[
    A \oplus A = \emptyset
  \]

  \[
    A \oplus B = B \oplus A
  \]

  \[
    \teal{\boxed{(A \oplus B) \oplus C = A \oplus (B \oplus C)}}
  \]
\end{frame}
%%%%%%%%%%%%%%%

%%%%%%%%%%%%%%%
\begin{frame}{}
  \[
    A \oplus B = \emptyset \iff A = B
  \]

  \[
    A \oplus B = \overline{A} \oplus \overline{B}
  \]

  \[
    A \cap (B \oplus C) = (A \cap B) \oplus (A \cap C)
  \]

  % \pause
  % \[
  %   A \cup (B \oplus C) \neq (A \cup B) \oplus (A \cup C)
  % \]

  \[
    \red{A \oplus B = A \oplus C \implies B = C}
  \]
\end{frame}
%%%%%%%%%%%%%%%

%%%%%%%%%%%%%%%
\begin{frame}{}
  \[
    A \oplus B = A \oplus C \implies B = C
  \]

  \pause
  \[
    B = (A \oplus A) \oplus B = A \oplus (A \oplus B)
      = A \oplus (A \oplus C) = (A \oplus A) \oplus C = C
  \]
\end{frame}
%%%%%%%%%%%%%%%