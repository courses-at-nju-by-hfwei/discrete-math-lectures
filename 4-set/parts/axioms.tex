% axioms.tex

%%%%%%%%%%%%%%%
\begin{frame}{}
  \begin{center}
    {\Large Axiomatic Set Theory \blue{(ZFC)}}
  \end{center}

  \vspace{0.80cm}
  \begin{columns}
    \column{0.45\textwidth}
      \fig{width = 0.50\textwidth}{figs/Zermelo}{\centerline{Ernst Zermelo (1871--1953)}}
    \column{0.45\textwidth}
      \fig{width = 0.48\textwidth}{figs/Fraenkel}{\centerline{Abraham Fraenkel (1891--1965)}}
  \end{columns}
\end{frame}
%%%%%%%%%%%%%%%

%%%%%%%%%%%%%%%
% \begin{frame}{}
%   \begin{center}
%     {\Large First-order Language}
%   \end{center}
% 
%   \begin{columns}
%     \column{0.30\textwidth}
%     \column{0.40\textwidth}
%       \begin{description}[Connectives:]
% 	\item[Parentheses:] $(,)$
% 	\item[Variables:] $x, y, z, \cdots$
% 	\item[Connectives:] $\land$, $\lor$, $\lnot$, $\to$, $\leftrightarrow$
% 	\item[Quantifiers:] $\forall$, $\exists$
% 	\item[Equality:] $=$
% 	  \pause \vspace{0.50cm}
% 	\item[Constants:] $a, b, c, \cdots$
% 	\item[Functions:] $f, g, h, \cdots$
% 	\item[Predicates:] $R, P, Q, \cdots$
%       \end{description}
%     \column{0.30\textwidth}
%   \end{columns}
% \end{frame}
%%%%%%%%%%%%%%%

%%%%%%%%%%%%%%%
\begin{frame}{}
  \begin{center}
    {\Large First-order Language for Sets \red{$\lset{} = \set{\in}$}}
  \end{center}

  \begin{columns}
    \column{0.30\textwidth}
    \column{0.40\textwidth}
      \begin{description}[Connectives:]
	\item[Parentheses:] $(,)$
	\item[Variables:] \textcolor<2->{red}{$x, y, z, \cdots$}
	\item[Connectives:] $\land$, $\lor$, $\lnot$, $\to$, $\leftrightarrow$
	\item[Quantifiers:] $\forall$, $\exists$
	\item[Equality:] $=$
	\vspace{0.50cm}
	\item[Constants:]
	\item[Functions:] 
	\item[Predicates:] \red{$\in$}
      \end{description}
    \column{0.30\textwidth}
  \end{columns}

  \uncover<3->{
    \begin{center}
      \red{\large Everything we consider in $\lset{}$ is a set.}
    \end{center}
  }
\end{frame}
%%%%%%%%%%%%%%%

%%%%%%%%%%%%%%%
\begin{frame}{}
  \begin{center}
    \red{$Q:$ What is \blue{``$\in$''}?} \\[10pt]
    \red{$Q:$ What are \blue{``sets''}?} \\[40pt]

    \pause
    We don't define them directly. \\[10pt]
    We only describe their properties in an \red{axiomatic} way.
  \end{center}
\end{frame}
%%%%%%%%%%%%%%%

%%%%%%%%%%%%%%%
\begin{frame}{}
  \fig{width = 0.30\textwidth}{figs/elements-ch}

  \begin{enumerate}[(1)]
    \item To draw a straight line from any point to any point.
    \item To extend a finite straight line continuously in a straight line.
    \item To describe a circle with any center and radius.
    \item That all right angles are equal to one another.
    \item The parallel postulate.
  \end{enumerate}
\end{frame}
%%%%%%%%%%%%%%%

%%%%%%%%%%%%%%%
\begin{frame}{}
  \begin{definition}[$\notin$]
    \[
      x \notin A \triangleq \lnot (x \in A).
    \]
  \end{definition}

  \pause
  \vspace{0.50cm}
  \begin{definition}[$\subseteq$]
    \[
      A \subseteq B \triangleq \forall x (x \in A \implies x \in B)
    \]
  \end{definition}
\end{frame}
%%%%%%%%%%%%%%%

%%%%%%%%%%%%%%%
% axioms-PURP.tex

%%%%%%%%%%%%%%%
\begin{frame}{}
  \begin{definition}[``$\bigcup A$'' (Arbitrary Union)]
    \[
      \bigcup A \triangleq \text{ the \teal{unique} set obtained by \blue{unioning} } A.
    \]
  \end{definition}

  \pause
  \begin{theorem}
    \[
      \bigcup \set{x, y} = x \cup y.
    \]
  \end{theorem}

  \pause
  \begin{theorem}
    \[
      \bigcup \emptyset = \emptyset.
    \]
  \end{theorem}
\end{frame}
%%%%%%%%%%%%%%%

%%%%%%%%%%%%%%%
\begin{frame}{}
  \begin{theorem}[``$\bigcap A$'' (Arbitrary Intersection)]
    \red{For any nonempty set $A$}, there is a unique set $B$ such that
    \[
      \forall x\; (x \in B \iff x \text{ belongs to every member of } A).
    \]
    \[
      \forall x\; \big(x \in B \iff \forall y \in A (x \in y) \big).
    \]
  \end{theorem}

  \pause
  \begin{proof}
    \begin{center}
      Let $c$ be a fixed member of $A$.
    \end{center}

    \pause
    \vspace{-0.50cm}
    \[
      \bigcap A \triangleq \set{x \in c \mid x \text{ belongs to every other member of } A}.
    \]
  \end{proof}

  \pause
  \begin{alertblock}{``$\bigcap \emptyset$''}
    \[
      \bigcap \emptyset \text{ is \red{\it not} a set}.
    \]
  \end{alertblock}
\end{frame}
%%%%%%%%%%%%%%%

%%%%%%%%%%%%%%%
\begin{frame}{}
  \begin{theorem}[No Universal Set]
    There is no universal set.

    \[
      \red{\nexists B} \big(\forall x (x \in B) \big).
    \]
  \end{theorem}

  \pause
  \begin{proof}
    \begin{center}
      \red{For any set $A$, we construct a set not in $A$.}
    \end{center}

    \pause
    \[
      \blue{B = \set{x \in A \mid x \notin x}}
    \]
    \pause
    \vspace{-0.30cm}
    \[
      B \in B \iff B \in A \land B \notin B
    \]

    \pause
    \[
      \red{\boxed{B \notin A}}
    \]

    \pause
    \vspace{-0.30cm}
    \[
      B \in A \implies (B \in B \iff B \notin B)
    \]
  \end{proof}
\end{frame}
%%%%%%%%%%%%%%%

%%%%%%%%%%%%%%%
\begin{frame}{}
  \begin{definition}[``$u \setminus v$'' \red{(Relative Complement)}]
    \[
      u \setminus v \triangleq \set{x \in u \mid x \notin v}.
    \]
  \end{definition}

  \pause
  \begin{theorem}[No ``Absolute Complement'']
    For any set $B$, the following is \red{\it not} a set:
    \[
      \set{x \mid x \notin B}.
    \]
  \end{theorem}

  \pause
  \begin{proof}
    \begin{center}
      \red{By Contradiction.}
    \end{center}

    \pause
    \vspace{-0.50cm}
    \[
      \set{x \mid x \notin B} \cup B \text{ would be a universal set}.
    \]
  \end{proof}

  \pause
  \begin{quote}
    We can never look for objects ``not in $B$'' \red{unless we know where to start looking}.
    \hfill --- UD (Chapter 6; Page 64)
  \end{quote}
\end{frame}
%%%%%%%%%%%%%%%

%%%%%%%%%%%%%%%
\begin{frame}{}
  \begin{definition}[Power Set Axiom]
    For any set $A$, there is a set whose members are the subsets of $A$:
    \[
      \forall A\; \red{\exists B}\; \forall x (x \in B \iff x \subseteq A).
    \]
  \end{definition}

  \pause
  \vspace{0.50cm}
  \begin{definition}[``$\mathcal{P}(A)$'']
    \[
      \mathcal{P}(A) \triangleq \text{ the \teal{unique} power set of } A.
    \]
  \end{definition}

  \pause
  \vspace{0.50cm}
  \begin{alertblock}{The is \red{\it not} correct!}
    \[
      \mathcal{P}(A) \triangleq \set{x \mid x \subseteq A}
    \]
  \end{alertblock}
\end{frame}
%%%%%%%%%%%%%%%
%%%%%%%%%%%%%%%