% axioms.tex

%%%%%%%%%%%%%%%
\begin{frame}{}
  \begin{center}
    {\Large Axiomatic Set Theory \blue{(ZFC)}}
  \end{center}

  \vspace{0.80cm}
  \begin{columns}
    \column{0.45\textwidth}
      \fig{width = 0.50\textwidth}{figs/Zermelo}{\centerline{Ernst Zermelo (1871--1953)}}
    \column{0.45\textwidth}
      \fig{width = 0.48\textwidth}{figs/Fraenkel}{\centerline{Abraham Fraenkel (1891--1965)}}
  \end{columns}
\end{frame}
%%%%%%%%%%%%%%%

%%%%%%%%%%%%%%%
\begin{frame}{}
  \begin{center}
    {\Large First-order Language for Sets \red{$\lset{} = \set{\in}$}}
  \end{center}

  \begin{columns}
    \column{0.30\textwidth}
    \column{0.40\textwidth}
      \begin{description}[Connectives:]
        \item[Parentheses:]
        \item[Variables:]
        \item[Connectives:]
        \item[Quantifiers:]
        \vspace{0.50cm}
        \item[Constants:]
        \item[Functions:]
        \item[Predicates:] \red{$\in$}
      \end{description}
    \column{0.30\textwidth}
  \end{columns}
\end{frame}
%%%%%%%%%%%%%%%

%%%%%%%%%%%%%%%
\begin{frame}{}
  \begin{center}
    \red{$Q:$ What is \blue{``$\in$''}?} \\[10pt]
    \red{$Q:$ What are \blue{``sets''}?} \\[40pt]

    \pause
    We don't define them directly. \\[10pt]
    We only describe their properties in an \red{axiomatic} way.
  \end{center}
\end{frame}
%%%%%%%%%%%%%%%

%%%%%%%%%%%%%%%
\begin{frame}{}
  \fig{width = 0.30\textwidth}{figs/elements-ch}

  \begin{enumerate}[(1)]
    \item To draw a straight line from any point to any point.
    \item To extend a finite straight line continuously in a straight line.
    \item To describe a circle with any center and radius.
    \item That all right angles are equal to one another.
    \item The parallel postulate.
  \end{enumerate}
\end{frame}
%%%%%%%%%%%%%%%

%%%%%%%%%%%%%%%
\begin{frame}{}
  \fig{width = 0.90\textwidth}{figs/axiomatic-set-theory-class}
  \begin{center}
    \teal{\url{https://www.youtube.com/watch?v=AAJB9l-HAZs}}
  \end{center}
\end{frame}
%%%%%%%%%%%%%%%