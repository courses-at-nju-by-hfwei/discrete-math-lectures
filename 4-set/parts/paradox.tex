% paradox.tex

%%%%%%%%%%%%%%%
\begin{frame}{}
  \begin{columns}
    \column{0.45\textwidth}
      \fig{width = 0.50\textwidth}{figs/Frege-old}{\centerline{Gottlob Frege (1848--1925)}}
    \column{0.45\textwidth}
      \fig{width = 0.40\textwidth}{figs/frege-arithmetic}
      \begin{center}
        {``Basic Laws of Arithmetic'' \\ (1893 \& 1903)}
      \end{center}
  \end{columns}

  \pause
  \vspace{0.80cm}
  \begin{quote}
    对于一个科学工作者来说,最不幸的事情莫过于:
    当他的工作接近完成时, 却发现那大厦的基础已经动摇。

    \hfill --- 《附录二》, 1902
  \end{quote}
\end{frame}
%%%%%%%%%%%%%%%

% \begin{quote}
%   就正直与风度而言,我认为在我所知的范围内,
%   无人可超越 Frege 对于真理的献身精神。
% \end{quote}

%%%%%%%%%%%%%%%
\begin{frame}{}
  \fig{width = 0.20\textwidth}{figs/Russell}{\vspace{-0.30cm}\centerline{Bertrand Russell (1872--1970)}}

  \begin{columns}
    \pause
    \column{0.30\textwidth}
      \fig{width = 0.60\textwidth}{figs/russell-philosophy}
    \pause
    \column{0.30\textwidth}
      \fig{width = 0.60\textwidth}{figs/russell-pm}
    \pause
    \column{0.30\textwidth}
      \fig{width = 0.80\textwidth}{figs/russell-nobel}
  \end{columns}
\end{frame}
%%%%%%%%%%%%%%%

%%%%%%%%%%%%%%%
\begin{frame}{}
  \begin{theorem}[概括原则]
    \blue{For any predicate $\psi(x)$}, there is a \red{set} $X$:
    \[
      X = \set{x \mid \psi(x)}.
    \]
  \end{theorem}

  \pause
  \begin{definition}[Russell's Paradox]
    \[
      \blue{\psi(x) \triangleq ``x \notin x"}
    \]

    \pause
    \[
      R = \set{x \mid x \notin x}
    \]

    \pause
    \[
      \red{Q: R \in R\;?}
    \]
  \end{definition}
\end{frame}
%%%%%%%%%%%%%%%

%%%%%%%%%%%%%%%
\begin{frame}{}
  \begin{center}
    \red{$Q:$ } 既然朴素集合论存在悖论,你是如何做作业的?
    \vspace{0.60cm}
    \fig{width = 0.20\textwidth}{figs/cannot-see}
  \end{center}
\end{frame}
%%%%%%%%%%%%%%%

%%%%%%%%%%%%%%%
\begin{frame}{}
  \begin{center}
    \fig{width = 0.40\textwidth}{figs/have-to-fix-it}

    \pause
    \vspace{0.20cm}
    \begin{theorem}[Russell's Paradox]
      \[
        \set{x \mid x \notin x} \text{ is \red{\it not} a set.}
      \]
    \end{theorem}
  \end{center}
\end{frame}
%%%%%%%%%%%%%%%