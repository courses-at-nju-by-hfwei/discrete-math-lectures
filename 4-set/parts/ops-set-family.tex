% ops-set-family.tex

%%%%%%%%%%%%%%%
\begin{frame}{}
  \begin{center}
    {\Large 集合的运算 (II)}
  \end{center}

  \[
    \text{\Large $\bigcap \qquad \bigcup$}
  \]
\end{frame}
%%%%%%%%%%%%%%%

%%%%%%%%%%%%%%%
\begin{frame}{}
  \begin{definition}[广义并 (Arbitrary Union)]
    设 $\mathbb{M}$ 是一组集合 (a {\it collection} of sets)
    \[
      \bigcup \mathbb{M} = \Bset{x \mid \exists A \in \mathbb{M}.\; x \in A}
    \]
  \end{definition}

  \pause
  \[
    \mathbb{M} = \set{\set{1, 2}, \set{\set{1, 2}, 3}, \set{4, 5}}
  \]

  \pause
  \[
    \bigcup \mathbb{M} = \pause\set{1, 2, 3, 4, 5, \red{\set{1, 2}}}
  \]

  \pause
  \[
    \bigcup \emptyset = \pause \emptyset
  \]
\end{frame}
%%%%%%%%%%%%%%%

%%%%%%%%%%%%%%%
\begin{frame}{}
  \[
    \bigcup_{j = 1}^{n} A_j \triangleq A_1 \cup A_2 \cup \cdots \cup A_n
  \]

  \pause
  \vspace{0.50cm}
  \[
    \bigcup_{j = 1}^{\infty} A_j \triangleq A_1 \cup A_2 \cup \cdots
  \]

  \pause
  \vspace{0.50cm}
  \[
    \bigcup_{\alpha \in I} A_{\alpha} \triangleq \Big\{x \mid \red{\exists} \alpha \in I: x \in A_{\alpha}\Big\}
  \]
\end{frame}
%%%%%%%%%%%%%%%

%%%%%%%%%%%%%%%
\begin{frame}{}
  \begin{definition}[广义交 (Arbitrary Intersection)]
    设 $\mathbb{M}$ 是一组集合 (a {\it collection} of sets)
    \[
      \bigcap \mathbb{M} = \Bset{x \mid \forall A \in \mathbb{M}.\; x \in A}
    \]
  \end{definition}
\end{frame}
%%%%%%%%%%%%%%%

%%%%%%%%%%%%%%%
\begin{frame}{}
  \[
    \bigcap_{j = 1}^{n} A_j \triangleq A_1 \cap A_2 \cap \cdots \cap A_n
  \]

  \pause
  \vspace{0.50cm}
  \[
    \bigcap_{j = 1}^{\infty} A_j \triangleq A_1 \cap A_2 \cap \cdots
  \]

  \pause
  \vspace{0.50cm}
  \[
    \bigcap_{\alpha \in I} A_{\alpha} \triangleq \Big\{x \mid \red{\forall} \alpha \in I: x \in A_{\alpha}\Big\}
  \]
\end{frame}
%%%%%%%%%%%%%%%

%%%%%%%%%%%%%%%
\begin{frame}{}
  \begin{theorem}[德摩根律]
    \[
      X \setminus \bigcup_{\alpha \in I} A_{\alpha} = \bigcap_{\alpha \in I} (X \setminus A_{\alpha})
    \]

    \[
      \teal{X \setminus \bigcap_{\alpha \in I} A_{\alpha} = \bigcup_{\alpha \in I} (X \setminus A_{\alpha})}
    \]
  \end{theorem}

  \pause
  \vspace{0.30cm}
  \begin{proof}
  \end{proof}
\end{frame}
%%%%%%%%%%%%%%%

%%%%%%%%%%%%%%%
\begin{frame}{}
  \begin{exampleblock}{德摩根律的应用}
    \[
      A = \mathbb{R} \setminus \bigcap_{n \in \mathbb{Z}^{+}} (\mathbb{R} \setminus \set{-n, -n+1, \cdots, 0, \cdots, n-1, n})
    \]
  \end{exampleblock}

  \pause
  \[
    X_n = \set{-n, -n+1, \cdots, 0, \cdots, n-1, n}
  \]

  \pause
  \begin{align*}
    \onslide<3->{A &= \mathbb{R} \setminus \bigcap_{n \in \mathbb{Z}^{+}} (\mathbb{R} \setminus X_n) \\}
      \onslide<4->{&= \mathbb{R} \setminus \Big(\mathbb{R} \setminus \bigcup_{n \in \mathbb{Z}^{+}} X_n \Big) \\}
      \onslide<5->{&= \mathbb{R} \setminus \Big(\mathbb{R} \setminus \mathbb{Z} \Big) \\}
      \onslide<6->{&= \mathbb{Z}}
  \end{align*}
\end{frame}
%%%%%%%%%%%%%%%