% predicate-logic-semantics.tex

%%%%%%%%%%%%%%%%%%%%
\begin{frame}{}
  \fig{width = 0.60\textwidth}{figs/syntax-semantics}
  \begin{center}
    一阶谓词逻辑的语义
  \end{center}
\end{frame}
%%%%%%%%%%%%%%%%%%%%

%%%%%%%%%%%%%%%%%%%%
\begin{frame}{}
  \begin{center}
    一阶谓词逻辑公式的\blue{\bf 语义}就是该公式的\red{``真 ($T/1/\top$)''、``假 ($F/0/\bot$)''}值。

    \[
      \forall x.\; Sx > {\bf 0}
    \]
    \pause
    \[
      \forall x.\; \exists y.\; (y < x)
    \]
    \pause
    \[
      x > {\bf 0}
    \]

    \pause
    \vspace{0.60cm}
    一阶谓词逻辑公式$\alpha$的真假值取决于
    \begin{columns}
      \column{0.15\textwidth}
      \column{0.70\textwidth}
        \begin{enumerate}[(1)]
          \setlength{\itemsep}{6pt}
          \item 对变元的解释
          \item 确定量词的论域
          \item 对常数符号、函数符号、谓词符号的解释
        \end{enumerate}
      \column{0.15\textwidth}
    \end{columns}

    \pause
    \vspace{0.60cm}
    \fbox{这种``\blue{\bf 解释}''将公式映射到一个\red{\bf 数学结构 $\U$}上, 决定了该公式的语义}
  \end{center}
\end{frame}
%%%%%%%%%%%%%%%%%%%%

%%%%%%%%%%%%%%%%%%%%
\begin{frame}{}
\end{frame}
%%%%%%%%%%%%%%%%%%%%

%%%%%%%%%%%%%%%%%%%%
\begin{frame}{}
\end{frame}
%%%%%%%%%%%%%%%%%%%%

%%%%%%%%%%%%%%%%%%%%
\begin{frame}{}
\end{frame}
%%%%%%%%%%%%%%%%%%%%