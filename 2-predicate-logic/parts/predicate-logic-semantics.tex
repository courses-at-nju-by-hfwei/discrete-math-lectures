% predicate-logic-semantics.tex

%%%%%%%%%%%%%%%%%%%%
\begin{frame}{}
  \begin{center}
    一阶谓词逻辑的语义

    \fig{width = 0.60\textwidth}{figs/syntax-semantics}
  \end{center}
\end{frame}
%%%%%%%%%%%%%%%%%%%%

%%%%%%%%%%%%%%%%%%%%
\begin{frame}{}
  \begin{center}
    一阶谓词逻辑公式的\blue{\bf 语义}就是该公式的\red{``真 ($T/1/\top$)''、``假 ($F/0/\bot$)''}值。

    \[
      \forall x.\; Sx > {\bf 0}
    \]
    \pause
    \[
      \forall x.\; \exists y.\; (y < x)
    \]
    \pause
    \[
      x > {\bf 0}
    \]

    \pause
    \vspace{0.60cm}
    一阶谓词逻辑公式$\alpha$的真假值取决于
    \begin{columns}
      \column{0.15\textwidth}
      \column{0.70\textwidth}
        \begin{enumerate}[(1)]
          \setlength{\itemsep}{6pt}
          \item 确定量词的论域
          \item 对常数符号、函数符号、谓词符号的解释
          \item 对\blue{自由}变元的解释 (赋值函数 $s$)
        \end{enumerate}
      \column{0.15\textwidth}
    \end{columns}

    \pause
    \vspace{0.60cm}
    \fbox{这种``\blue{\bf 解释}''将公式映射到一个\red{\bf 数学结构 $\U$}上, 决定了该公式的语义}
  \end{center}
\end{frame}
%%%%%%%%%%%%%%%%%%%%

%%%%%%%%%%%%%%%%%%%%
\begin{frame}{}
  \begin{definition}[$(\U, s) \models \alpha$]
    \begin{center}
      $\U$ 与 $s$ 满足公式 $\alpha$:
    \end{center}
    \[
      (\U, s) \models \alpha
    \]
    \begin{itemize}
      \item 将$\alpha$中的常数符号、函数符号、谓词符号按照数学结构 $\U$ 进行解释
      \item 将量词的论域限制在集合 $|\U|$ 上,
      \item 将自由变元 $x$ 解释为 $s(x)$,
      \item 这样就将公式 $\alpha$ 翻译成了某个数学领域中的命题
      \item 使用数学领域知识我们知道该命题成立
    \end{itemize}
  \end{definition}
\end{frame}
%%%%%%%%%%%%%%%%%%%%

%%%%%%%%%%%%%%%%%%%%
\begin{frame}{}
  \[
    \alpha: \forall x.\; (x \times x \neq 1 + 1)
  \]

  \pause
  \vspace{0.80cm}
  \begin{center}
    $\alpha$ 在数学结构 $\U = \Q$ 中为真

    \vspace{0.60cm}

    $\alpha$ 在数学结构 $\U = \R$ 中为假
  \end{center}
\end{frame}
%%%%%%%%%%%%%%%%%%%%

%%%%%%%%%%%%%%%%%%%%
\begin{frame}{}
  \begin{definition}[语义蕴含 (Logically Imply)]
    \begin{center}
      令 $\Sigma$ 为一个公式集, $\alpha$ 为一个公式。 \\[8pt]
      $\Sigma$ \red{\bf 语义蕴含} $\alpha$, 记为 $\Sigma \models \alpha$, \\[8pt]
      如果\red{每个}满足$\Sigma$中\blue{所有}公式的\red{结构$\U$与赋值$s$}都满足$\alpha$。
    \end{center}
  \end{definition}

  \[
    \forall x.\; P(x) \models P(y)
  \]

  \pause
  \[
    P(y) \not\models \forall x.\; P(x)
  \]
\end{frame}
%%%%%%%%%%%%%%%%%%%%

%%%%%%%%%%%%%%%%%%%%
\begin{frame}{}
  \[
    \alpha: \forall x \forall y \forall z ((Pxy \land Pyz) \to Pxz)
  \]

  \[
    \beta: \forall x \forall y ((Pxy \land Pyx) \to x = y)
  \]

  \[
    \gamma: \forall x \exists y Pxy \to \exists y \forall x Pxy
  \]

  \pause
  \vspace{0.60cm}
  \[
    \set{\alpha, \beta} \models \gamma\; ?
  \]
\end{frame}
%%%%%%%%%%%%%%%%%%%%

%%%%%%%%%%%%%%%%%%%%
\begin{frame}{}
  \[
    \set{\alpha} \models \beta \text{ 简记为 } \;\alpha \models \beta
  \]

  \pause
  \vspace{0.50cm}
  \begin{definition}[语义等价 (Logically Equivalent)]
    如果 $\alpha \models \beta$ 且 $\beta \models \alpha$,
    则称 $\alpha$ 与 $\beta$ \red{\bf 语义等价}, 记为 $\alpha \equiv \beta$。
  \end{definition}
\end{frame}
%%%%%%%%%%%%%%%%%%%%

%%%%%%%%%%%%%%%%%%%%
\begin{frame}{}
  \begin{definition}[普遍有效的 (Valid)]
    如果 $\emptyset \models \alpha$, 则称 $\alpha$ 是\red{\bf 普遍有效的},
    记为 $\models \alpha$。
  \end{definition}

  \vspace{0.60cm}
  \begin{center}
    普遍有效的公式在\red{所有可能的结构$\U$与所有可能的赋值$s$}下均为真。

    \[
      \forall x.\; P(x) \models P(y)
    \]

    \[
      \forall x.\; P(x) \to P(y)
    \]
  \end{center}
\end{frame}
%%%%%%%%%%%%%%%%%%%%

%%%%%%%%%%%%%%%%%%%%
\begin{frame}{}
  \[
    \lnot \forall x \alpha \leftrightarrow \exists x \lnot \alpha
  \]

  \[
    \lnot \exists x \alpha \leftrightarrow \forall x \lnot \alpha
  \]

  \[
    \lnot (\forall x \in A.\; \alpha) \leftrightarrow \exists x \in A.\; \lnot \alpha
  \]
\end{frame}
%%%%%%%%%%%%%%%%%%%%

%%%%%%%%%%%%%%%%%%%%
\begin{frame}{}
  \[
    \forall x \forall y \alpha \leftrightarrow \forall y \forall x \alpha
  \]

  \[
    \exists x \exists y \alpha \leftrightarrow \exists y \exists x \alpha
  \]
\end{frame}
%%%%%%%%%%%%%%%%%%%%

%%%%%%%%%%%%%%%%%%%%
\begin{frame}{}
  \[
    \forall x \alpha \land \forall x \beta
      \leftrightarrow \forall x (\alpha \land \beta)
  \]

  \[
    \exists x \alpha \lor \exists x \beta
      \leftrightarrow \exists x (\alpha \lor \beta)
  \]
\end{frame}
%%%%%%%%%%%%%%%%%%%%

%%%%%%%%%%%%%%%%%%%%
\begin{frame}{}
  \[
    \forall x \alpha \to \exists x \alpha
  \]

  \[
    \exists x \forall y \alpha \to \forall y \exists x \alpha
  \]

  \pause
  \[
    \forall x \alpha \lor \forall x \beta \to \forall x (\alpha \lor \beta)
  \]

  \[
    \exists x (\alpha \land \beta) \to \exists x \alpha \land \exists x \beta
  \]

  \pause
  \[
    \forall x (\alpha \to \beta) \to (\exists x \alpha \to \exists x \beta)
  \]
\end{frame}
%%%%%%%%%%%%%%%%%%%%

%%%%%%%%%%%%%%%%%%%%
\begin{frame}{}
  \begin{exampleblock}{}
    给定如下``前提'', 请判断``结论''是否有效, 并说明理由。\\
    前提如下:
    \begin{enumerate}[(1)]
      \item 每个人或者喜欢美剧, 或者喜欢韩剧 (可以同时喜欢二者);
      \item 任何人如果他喜欢抗日神剧, 他就不喜欢美剧;
      \item 有的人不喜欢韩剧。
    \end{enumerate}

    结论: 有的人不喜欢抗日神剧 (\emph{幸亏如此})。
  \end{exampleblock}
\end{frame}
%%%%%%%%%%%%%%%%%%%%

%%%%%%%%%%%%%%%%%%%%
\begin{frame}{}
  例子
\end{frame}
%%%%%%%%%%%%%%%%%%%%