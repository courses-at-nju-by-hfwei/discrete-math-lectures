% predicate-logic-syntax.tex

%%%%%%%%%%%%%%%%%%%%
\begin{frame}{}
  \begin{center}
    \fig{width = 0.60\textwidth}{figs/syntax-semantics}

    \vspace{0.60cm}
    语法与语义是``对立统一''的
  \end{center}
\end{frame}
%%%%%%%%%%%%%%%%%%%%

%%%%%%%%%%%%%%%%%%%%
\begin{frame}{}
  \begin{definition}[命题逻辑的语言]
    一阶谓词逻辑的语言包括以下7部分: \\[5pt]
    \begin{description}
      \setlength{\itemsep}{5pt}
      \item [逻辑联词:] $\lnot, \land, \lor, \to, \leftrightarrow$
      \item [\red{量词符号:}] $\forall$ (全称量词), $\exists$ (存在量词)
      \item [变元符号:] $x, y, z, \dots$
      \item [左右括号:] $(, )$
      \item [\cyan{常数符号:}] 零个或多个常数符号
      \item [\blue{函数符号:}] $n$-元函数符号 ($n \in \N^{+}$)
      \item [\red{谓词符号:}] $n$-元谓词符号 ($n \in \N^{+}$)
    \end{description}
  \end{definition}

  \pause
  \begin{center}
    $Q:$ 为什么没有命题符号 $P, Q, \dots$?
  \end{center}
\end{frame}
%%%%%%%%%%%%%%%%%%%%

%%%%%%%%%%%%%%%%%%%%
\begin{frame}{}
  \begin{center}
    初等数论的语言 $L = \set{\textbf{\cyan{0}},
      \blue{S}, \blue{+}, \blue{\times}, \red{<}, \red{=}}$
  \end{center}
\end{frame}
%%%%%%%%%%%%%%%%%%%%

%%%%%%%%%%%%%%%%%%%%
\begin{frame}{}
  \begin{definition}[项 (Item)]
    \begin{enumerate}[(1)]
      \setlength{\itemsep}{8pt}
      \item 每个变元 $x, y, z, \dots$ 都是一个项;
      \item 每个常数符号都是一个项;
      \item 如果 $t_{1}, t_{2}, \dots, t_{n}$ 是项,
        且 $f$ 为一个 $n$ 元函数符号, \\
        则 $f(t_{1}, t_{2}, \dots, t_{n})$ 也是项;
      \item 除此之外, 别无其它。
    \end{enumerate}
  \end{definition}

  \[
    x
  \]
  \[
    {\bf 0}
  \]
  \[
    S{\bf 0} \qquad +(x, SSS0) \qquad \times(+(x, SSS0), y)
  \]
\end{frame}
%%%%%%%%%%%%%%%%%%%%

%%%%%%%%%%%%%%%%%%%%
\begin{frame}{}
  \begin{definition}[公式 (Formula)]
    \begin{enumerate}[(1)]
      \setlength{\itemsep}{8pt}
      \item 如果 $t_{1}, \dots, t_{n}$ 是项, 且 $P$ 是一个 $n$ 元谓词符号, \\
        则 $P(t_{1}, \dots, t_{n})$ 为公式, 称为{\bf 原子公式};
      \item 如果 $\alpha$ 与 $\beta$ 都是公式, 则 $(\lnot \alpha)$
        与 $(\alpha \;\red{\ast}\; \beta)$ 都是公式;
      \item 如果 $\alpha$ 是公式, 则 \red{$\forall x.\; \alpha$}
        与 \red{$\exists x.\; \alpha$} 也是公式;
      \item 除此之外, 别无其它。
    \end{enumerate}
  \end{definition}

  \begin{center}
    约定:
  \end{center}
\end{frame}
%%%%%%%%%%%%%%%%%%%%

%%%%%%%%%%%%%%%%%%%%
\begin{frame}{}
  \begin{center}
    初等数论的语言 $L = \set{\textbf{\cyan{0}},
      \blue{S}, \blue{+}, \blue{\times}, \red{<}, \red{=}}$
  \end{center}

  \begin{enumerate}[<+->][(1)]
    \setlength{\itemsep}{5pt}
    \item 0 不是任何自然数的后继
      \[
        \forall x. \lnot (Sx = 0)
      \]
    \item 两个自然数相等当且进当它们的后继相等
      \[
        \forall x, y.\; (x = y \leftrightarrow Sx = Sy)
      \]
    \item $x$ 是素数 \\
      ($x > 1$ 且 $x$ 没有除自身和1之外的因子)
      \[
        S0 < x \land \forall y, z.\; (y < x \land z < x) \to \lnot (y \times z = x)
      \]
    \item 给定任何性质 (谓词) $P(x)$, 自然数上的数学归纳原理
      \[
        (P(0) \land \forall x.\; (P(x) \to P(Sx))) \to (\forall x.\; P(x))
      \]
    \item 哥德巴赫猜想 (任一大于2的偶数, 都可表示成两个素数之和)
  \end{enumerate}
\end{frame}
%%%%%%%%%%%%%%%%%%%%

%%%%%%%%%%%%%%%%%%%%
\begin{frame}{}
  \[
    \lim_{x \to a} f(x) = l
  \]
  \[
    \forall \epsilon \in \R^{+}.\; \exists \delta \in \R^{+}.\;
      \forall x \in \R^{+}.\; (0 < |x - a| < \delta \to |f(x) - l| < \epsilon)
  \]
\end{frame}
%%%%%%%%%%%%%%%%%%%%

%%%%%%%%%%%%%%%%%%%%
\begin{frame}{}
  A function $f$ from $\mathbb{R}$ to $\mathbb{R}$ is called
  \begin{itemize}
    \item \emph{pointwise continuous} if
      for every $x \in \mathbb{R}$
      and every real number $\epsilon > 0$,
      there exists real $\delta > 0$ such that
      for every $y \in \mathbb{R}$ with $|x - y| < \delta$,
      we have that $|f(x) -  f(y)|< \epsilon$.
    \item \emph{uniformly continuous} if
      for every real number $\epsilon > 0$,
      there exists real $\delta > 0$ such that
      for every $x, y \in \mathbb{R}$ with $|x - y| < \delta$,
      we have that $|f(x) -  f(y)|< \epsilon$.
  \end{itemize}
\end{frame}
%%%%%%%%%%%%%%%%%%%%

%%%%%%%%%%%%%%%%%%%%
\begin{frame}{}
\end{frame}
%%%%%%%%%%%%%%%%%%%%