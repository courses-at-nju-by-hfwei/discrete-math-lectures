% predicate-logic-syntax.tex

%%%%%%%%%%%%%%%%%%%%
\begin{frame}{}
  \begin{center}
    一阶谓词逻辑的语法

    \fig{width = 0.60\textwidth}{figs/syntax-semantics}
  \end{center}
\end{frame}
%%%%%%%%%%%%%%%%%%%%

%%%%%%%%%%%%%%%%%%%%
\begin{frame}{}
  \begin{exampleblock}{例子: 谓词 (Predicate)}
    \pause
    \begin{enumerate}[<+->][(1)]
      \setlength{\itemsep}{6pt}
      \item 张三是个法外狂徒
      \item 5 大于 3
      \item 地球绕着太阳转
      \item 点 $a$ 在点 $b$ 与点 $c$ 之间
    \end{enumerate}
  \end{exampleblock}

  \pause
  \vspace{0.30cm}
  \begin{center}
    一元谓词表达了个体的性质 \\[6pt]
    多元谓词表达了个体之间的关系 \\[6pt]
    零元谓词即命题逻辑中的命题 ($P:$ 张三是个法外狂徒)
  \end{center}
\end{frame}
%%%%%%%%%%%%%%%%%%%%

%%%%%%%%%%%%%%%%%%%%
\begin{frame}{}
  \begin{definition}[一阶谓词逻辑的语言 (Language)]
    一阶谓词逻辑的语言包括以下7部分: \\[5pt]
    \begin{description}
      \setlength{\itemsep}{5pt}
      \item [逻辑联词:] $\lnot, \land, \lor, \to, \leftrightarrow$
      \item [\purple{量词符号:}] $\forall$ (forall; 全称量词), $\exists$ (exists; 存在量词)
      \item [变元符号:] $x, y, z, \dots$
      \item [左右括号:] $(, )$
      \pause
      \vspace{0.50cm}
      \item [\cyan{常数符号:}] 零个或多个常数符号 $a, b, c, \dots$, \teal{表达特殊的个体}
      \item [\blue{函数符号:}] $n$-元函数符号 $f, g, h, \dots$ ($n \in \N^{+}$), \teal{表达个体上的运算}
      \item [\red{谓词符号:}] $n$-元谓词符号 $P, Q, R, \dots$ ($n \in \N$), \teal{表达个体的性质与关系}
    \end{description}
  \end{definition}
\end{frame}
%%%%%%%%%%%%%%%%%%%%

%%%%%%%%%%%%%%%%%%%%
\begin{frame}{}
  \begin{center}
    初等数论的语言
    \[
      L = \set{\textbf{\cyan{0}}, \blue{S}, \blue{+}, \blue{\times}, \red{<}, \red{=}}
    \]
  \end{center}
\end{frame}
%%%%%%%%%%%%%%%%%%%%

%%%%%%%%%%%%%%%%%%%%
\begin{frame}{}
  \begin{center}
    ``项''是一阶谓词逻辑要讨论的\red{个体}对象, 它本身无所谓真假。
  \end{center}

  \begin{definition}[项 (Term)]
    \begin{enumerate}[(1)]
      \setlength{\itemsep}{8pt}
      \item 每个变元 $x, y, z, \dots$ 都是一个项;
      \item 每个常数符号都是一个项;
      \item 如果 $t_{1}, t_{2}, \dots, t_{n}$ 是项,
        且 $f$ 为一个 $n$-元函数符号, \\
        则 $f(t_{1}, t_{2}, \dots, t_{n})$ 也是项;
      \item 除此之外, 别无其它。
    \end{enumerate}
  \end{definition}
  \[
    L = \set{\textbf{\cyan{0}}, \blue{S}, \blue{+}, \blue{\times}, \red{<}, \red{=}}
  \]
  \[
    x
  \]
  \[
    {\bf 0}
  \]
  \[
    S{\bf 0} \qquad x + SSS0 \qquad (x + SSS0) \times y
  \]
\end{frame}
%%%%%%%%%%%%%%%%%%%%

%%%%%%%%%%%%%%%%%%%%
\begin{frame}{}
  \begin{center}
    公式刻画了\red{个体的性质}或者\red{个体之间的关系}, 它的语义就是它的真假值。
  \end{center}

  \begin{definition}[公式 (Formula)]
    \begin{enumerate}[(1)]
      \setlength{\itemsep}{8pt}
      \item 如果 $t_{1}, \dots, t_{n}$ 是项, 且 $P$ 是一个 $n$ 元谓词符号, \\
        则 $P(t_{1}, \dots, t_{n})$ 为公式, 称为\teal{\bf 原子公式};
      \item 如果 $\alpha$ 与 $\beta$ 都是公式, 则 $(\lnot \alpha)$
        与 $(\alpha \;\red{\ast}\; \beta)$ 都是公式;
      \item 如果 $\alpha$ 是公式, 则 \red{$\forall x.\; \alpha$}
        与 \red{$\exists x.\; \alpha$} 也是公式;
      \item 除此之外, 别无其它。
    \end{enumerate}
  \end{definition}

  \pause
  \vspace{0.20cm}
  \begin{center}
    \blue{\bf 约定:} 量词符号 $\forall$ 与 $\exists$ 的管辖范围尽可能短
    \[
      \forall x.\; \alpha \to \beta:
        \text{表示}\; (\forall x.\; \alpha) \to \beta
        \quad \gray{\text{不表示}\; \forall x.\; (\alpha \to \beta)}
    \]
  \end{center}
\end{frame}
%%%%%%%%%%%%%%%%%%%%

%%%%%%%%%%%%%%%%%%%%
\begin{frame}{}
  \[
    L = \set{\textbf{\cyan{0}}, \blue{S}, \blue{+}, \blue{\times}, \red{<}, \red{=}}
  \]

  \begin{enumerate}[<+->][(1)]
    \setlength{\itemsep}{6pt}
    \item 0 不是任何自然数的后继
      \[
        \forall x.\; \lnot (Sx = 0)
      \]
    \item 两个自然数相等当且仅当它们的后继相等
      \[
        \forall x.\; \forall y.\; (x = y \leftrightarrow Sx = Sy)
      \]
    \item $x$ 是素数 ($x > 1$ 且 $x$ 没有除自身和1之外的因子)
      \[
        \text{Prime}(x): S0 < x \land \forall y.\; \forall z.\; (y < x \land z < x) \to \lnot (y \times z = x)
      \]
    \item 哥德巴赫猜想 (任一大于2的偶数, 都可表示成两个素数之和)
      \begin{align*}
        &\forall x.\; (SS0 < 2 \land (\exists y.\; 2 \times y = x)) \to \\
          &\quad (\exists x_{1}.\; \exists x_{2}.\; \text{Prime}(x_{1}) \land \text{Prime}(x_{2})
            \land x_{1} + x_{2} = x)
      \end{align*}
  \end{enumerate}
\end{frame}
%%%%%%%%%%%%%%%%%%%%

%%%%%%%%%%%%%%%%%%%%
\begin{frame}{}
  \[
    \blue{\boxed{\lim_{x \to a} f(x) = b}}
  \]

  \pause
  \vspace{0.30cm}
  \begin{center}
    对于任意的正实数 $\epsilon$, 存在一个正实数 $\delta$, \\[6pt]
    使得对于任意的 $x$, 当 $0 < |x - a| < \delta$ 时, 都有 $|f(x) - b| < \epsilon$ 成立。
  \end{center}

  \pause
  \[
    \forall \epsilon \in \R^{+}.\; \exists \delta \in \R^{+}.\;
      \forall x \in \R^{+}.\; (0 < |x - a| < \delta \to |f(x) - b| < \epsilon)
  \]
\end{frame}
%%%%%%%%%%%%%%%%%%%%

%%%%%%%%%%%%%%%%%%%%
\begin{frame}{}
  \[
    \blue{\forall x \in A.\; \alpha} \quad\text{实际上是}\quad
      \blue{\forall x.\; (x \in A \;\red{\to}\; \alpha)} \quad\text{的简记}
  \]

  \[
    \blue{\exists x \in A.\; \alpha} \quad\text{实际上是}\quad
      \blue{\exists x.\; (x \in A \;\red{\land}\; \alpha)} \quad\text{的简记}
  \]
\end{frame}
%%%%%%%%%%%%%%%%%%%%

%%%%%%%%%%%%%%%%%%%%
\begin{frame}{}
  \begin{exampleblock}{本周作业之一: 使用一阶谓词逻辑公式表达下列概念}
    A function $f$ from $\mathbb{R}$ to $\mathbb{R}$ is called
    \begin{enumerate}[(1)]
      \setlength{\itemsep}{10pt}
      \item \blue{\it pointwise continuous} (连续的) if
        for every $x \in \mathbb{R}$
        and every real number $\epsilon > 0$,
        there exists real $\delta > 0$ such that
        for every $y \in \mathbb{R}$ with $|x - y| < \delta$,
        we have that $|f(x) -  f(y)|< \epsilon$.
      \item \blue{\it uniformly continuous} (一致连续的) if
        for every real number $\epsilon > 0$,
        there exists real $\delta > 0$ such that
        for every $x, y \in \mathbb{R}$ with $|x - y| < \delta$,
        we have that $|f(x) -  f(y)|< \epsilon$.
    \end{enumerate}
  \end{exampleblock}
\end{frame}
%%%%%%%%%%%%%%%%%%%%

%%%%%%%%%%%%%%%%%%%%
\begin{frame}{}
  \begin{center}
    以下概念与程序设计语言中相应的概念类似, 我们举例说明, 不作定义
  \end{center}

  \begin{definition}[作用域 (Scope)、约束变元 (Bind)、自由变元 (Free)]
    \begin{enumerate}[(1)]
      \setlength{\itemsep}{6pt}
      \item $\forall x.\; (P(x) \to Q(x))$
      \item $(\forall x.\; P(x)) \to Q(x)$
      \item $\forall x.\; \Big(P(x) \to \big(\exists y.\; R(x, y)\big)\Big)$
      \item $\Big(\forall x.\; \forall y.\; \big(P(x, y) \land Q(y, z)\big)\Big) \land \exists x.\; P(x, y)$
    \end{enumerate}
  \end{definition}

  \pause
  \vspace{0.30cm}
  \begin{definition}[改名 (Rename)]
    为尽量避免重名, 可将约束变元或自由变元\red{\bf 改名}为\blue{新鲜(fresh)}变元
  \end{definition}
\end{frame}
%%%%%%%%%%%%%%%%%%%%