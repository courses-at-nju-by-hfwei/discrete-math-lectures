% history.tex

%%%%%%%%%%%%%%%%%%%%
\begin{frame}{}
  \fig{width = 0.80\textwidth}{figs/history}
\end{frame}
%%%%%%%%%%%%%%%%%%%%

%%%%%%%%%%%%%%%%%%%%
\begin{frame}{}
  \begin{center}
    \fig{width = 0.70\textwidth}{figs/three}

    Socrates $\qquad\quad$ Plato $\qquad\quad$ Aristotle
  \end{center}
\end{frame}
%%%%%%%%%%%%%%%%%%%%

%%%%%%%%%%%%%%%%%%%%
\begin{frame}{}
  \[
    \ndnoname{\text{所有人都是必死的} \qquad \text{所有希腊人都是人}}{\text{所有希腊人都是必死的}}
  \]

  \pause
  \vspace{0.50cm}
  \[
    \ndnoname{\text{所有人都是必死的} \qquad \text{苏格拉底是人}}{\text{苏格拉底是必死的}}
  \]

  \pause
  \vspace{0.50cm}
  \[
    \ndnoname{\text{金属可以导电} \qquad \text{铜是金属}}{\text{铜可以导电}}
  \]

  \pause
  \vspace{0.50cm}
  \[
    \ndnoname{\forall x.\; \mathit{People}(x) \to \mathit{Die}(x)
      \qquad \mathit{People}(y)}{\mathit{Die}(y)}
  \]
\end{frame}
%%%%%%%%%%%%%%%%%%%%

%%%%%%%%%%%%%%%
\begin{frame}{}
  \begin{center}
    \fig{width = 0.95\textwidth}{figs/ABC}

    \vspace{0.30cm}
    亚里士多德总结的24种三段论
  \end{center}
\end{frame}
%%%%%%%%%%%%%%%%%%%%

%%%%%%%%%%%%%%%
\begin{frame}{}
  \fig{width = 0.60\textwidth}{figs/2000-years}
\end{frame}
%%%%%%%%%%%%%%%%%%%%

%%%%%%%%%%%%%%%
\begin{frame}{}
  \fig{width = 0.35\textwidth}{figs/Leibniz.jpg}

  \pause
  \begin{center}
    Gottfried Wilhelm Leibniz (莱布尼茨 1646 -- 1716)

    \vspace{0.30cm}
    ``17世纪的亚里士多德''
  \end{center}
\end{frame}
%%%%%%%%%%%%%%%

%%%%%%%%%%%%%%%
\begin{frame}{``我有一个梦想 $\ldots$''}
  \begin{quote}
    建立一个能够涵盖人类思维活动的\red{\large ``通用符号演算系统''},\\
    让人们的思维方式变得像数学运算那样清晰。\\[8pt]

    一旦有争论, 不管是科学上的还是哲学上的,
    人们只要坐下来\red{\LARGE 算一算},
    就可以毫不费力地辨明谁是对的。
  \end{quote}

  \vspace{0.80cm}
  \begin{quote}
    \centerline{\red{\LARGE Let us calculate [calculemus].}}
  \end{quote}
\end{frame}
%%%%%%%%%%%%%%%

%%%%%%%%%%%%%%%%%%%%
% \begin{frame}{}
%   \fig{width = 0.50\textwidth}{figs/200-years-later}
% \end{frame}
%%%%%%%%%%%%%%%

%%%%%%%%%%%%%%%%%%%%
\begin{frame}{}
  \begin{center}
    {\bf 两个重要的符号逻辑系统:} \red{\bf ``命题逻辑''} 与 \blue{\bf ``一阶谓词逻辑''}

    \fig{width = 0.50\textwidth}{figs/2-logic-systems}

    \vspace{0.50cm}
    \cyan{\bf ``推理即(符号)计算''}
  \end{center}
\end{frame}
%%%%%%%%%%%%%%%

%%%%%%%%%%%%%%%%%%%%
\begin{frame}{}
  \begin{center}
    今天, 我们学习\red{\bf 命题逻辑}

    \fig{width = 0.60\textwidth}{figs/syntax-semantics}

    下周, 我们学习\blue{\bf 一阶谓词逻辑}
  \end{center}
\end{frame}
%%%%%%%%%%%%%%%%%%%%