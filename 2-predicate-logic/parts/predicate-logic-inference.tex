% predicate-logic-inference.tex

%%%%%%%%%%%%%%%%%%%%
\begin{frame}{}
  \begin{center}
    {\Large 一阶谓词逻辑的自然推理 (演绎; 推演) 系统}
  \end{center}
\end{frame}
%%%%%%%%%%%%%%%%%%%%

%%%%%%%%%%%%%%%%%%%%
\begin{frame}{$\forall\text{-elim}$}
  \begin{center}
    \[
      \nd{\forall x \alpha}{\alpha[t/x]}{\forall x\text{-elim}}
    \]

    where $t$ is \red{\bf free} for $x$ in $\alpha$
  \end{center}

  \pause
  \[
    \alpha: \exists y (x < y)
  \]
  \[
    \forall x \exists y (x < y)
  \]

  \pause
  \[
    \forall x \exists y (x < y) \;\to\; \exists y (\blue{t} < y)
  \]

  \pause
  \[
    \xout{\forall x \exists y (x < y) \;\to\; \exists y (\red{y} < y)}
  \]
\end{frame}
%%%%%%%%%%%%%%%%%%%%

%%%%%%%%%%%%%%%%%%%%
\begin{frame}{$\forall$\text{-intro}}
  \begin{center}
    \[
      \nd{\stackrel{[t]}{\stackrel{\vdots}{\alpha[t/x]}}}
        {\forall x \alpha}{\forall x\text{-intro}}
    \]
    where, $t$ is \red{\bf fresh} variable never used elsewhere
  \end{center}
\end{frame}
%%%%%%%%%%%%%%%%%%%%

%%%%%%%%%%%%%%%%%%%%
\begin{frame}{}
  \begin{exampleblock}{$\forall$-推理规则的应用}
    \[
      \set{\forall x (P(x) \to Q(x)), \forall x P(x)} \vdash \forall x Q(x)
    \]
  \end{exampleblock}
\end{frame}
%%%%%%%%%%%%%%%%%%%%

%%%%%%%%%%%%%%%%%%%%
\begin{frame}{}
  \begin{exampleblock}{$\forall$-推理规则的应用}
    \[
      \set{P(t), \forall x (P(x) \to \lnot Q(x))} \vdash \lnot Q(t)
    \]
  \end{exampleblock}
\end{frame}
%%%%%%%%%%%%%%%%%%%%

%%%%%%%%%%%%%%%%%%%%
\begin{frame}{$\exists\text{-intro}$}
  \begin{center}
    \[
      \nd{\alpha[t/x]}{\exists x \alpha}{\exists x\text{-intro}}
    \]
    where $t$ is \red{\bf free} for $x$ in $\alpha$
  \end{center}

  \pause
  \[
    \alpha: \forall y (x = y)
  \]

  \pause
  \[
  \]

  \pause
  \[
    \xout{\exists x \forall y (x = y)}
  \]
\end{frame}
%%%%%%%%%%%%%%%%%%%%

%%%%%%%%%%%%%%%%%%%%
\begin{frame}{$\exists\text{-elim}$}
  \begin{center}
    \[
      \nd{\exists x \alpha \qquad \stackrel{\big[\alpha[t/x]\big]}{\stackrel{\vdots}{\beta}}}{\beta}{\exists\text{-elim}}
    \]
  where $t$ does not occur in $\beta$
  \end{center}
\end{frame}
%%%%%%%%%%%%%%%%%%%%

%%%%%%%%%%%%%%%%%%%%
\begin{frame}{$\exists$\text{-推理规则的应用}}
  \[
    \forall x P(x) \vdash \exists x P(x)
  \]
\end{frame}
%%%%%%%%%%%%%%%%%%%%

%%%%%%%%%%%%%%%%%%%%
\begin{frame}{$\exists$\text{-推理规则的应用}}
  \[
    \set{\forall x (P(x) \to Q(x)), \exists x P(x)} \vdash \exists x Q(x)
  \]
\end{frame}
%%%%%%%%%%%%%%%%%%%%

%%%%%%%%%%%%%%%%%%%%
\begin{frame}{$\forall, \exists$-推理规则的应用}
  \[
    \set{\exists x P(x), \forall x \forall y (P(x) \to Q(y))} \vdash \forall y Q(y)
  \]
\end{frame}
%%%%%%%%%%%%%%%%%%%%

%%%%%%%%%%%%%%%%%%%%
\begin{frame}{$\forall, \exists$-推理规则的应用}
  \begin{exampleblock}{}
    给定如下``前提'', 请判断``结论''是否有效, 并说明理由。\\
    前提如下:
    \begin{enumerate}[(1)]
      \item 每个人或者喜欢美剧, 或者喜欢韩剧 (可以同时喜欢二者);
      \item 任何人如果他喜欢抗日神剧, 他就不喜欢美剧;
      \item 有的人不喜欢韩剧。
    \end{enumerate}

    结论: 有的人不喜欢抗日神剧 (\emph{幸亏如此})。
  \end{exampleblock}
\end{frame}
%%%%%%%%%%%%%%%%%%%%

%%%%%%%%%%%%%%%%%%%%
\begin{frame}{$\forall, \exists$-推理规则的应用}
\end{frame}
%%%%%%%%%%%%%%%%%%%%