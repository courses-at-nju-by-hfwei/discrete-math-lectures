% predicate-logic-inference.tex

%%%%%%%%%%%%%%%%%%%%
\begin{frame}{}
  \begin{center}
    {\Large 一阶谓词逻辑的自然推理 (演绎; 推演) 系统} \\[10pt]
    \teal{\large (简化版本)}
  \end{center}
\end{frame}
%%%%%%%%%%%%%%%%%%%%

%%%%%%%%%%%%%%%%%%%%
\begin{frame}{$\forall\text{-elim}$}
  \begin{center}
    \[
      \nd{\forall x.\; \alpha}{\alpha[t/x]}{\forall x\text{-elim}}
    \]

    where $t$ is \red{free} for $x$ in $\alpha$
  \end{center}

  \pause
  \[
    \forall x.\; \blue{P(x)} \vdash P(\blue{c}) \quad (c \;\text{是任意常元符号})
  \]

  \pause
  \[
    \forall x.\; \blue{\exists y.\; (x < y)} \vdash \exists y.\; (\blue{c} < y)
      \quad (c \;\text{是任意常元符号})
  \]

  \pause
  \[
    \forall x.\; \blue{\exists y.\; (x < y)} \vdash \exists y.\; (\blue{z} < y)
      \quad (z \neq y \;\text{是任意变元符号})
  \]

  \pause
  \[
    \forall x.\; \blue{\exists y.\; (x < y)} \nvdash \exists y.\; (\red{y} < y)
      \quad (y \;\text{is {\it not} free for } x \text{ in } \alpha)
  \]
\end{frame}
%%%%%%%%%%%%%%%%%%%%

%%%%%%%%%%%%%%%%%%%%
\begin{frame}{$\forall$\text{-elim}推理规则的应用}
  \[
    \ndnoname{\text{每个人都是要死的 \qquad 苏格拉底是人}}{\text{苏格拉底是要死的}}
  \]

  \vspace{0.80cm}
  \[
    \ndnoname{\forall x.\; (H(x) \to M(x)) \qquad H(s)}{M(s)}
  \]
\end{frame}
%%%%%%%%%%%%%%%%%%%%

%%%%%%%%%%%%%%%%%%%%
\begin{frame}{$\forall$\text{-intro}}
  \begin{center}
    \[
      \nd{\stackrel{[t]}{\stackrel{\vdots}{\alpha[t/x]}}}
        {\forall x.\; \alpha}{\forall x\text{-intro}}
    \]
    where, $t$ is a \red{fresh} variable

    \vspace{0.80cm}
    ``任取 $t$, 如果能证明 $\alpha$ 对 $t$ 成立, 则 $\alpha$ 对所有 $x$ 成立''
  \end{center}
\end{frame}
%%%%%%%%%%%%%%%%%%%%

%%%%%%%%%%%%%%%%%%%%
\begin{frame}{}
  \begin{exampleblock}{$\forall$-推理规则的应用}
    \[
      \Big\{P(t), \forall x (P(x) \to \lnot Q(x))\Big\} \vdash \lnot Q(t)
    \]
  \end{exampleblock}

  \pause
  \setcounter{equation}{0}
  \begin{align}
    P(t) & \quad (\text{前提}) \label{eq:1} \\[6pt]
    \forall x.\; (P(x) \to \lnot Q(x)) & \quad \text{(前提)} \label{eq:2} \\[6pt]
    \uncover<3->{P(t) \to \lnot Q(t) & \quad (\forall\text{-elim}, (\ref{eq:2})) \label{eq:3} \\[6pt]}
    \uncover<4->{\lnot Q(t) & \quad (\to\text{-elim}, (\ref{eq:1}), (\ref{eq:3})) \label{eq:4}}
  \end{align}
\end{frame}
%%%%%%%%%%%%%%%%%%%%

%%%%%%%%%%%%%%%%%%%%
\begin{frame}{}
  \begin{exampleblock}{$\forall$-推理规则的应用}
    \[
      \Big\{\forall x.\; (P(x) \to Q(x)), \forall x.\; P(x)\Big\} \vdash \forall x.\; Q(x)
    \]
  \end{exampleblock}

  \pause
  \setcounter{equation}{0}
  \begin{align}
    \forall x.\; (P(x) \to Q(x)) & \quad (\text{前提}) \label{eq:1} \\[6pt]
    \forall x.\; P(x) & \quad (\text{前提}) \label{eq:2} \\[6pt]
    \uncover<3->{[x_{0}] & \quad (\text{引入变量}) \label{eq:3} \\[6pt]}
    \uncover<4->{P(x_{0}) \to Q(x_{0}) &
      \quad (\forall\text{-elim}, (\ref{eq:1}), (\ref{eq:3})) \label{eq:4} \\[6pt]}
    \uncover<5->{P(x_{0}) & \quad (\forall\text{-elim}, (\ref{eq:2}), (\ref{eq:3})) \label{eq:5} \\[6pt]}
    \uncover<6->{Q(x_{0}) & \quad (\to\text{-elim}, (\ref{eq:4}), (\ref{eq:5})) \label{eq:6} \\[6pt]}
    \uncover<7->{\forall x.\; Q(x) & \quad (\forall\text{-intro}, (\ref{eq:3})-(\ref{eq:6})) \label{eq:7}}
  \end{align}
\end{frame}
%%%%%%%%%%%%%%%%%%%%

%%%%%%%%%%%%%%%%%%%%
\begin{frame}{$\exists\text{-intro}$}
  \begin{center}
    \[
      \nd{\alpha[t/x]}{\exists x.\; \alpha}{\exists x\text{-intro}}
    \]
    where $t$ is \red{free} for $x$ in $\alpha$
  \end{center}

  \vspace{0.30cm}
  \begin{center}
    ``如果 $\alpha$ 对某个项 $t$ 成立, 则 $\exists x.\; \alpha$ 成立。''
  \end{center}

  \pause
  \[
    P(\blue{c}) \vdash \exists x.\; P(x)
      \quad c \text{ 是任意常元符号}
  \]

  \pause
  \[
    \teal{\forall y.\; (y = y)} \nvdash \exists x.\; \teal{\forall y.\; (x = y)}
      \quad (y \text{ is \red{\it not} free for } x \text{ in } \teal{\alpha})
  \]
\end{frame}
%%%%%%%%%%%%%%%%%%%%

%%%%%%%%%%%%%%%%%%%%
\begin{frame}{$\exists\text{-elim}$}
  \begin{center}
    % \[
    %   \nd{\exists x.\; \alpha \qquad [x_{0}] \qquad
    %     \stackrel{\big[\alpha[x_{0}/x]\big]}{\stackrel{\vdots}{\beta}}}{\beta}{\exists\text{-elim}}
    % \]
    \[
      \nd{\exists x.\; \alpha}{\alpha[x_{0}/x]}{\exists\text{-elim}}
    \]
  where $x_{0}$ is \red{free} for $x$ in $\alpha$
  \end{center}
\end{frame}
%%%%%%%%%%%%%%%%%%%%

%%%%%%%%%%%%%%%%%%%%
\begin{frame}{$\exists\text{-elim}$}
  \begin{center}
    \[
      \nd{\exists x.\; \alpha \qquad [x_{0}] \qquad
        \stackrel{\big[\alpha[x_{0}/x]\big]}{\stackrel{\vdots}{\beta}}}{\beta}{\exists\text{-elim}}
    \]
    where $x_{0}$ is \red{free} for $x$ in $\alpha$

    \vspace{1.00cm}
    ``\red{假设} $x_{0}$ 使得 $\alpha$ 成立, 如果从 $\alpha[x_{0}/x]$ 可以推导出 $\beta$, \\[6pt]
      则从 $\exists x.\; \alpha$ 可以推导出 $\beta$''
  \end{center}
\end{frame}
%%%%%%%%%%%%%%%%%%%%

%%%%%%%%%%%%%%%%%%%%
\begin{frame}{}
  \begin{exampleblock}{$\exists$\text{-推理规则的应用}}
    \[
      \forall x.\; P(x) \vdash \exists x.\; P(x)
    \]
  \end{exampleblock}

  \pause
  \setcounter{equation}{0}
  \begin{align}
    \forall x.\; P(x) & \quad (\text{前提})
      \label{eq:1} \\[6pt]
    \uncover<3->{[x_{0}] & \quad (\text{引入变量})
      \label{eq:2} \\[6pt]}
    \uncover<4->{P(x_{0}) & \quad (\forall\text{-elim}, (\ref{eq:1}), (\ref{eq:2}))
      \label{eq:3} \\[6pt]}
    \uncover<5->{\exists x.\; P(x) & \quad (\exists\text{-intro}, (\ref{eq:3}))
      \label{eq:4}}
  \end{align}
\end{frame}
%%%%%%%%%%%%%%%%%%%%

%%%%%%%%%%%%%%%%%%%%
\begin{frame}{}
  \begin{exampleblock}{$\forall, \exists$\text{-推理规则的应用}}
    \[
      \Big\{\forall x.\; (P(x) \to Q(x)), \exists x.\; P(x)\Big\}
        \vdash \exists x.\; Q(x)
    \]
  \end{exampleblock}

  \pause
  \setcounter{equation}{0}
  \begin{align}
    \forall x.\; (P(x) \to Q(x)) & \quad (\text{前提})
      \label{eq:1} \\[6pt]
    \exists x.\; P(x) & \quad (\text{前提})
      \label{eq:2} \\[6pt]
    \uncover<3->{\red{[x_{0}] \quad [P(x_{0})]} & \quad (\text{引入变量与假设})
      \label{eq:3} \\[6pt]}
    \uncover<4->{P(x_{0}) \to Q(x_{0}) & \quad (\forall\text{-elim}, (\ref{eq:1}), (\ref{eq:3}))
      \label{eq:4} \\[6pt]}
    \uncover<5->{Q(x_{0}) & \quad (\to\text{-elim}, (\ref{eq:3}), (\ref{eq:4}))
      \label{eq:5} \\[6pt]}
    \uncover<6->{\red{\exists x.\; Q(x)} & \quad (\exists\text{-intro}, (\ref{eq:5}))
      \label{eq:6} \\[6pt]}
    \uncover<7->{\exists x.\; Q(x) & \quad (\exists\text{-elim}, (\ref{eq:2}),
      (\ref{eq:3})-(\ref{eq:6})) \label{eq:7}}
  \end{align}
\end{frame}
%%%%%%%%%%%%%%%%%%%%

%%%%%%%%%%%%%%%%%%%%
\begin{frame}{}
  \begin{exampleblock}{$\forall, \exists$-推理规则的应用}
    \[
      \Big\{\exists x.\; P(x), \forall x.\; \forall y.\; (P(x) \to Q(y))\Big\}
        \vdash \forall y.\; Q(y)
    \]
  \end{exampleblock}

  \pause
  \vspace{-0.60cm}
  \setcounter{equation}{0}
  \begin{align}
    \exists x.\; P(x) & \quad \text{(前提)}
      \label{eq:1} \\[6pt]
    \forall x.\; \forall y.\; (P(x) \to Q(y)) & \quad \text{(前提)}
      \label{eq:2} \\[6pt]
    \uncover<3->{\red{[x_{0}] \quad [P(x_{0})]} & \quad \text{(引入变量与假设)}
      \label{eq:3} \\[6pt]}
    \uncover<4->{\forall y.\; (P(x_{0}) \to Q(y)) & \quad (\forall\text{-elim}, (\ref{eq:2}), (\ref{eq:3}))
      \label{eq:4} \\[6pt]}
    \uncover<5->{\blue{[y_{0}]} & \quad \text{(引入变量)}
      \label{eq:5} \\[6pt]}
    \uncover<6->{P(x_{0}) \to Q(y_{0}) & \quad (\forall\text{-elim}, (\ref{eq:4}), (\ref{eq:5}))
      \label{eq:6} \\[6pt]}
    \uncover<7->{Q(y_{0}) & \quad (\to\text{-elim}, (\ref{eq:3}), (\ref{eq:6}))
      \label{eq:7} \\[6pt]}
    \uncover<8->{\red{Q(y_{0})} & \quad (\exists\text{-elim}, (\ref{eq:1}), (\ref{eq:3})-(\ref{eq:7}))
      \label{eq:8} \\[6pt]}
    \uncover<9->{\forall y.\; Q(y) & \quad (\forall\text{-intro}, (\ref{eq:5})-(\ref{eq:8}))}
  \end{align}
\end{frame}
%%%%%%%%%%%%%%%%%%%%

%%%%%%%%%%%%%%%%%%%%
\begin{frame}{$\forall, \exists$-推理规则的应用}
  \begin{exampleblock}{请使用一阶谓词逻辑公式表达下列命题与推理}
    给定如下``前提'', 请判断``结论''是否有效, 并说明理由。\\[10pt]
    {\bf 前提:}
    \begin{enumerate}[(1)]
      \item 每个人或者喜欢美剧, 或者喜欢韩剧 (可以同时喜欢二者);
      \item 任何人如果他喜欢抗日神剧, 他就不喜欢美剧;
      \item 有的人不喜欢韩剧。
    \end{enumerate}

    \vspace{0.20cm}
    {\bf 结论:} 有的人不喜欢抗日神剧 ({\it 幸亏如此})。
  \end{exampleblock}

  \pause
  \vspace{0.50cm}
  \[
    \ndnoname{\forall x.\; A(x) \lor K(x) \qquad
      \forall x.\; J(x) \to \lnot A(x) \qquad
      \exists x.\; \lnot K(x)}{\exists x.\; \lnot J(x)}
  \]
\end{frame}

%%%%%%%%%%%%%%%%%%%%
\begin{frame}{}
  \setcounter{equation}{0}
  \begin{align}
    \forall x.\; A(x) \lor K(x) & \quad (\text{前提})
      \label{eq:1} \\[6pt]
    \forall x.\; J(x) \to \lnot A(x) & \quad (\text{前提})
      \label{eq:2} \\[6pt]
    \exists x.\; \lnot K(x) & \quad (\text{前提})
      \label{eq:3} \\[6pt]
    \uncover<2->{\red{[x_{0}] \quad [\lnot K(x_{0})]} & \quad (\text{引入变量与假设})
      \label{eq:4} \\[6pt]}
    \uncover<3->{A(x_{0}) \lor K(x_{0}) & \quad (\forall\text{-elim}, (\ref{eq:1}), (\ref{eq:4}))
      \label{eq:5} \\[6pt]}
    \uncover<4->{A(x_{0}) & \quad ((\ref{eq:4}), (\ref{eq:5}))
      \label{eq:6} \\[6pt]}
    \uncover<5->{J(x_{0}) \to \lnot A(x_{0}) & \quad (\forall\text{-elim}, (\ref{eq:2}), (\ref{eq:4}))
      \label{eq:7} \\[6pt]}
    \uncover<6->{\lnot J(x_{0}) & \quad ((\ref{eq:6}), (\ref{eq:7}))
      \label{eq:8} \\[6pt]}
    \uncover<7->{\red{\exists x.\; \lnot J(x)} & \quad (\exists\text{-intro}, (\ref{eq:8}))
      \label{eq:9} \\[6pt]}
    \uncover<8->{\exists x.\; \lnot J(x) & \quad (\exists\text{-elim}, (\ref{eq:3})-(\ref{eq:8}))
      \label{eq:10}}
  \end{align}
\end{frame}
%%%%%%%%%%%%%%%%%%%%

%%%%%%%%%%%%%%%%%%%%
\begin{frame}{}
  \setcounter{equation}{0}
  \begin{align}
    \forall x.\; A(x) \lor K(x) \label{eq:1} \\[3pt]
    \forall x.\; J(x) \to \lnot A(x) \label{eq:2} \\[3pt]
    \exists x.\; \lnot K(x) \label{eq:3}
  \end{align}

  \pause
  \begin{center}
    根据 (\ref{eq:3}), 不妨设 $\lnot K(x)$ 对 $x_{0}$ 成立:
    \begin{align}
      \lnot K(x_{0}) \label{eq:4}
    \end{align}
    \begin{columns}
      \column{0.45\textwidth}
        \pause
        根据 (\ref{eq:1}), 有
        \begin{align}
          A(x_{0}) \lor K(x_{0}) \label{eq:5}
        \end{align}
        \pause
        根据 (\ref{eq:4}) 与 (\ref{eq:5}), 有
        \begin{align}
          A(x_{0}) \label{eq:6}
        \end{align}
      \column{0.45\textwidth}
        \pause
        根据 (\ref{eq:2}), 有
        \begin{align}
          J(x_{0}) \to \lnot A(x_{0}) \label{eq:7}
        \end{align}
        \pause
        根据 (\ref{eq:6}) 与 (\ref{eq:7}), 有
        \begin{align}
          \lnot J(x_{0}) \label{eq:8}
        \end{align}
        \pause
        因此, $\exists x.\; \lnot J(x)$
    \end{columns}
  \end{center}
\end{frame}
%%%%%%%%%%%%%%%%%%%%