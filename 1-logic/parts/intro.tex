% intro.tex

%%%%%%%%%%%%%%%%%%%%
\begin{frame}{}
  \fig{width = 0.60\textwidth}{figs/euler-logic}
\end{frame}
%%%%%%%%%%%%%%%%%%%%

%%%%%%%%%%%%%%%%%%%%
\begin{frame}{}
  \begin{center}
    数理逻辑的萌芽时期

    \fig{width = 0.70\textwidth}{figs/three}
    Socrates $\qquad\quad$ Plato $\qquad\quad$ Aristotle
  \end{center}
\end{frame}
%%%%%%%%%%%%%%%%%%%%

%%%%%%%%%%%%%%%%%%%%
\begin{frame}{}
\end{frame}
%%%%%%%%%%%%%%%%%%%%

%%%%%%%%%%%%%%%
\begin{frame}{}
  \fig{width = 0.35\textwidth}{figs/Leibniz.jpg}

  \pause
  \begin{center}
    Gottfried Wilhelm Leibniz (莱布尼茨 1646 -- 1716)
  \end{center}
\end{frame}
%%%%%%%%%%%%%%%

%%%%%%%%%%%%%%%
\begin{frame}{``我有一个梦想 $\ldots$''}
  \begin{quote}
    建立一个能够涵盖人类思维活动的\red{\large ``通用符号演算系统''},\\
    让人们的思维方式变得像数学运算那样清晰。\\[8pt]

    一旦有争论, 不管是科学上的还是哲学上的,
    人们只要坐下来\red{\LARGE 算一算},
    就可以毫不费力地辨明谁是对的。
  \end{quote}

  \vspace{0.80cm}
  \begin{quote}
    \centerline{\red{\LARGE Let us calculate [calculemus].}}
  \end{quote}
\end{frame}
%%%%%%%%%%%%%%%

%%%%%%%%%%%%%%%%%%%%
\begin{frame}{}
  \fig{width = 0.60\textwidth}{figs/syntax-semantics}
\end{frame}
%%%%%%%%%%%%%%%%%%%%