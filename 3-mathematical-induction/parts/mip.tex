% mip.tex

%%%%%%%%%%%%%%%%%%%%%%%%%%%%%%
\begin{frame}{}
  \begin{theorem}[Fermat's Little Theorem]
    对于任意自然数 $a$ 与素数 $p$,
    \[
      a^{p} \equiv a \;(\mathrm{mod}\; p).
    \]
  \end{theorem}

  \pause
  \[
    (a+1)^p = a^p+ \binom{p}{1} a^{p-1} + \binom{p}{2} a^{p-2}
      + \dots + \binom{p}{p-1}a + 1
  \]

  \pause
  \[
    \binom{p}{k} = \left( \frac{p!}{k!(p-k)!} \right)
  \]
\end{frame}
%%%%%%%%%%%%%%%%%%%%%%%%%%%%%%

%%%%%%%%%%%%%%%%%%%%%%%%%%%%%%
\begin{frame}{}
  \begin{exampleblock}{Tiling Puzzle}
    任何一个\blue{缺失了一格的} $2^{n} \times 2^{n}$ 的网格都可以被 $L$ 型填满。
  \end{exampleblock}

  \fig{width = 0.70\textwidth}{figs/L-tiling-case-2}
\end{frame}
%%%%%%%%%%%%%%%%%%%%%%%%%%%%%%

%%%%%%%%%%%%%%%%%%%%%%%%%%%%%%
\begin{frame}{}
  \begin{definition}[Line Map]
    \begin{itemize}
      \item A blank circle is a line map;
      \item A line map with a straight line drawn across it is a line map.
    \end{itemize}
  \end{definition}

  \fig{width = 0.40\textwidth}{figs/n=4-color}

  \pause
  \vspace{-0.50cm}
  \begin{theorem}
    Any line map can be two-colored.
  \end{theorem}
\end{frame}
%%%%%%%%%%%%%%%%%%%%%%%%%%%%%%

%%%%%%%%%%%%%%%%%%%%%%%%%%%%%%
\begin{frame}{}
  \begin{exampleblock}{The Tower of Hanoi}
    \fig{width = 0.50\textwidth}{figs/hanoi}
  \end{exampleblock}
\end{frame}
%%%%%%%%%%%%%%%%%%%%%%%%%%%%%%

%%%%%%%%%%%%%%%%%%%%%%%%%%%%%%
\begin{frame}{}
  \fig{width = 0.50\textwidth}{figs/hanoi-solution}
\end{frame}
%%%%%%%%%%%%%%%%%%%%%%%%%%%%%%

%%%%%%%%%%%%%%%%%%%%%%%%%%%%%%
\begin{frame}{}
\end{frame}
%%%%%%%%%%%%%%%%%%%%%%%%%%%%%%

%%%%%%%%%%%%%%%%%%%%%%%%%%%%%%
\begin{frame}{}
  \begin{exampleblock}{Josephus Problem}
    \fig{width = 0.50\textwidth}{figs/Josephus}
  \end{exampleblock}
\end{frame}
%%%%%%%%%%%%%%%%%%%%%%%%%%%%%%

%%%%%%%%%%%%%%%%%%%%%%%%%%%%%%
\begin{frame}{}
  \begin{exampleblock}{}
    \[
      \binom{n}{k} = \frac{n!}{k! (n-k)!}
    \]
  \end{exampleblock}
\end{frame}
%%%%%%%%%%%%%%%%%%%%%%%%%%%%%%

%%%%%%%%%%%%%%%%%%%%%%%%%%%%%%
\begin{frame}{}
  \begin{exampleblock}{Horse Paradox}
    所有马的颜色都相同。
  \end{exampleblock}

  \pause
  \vspace{0.50cm}
  \begin{center}
    对马的数目 $n$ 作归纳

    \pause
    \fig{width = 0.50\textwidth}{figs/induction-horses}
  \end{center}
\end{frame}
%%%%%%%%%%%%%%%%%%%%%%%%%%%%%%