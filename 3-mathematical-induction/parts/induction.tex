% induction.tex

%%%%%%%%%%%%%%%%%%%%%%%%%%%%%%
\begin{frame}{}
  \begin{theorem}[第一数学归纳法 (The First Mathematical Induction)]
    令 $P(n)$ 表示关于自然数 $n$ 的某个性质。
    如果
    \begin{enumerate}[(i)]
      \setlength{\itemsep}{8pt}
      \item $P(0)$ 成立;
      \item 对任意自然数 $n$, 如果 $P(n)$ 成立, 则 $P(n+1)$ 成立。
    \end{enumerate}
    那么, $P(n)$ 对所有自然数 $n$ 都成立。
  \end{theorem}

  \pause
  \[
    \nd{P(0) \qquad \forall n \in \N.\; \Big(P(n) \to P(n+1) \Big)}{
      \forall n \in \N.\; P(n)}{\text{第一数学归纳法}}
  \]

  \pause
  \[
    \biggl(P(0) \land \forall n \in \N.\; \Big(P(n) \to P(n+1) \Big) \biggr)
	  \to \forall n \in \N.\; P(n).
  \]
\end{frame}
%%%%%%%%%%%%%%%%%%%%%%%%%%%%%%

%%%%%%%%%%%%%%%%%%%%%%%%%%%%%%
\begin{frame}{}
  \begin{theorem}[第二数学归纳法 (The Second Mathematical Induction)]
    令 $Q(n)$ 表示关于自然数 $n$ 的某个性质。
    如果
    \begin{enumerate}[(i)]
      \setlength{\itemsep}{8pt}
      \item $Q(0)$ 成立;
      \item 对任意自然数 $n$, 如果 $Q(1), Q(2), \dots, Q(n)$ 都成立, 则 $Q(n+1)$ 成立。
    \end{enumerate}
    那么, $Q(n)$ 对所有自然数 $n$ 都成立。
  \end{theorem}

  \pause
  \[
    \nd{Q(0) \;\; \forall n \in \N.\; \Big(\big(Q(1) \land \dots \land Q(n)\big) \to Q(n+1) \Big)}{
      \forall n \in \N.\; Q(n)}{\text{第二数学归纳法}}
  \]

  \pause
  \[
    \biggl(Q(0) \land \forall n \in \N.\; \Big(\big(Q(1) \land \cdots \land Q(n)\big) \to Q(n+1) \Big) \biggr)
      \to \forall n \in \N.\; Q(n).
  \]
\end{frame}
%%%%%%%%%%%%%%%%%%%%%%%%%%%%%%

%%%%%%%%%%%%%%%%%%%%%%%%%%%%%%
\begin{frame}{}
  \begin{theorem}[数学归纳法]
    第一数学归纳法与第二数学归纳法等价。
  \end{theorem}

  \pause
  \vspace{0.50cm}
  \begin{center}
    {\red{$Q:$} 第二数学归纳法也被称为\red{\bf ``强'' (Strong)} 数学归纳法, 它强在何处?}
  \end{center}
\end{frame}
%%%%%%%%%%%%%%%%%%%%%%%%%%%%%%

%%%%%%%%%%%%%%%%%%%%%%%%%%%%%%
\begin{frame}{}
  \begin{lemma}
    第二数学归纳法蕴含第一数学归纳法。
  \end{lemma}
\end{frame}
%%%%%%%%%%%%%%%%%%%%%%%%%%%%%%

%%%%%%%%%%%%%%%%%%%%%%%%%%%%%%
\begin{frame}{}
  \begin{lemma}
    第一数学归纳法蕴含第二数学归纳法。
  \end{lemma}
\end{frame}
%%%%%%%%%%%%%%%%%%%%%%%%%%%%%%

%%%%%%%%%%%%%%%%%%%%%%%%%%%%%%
\begin{frame}{}
  \begin{center}
    \red{\bf \Large 数学归纳法为何成立?}
  \end{center}
\end{frame}
%%%%%%%%%%%%%%%%%%%%%%%%%%%%%%

%%%%%%%%%%%%%%%%%%%%%%%%%%%%%%
\begin{frame}{}
  \begin{center}
    Peano 公理体系刻画了自然数的递归结构
  \end{center}

  \begin{columns}
    \column{0.70\textwidth}
      \begin{definition}{Peano Axioms}
        \begin{enumerate}[(1)]
          \item 
        \end{enumerate}
      \end{definition}
    \column{0.30\textwidth}
      \fig{width = 0.80\textwidth}{figs/Peano}
  \end{columns}
\end{frame}
%%%%%%%%%%%%%%%%%%%%%%%%%%%%%%

%%%%%%%%%%%%%%%%%%%%%%%%%%%%%%
\begin{frame}{}
  \begin{definition}[良序原理 (The Well-Ordering Principle)]
    \red{自然数集}的任意\blue{非空}子集都有一个最小元。
  \end{definition}
\end{frame}
%%%%%%%%%%%%%%%%%%%%%%%%%%%%%%

%%%%%%%%%%%%%%%%%%%%%%%%%%%%%%
\begin{frame}{}
  \begin{theorem}{}
  \end{theorem}
\end{frame}
%%%%%%%%%%%%%%%%%%%%%%%%%%%%%%