% special-elements.tex

%%%%%%%%%%%%%%%
\begin{frame}{}
  \begin{definition}[极大元与极小元 (Maximal/Minimal)]
    令 $R \subseteq X \times X$ 是 $X$ 上的偏序。
    设 $a \in X$。\\[5pt]

    \pause
    如果
    \[
      \forall x \in X.\; \lnot \big(x \neq a \land (a, x) \in R \big),
    \]
    \[
      \forall x \in X.\; \big(x \neq a \to (a, x) \notin R \big),
    \]
    则称 $a$ 是 $X$ 的\red{极大元}。\\[10pt]

    \pause
    如果
    \[
      \forall x \in X.\; \lnot \big(x \neq a \land (x, a) \in R \big),
    \]
    \[
      \forall x \in X.\; \big(x \neq a \to (x, a) \notin R \big),
    \]
    则称 $a$ 是 $X$ 的\red{极小元}。
  \end{definition}

  \pause
  \begin{center}
    \red{$Q:$ 极大/极小元是否一定存在? 如果存在, 是否唯一?}
  \end{center}
\end{frame}
%%%%%%%%%%%%%%%

%%%%%%%%%%%%%%%
\begin{frame}{}
  \begin{exampleblock}{}
    \[
      (\R, \le)
    \]
    \pause
    \begin{center}
      无极大元、无极小元
    \end{center}
  \end{exampleblock}

  \pause
  \begin{exampleblock}{}
    \[
      (\N, \le)
    \]
    \pause
    \begin{center}
      无极大元、有唯一极小元 $0$
    \end{center}
  \end{exampleblock}

  \pause
  \begin{exampleblock}{}
    \[
      (\set{2, 3, 4, \dots} = \N \setminus \set{0, 1}, \mid)
    \]
    \pause
    \begin{center}
      无极大元、有无穷多个极小元 (所有的素数)
    \end{center}
  \end{exampleblock}
\end{frame}
%%%%%%%%%%%%%%%

%%%%%%%%%%%%%%%
\begin{frame}{}
  \begin{definition}[最大元与最小元 (Greatest/Least)]
    令 $R \subseteq X \times X$ 是 $X$ 上的偏序。
    设 $a \in X$。\\[5pt]

    \pause
    如果
    \[
      \forall x \in X.\; (x, a) \in R,
    \]
    则称 $a$ 是 $X$ 的\red{最大元}。\\[10pt]

    \pause
    如果
    \[
      \forall x \in X.\; (a, x) \in R,
    \]
    则称 $a$ 是 $X$ 的\red{最小元}。
  \end{definition}

  \pause
  \begin{center}
    \red{$Q:$ 最大/最小元是否一定存在? 如果存在, 是否唯一?}
  \end{center}
\end{frame}
%%%%%%%%%%%%%%%

%%%%%%%%%%%%%%%
\begin{frame}{}
  \begin{theorem}
    偏序集 $(X, R)$ 如果有最大元或最小元, 则它们是唯一的。
  \end{theorem}
\end{frame}
%%%%%%%%%%%%%%%

%%%%%%%%%%%%%%%
\begin{frame}{}
\end{frame}
%%%%%%%%%%%%%%%

%%%%%%%%%%%%%%%
\begin{frame}{}
\end{frame}
%%%%%%%%%%%%%%%

%%%%%%%%%%%%%%%
\begin{frame}{}
\end{frame}
%%%%%%%%%%%%%%%

%%%%%%%%%%%%%%%
\begin{frame}{}
\end{frame}
%%%%%%%%%%%%%%%