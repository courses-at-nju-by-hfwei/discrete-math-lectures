% special-elements.tex

%%%%%%%%%%%%%%%
\begin{frame}{}
  \begin{definition}[极大元与极小元 (Maximal/Minimal)]
    令 $\preceq \;\subseteq X \times X$ 是 $X$ 上的偏序。
    设 $a \in X$。\\[5pt]

    \pause
    如果
    \[
      \forall x \in X.\; \lnot \big(a \prec x \big),
    \]
    则称 $a$ 是 $X$ 的\red{极大元}。\\[10pt]

    \pause
    如果
    \[
      \forall x \in X.\; \lnot \big(x \prec a \big),
    \]
    则称 $a$ 是 $X$ 的\red{极小元}。
  \end{definition}

  \pause
  \vspace{0.30cm}
  \begin{center}
    \red{$Q:$ 极大/极小元是否一定存在? 如果存在, 是否唯一?}
  \end{center}
\end{frame}
%%%%%%%%%%%%%%%

%%%%%%%%%%%%%%%
\begin{frame}{}
  \begin{exampleblock}{}
    \[
      (\R, \le)
    \]
    \pause
    \begin{center}
      无极大元、无极小元
    \end{center}
  \end{exampleblock}

  \pause
  \begin{exampleblock}{}
    \[
      (\N, \le)
    \]
    \pause
    \begin{center}
      无极大元、有唯一极小元 $0$
    \end{center}
  \end{exampleblock}

  \pause
  \begin{exampleblock}{}
    \[
      (\set{2, 3, 4, \dots} = \N \setminus \set{0, 1}, \mid)
    \]
    \pause
    \begin{center}
      无极大元、有无穷多个极小元 (所有的素数)
    \end{center}
  \end{exampleblock}
\end{frame}
%%%%%%%%%%%%%%%

%%%%%%%%%%%%%%%
\begin{frame}{}
  \begin{definition}[最大元与最小元 (Maximum/Minimum; Greatest/Least; Largest/Smallest)]
    令 $\preceq \;\subseteq X \times X$ 是 $X$ 上的偏序。
    设 $a \in X$。\\[5pt]

    \pause
    如果
    \[
      \forall x \in X.\; x \preceq a,
    \]
    则称 $a$ 是 $X$ 的\red{最大元}。\\[10pt]

    \pause
    如果
    \[
      \forall x \in X.\; a \preceq x,
    \]
    则称 $a$ 是 $X$ 的\red{最小元}。
  \end{definition}

  \pause
  \vspace{0.30cm}
  \begin{center}
    \red{$Q:$ 最大/最小元是否一定存在? 如果存在, 是否唯一?}
  \end{center}
\end{frame}
%%%%%%%%%%%%%%%

%%%%%%%%%%%%%%%
\begin{frame}{}
  \begin{theorem}
    偏序集 $(X, \preceq)$ 如果有最大元或最小元, 则它们是唯一的。
  \end{theorem}

  \pause
  \vspace{0.30cm}
  \begin{center}
    \red{假设存在两个最大元 $x$ 与 $y$。}

    \pause
    \[
      x \preceq y \land y \preceq x \pause \implies x = y
    \]
  \end{center}
\end{frame}
%%%%%%%%%%%%%%%

%%%%%%%%%%%%%%%
\begin{frame}{}
  \begin{definition}[线性拓展 (Linear Extension)]
    设 $(X, \preceq)$ 是偏序集, $(X, \preceq')$ 是全序集。
    如果
    \[
      \forall x, y \in X.\; x \preceq y \to x \preceq' y,
    \]
    则称 $\preceq'$ 是 $\preceq$ 的\red{线性拓展}。
  \end{definition}

  \fig{width = 0.40\textwidth}{figs/dag-topo}

  \pause
  \vspace{-0.50cm}
  \[
    \blue{5, 7, 3, 11, 8, 2, 9, 10} \qquad \cyan{3, 5, 7, 8, 11, 2, 9, 10}
  \]
\end{frame}
%%%%%%%%%%%%%%%

%%%%%%%%%%%%%%%
\begin{frame}{}
  \begin{theorem}
    设 $(X, \preceq)$ 是偏序集且 \red{$X$是有限集}, 则 $\preceq$ 的线性拓展必定存在。
  \end{theorem}

  \fig{width = 0.40\textwidth}{figs/dag-topo}

  \pause
  \vspace{0.30cm}
  \begin{theorem}
    设 $(X, \preceq)$ 是偏序集且 \red{$X$是有限集}, 则极小元一定存在。
  \end{theorem}
\end{frame}
%%%%%%%%%%%%%%%