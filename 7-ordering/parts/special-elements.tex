% special-elements.tex

%%%%%%%%%%%%%%%
\begin{frame}{}
  \begin{definition}[极大元与极小元 (Maximal/Minimal)]
    令 $\preceq \;\subseteq X \times X$ 是 $X$ 上的偏序。
    设 $a \in X$。\\[5pt]

    \pause
    如果
    \[
      \forall x \in X.\; \lnot \big(a \prec x \big),
    \]
    则称 $a$ 是 $X$ 的\red{极大元}。\\[10pt]

    \pause
    如果
    \[
      \forall x \in X.\; \lnot \big(x \prec a \big),
    \]
    则称 $a$ 是 $X$ 的\red{极小元}。
  \end{definition}

  \pause
  \vspace{0.30cm}
  \begin{center}
    \red{$Q:$ 极大/极小元是否一定存在? 如果存在, 是否唯一?}
  \end{center}
\end{frame}
%%%%%%%%%%%%%%%

%%%%%%%%%%%%%%%
\begin{frame}{}
  \fig{width = 0.60\textwidth}{figs/powerset}
\end{frame}
%%%%%%%%%%%%%%%

%%%%%%%%%%%%%%%
\begin{frame}{}
  \begin{exampleblock}{}
    \[
      (\R, \le)
    \]
    \pause
    \begin{center}
      无极大元、无极小元
    \end{center}
  \end{exampleblock}

  \pause
  \vspace{0.30cm}
  \begin{exampleblock}{}
    \[
      (\N, \le)
    \]
    \pause
    \begin{center}
      无极大元、有唯一极小元 $0$
    \end{center}
  \end{exampleblock}
\end{frame}
%%%%%%%%%%%%%%%

%%%%%%%%%%%%%%%
\begin{frame}{}
  \begin{exampleblock}{}
    \[
      (\set{2, 3, 4, \dots} = \N \setminus \set{0, 1}, \mid)
    \]
    \pause
    \begin{center}
      无极大元、有无穷多个极小元 (所有的素数)
    \end{center}
  \end{exampleblock}

  \pause
  \fig{width = 0.40\textwidth}{figs/infinite-divisors}
\end{frame}
%%%%%%%%%%%%%%%

%%%%%%%%%%%%%%%
\begin{frame}{}
  \begin{definition}[最大元与最小元 (Maximum/Minimum; Greatest/Least; Largest/Smallest)]
    令 $\preceq \;\subseteq X \times X$ 是 $X$ 上的偏序。
    设 $a \in X$。\\[5pt]

    \pause
    如果
    \[
      \forall x \in X.\; x \preceq a,
    \]
    则称 $a$ 是 $X$ 的\red{最大元}。\\[10pt]

    \pause
    如果
    \[
      \forall x \in X.\; a \preceq x,
    \]
    则称 $a$ 是 $X$ 的\red{最小元}。
  \end{definition}

  \pause
  \vspace{0.30cm}
  \begin{center}
    \red{$Q:$ 最大/最小元是否一定存在? 如果存在, 是否唯一?}
  \end{center}
\end{frame}
%%%%%%%%%%%%%%%

%%%%%%%%%%%%%%%
\begin{frame}{}
  \begin{theorem}
    偏序集 $(X, \preceq)$ 如果有最大元或最小元, 则它们是唯一的。
  \end{theorem}

  \pause
  \vspace{0.30cm}
  \begin{center}
    \red{假设存在两个最大元 $x$ 与 $y$。}

    \pause
    \[
      x \preceq y \land y \preceq x \pause \implies x = y
    \]
    \vspace{-0.60cm}
    \begin{center}
      \cyan{(反对称性)}
    \end{center}
  \end{center}
\end{frame}
%%%%%%%%%%%%%%%

%%%%%%%%%%%%%%%
\begin{frame}{}
  \begin{definition}[线性拓展 (Linear Extension)]
    设 $(X, \preceq)$ 是偏序集, $(X, \preceq')$ 是全序集。
    如果
    \[
      \forall x, y \in X.\; x \preceq y \to x \preceq' y,
    \]
    则称 $\preceq'$ 是 $\preceq$ 的\red{线性拓展}。
  \end{definition}

  \fig{width = 0.40\textwidth}{figs/dag-topo}

  \pause
  \vspace{-0.50cm}
  \[
    \blue{5, 7, 3, 11, 8, 2, 9, 10} \qquad \cyan{3, 5, 7, 8, 11, 2, 9, 10}
  \]
\end{frame}
%%%%%%%%%%%%%%%

%%%%%%%%%%%%%%%
\begin{frame}{}
  \begin{theorem}
    设 $(X, \preceq)$ 是偏序集且 \red{$X$是有限集}, 则 $\preceq$ 的线性拓展必定存在。
  \end{theorem}

  \fig{width = 0.40\textwidth}{figs/dag-topo}

  \pause
  \vspace{0.30cm}
  \begin{theorem}
    设 $(X, \preceq)$ 是偏序集且 \red{$X$是有限集}, 则极小元一定存在。
  \end{theorem}
\end{frame}
%%%%%%%%%%%%%%%

%%%%%%%%%%%%%%%
\begin{frame}{}
  \begin{definition}[上界 (Upper Bound) 与下界 (Lower Bound)]
    设 $(X, \preceq)$ 是\blue{偏序集}。对于 $Y \subseteq X$, 如果
    \[
      \exists \red{x \in X}.\; \forall y \in Y.\; y \preceq x,
    \]
    则称 $x$ 为 $Y$ 的\red{上界}。

    \vspace{0.30cm}
    类似地, 如果
    \[
      \exists \red{x \in X}.\; \forall y \in Y.\; x \preceq y,
    \]
    则称 $x$ 为 $Y$ 的\red{下界}。
  \end{definition}

  \fig{width = 0.55\textwidth}{figs/bound}
\end{frame}
%%%%%%%%%%%%%%%

%%%%%%%%%%%%%%%
\begin{frame}{}
  \begin{definition}[最小上界 (Least Upper Bound; LUB)]
    设 $(X, \preceq)$ 是\blue{偏序集}, $Y \subseteq X$ 是 $X$ 的子集,
    $x \in X$ 是 $Y$ 的任一\cyan{上界}。 \\[5pt]
    如果对于 $B$ 的所有\cyan{下界} $x'$, 均有 $x \preceq x'$,
    则称 $x$ 是 $Y$ 的\red{最小上界}。
  \end{definition}

  \fig{width = 0.55\textwidth}{figs/bound}

  \begin{definition}[最大下界 (Greatest Lower Bound; GLB)]
    设 $(X, \preceq)$ 是\blue{偏序集}, $Y \subseteq X$ 是 $X$ 的子集,
    $x \in X$ 是 $Y$ 的任一\cyan{下界}。 \\[5pt]
    如果对于 $B$ 的所有\cyan{下界} $x'$, 均有 $x' \preceq x$,
    则称 $x$ 是 $Y$ 的\red{最大下界}。
  \end{definition}
\end{frame}
%%%%%%%%%%%%%%%

%%%%%%%%%%%%%%%
\begin{frame}{}
  \begin{definition}[格 (Lattice)]
    设 $(X, \preceq)$ 是\blue{偏序集}。\\[6pt]
    如果任意两个元素都有最小上界与最大下界, 则称 $(X, \preceq)$ 为\red{格}。
  \end{definition}

  \begin{columns}
    \column{0.50\textwidth}
      \pause
      \fig{width = 0.80\textwidth}{figs/Hasse-Powerset}
    \column{0.50\textwidth}
      \pause
      \fig{width = 0.90\textwidth}{figs/60-poset}
  \end{columns}

  \pause
  \vspace{-0.60cm}
  \fig{width = 0.25\textwidth}{figs/lattice-integer}
\end{frame}
%%%%%%%%%%%%%%%

%%%%%%%%%%%%%%%
\begin{frame}{}
  \fig{width = 0.50\textwidth}{figs/stop-here}
\end{frame}
%%%%%%%%%%%%%%%