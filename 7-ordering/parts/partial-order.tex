% partial-order.tex

%%%%%%%%%%%%%%%
\begin{frame}{}
  \[
    X = \set{1, 2, 3, 4, 5, 6, 10, 12, 15, 20, 30, 60}
  \]
  \[
    X \text{ 上的\red{整除}关系}
  \]

  \fig{width = 0.70\textwidth}{figs/60-poset}
\end{frame}
%%%%%%%%%%%%%%%

%%%%%%%%%%%%%%%
\begin{frame}{}
  \[
    S = \set{x, y, z}
  \]
  \[
    \ps{S} \text{ 上的\red{包含 $\subseteq$}关系}
  \]

  \pause
  \fig{width = 0.60\textwidth}{figs/Hasse-Powerset}
\end{frame}
%%%%%%%%%%%%%%%

%%%%%%%%%%%%%%%
\begin{frame}{}
  \begin{definition}[偏序关系 (Partial Order/Ordering)]
    令 $\red{\le} \;\subseteq X \times X$ 是 $X$ 上的二元关系。\\[3pt]
    如果 $R$ 满足以下条件, 则称 $R$ 是 $X$ 上的 \red{\bf 偏序 (关系)}: \\[6pt]
    \begin{enumerate}[(1)]
      \setlength{\itemsep}{6pt}
      \item $R$ 是\red{自反 (irreflexive)}的。
        \[
          \forall x \in X.\; x R x.
        \]
      \item $R$ 是\red{反对称 (antisymmetric)}的。
        \[
          \forall x, y \in X.\; x R y \land y R x \to x = y.
        \]
      \item $R$ 是\red{传递 (transitive)}的。
        \[
          \forall x, y, z \in X.\; x R y \land y R z \to x R z.
        \]
    \end{enumerate}
  \end{definition}
\end{frame}
%%%%%%%%%%%%%%%

%%%%%%%%%%%%%%%
\begin{frame}{}
  \[
    R \qquad \text{\it vs.} \qquad \le
  \]

  \pause
  \[
    R^{-1} \qquad \text{\it vs.} \qquad \ge \;\triangleq\; \le^{-1}
  \]

  \pause
  \[
    a < b \;\triangleq a \le b \land a \neq b
  \]
  \[
    a R b \land a \neq b
  \]
\end{frame}
%%%%%%%%%%%%%%%

%%%%%%%%%%%%%%%
\begin{frame}{}
  \begin{definition}[严格偏序关系 (Strict Partial Order/Ordering)]
    令 $R \subseteq X \times X$ 是 $X$ 上的二元关系。\\[3pt]
    如果 $R$ 满足以下条件, 则称 $R$ 是 $X$ 上的 \red{\bf 严格偏序 (关系)}: \\[6pt]
    \begin{enumerate}[(1)]
      \setlength{\itemsep}{6pt}
      \item $R$ 是\red{反自反 (irreflexive)}的。
        \[
          \forall x \in X.\; x \overline{R} x.
        \]
      \item \gray{$R$ 是\textcolor{red!20!gray}{反对称 (antisymmetric)}的。
        \[
          \forall x, y \in X.\; x R y \land y R x \to x = y.
        \]}
      \item $R$ 是\red{传递 (transitive)}的。
        \[
          \forall x, y, z \in X.\; x R y \land y R z \to x R z.
        \]
    \end{enumerate}
  \end{definition}

  \pause
  \vspace{-0.20cm}
  \[
    (1) + (3) \implies (2)
  \]
\end{frame}
%%%%%%%%%%%%%%%

%%%%%%%%%%%%%%%
\begin{frame}{}
  \begin{theorem}
    设 $R \subseteq X \times X$ 是 $X$ 上的\red{严格偏序}。\\[5pt]
    对于任意 $x, y, z \in X$: \\[6pt]
    \begin{enumerate}[(1)]
      \setlength{\itemsep}{6pt}
      \item $x < y$, $x = y$, $y < x$ 三者中\blue{\bf 至多有一个}成立;
      \item $x \le y \le x \to x = y$。
    \end{enumerate}
  \end{theorem}
\end{frame}
%%%%%%%%%%%%%%%

%%%%%%%%%%%%%%%
\begin{frame}{}
  \fig{width = 0.30\textwidth}{figs/lamport}

  \begin{center}
    \teal{Leslie Lamport (1941 $\sim$)} \\[3pt]
  \end{center}

  \begin{quote}
    For ``fundamental contributions to the theory and practice of
      \red{\bf distributed and concurrent systems},
      notably the invention of concepts
      such as causality and logical clocks,
      safety and liveness, replicated state machines,
      and sequential consistency''

    \hfill  --- \blue{Turing Award (2013)}
  \end{quote}
\end{frame}
%%%%%%%%%%%%%%%

%%%%%%%%%%%%%%%
\begin{frame}{}
  \fig{width = 0.75\textwidth}{figs/lamport-ordering}
  \begin{center}
    \red{引用量 $\ge 12525$}
  \end{center}
\end{frame}
%%%%%%%%%%%%%%%

%%%%%%%%%%%%%%%
\begin{frame}{}
  \fig{width = 0.65\textwidth}{figs/lamport-std}
\end{frame}
%%%%%%%%%%%%%%%

%%%%%%%%%%%%%%%
\begin{frame}{}
  \begin{definition}[事件间的``先于''关系 (``Happened Before'' Relation on Events)]
    令 $E$ 是表示事件的集合。\\[3pt]
    $\to \;\subseteq E \times E$ 是 $E$ 上的\red{``先于''}关系,
    如果它是满足以下条件的\red{最小}关系: \\[5pt]
    \begin{itemize}
      \setlength{\itemsep}{8pt}
      \item 对任何事件 $a \in E$, $a \not\to a$;
      \item 如果 $a, b$ 事件属于同一个进程且 $a$ 在 $b$ 之前发生, 则 $a \to b$;
      \item 如果 $a, b$ 分别表示同一个消息的发送事件与接收事件, 则 $a \to b$;
      \item 如果 $a \to b$ 且 $b \to c$, 则 $a \to c$。
    \end{itemize}
  \end{definition}

  \pause
  \vspace{0.30cm}
  \begin{theorem}
    $\to \;\subseteq E \times E$ 是 $E$ 上的严格偏序。
  \end{theorem}
\end{frame}
%%%%%%%%%%%%%%%

%%%%%%%%%%%%%%%
\begin{frame}{}
  \begin{definition}[偏序集 (Partially Ordered Set; Poset)]
    如果 $R \subseteq X \times X$ 是 $X$ 上的偏序,
    则称 $(X, R)$ 为偏序集。
  \end{definition}
\end{frame}
%%%%%%%%%%%%%%%