% partial-order.tex

%%%%%%%%%%%%%%%
\begin{frame}{}
  \[
    X = \set{1, 2, 3, 4, 5, 6, 10, 12, 15, 20, 30, 60}
  \]
  \[
    X \text{ 上的\red{整除}关系}
  \]

  \fig{width = 0.70\textwidth}{figs/60-poset}

  \pause
  \begin{center}
    \blue{自反 + 反对称 + 传递}
  \end{center}
\end{frame}
%%%%%%%%%%%%%%%

%%%%%%%%%%%%%%%
\begin{frame}{}
  \[
    S = \set{x, y, z}
  \]
  \[
    \ps{S} \text{ 上的\red{包含 $\subseteq$}关系}
  \]

  \pause
  \fig{width = 0.60\textwidth}{figs/Hasse-Powerset}

  \pause
  \begin{center}
    \blue{自反 + 反对称 + 传递}
  \end{center}
\end{frame}
%%%%%%%%%%%%%%%

%%%%%%%%%%%%%%%
\begin{frame}{}
  \begin{definition}[偏序关系 (Partial Order)]
    令 $\red{\preceq} \;\subseteq X \times X$ 是 $X$ 上的二元关系。\\[3pt]
    如果 $\preceq$ 满足以下条件, 则称 $\preceq$ 是 $X$ 上的\red{偏序关系}, \\[3pt]
    并称 $(X, \preceq)$ 为\red{偏序集 (poset; Partially Ordered Set)}: \\[6pt]
    \begin{enumerate}[(1)]
      \setlength{\itemsep}{6pt}
      \item $\preceq$ 是\blue{自反 (irreflexive)}的。
        \[
          \forall x \in X.\; x \preceq x.
        \]
      \item $\preceq$ 是\blue{反对称 (antisymmetric)}的。
        \[
          \forall x, y \in X.\; x \preceq y \land y \preceq x \to x = y.
        \]
      \item $\preceq$ 是\blue{传递 (transitive)}的。
        \[
          \forall x, y, z \in X.\; x \preceq y \land y \preceq z \to x \preceq z.
        \]
    \end{enumerate}
  \end{definition}
\end{frame}
%%%%%%%%%%%%%%%

%%%%%%%%%%%%%%%
\begin{frame}{}
  \fig{width = 0.35\textwidth}{figs/dag}

  \begin{center}
    有向无环图 (DAG; Directed Acyclic Graph) 上的 \\[3pt]
    \red{可达 (reachability) 关系}
  \end{center}
\end{frame}
%%%%%%%%%%%%%%%

%%%%%%%%%%%%%%%
\begin{frame}{}
  \begin{definition}
    设 $(X, \preceq)$ 是偏序集。
    对任意 $a, b \in X$,

    \begin{description}
      \item[严格小于 (strictly less than):]
        \[
          a \prec b \triangleq a \preceq b \land a \neq b
        \]
      \item[$a$被$b$覆盖 (covered by):]
        \[
          a \prec b \land
            (\forall c \in X.\; (c \neq a \land c \neq b) \to \lnot (a \preceq c \preceq b))
        \]
    \end{description}
  \end{definition}

  \fig{width = 0.40\textwidth}{figs/Hasse-Powerset}
\end{frame}
%%%%%%%%%%%%%%%

%%%%%%%%%%%%%%%
\begin{frame}{}
  \begin{definition}
    设 $(X, \preceq)$ 是偏序集。
    对 $a, b \in X$,

    \begin{description}
      \item[可比的 (Comparable):]
        \[
          a \preceq b \lor b \preceq a
        \]
      \item[不可比的 (Incomparable):]
        \[
          \lnot(a \preceq b \lor b \preceq a)
        \]
    \end{description}
  \end{definition}

  \fig{width = 0.50\textwidth}{figs/Hasse-Powerset}
\end{frame}
%%%%%%%%%%%%%%%

%%%%%%%%%%%%%%%
\begin{frame}{}
  \begin{definition}[链与反链 (Chain; Antichain)]
    设 $(X, \preceq)$ 是偏序集。

    \begin{itemize}
      \setlength{\itemsep}{6pt}
      \item 设 $S \subseteq X$ 且 $S$ 中元素两两\blue{可比}, 则称 $S$ 是\red{链}。
      \item 设 $S \subseteq X$ 且 $S$ 中元素两两\blue{不可比}, 则称 $S$ 是\red{反链}。
    \end{itemize}
  \end{definition}

  \begin{center}
    规定: 单元素集合既是链, 也是反链
  \end{center}

  \fig{width = 0.50\textwidth}{figs/Hasse-Powerset}
\end{frame}
%%%%%%%%%%%%%%%

%%%%%%%%%%%%%%%
\begin{frame}{}
  \[
    \set{\blue{\set{\emptyset, \set{x}, \set{x, y}, \set{x, y, z}}},
      \red{\set{\set{y}, \set{y, z}}}, \cyan{\set{\set{z}, \set{x, z}}}}
  \]
  \fig{width = 0.50\textwidth}{figs/Hasse-Powerset}
  \[
    \purple{\set{\set{x}, \set{y}, \set{z}}}
  \]

  \begin{theorem}[Dilworth's Theorem]
    \red{最大}\blue{反链}的大小 = \red{最小}\blue{链分解}中链的条数
  \end{theorem}
\end{frame}
%%%%%%%%%%%%%%%

%%%%%%%%%%%%%%%
\begin{frame}{}
  \begin{definition}[严格偏序关系 (Strict Partial Order)]
    令 $\prec\; \subseteq X \times X$ 是 $X$ 上的二元关系。\\[3pt]
    如果 $\prec$ 满足以下条件, 则称 $\prec$ 是 $X$ 上的\red{严格偏序关系}: \\[6pt]

    \begin{enumerate}[(1)]
      \setlength{\itemsep}{6pt}
      \item $\prec$ 是\purple{反自反 (irreflexive)}的。
        \[
          \forall x \in X.\; \lnot(x \prec x).
        \]
      \item \gray{$\prec$ 是\textcolor{blue!20!gray}{非对称 (asymmetric)}的。}
        \[
          \gray{\forall x, y \in X.\; x \prec y \to \lnot (y \prec x).}
        \]
      \item $\prec$ 是\blue{传递 (transitive)}的。
        \[
          \forall x, y, z \in X.\; x \prec y \land y \prec z \to x \prec z.
        \]
    \end{enumerate}
  \end{definition}
  \vspace{-0.20cm}
  \[
    (\ps{X}, \subset)
  \]

  \pause
  \vspace{-0.30cm}
  \[
    (1) + (3) \implies (2)
  \]
\end{frame}
%%%%%%%%%%%%%%%

%%%%%%%%%%%%%%%
\begin{frame}{}
  \begin{theorem}
    设 $\prec\; \subseteq X \times X$ 是 $X$ 上的\red{严格偏序关系}。\\[5pt]
    对于任意 $x, y, z \in X$: \\[6pt]
    \begin{enumerate}[(1)]
      \setlength{\itemsep}{6pt}
      \item $x \prec y$, $x = y$, $y \prec x$ 三者中\blue{至多有一个}成立;
      \item $(x \preceq y \land y \preceq x) \to x = y$。
        \[
          (x \preceq y \triangleq x \prec y \lor x = y)
        \]
    \end{enumerate}
  \end{theorem}

  \pause
  \begin{align*}
    & x \preceq y \land y \preceq x \\[5pt]
    \iff & (x \prec y \lor x = y) \land (y \prec x \lor x = y) \\[5pt]
    \iff & x = y \lor (x \prec y \land y \prec x) \\[5pt]
    \implies & x = y \lor \text{False} \\[5pt]
    \implies & x = y
  \end{align*}
\end{frame}
%%%%%%%%%%%%%%%